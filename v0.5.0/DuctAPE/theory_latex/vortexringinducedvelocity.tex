\section{Detailed derivation of vortex ring induced velocity}
\label{app:ringvortexinducedvelocities}

As mentioned at the end of \cref{ssec:ringvortices}, in this section we go through the somewhat tedious algebra required to take the partial derivatives of the vector potential in order to obtain the induced velocity from a vortex ring.
%
We begin with the terms remaining in \cref{ssec:ringvortices}:

\begin{align}
    v_r &= -\pd{\psi_\theta}{z}, \\
    v_z &= \frac{1}{r}\pd{(r\psi_\theta)}{r}.
\end{align}

Now we need to take these partial derivatives to arrive at our final expressions of induced velocity.
%
Because our current vector potential expressions are in terms of \(m\) and normalized values, we will require the application of the chain rule.
%
Therefore it will be important to have the expressions for the various partial derivatives along the way.
%
The derivative of the elliptic integral of the first kind with respect to \(m\) is

\begin{equation}
    \label{eqn:dkdm}
    \pd{\mathcal{K}(m)}{m} = \frac{\mathcal{E}(m)}{2m(1-m)} - \frac{\mathcal{K}(m)}{2m}.
\end{equation}

\noindent The derivative of the elliptic integral of the second kind is

\begin{equation}
    \label{eqn:dkdm2}
    \pd{\mathcal{E}(m)}{m} = \frac{\mathcal{E}(m)}{2m} - \frac{\mathcal{K}(m)}{2m}.
\end{equation}

\noindent The partial of \(m\) with respect to \(\xi\) is

\begin{equation}
    \pd{m}{\xi} = - \frac{8 \rho \xi }{\left((\rho+1)^2+\xi^2\right)^2}.
\end{equation}

\noindent The partial of \(\xi\) with respect to \(z\) is

\begin{equation}
    \pd{\xi}{z} = \frac{1}{r_o}.
\end{equation}


\noindent The partial of \(m\) with respect to \(\rho\) is

\begin{equation}
    \pd{m}{\rho} = \frac{4\left(-\rho^2 + \xi^2 + 1\right)}{\left((\rho +1)^2 + \xi^2 \right)^2}.
\end{equation}

\noindent The partial of \(\rho\) with respect to \(r\) is

\begin{equation}
    \pd{\rho}{r} = \frac{1}{r_o}.
\end{equation}

Though simple to write symbolically, the overall derivatives become very cumbersome.
%
To keep things manageable, let us separate out the expression for \(\psi_\theta\) into the constant (\(\mathcal{C}\)), numerator (\(\mathcal{N}\)), and denominator (\(\mathcal{D}\)) portions, respectively:

\begin{equation}
    \label{eqn:cnd}
\begin{aligned}
    \mathcal{\mathcal{C}} &= -\frac{\Gamma}{\pi } \\
    \mathcal{\mathcal{N}} &= \frac{2}{m}\mathcal{E}(m) - \frac{2}{m}\mathcal{K}(m) + \mathcal{K}(m) \\
    \mathcal{\mathcal{D}} &= \left[(\rho+1)^2+\xi^2\right]^{1/2}.
\end{aligned}
\end{equation}

\noindent The partial of the numerator with respect to \(z\) is

\begin{equation}
    \begin{split}
        \pd{\mathcal{N}}{z} =& -\pd{m}{\xi}\pd{\xi}{z} \bigg[ \frac{\mathcal{K}(m)+\mathcal{E}(m)}{m^2}\\
         &- \frac{3\mathcal{K}(m)(m-1)+\mathcal{E}(m)}{m^2(m-1)}\\
         &+ \frac{\mathcal{K}(m)(m-1)+\mathcal{E}(m)}{2m(m-1)} \bigg].
    \end{split}
\end{equation}

\noindent The partial of the denominator with respect to \(z\) is

\begin{equation}
    \pd{\mathcal{D}}{z} = \pd{\xi}{z} \frac{\xi}{\mathcal{D}} = \frac{\xi}{r_o\mathcal{D}}.
\end{equation}

\noindent The partial of the numerator with respect to \(r\) is

\begin{equation}
    \begin{split}
        \pd{\mathcal{N}}{r} =& \pd{m}{\rho}\pd{\rho}{r} \bigg[ -\frac{3\mathcal{E}(m) + (m-5)\mathcal{K}(m)}{2m(m-1)} \\
         &+ \frac{2\mathcal{E}(m) - 2\mathcal{K}(m)}{m^2(m-1)} \bigg]\\
    \end{split}
\end{equation}

\noindent The partial of the denominator with respect to \(r\) is

\begin{equation}
    \pd{\mathcal{D}}{r} = \pd{\rho}{r}\frac{\rho+1}{\mathcal{D}} = \frac{\rho+1}{r_o\mathcal{D}}.
\end{equation}

\noindent Putting things together for \(v_r\) with the quotient rule gives

\begin{equation}
    \label{eqn:vrlong}
        v_r = -\pd{\psi_\theta}{z} = -\mathcal{C}\frac{\pd{\mathcal{N}}{z}\mathcal{D} - \mathcal{N}\pd{\mathcal{D}}{z}}{\mathcal{D}^2}.
\end{equation}

\noindent Putting things together for \(v_z\) we start with the quotient rule, then apply the product rule to arrive at

\begin{equation}
    \label{eqn:vzlong}
    \begin{aligned}
        v_z =& \frac{1}{r}\pd{(r\psi_\theta)}{r} \\
        =& \frac{\mathcal{C}}{r}\frac{\pd{(\mathcal{N}r)}{r}\mathcal{D} - (\mathcal{N}r)\pd{\mathcal{D}}{r}}{\mathcal{D}^2} \\
        =& \mathcal{C}\frac{\left(\mathcal{N} + r\pd{\mathcal{N}}{r}\right)\mathcal{D} - (\mathcal{N}r)\pd{\mathcal{D}}{r}}{r\mathcal{D}^2}.
\end{aligned}
\end{equation}

\subsubsection{Radially Induced Velocity Component}

Now let's see what we can do about simplifying these expressions.
%
We'll start with \cref{eqn:vrlong}.
%
To get started, we'll split up the fraction, and expand out the partial of \(\mathcal{D}\),

\begin{equation}
    \label{eqn:vr2}
    v_r  = -C \left[ \underbrace{\frac{\pd{\mathcal{N}}{z}}{\mathcal{D}}}_\text{Term 1}
    -\underbrace{\frac{\mathcal{N}\xi}{r_oD^3}}_\text{Term 2}\right]
\end{equation}

\noindent We are going to look at each term in the brackets of \cref{eqn:vr2} separately first, then bring them together.
%
We'll start with Term 2.
%
Expanding gives

\begin{equation}
    \label{eqn:vrt2.1}
    \frac{N\xi}{r_oD^3} =  \frac{\xi}{r_oD^3} \left[\frac{2}{m}\mathcal{E}(m) + \left(1-\frac{2}{m}\right)\mathcal{K}(m)\right]
\end{equation}

\noindent Let us address the \(m\)'s in the denominators by realizing that a comparison of \cref{eqn:mdef,eqn:cnd} indicates that

\begin{equation}
    \mathcal{D}^2 = \frac{4\rho}{m}.
\end{equation}

\noindent Making this replacement in \cref{eqn:vrt2.1} gives

\begin{equation}
    \label{eqn:term2simple}
    \begin{aligned}
        \frac{N\xi}{r_oD^3} &= \frac{\xi \cancel{m}}{4\rho r_o D} \left[ \frac{2}{\cancel{m}}\mathcal{E}(m) + \left(\cancelto{m}{1} - \frac{2}{\cancel{m}} \right)\mathcal{K}(m) \right] \\
        &= \frac{\xi/\rho }{4 r_o D} \left[ 2\mathcal{E}(m) + \left(m - 2 \right)\mathcal{K}(m) \right]
    \end{aligned}
\end{equation}

Now let's look at Term 1 from \cref{eqn:vrlong}.
%
Expanding out gives

\begin{equation}
    \begin{split}
        \frac{\pd{\mathcal{N}}{z}}{\mathcal{D}} =& \frac{8\rho \xi}{r_o D \left((\rho+1)^2 + \xi^2\right)^2} \bigg[ \frac{\mathcal{K}(m)+\mathcal{E}(m)}{m^2}\\
         &- \frac{3\mathcal{K}(m)(m-1)+\mathcal{E}(m)}{m^2(m-1)}\\
         &+ \frac{\mathcal{K}(m)(m-1)+\mathcal{E}(m)}{2m(m-1)} \bigg].
    \end{split}
\end{equation}

\noindent We can see right away that part of the fraction outside of the brackets closely resembles the parameter \(m^2\), all we're missing is \(2\rho\) in the numerator, so we'll multiply and divide by \(2 \rho\).

\begin{equation}
    \begin{split}
        \frac{\pd{\mathcal{N}}{z}}{\mathcal{D}} =& \frac{(2\rho)8\rho \xi}{2\rho r_o D \left((\rho+1)^2 + \xi^2\right)^2} \bigg[ \frac{\mathcal{K}(m)+\mathcal{E}(m)}{m^2}\\
         &- \frac{3\mathcal{K}(m)(m-1)+\mathcal{E}(m)}{m^2(m-1)}\\
         &+ \frac{\mathcal{K}(m)(m-1)+\mathcal{E}(m)}{2m(m-1)} \bigg].
    \end{split}
\end{equation}

\noindent which allows us to remove some of the \(m^2\) denominators inside the brackets

\begin{equation}
    \begin{split}
        \frac{\pd{\mathcal{N}}{z}}{\mathcal{D}} =& \frac{\xi\cancel{m^2}}{2\rho r_o D} \bigg[ \frac{\mathcal{K}(m)+\mathcal{E}(m)}{\cancel{m^2}}\\
         &- \frac{3\mathcal{K}(m)(m-1)+\mathcal{E}(m)}{\cancel{m^2}(m-1)}\\
         &+ \frac{m(\mathcal{K}(m)(m-1)+\mathcal{E}(m))}{2\cancel{m}(m-1)} \bigg].
    \end{split}
\end{equation}

\begin{equation}
    \begin{split}
        \frac{\pd{\mathcal{N}}{z}}{\mathcal{D}} =& \frac{\xi/\rho}{2r_o D} \bigg[ \mathcal{K}(m)+\mathcal{E}(m)\\
         &- \frac{3\mathcal{K}(m)(m-1)+\mathcal{E}(m)}{m-1}\\
         &+ \frac{m(\mathcal{K}(m)(m-1)+\mathcal{E}(m))}{2(m-1)} \bigg].
    \end{split}
\end{equation}

\noindent Splitting up the fractions inside the brackets will let us simplify further.

\begin{equation}
    \begin{split}
        \frac{\pd{\mathcal{N}}{z}}{\mathcal{D}} =& \frac{\xi/\rho}{2r_o D} \bigg[ \mathcal{K}(m)+\mathcal{E}(m)\\
         &- \frac{3\mathcal{K}(m)\cancel{(m-1)}}{\cancel{m-1}} - \frac{\mathcal{E}(m)}{m-1}\\
         &+ \frac{m\mathcal{K}(m)\cancel{(m-1)}}{2\cancel{(m-1)}} + \frac{m\mathcal{E}(m)}{2(m-1)} \bigg].
\end{split}
\end{equation}

\begin{equation}
    \begin{split}
        \frac{\pd{\mathcal{N}}{z}}{\mathcal{D}} =& \frac{\xi/\rho}{2r_o D} \bigg[ \mathcal{K}(m)+\mathcal{E}(m) - 3\mathcal{K}(m) - \frac{1}{m-1}\mathcal{E}(m)\\
         &+ \frac{m}{2}\mathcal{K}(m) + \frac{m}{2(m-1)} \mathcal{E}(m) \bigg].
\end{split}
\end{equation}

\noindent Grouping like terms

\begin{equation}
    \begin{split}
        \frac{\pd{\mathcal{N}}{z}}{\mathcal{D}} =& \frac{\xi/\rho}{2r_o D} \bigg[ \left(1- \frac{1}{m-1}+ \frac{m}{2(m-1)} \right)\mathcal{E}(m) \\
         &+ \left(\frac{m}{2}-2\right)\mathcal{K}(m) \bigg].
\end{split}
\end{equation}

\noindent Simplifying the gathered terms for \(\mathcal{E}(m)\)

\begin{equation}
    \begin{split}
        \frac{\pd{\mathcal{N}}{z}}{\mathcal{D}} =& \frac{\xi/\rho}{2r_o D} \bigg[ \left(\frac{2(m-1)- 2+ m}{2(m-1)} \right)\mathcal{E}(m) \\
         &- \left(\frac{m}{2}-2\right)\mathcal{K}(m) \bigg].
\end{split}
\end{equation}

\begin{equation}
    \begin{split}
        \frac{\pd{\mathcal{N}}{z}}{\mathcal{D}} =& \frac{\xi/\rho}{2r_o D} \bigg[ \left(\frac{3m-4}{2(m-1)} \right)\mathcal{E}(m) \\
         &- \left(\frac{m}{2}-2\right)\mathcal{K}(m) \bigg].
\end{split}
\end{equation}

\noindent Multiplying and dividing by 2

\begin{equation}
    \label{eqn:term1simple}
        \frac{\pd{\mathcal{N}}{z}}{\mathcal{D}} = \frac{\xi/\rho}{4 r_o D} \left[ \left(\frac{3m-4}{m-1} \right)\mathcal{E}(m) - (m-4)\mathcal{K}(m) \right].
\end{equation}

Noting that the fractions outside of the brackets are now the same for both of the simplified expressions for Term 1 (see \cref{eqn:term1simple}) and Term 2 (see \cref{eqn:term2simple}, we'll substitute the expression for Term 2 in \cref{eqn:term2simple} and the expression for Term 1 in \cref{eqn:term1simple} back in to \cref{eqn:vr2}.
%

\begin{equation}
    \begin{split}
\frac{\pd{\mathcal{N}}{z}\mathcal{D} - \mathcal{N}\pd{\mathcal{D}}{z}}{\mathcal{D}^2} = \frac{\xi/\rho}{4 r_o D} \left[ \left(\frac{3m-4}{m-1} \right)\mathcal{E}(m) - (m-4)\mathcal{K}(m) \right] \\
        - \frac{\xi/\rho }{4 r_o D} \left[ 2\mathcal{E}(m) + \left(m - 2 \right)\mathcal{K}(m) \right].
    \end{split}
\end{equation}

\noindent Let's first look at just the difference of the \(\mathcal{K}(m)\) terms:

\begin{equation}
    \label{eqn:kterms}
    ((m-1) - (m-2))K = (m - 4 - m + 2)\mathcal{K}(m) = -2\mathcal{K}(m)
\end{equation}

\noindent Now just looking at the \(\mathcal{E}(m)\) terms:

\begin{equation}
    \label{eqn:eterms}
    \begin{aligned}
        \left(\frac{3m-4}{m-1} - 2\right)\mathcal{E}(m) \\
        \left(\frac{3m-4-2(m-1)}{m-1} \right)\mathcal{E}(m) && \text{(common denominators)} \\
        \left(\frac{3m-4-2m+2}{m-1} \right)\mathcal{E}(m) && \text{(expand)} \\
        \left(\frac{m-2}{m-1} \right)\mathcal{E}(m) && \text{(simplify)}
    \end{aligned}
\end{equation}

\noindent Applying the definition of \(m\) in \cref{eqn:mdef}

\begin{equation}
    \label{eqn:eterms2}
    \begin{aligned}
        \left(\frac{\frac{4\rho}{\mathcal{D}^2}-2}{\frac{4\rho}{\mathcal{D}^2}-1} \right)\mathcal{E}(m) \\
        \left(\frac{\frac{4\rho-2\mathcal{D}^2}{\cancel{\mathcal{D}^2}}}{\frac{4\rho-\mathcal{D}^2}{\cancel{\mathcal{D}^2}}} \right)\mathcal{E}(m) && \text{(common denominators)}\\
        \left(\frac{4\rho-2\mathcal{D}^2}{4\rho-\mathcal{D}^2} \right)\mathcal{E}(m) && \text{(divide)} \\
        2\left(\frac{2\rho-\mathcal{D}^2}{4\rho-\mathcal{D}^2} \right)\mathcal{E}(m) && \text{(pull out a 2)}
    \end{aligned}
\end{equation}

\noindent Expanding out the \(\mathcal{D}\) terms

\begin{equation}
    \label{eqn:eterms3}
    \begin{aligned}
        2\left(\frac{2\rho-((\rho+1)^2+\xi^2)}{4\rho-((\rho+1)^2+\xi^2)} \right)\mathcal{E}(m) \\
        2\left(\frac{\cancel{2\rho}-\rho^2-\cancel{2\rho}-1-\xi^2}{\cancelto{2\rho}{4\rho}-\rho^2 - \cancel{2\rho} -1 -\xi^2} \right)\mathcal{E}(m) && \text{(expand and cancel)} \\
        2\left(\frac{-\rho^2-1-\xi^2}{2\rho-\rho^2 - 1 -\xi^2} \right)\mathcal{E}(m)\\
        -2\left(\frac{-\rho^2-1-\xi^2}{(\rho- 1)^2 +\xi^2} \right)\mathcal{E}(m) && \text{(multiply by -1 and simplify)} \\
        -2\left(\frac{-\rho^2-1-\xi^2 + 2\rho - 2\rho}{(\rho- 1)^2 +\xi^2} \right)\mathcal{E}(m) && \text{(add and subtract }2\rho)\\
        2\left(\frac{(\rho-1)^2+\xi^2 + 2\rho}{(\rho- 1)^2 +\xi^2} \right)\mathcal{E}(m) && \text{(consolidate numerator)}\\
        2\left(\cancelto{1}{\frac{(\rho-1)^2+\xi^2 }{(\rho- 1)^2 +\xi^2}} +\frac{ 2\rho}{(\rho- 1)^2 +\xi^2} \right)\mathcal{E}(m) && \text{(split fraction and cancel)} \\
        2\left(1 + \frac{2\rho}{\xi^2 + (\rho- 1)^2} \right)\mathcal{E}(m) \\
    \end{aligned}
\end{equation}

\noindent Now putting the \(\mathcal{K}(m)\) terms from \cref{eqn:kterms} and the \(\mathcal{E}(m)\) terms from \cref{eqn:eterms3} together in the difference, remembering the fraction out front of both Term 1 and Term 2:

\begin{equation}
    v_r = -\mathcal{C} \left[\frac{\xi/\rho}{\cancelto{2}{4}\mathcal{D}r_o} \left(-\cancel{2}\mathcal{K}(m)+\cancel{2}\left[1+\frac{2\rho}{\xi^2+(\rho-1)^2}\right]\mathcal{E}(m) \right) \right]
\end{equation}

\noindent And finally expanding out \(\mathcal{C}\) and \(\mathcal{D}\) as well as some minor cleanup and rearranging we have the expression presented in \cref{eqn:ringvortexinducedvelocityradial}:

\begin{equation}
    \eqbox{
    v_r = -\frac{\Gamma}{2\pi r_o}\frac{\xi/\rho}{[\xi^2+(\rho+1)^2]^{1/2}} \left(\mathcal{K}(m)-\left[1+\frac{2\rho}{\xi^2+(\rho-1)^2}\right]\mathcal{E}(m) \right)
}
\end{equation}


\subsubsection{Axially Induced Velocity Component}

Next let's simplify \cref{eqn:vzlong}

\begin{equation}
v_z = \mathcal{C}\left[\underbrace{\frac{\mathcal{N}\mathcal{D}}{r\mathcal{D}^2}}_\text{Term 1} + \underbrace{\frac{r\pd{\mathcal{N}}{r}\mathcal{D}}{rD^2}}_\text{Term 2} - \underbrace{\frac{(\mathcal{N}r)\pd{\mathcal{D}}{r}}{r\mathcal{D}^2}}_\text{Term 3}\right].
\end{equation}

\noindent We'll first expand the partials.
%
Term 1 has no partials to expand:

\begin{equation}
    \text{Term 1} = \frac{\mathcal{N}\mathcal{D}}{r\mathcal{D}^2}.
\end{equation}

\noindent Term 2 has several sets of partials to expand:

\begin{equation}
    \begin{split}
    \text{Term 2} =& \frac{r\mathcal{D}}{r\mathcal{D}^2} \pd{m}{\rho}\pd{\rho}{r} \bigg[ -\frac{3\mathcal{E}(m) + (m-5)\mathcal{K}(m)}{2m(m-1)} \\
         &+ \frac{2\mathcal{E}(m) - 2\mathcal{K}(m)}{m^2(m-1)} \bigg],
    \end{split}
\end{equation}

\begin{equation}
    \begin{split}
        \text{Term 2} =& \frac{r\mathcal{D}}{r\mathcal{D}^2} \frac{4(-\rho^2+\xi^2 +1)}{\mathcal{D}^4}\frac{1}{r_o} \bigg[ -\frac{3\mathcal{E}(m) + (m-5)\mathcal{K}(m)}{2m(m-1)} \\
         &+ \frac{2\mathcal{E}(m) - 2\mathcal{K}(m)}{m^2(m-1)} \bigg].
    \end{split}
\end{equation}

\noindent Term 3 also has a couple sets of partials to expand:

\begin{equation}
        \text{Term 3} = -\frac{\mathcal{N}r}{r\mathcal{D}^2}\pd{\rho}{r}\frac{\rho+1}{\mathcal{D}},
\end{equation}

\begin{equation}
        \text{Term 3} = -\frac{\mathcal{N}r}{r\mathcal{D}^2}\frac{1}{r_o}\frac{\rho+1}{\mathcal{D}}.
\end{equation}

Next let's expand out the \(\mathcal{N}\)'s.
%
For Term 1

\begin{equation}
    \text{Term 1} = \frac{\mathcal{D}}{r\mathcal{D}^2}\left[\frac{2}{m}\mathcal{E}(m) + \frac{m-2}{m}\mathcal{K}(m)\right].
\end{equation}

\noindent Term 2 is already expanded, but let us gather the \(\mathcal{E}(m)\) and \(\mathcal{K}(m)\) terms.

\begin{equation}
    \begin{split}
        \text{Term 2} = \frac{r\mathcal{D}}{r\mathcal{D}^2} \frac{4(-\rho^2+\xi^2 +1)}{\mathcal{D}^4}\frac{1}{r_o} \bigg[ -&\frac{3m-4}{2m^2(m-1)}\mathcal{E}(m) \\
                   &- \frac{(m-4)(m-1)}{2m^2(m-1)}\mathcal{K}(m) \bigg].
    \end{split}
\end{equation}

\noindent For Term 3:

\begin{equation}
        \text{Term 3} = -\frac{r}{r\mathcal{D}^2}\frac{1}{r_o}\frac{\rho+1}{\mathcal{D}}\left[\frac{2}{m}\mathcal{E}(m) + \frac{m-2}{m}\mathcal{K}(m)\right].
\end{equation}

In order to add the terms together, we require a common denominator.
%
Let us gather the multipliers of each of the terms to see what we're working with and decide what common denominator to choose.

\begin{align}
    \text{Term 1 Multiplier} &= \frac{\mathcal{D}}{r\mathcal{D}^2m}; \\
    \text{Term 2 Multiplier} &= \frac{4r\mathcal{D}(-\rho^2+\xi^2+1)}{2rr_o\mathcal{D}^6m^2(m-1)}; \\
    \text{Term 3 Multiplier} &= \frac{r(\rho+1)}{rr_o\mathcal{D}^3m}.
\end{align}

%
\noindent We may expect our final expression to look similar to the expression for \(v_r\), so we may want to make sure to keep a \(r_oD\) in the denominator as we go forward.
%
Therefore we'll start by multiplying Term 1 by \(r_o/r_o\):

\begin{align}
    \text{Term 1 Multiplier} &= \frac{\mathcal{D}r_o}{rr_o\mathcal{D}^2m}; \\
    \text{Term 2 Multiplier} &= \frac{4r\mathcal{D}(-\rho^2+\xi^2+1)}{2rr_o\mathcal{D}^6m^2(m-1)}; \\
    \text{Term 3 Multiplier} &= \frac{r(\rho+1)}{rr_o\mathcal{D}^3m}.
\end{align}

\noindent Term 2 seems to have some extraneous values, so let's divide out the 2 and one of the \(\mathcal{D}\)'s,

\begin{align}
    \text{Term 1 Multiplier} &= \frac{\mathcal{D}r_o}{rr_o\mathcal{D}^2m}; \\
    \text{Term 2 Multiplier} &= \frac{2r(-\rho^2+\xi^2+1)}{rr_o\mathcal{D}^5m^2(m-1)}; \\
    \text{Term 3 Multiplier} &= \frac{r(\rho+1)}{rr_o\mathcal{D}^3m}.
\end{align}

\noindent Now we just need to multiply the top and bottom of Term 1 by \(\mathcal{D}^3m(m-1)\) and the top and bottom of Term 3 by \(\mathcal{D}^2m(m-1)\) to get a common denominator between the terms.

\begin{align}
    \label{eqn:vzt1m}
    \text{Term 1 Multiplier} &= \frac{r_o\mathcal{D}^4m(m-1)}{rr_o\mathcal{D}^5m^2(m-1)}; \\
    \label{eqn:vzt2m}
    \text{Term 2 Multiplier} &= \frac{2r(-\rho^2+\xi^2+1)}{rr_o\mathcal{D}^5m^2(m-1)}; \\
    \label{eqn:vzt3m}
    \text{Term 3 Multiplier} &= \frac{r(\rho+1)\mathcal{D}^2m(m-1)}{rr_o\mathcal{D}^5m^2(m-1)}.
\end{align}

With a common denominator in place, we can start to add the various terms together.
%
Let us try to begin with the \(\mathcal{K}(m)\) terms:

\begin{equation}
    \begin{aligned}
        \frac{\mathcal{K}(m)}{rr_o\mathcal{D}^5m^2(m-1)} \bigg[&  r_o\mathcal{D}^4m(m-1)(m-2)  && \text{(from Term 1)} \\
      &- 2r(-\rho^2+\xi^2+1)(m-4)(m-1) && \text{(from Term 2)} \\
      &- r(\rho+1)\mathcal{D}^2m(m-1)(m-2) \bigg]. && \text{(from Term 3)} \\
    \end{aligned}
\end{equation}

\noindent We see immediately that we can cancel out the \((m-1)\) from all the terms.

\begin{equation}
    \begin{aligned}
        \frac{\mathcal{K}(m)}{rr_o\mathcal{D}^5m^2} \bigg[&  r_o\mathcal{D}^4m(m-2)  && \text{(from Term 1)} \\
      &- 2r(-\rho^2+\xi^2+1)(m-4) && \text{(from Term 2)} \\
      &- r(\rho+1)\mathcal{D}^2m(m-2) \bigg]. && \text{(from Term 3)} \\
    \end{aligned}
\end{equation}

\noindent Unfortunately, that appears to be the only obvious cancellation to make right away.
%
Perhaps expanding things out more will help.
%
Let us expand the \(m\)'s out next, remembering that \(m=4\rho/\mathcal{D}^2\).

\begin{equation}
    \begin{aligned}
        \frac{\mathcal{K}(m)\cancel{\mathcal{D}^4}}{rr_o\mathcal{D}^{\cancel{5}} \cancelto{4}{16}\rho^2} \bigg[&  r_o\mathcal{D}^{\cancelto{2}{4}}\frac{\cancel{4}\rho}{\cancel{\mathcal{D}^2}}\left(\frac{4\rho}{\mathcal{D}^2}-2\right)  && \text{(from Term 1)} \\
        &- 2r(-\rho^2+\xi^2+1)\left(\frac{\cancel{4}\rho}{\mathcal{D}^2}-\cancel{4}\right) && \text{(from Term 2)} \\
        &- r(\rho+1)\cancel{\mathcal{D}^2}\frac{\cancel{4}\rho}{\cancel{\mathcal{D}^2}}\left(\frac{4\rho}{\mathcal{D}^2}-2\right) \bigg]; && \text{(from Term 3)} \\
    \end{aligned}
\end{equation}

\noindent then canceling out the obvious items and cleaning up:

\begin{equation}
    \begin{aligned}
        \frac{\mathcal{K}(m)}{4\rho^2 rr_o\mathcal{D} } \bigg[&  \rho r_o\mathcal{D}^2\left(\frac{4\rho}{\mathcal{D}^2}-2\right)  && \text{(from Term 1)} \\
      &- 2r(-\rho^2+\xi^2+1)\left(\frac{\rho}{\mathcal{D}^2}-1\right) && \text{(from Term 2)} \\
      &- \rho r(\rho+1)\left(\frac{4\rho}{\mathcal{D}^2}-2\right) \bigg]. && \text{(from Term 3)} \\
    \end{aligned}
\end{equation}

\noindent We see that \(r_o\rho=r\), which gives us a mutual \(r\) that we can cancel out of all the terms.

\begin{equation}
    \begin{aligned}
        \frac{\mathcal{K}(m)}{4\rho^2 r_o\mathcal{D} } \bigg[&  \mathcal{D}^2 \left(\frac{4\rho}{\mathcal{D}^2}-2\right)  && \text{(from Term 1)} \\
      &- 2(-\rho^2+\xi^2+1)\left(\frac{\rho}{\mathcal{D}^2}-1\right) && \text{(from Term 2)} \\
      &- \rho (\rho+1)\left(\frac{4\rho}{\mathcal{D}^2}-2\right) \bigg]. && \text{(from Term 3)} \\
    \end{aligned}
\end{equation}

\noindent We also see that we can cancel out an additional 2 from everything.

\begin{equation}
    \begin{aligned}
        \frac{\mathcal{K}(m)}{2\mathcal{D}r_o \rho^2} \bigg[&  \mathcal{D}^2 \left(\frac{2\rho}{\mathcal{D}^2}-1\right)  && \text{(from Term 1)} \\
      &- (-\rho^2+\xi^2+1)\left(\frac{\rho}{\mathcal{D}^2}-1\right) && \text{(from Term 2)} \\
      &- \rho (\rho+1)\left(\frac{2\rho}{\mathcal{D}^2}-1\right) \bigg]. && \text{(from Term 3)} \\
    \end{aligned}
\end{equation}

We've found ourselves with some more uncommon denominators, so let's expand and gather terms.

\begin{equation}
    \begin{aligned}
        \frac{\mathcal{K}(m)}{2\mathcal{D}r_o \rho^2 } \bigg[&   2\rho-\mathcal{D}^2  && \text{(from Term 1)} \\
        &- \frac{-\rho^3+\rho\xi^2+\rho}{\mathcal{D}^2}-\rho^2+\xi^2+1 && \text{(from Term 2)} \\
      &- \frac{2\rho^3 +2\rho^2}{\mathcal{D}^2}+\rho^2 + \rho \bigg]; && \text{(from Term 3)} \\
    \end{aligned}
\end{equation}

\begin{equation}
    \begin{aligned}
        \frac{\mathcal{K}(m)}{2\mathcal{D}r_o \rho^2 } \bigg[&   2\rho-\mathcal{D}^2 -\cancel{\rho^2} +\xi^2 +1 +\cancel{\rho^2} + \rho \\
        &- \frac{-\rho^3+\rho\xi^2+\rho}{\mathcal{D}^2} - \frac{2\rho^3 +2\rho^2}{\mathcal{D}^2} \bigg].
    \end{aligned}
\end{equation}

\noindent Expanding out the \(\mathcal{D}\) in the numerator:

\begin{equation}
    \begin{aligned}
        \frac{\mathcal{K}(m)}{2\mathcal{D}r_o \rho^2 } \bigg[&   \cancel{3}\rho+\cancel{\xi^2} +\cancel{1} -(\rho^2 +\cancel{2\rho} +\cancel{1} +\cancel{\xi})  \\
        &- \frac{-\rho^3+\rho\xi^2+\rho}{\mathcal{D}^2} - \frac{2\rho^3 +2\rho^2}{\mathcal{D}^2} \bigg].
    \end{aligned}
\end{equation}

\noindent Now let's get a common denominator again, pulling the \(\rho^2\) inside the brackets.

\begin{equation}
    \frac{\mathcal{K}(m)}{2\mathcal{D}r_o } \left[\frac{\rho\mathcal{D}^2 -\rho^2\mathcal{D}^2 + \cancel{\rho^3}-\rho\xi^2-\rho - \cancel{2}\rho^3 -2\rho^2}{\mathcal{D}^2\rho^2} \right].
\end{equation}

\noindent We can immediately cancel out a \(\rho\):

\begin{equation}
    \frac{\mathcal{K}(m)}{2\mathcal{D}r_o } \left[ \frac{\mathcal{D}^2 -(\rho^2 +2\rho +1 +\xi^2)-\rho\mathcal{D}^2}{\mathcal{D}^2\rho} \right].
\end{equation}

\noindent We also see that \(\mathcal{D}^2 = \rho^2 + 2\rho +1 +\xi^2\), which cancels in the numerator.

\begin{equation}
    \frac{\mathcal{K}(m)}{2\mathcal{D}r_o } \left[\frac{-\rho\mathcal{D}^2}{\rho\mathcal{D}^2} \right].
\end{equation}

\noindent We are finally left with
\begin{equation}
    \label{eqn:vzksimple}
    -\frac{\mathcal{K}(m)}{2\mathcal{D}r_o }.
\end{equation}

Now let's look at the \(\mathcal{E}(m)\) terms start back with the term multipliers with common denominators: \cref{eqn:vzt1m,eqn:vzt2m,eqn:vzt3m}.


\begin{equation}
    \begin{aligned}
        \frac{\mathcal{E}(m)}{rr_o\mathcal{D}^5m^2(m-1)} \bigg[&  2r_o\mathcal{D}^4m(m-1)  && \text{(from Term 1)} \\
      &- 2r(-\rho^2+\xi^2+1)(3m-4) && \text{(from Term 2)} \\
      &- 2r(\rho+1)\mathcal{D}^2m(m-1) \bigg]. && \text{(from Term 3)}
    \end{aligned}
\end{equation}

\noindent Unlike the \(\mathcal{K}(m)\) terms, it doesn't appear as though anything will cancel out immediately.
%
Let's take a similar approach as before and expand out the \(m\) terms.
%

\begin{equation}
    \begin{aligned}
        \frac{\mathcal{E}(m)}{rr_o\mathcal{D}^5m^2(m-1)} \bigg[&  2r_o\mathcal{D}^4\left(\frac{4\rho}{\mathcal{D}^2}\right)\left(\frac{4\rho}{\mathcal{D}^2}\right)-1)  && \text{(from Term 1)} \\
      &- 2r(-\rho^2+\xi^2+1)\left(3\left(\frac{4\rho}{\mathcal{D}^2}\right)-4\right) && \text{(from Term 2)} \\
      &- 2r(\rho+1)\mathcal{D}^2\left(\frac{4\rho}{\mathcal{D}^2}\right)\left(\frac{4\rho}{\mathcal{D}^2}\right)-1) \bigg]. && \text{(from Term 3)}
    \end{aligned}
\end{equation}

\noindent Again noting that \(r_o\rho = r\), we can cancel out an \(r\).

\begin{equation}
    \begin{aligned}
        \frac{\mathcal{E}(m)}{r_o\mathcal{D}^5m^2(m-1)} \bigg[&  2\mathcal{D}^4\left(\frac{4}{\mathcal{D}^2}\right)\left(\frac{4\rho}{\mathcal{D}^2}\right)-1)  && \text{(from Term 1)} \\
      &- 2(-\rho^2+\xi^2+1)\left(3\left(\frac{4\rho}{\mathcal{D}^2}\right)-4\right) && \text{(from Term 2)} \\
      &- 2(\rho+1)\mathcal{D}^2\left(\frac{4\rho}{\mathcal{D}^2}\right)\left(\frac{4\rho}{\mathcal{D}^2}\right)-1) \bigg]. && \text{(from Term 3)}
    \end{aligned}
\end{equation}

\noindent We may also want to expand the \(m^2\) on the outside---

\begin{equation}
    \frac{1}{r_o\mathcal{D}^5m^2(m-1)} = \frac{\cancel{\mathcal{D}^4}}{r_o\mathcal{D}^{\cancel{5}}16\rho^2(m-1)}
\end{equation}

\noindent ---which leaves us with

\begin{equation}
    \begin{aligned}
        \frac{\mathcal{E}(m)}{r_o\mathcal{D}16\rho^2(m-1)} \bigg[&  2\mathcal{D}^4\left(\frac{4}{\mathcal{D}^2}\right)\left(\frac{4\rho}{\mathcal{D}^2}\right)-1)  && \text{(from Term 1)} \\
      &- 2(-\rho^2+\xi^2+1)\left(3\left(\frac{4\rho}{\mathcal{D}^2}\right)-4\right) && \text{(from Term 2)} \\
      &- 2(\rho+1)\mathcal{D}^2\left(\frac{4\rho}{\mathcal{D}^2}\right)\left(\frac{4\rho}{\mathcal{D}^2}\right)-1) \bigg]. && \text{(from Term 3)}
    \end{aligned}
\end{equation}

\noindent We can now take an 8 out of everything:

\begin{equation}
    \begin{aligned}
        \frac{\mathcal{E}(m)}{2\mathcal{D}r_o\rho^2(m-1)} \bigg[&  \mathcal{D}^{\cancelto{2}{4}}\cancel{\left(\frac{1}{\mathcal{D}^2}\right)}\left(\frac{4\rho}{\mathcal{D}^2}\right)-1)  && \text{(from Term 1)} \\
      &- (-\rho^2+\xi^2+1)\left(3\left(\frac{\rho}{\mathcal{D}^2}\right)-1\right) && \text{(from Term 2)} \\
      &- (\rho+1)\cancel{\mathcal{D}^2}\left(\frac{\rho}{\cancel{\mathcal{D}^2}}\right)\left(\frac{4\rho}{\mathcal{D}^2}\right)-1) \bigg]. && \text{(from Term 3)}
    \end{aligned}
\end{equation}

\noindent Cleaning up a bit:

\begin{equation}
    \begin{aligned}
        \frac{\mathcal{E}(m)}{2\mathcal{D}r_o\rho^2(m-1)} \bigg[&  \mathcal{D}^2\left(\frac{4\rho}{\mathcal{D}^2}\right)-1)  && \text{(from Term 1)} \\
      &- (-\rho^2+\xi^2+1)\left(3\left(\frac{\rho}{\mathcal{D}^2}\right)-1\right) && \text{(from Term 2)} \\
      &- (\rho+1)\rho\left(\frac{4\rho}{\mathcal{D}^2}\right)-1) \bigg]. && \text{(from Term 3)}
    \end{aligned}
\end{equation}

\noindent Let's next expand out the multiplications.

\begin{equation}
    \begin{aligned}
        \frac{\mathcal{E}(m)}{2\mathcal{D}r_o\rho^2(m-1)} \bigg[&  4\rho-\mathcal{D}^2  && \text{(from Term 1)} \\
      &- \frac{-3\rho^3+3\rho\xi^2+3\rho}{\mathcal{D}^2} -\cancel{\rho^2}+\xi^2 +1 && \text{(from Term 2)} \\
      &- \frac{4\rho^3 + 4\rho^2}{\mathcal{D}^2}+\cancel{\rho^2}+\rho \bigg]. && \text{(from Term 3)}
    \end{aligned}
\end{equation}

\noindent Gathering terms:

\begin{equation}
    \begin{aligned}
        \frac{\mathcal{E}(m)}{2\mathcal{D}r_o\rho^2(m-1)} \bigg[&  5\rho+\xi+1-\mathcal{D}^2
      &- \frac{-3\rho^3+3\rho\xi^2+3\rho}{\mathcal{D}^2}- \frac{4\rho^3 + 4\rho^2}{\mathcal{D}^2} \bigg].
    \end{aligned}
\end{equation}

\noindent Expanding the \(\mathcal{D}\) in the numerator:

\begin{equation}
    \begin{aligned}
        \frac{\mathcal{E}(m)}{2\mathcal{D}r_o\rho^2(m-1)} \bigg[&  \cancelto{3}{5}\rho+\cancel{\xi}+\cancel{1}-(\rho^2 + \cancel{2\rho} + \cancel{1} +\cancel{\xi})
      &- \frac{-\cancel{3\rho^3}+3\rho\xi^2+3\rho+\cancel{4}\rho^3 + 4\rho^2}{\mathcal{D}^2} \bigg].
    \end{aligned}
\end{equation}

\noindent We can take a \(\rho\) out of everything now as well.

\begin{equation}
    \begin{aligned}
        \frac{\mathcal{E}(m)}{2\mathcal{D}r_o\rho(m-1)} \bigg[&  3-\rho
      &- \frac{3\xi^2+3+\rho^2 + 4\rho}{\mathcal{D}^2} \bigg].
    \end{aligned}
\end{equation}

\noindent Let's move the 3 into the fraction and expand the \(\mathcal{D}\) that will appear in the numerator.

\begin{equation}
    \frac{\mathcal{E}(m)}{2\mathcal{D}r_o\rho(m-1)} \left[  -\rho - \frac{-\cancelto{2}{3}\rho^2 - \cancelto{2}{6}\rho - \cancel{3} -\cancel{3\xi} + \cancel{3\xi^2}+\cancel{3}+\cancel{\rho^2} + \cancel{4\rho}}{\mathcal{D}^2} \right].
\end{equation}

\noindent After cleaning up the various cancelations, we can take another \(\rho\) out of everything.

\begin{equation}
    \frac{\mathcal{E}(m)}{2\mathcal{D}r_o(m-1)} \left[ -1 + \frac{2(\rho + 1) }{\mathcal{D}^2} \right].
\end{equation}

\noindent Now let's move the \((m-1)\) into the inside and expand out the \(m\) and \(\mathcal{D}\) terms.

\begin{equation}
    \frac{\mathcal{E}(m)}{2\mathcal{D}r_o} \left[ \frac{-1}{\frac{4\rho}{(\rho+1)^2+\xi^2}-1} + \frac{2(\rho + 1) }{\left(\frac{4\rho}{(\rho+1)^2+\xi^2}-1\right)\left((\rho+1)^2+\xi^2\right)} \right].
\end{equation}

\noindent Combining fractions:

\begin{equation}
    \frac{\mathcal{E}(m)}{2\mathcal{D}r_o} \left[ \frac{2\rho + 2  -(\rho+1)^2 - \xi^2}{\left(\frac{4\rho}{(\rho+1)^2+\xi^2}-1\right)\left((\rho+1)^2+\xi^2\right)} \right].
\end{equation}

\noindent Expanding the numerator:

\begin{equation}
    \frac{\mathcal{E}(m)}{2\mathcal{D}r_o} \left[ \frac{\cancel{2\rho} + \cancelto{1}{2}  -\rho^2-\cancel{2\rho}-\cancel{1} - \xi^2}{\left(\frac{4\rho}{(\rho+1)^2+\xi^2}-1\right)\left((\rho+1)^2+\xi^2\right)} \right];
\end{equation}

\begin{equation}
    \frac{\mathcal{E}(m)}{2\mathcal{D}r_o} \left[ \frac{1  -\rho^2 - \xi^2}{\left(\frac{4\rho}{(\rho+1)^2+\xi^2}-1\right)\left((\rho+1)^2+\xi^2\right)} \right].
\end{equation}

\noindent Getting a common denominator in the denominator:

\begin{equation}
    \frac{\mathcal{E}(m)}{2\mathcal{D}r_o} \left[ \frac{1  -\rho^2 - \xi^2}{4\rho-(\rho+1)^2-\xi^2} \right].
\end{equation}

\noindent Expanding then simplifying the denominator:

\begin{equation}
    \frac{\mathcal{E}(m)}{2\mathcal{D}r_o} \left[ \frac{1  -\rho^2 - \xi^2}{\cancelto{2}{4}\rho-\rho^2-\cancel{2\rho}-1-\xi^2} \right];
\end{equation}

\begin{equation}
    \frac{\mathcal{E}(m)}{2\mathcal{D}r_o} \left[ \frac{\rho^2 - 1 + \xi^2}{(\rho-1)^2+\xi^2} \right].
\end{equation}

\noindent Adding and subtracting \(2\rho + 1\) to the numerator:

\begin{equation}
    \frac{\mathcal{E}(m)}{2\mathcal{D}r_o} \left[ \frac{\rho^2 -2\rho + 1 + \xi^2 + 2\rho - 2}{(\rho-1)^2+\xi^2} \right].
\end{equation}

\noindent Simplifying:

\begin{equation}
    \frac{\mathcal{E}(m)}{2\mathcal{D}r_o} \left[ \frac{(\rho- 1)^2 + \xi^2 + 2(\rho - 1)}{(\rho-1)^2+\xi^2} \right].
\end{equation}

\noindent Splitting the fraction:

\begin{equation}
    \frac{\mathcal{E}(m)}{2\mathcal{D}r_o} \left[ \cancel{\frac{(\rho- 1)^2 + \xi^2 }{(\rho-1)^2+\xi^2}} + \frac{2(\rho - 1)}{(\rho-1)^2+\xi^2} \right].
\end{equation}

\noindent Finally, we are left with

\begin{equation}
    \label{eqn:vzesimple}
    \frac{\mathcal{E}(m)}{2\mathcal{D}r_o} \left[ 1 + \frac{2(\rho - 1)}{(\rho-1)^2+\xi^2} \right].
\end{equation}

Now combining our \(\mathcal{K}(m)\) and \(\mathcal{E}(m)\) terms from \cref{eqn:vzksimple} and \cref{eqn:vzesimple}, respectively, we arrive at

\begin{equation}
    v_z = \mathcal{C}\frac{1}{2\mathcal{D}r_o}\left[-\mathcal{K}(m) +\left(1 + \frac{2(\rho - 1)}{(\rho-1)^2+\xi^2}\right)\mathcal{E}(m)\right].
\end{equation}

\noindent Expanding out \(\mathcal{C}\) and \(\mathcal{D}\) gives us our final expression as presented in \cref{eqn:ringvortexinducedvelocityaxial}:

\begin{equation}
    \eqbox{
        v_z = \frac{\Gamma}{2\pi r_o}\frac{1}{\left[\xi^2+(\rho+1)^2\right]^{1/2}}\left[\mathcal{K}(m) -\left(1 + \frac{2(\rho - 1)}{\xi^2 + (\rho-1)^2}\right)\mathcal{E}(m)\right].
}
\end{equation}
