
\section{Verification of Full Solver Implementations in DuctAPE}
\label{ssec:verification}

As we have established, the methodology behind DuctAPE is based heavily on DFDC.
%
Therefore, we take the opportunity to provide a set of comparisons between DuctAPE and DFDC.
%
We compared an example available in the DFDC source code using a single ducted rotor across a range of operating conditions, specifically across a range of advance ratios including a hover condition.

The geometry used in the single ducted rotor example case is shown in \cref{fig:singlerotorgeom}.
%
For this verification case, we used a rotor with tip radius of 0.15572 meters located 0.12 meters aft of the center body leading edge.
%
The wake extended 0.8 times the length of the duct (roughly 0.3 meters) past the duct trailing edge.
%
We used 10 blade elements associated with 11 wake sheets to model the rotor.
%
We set the rotor rotation rate constant at 8000 revolutions per minute and adjust the freestream velocity in order to sweep across advance ratios from 0.0 to 2.0 by increments of 0.1.
%
We assumed sea level conditions for reference values.

\begin{figure}[h!]
     \centering
% \tikzsetnextfilename{solvers/ductape_single_rotor_geometry}
     % Recommended preamble:
% \usetikzlibrary{arrows.meta}
% \usetikzlibrary{backgrounds}
% \usepgfplotslibrary{patchplots}
% \usepgfplotslibrary{fillbetween}
% \pgfplotsset{%
%     layers/standard/.define layer set={%
%         background,axis background,axis grid,axis ticks,axis lines,axis tick labels,pre main,main,axis descriptions,axis foreground%
%     }{
%         grid style={/pgfplots/on layer=axis grid},%
%         tick style={/pgfplots/on layer=axis ticks},%
%         axis line style={/pgfplots/on layer=axis lines},%
%         label style={/pgfplots/on layer=axis descriptions},%
%         legend style={/pgfplots/on layer=axis descriptions},%
%         title style={/pgfplots/on layer=axis descriptions},%
%         colorbar style={/pgfplots/on layer=axis descriptions},%
%         ticklabel style={/pgfplots/on layer=axis tick labels},%
%         axis background@ style={/pgfplots/on layer=axis background},%
%         3d box foreground style={/pgfplots/on layer=axis foreground},%
%     },
% }

\begin{tikzpicture}[/tikz/background rectangle/.style={fill={rgb,1:red,0.0;green,0.0;blue,0.0}, fill opacity={0.0}, draw opacity={0.0}}, show background rectangle]
\begin{axis}[point meta max={nan}, point meta min={nan}, legend cell align={left}, legend columns={1}, title={}, title style={at={{(0.5,1)}}, anchor={south}, font={{\fontsize{14 pt}{18.2 pt}\selectfont}}, color={rgb,1:red,0.0;green,0.0;blue,0.0}, draw opacity={1.0}, rotate={0.0}, align={center}}, legend style={color={rgb,1:red,0.0;green,0.0;blue,0.0}, draw opacity={0.0}, line width={1}, solid, fill={rgb,1:red,0.0;green,0.0;blue,0.0}, fill opacity={0.0}, text opacity={1.0}, font={{\fontsize{8 pt}{10.4 pt}\selectfont}}, text={rgb,1:red,0.0;green,0.0;blue,0.0}, cells={anchor={center}}, at={(1.02, 1)}, anchor={north west}}, axis background/.style={fill={rgb,1:red,0.0;green,0.0;blue,0.0}, opacity={0.0}}, anchor={north west}, xshift={5.0mm}, yshift={-5.0mm}, width={131.304mm}, height={50.96mm}, scaled x ticks={false}, xlabel={$z~\mathrm{(m)}$}, x tick style={color={rgb,1:red,0.0;green,0.0;blue,0.0}, opacity={1.0}}, x tick label style={color={rgb,1:red,0.0;green,0.0;blue,0.0}, opacity={1.0}, rotate={0}}, xlabel style={at={(ticklabel cs:0.5)}, anchor=near ticklabel, at={{(ticklabel cs:0.5)}}, anchor={near ticklabel}, font={{\fontsize{11 pt}{14.3 pt}\selectfont}}, color={rgb,1:red,0.0;green,0.0;blue,0.0}, draw opacity={1.0}, rotate={0.0}}, xmajorgrids={true}, xmin={-0.01654556446238531}, xmax={0.5680643798752283}, xticklabels={{0.00,0.12,0.31,0.55}}, xtick={{0.0,0.12,0.30637899,0.5514821820000002}}, xtick align={inside}, xticklabel style={font={{\fontsize{8 pt}{10.4 pt}\selectfont}}, color={rgb,1:red,0.0;green,0.0;blue,0.0}, draw opacity={1.0}, rotate={0.0}}, x grid style={color={rgb,1:red,0.0;green,0.0;blue,0.0}, draw opacity={0.1}, line width={0.5}, solid}, axis x line*={left}, x axis line style={color={rgb,1:red,0.0;green,0.0;blue,0.0}, draw opacity={1.0}, line width={1}, solid}, scaled y ticks={false}, ylabel={$r~\mathrm{(m)}$}, y tick style={color={rgb,1:red,0.0;green,0.0;blue,0.0}, opacity={1.0}}, y tick label style={color={rgb,1:red,0.0;green,0.0;blue,0.0}, opacity={1.0}, rotate={0}}, ylabel style={{rotate=-90}}, ymajorgrids={true}, ymin={-0.018897937641611184}, ymax={0.23330880956173}, yticklabels={{0.00,0.04,0.16,0.18,0.21}}, ytick={{0.0,0.044952523,0.155720815,0.17818011312626458,0.21441087192011882}}, ytick align={inside}, yticklabel style={font={{\fontsize{8 pt}{10.4 pt}\selectfont}}, color={rgb,1:red,0.0;green,0.0;blue,0.0}, draw opacity={1.0}, rotate={0.0}}, y grid style={color={rgb,1:red,0.0;green,0.0;blue,0.0}, draw opacity={0.1}, line width={0.5}, solid}, axis y line*={left}, y axis line style={color={rgb,1:red,0.0;green,0.0;blue,0.0}, draw opacity={1.0}, line width={1}, solid}, colorbar={false}]
    \addplot[color={rgb,1:red,0.502;green,0.502;blue,0.502}, name path={ad3ad9cd-3fec-45f0-b5e6-88e87c686960}, draw opacity={0.25}, line width={0.1}, solid, forget plot]
        table[row sep={\\}]
        {
            \\
            0.12  0.04495252299071941  \\
            0.12  0.05602935219164747  \\
            0.12  0.06710618139257553  \\
            0.12  0.07818301059350359  \\
            0.12  0.08925983979443165  \\
            0.12  0.10033666899535972  \\
            0.12  0.11141349819628778  \\
            0.12  0.12249032739721583  \\
            0.12  0.1335671565981439  \\
            0.12  0.14464398579907195  \\
            0.12  0.155720815  \\
        }
        ;
    \addplot[color={rgb,1:red,0.502;green,0.502;blue,0.502}, name path={68b01e25-813b-4b02-a6a7-910eb86f5f9e}, draw opacity={0.25}, line width={0.1}, solid, forget plot]
        table[row sep={\\}]
        {
            \\
            0.12615140016666665  0.04494248592934282  \\
            0.12615140019524057  0.0560404395104915  \\
            0.1261514002210339  0.06713071695180639  \\
            0.126151400241577  0.07821950267194142  \\
            0.1261514002549012  0.08930768769610237  \\
            0.12615140025972063  0.10039538250716509  \\
            0.12615140025554467  0.11148236259103259  \\
            0.12615140024271357  0.12256798715099464  \\
            0.12615140022235488  0.1336508446344345  \\
            0.1261514001962611  0.1447277691363484  \\
            0.12615140016666665  0.15579077631842816  \\
        }
        ;
    \addplot[color={rgb,1:red,0.502;green,0.502;blue,0.502}, name path={8118e2da-b4ae-4e8d-b030-3c5f03bbcb41}, draw opacity={0.25}, line width={0.1}, solid, forget plot]
        table[row sep={\\}]
        {
            \\
            0.13230280033333333  0.04494776043555422  \\
            0.1323028003913925  0.05605296683605158  \\
            0.13230280044378667  0.0671553292278602  \\
            0.13230280048548837  0.0782559880131186  \\
            0.13230280051249943  0.08935560960449251  \\
            0.13230280052221355  0.10045432517316684  \\
            0.13230280051364623  0.11155172607354069  \\
            0.13230280048750778  0.12264664646709793  \\
            0.13230280044611692  0.13373657213266293  \\
            0.132302800393164  0.14481614363284223  \\
            0.13230280033333333  0.1558732022935015  \\
        }
        ;
    \addplot[color={rgb,1:red,0.502;green,0.502;blue,0.502}, name path={96373f63-31b8-4647-9a2f-beaceca27e5c}, draw opacity={0.25}, line width={0.1}, solid, forget plot]
        table[row sep={\\}]
        {
            \\
            0.13845420049999999  0.044949764153980734  \\
            0.13845420058934013  0.05606501577684807  \\
            0.13845420066992722  0.06717967633249355  \\
            0.13845420073402734  0.07829231617435715  \\
            0.13845420077549261  0.08940353188882594  \\
            0.13845420079032575  0.10051354088230255  \\
            0.13845420077703385  0.11162185072557691  \\
            0.13845420073673972  0.12272697691396277  \\
            0.13845420067305042  0.13382579732752276  \\
            0.1384542005917003  0.14491206495073028  \\
            0.13845420049999999  0.1559729329828782  \\
        }
        ;
    \addplot[color={rgb,1:red,0.502;green,0.502;blue,0.502}, name path={3751e22c-a163-4492-98c5-d421a78bab84}, draw opacity={0.25}, line width={0.1}, solid, forget plot]
        table[row sep={\\}]
        {
            \\
            0.14460560066666667  0.044942278699579535  \\
            0.14460560079001664  0.05607536754340089  \\
            0.1446056009012248  0.06720344097092065  \\
            0.14460560098962408  0.07832831538807132  \\
            0.144605601046736  0.0894513380666458  \\
            0.14460560106706402  0.10057299701664563  \\
            0.14460560104858047  0.11169285335570084  \\
            0.14460560099287698  0.12280934155347033  \\
            0.14460560090498  0.13391924135262664  \\
            0.14460560079285847  0.1450166051869055  \\
            0.14460560066666667  0.15609102041875295  \\
        }
        ;
    \addplot[color={rgb,1:red,0.502;green,0.502;blue,0.502}, name path={3184714d-b5d8-45f1-a2fe-86396167fc4f}, draw opacity={0.25}, line width={0.1}, solid, forget plot]
        table[row sep={\\}]
        {
            \\
            0.15075700083333332  0.04493204538203043  \\
            0.1507570009944323  0.056084531918666  \\
            0.1507570011396005  0.06722652798138998  \\
            0.15075700125491626  0.07836381408209799  \\
            0.15075700132932401  0.0894988566340506  \\
            0.15075700135567988  0.10063255991642192  \\
            0.1507570013313835  0.11176466904790242  \\
            0.1507570012585672  0.12289374982477055  \\
            0.15075700114384671  0.13401687233765402  \\
            0.15075700099767186  0.14512903043874306  \\
            0.15075700083333332  0.15622265571003757  \\
        }
        ;
    \addplot[color={rgb,1:red,0.502;green,0.502;blue,0.502}, name path={47116a75-9c82-4992-b8f7-50dc81a3b762}, draw opacity={0.25}, line width={0.1}, solid, forget plot]
        table[row sep={\\}]
        {
            \\
            0.156908401  0.044923127246484104  \\
            0.15690840120370972  0.0560932491500544  \\
            0.15690840138718576  0.06724891750798433  \\
            0.15690840153282662  0.07839861751686146  \\
            0.15690840162667963  0.08954584775223029  \\
            0.15690840165976214  0.10069198275589758  \\
            0.15690840162885206  0.1118370460812192  \\
            0.15690840153671326  0.12297988993388287  \\
            0.15690840139176687  0.1341181109398657  \\
            0.15690840120725882  0.145247877297928  \\
            0.156908401  0.15636407383589124  \\
        }
        ;
    \addplot[color={rgb,1:red,0.502;green,0.502;blue,0.502}, name path={49c91940-de59-427d-84c8-75ae91518b16}, draw opacity={0.25}, line width={0.1}, solid, forget plot]
        table[row sep={\\}]
        {
            \\
            0.16305980116666666  0.04491727692121097  \\
            0.16305980141911983  0.05610185389778746  \\
            0.16305980164638673  0.06727051141803732  \\
            0.16305980182664898  0.07843247184616652  \\
            0.16305980194265315  0.08959199140521185  \\
            0.1630598019833388  0.10075090393602434  \\
            0.16305980194480232  0.11190956234905504  \\
            0.16305980183055457  0.12306719467988293  \\
            0.1630598016510932  0.13422203132650048  \\
            0.16305980142285748  0.1453714339695169  \\
            0.16305980116666666  0.15651236616271635  \\
        }
        ;
    \addplot[color={rgb,1:red,0.502;green,0.502;blue,0.502}, name path={6b0e2fcb-5ad8-461c-8a6f-51a89cddcb37}, draw opacity={0.25}, line width={0.1}, solid, forget plot]
        table[row sep={\\}]
        {
            \\
            0.16921120133333334  0.04491196172254526  \\
            0.16921120164212095  0.056110045676790836  \\
            0.16921120191995728  0.06729106515668723  \\
            0.16921120214014587  0.07846504059891733  \\
            0.1692112022816338  0.08963688056691509  \\
            0.16921120233099135  0.10080885315695454  \\
            0.16921120228356445  0.11198165270245582  \\
            0.16921120214375715  0.12315491172194172  \\
            0.1692112019244811  0.13432750486434455  \\
            0.1692112016458617  0.14549789279093822  \\
            0.16921120133333334  0.15666486336534555  \\
        }
        ;
    \addplot[color={rgb,1:red,0.502;green,0.502;blue,0.502}, name path={3703a70e-d83b-4223-a00d-7ae05d9cb1b3}, draw opacity={0.25}, line width={0.1}, solid, forget plot]
        table[row sep={\\}]
        {
            \\
            0.1753626015  0.044906620646925216  \\
            0.17536260187440036  0.05611747133702889  \\
            0.1753626022110816  0.06731023233667703  \\
            0.17536260247765992  0.07849590014122762  \\
            0.1753626026486766  0.0896800198575199  \\
            0.1753626027079842  0.10086526204915033  \\
            0.17536260265010248  0.11205263965240041  \\
            0.17536260248051416  0.12324216719187772  \\
            0.17536260221496122  0.13443330414778515  \\
            0.17536260187785932  0.14562542051196536  \\
            0.1753626015  0.15681857836445  \\
        }
        ;
    \addplot[color={rgb,1:red,0.502;green,0.502;blue,0.502}, name path={a49c7b81-dcad-43b7-a78b-09066fa71c87}, draw opacity={0.25}, line width={0.1}, solid, forget plot]
        table[row sep={\\}]
        {
            \\
            0.18151400166666667  0.04490069966154803  \\
            0.1815140021179268  0.056123773948205766  \\
            0.1815140025234716  0.06732758926788507  \\
            0.1815140028442434  0.07852453904258988  \\
            0.181514003049649  0.08972082667202388  \\
            0.1815140031204133  0.10091947637043401  \\
            0.18151400305015145  0.11212176274436124  \\
            0.1815140028456599  0.12332801937251056  \\
            0.18151400252602115  0.1345381905217773  \\
            0.18151400212067537  0.1457523440446712  \\
            0.18151400166666667  0.15697133722994003  \\
        }
        ;
    \addplot[color={rgb,1:red,0.502;green,0.502;blue,0.502}, name path={e1147850-e212-4918-ac80-e315a58a1a65}, draw opacity={0.25}, line width={0.1}, solid, forget plot]
        table[row sep={\\}]
        {
            \\
            0.18766540183333333  0.04489448945657971  \\
            0.18766540237501228  0.05612866310351651  \\
            0.18766540286148156  0.067342641873836  \\
            0.18766540324580985  0.07855035379718306  \\
            0.18766540349140065  0.08975863172261479  \\
            0.18766540357537678  0.10097076746037408  \\
            0.1876654034903738  0.11218820383875444  \\
            0.18766540324479908  0.1234114987789757  \\
            0.18766540286169772  0.1346409642309278  \\
            0.1876654023764194  0.1458771844962717  \\
            0.18766540183333333  0.15712150120938215  \\
        }
        ;
    \addplot[color={rgb,1:red,0.502;green,0.502;blue,0.502}, name path={4d544e49-f094-46ae-84c9-40633031beae}, draw opacity={0.25}, line width={0.1}, solid, forget plot]
        table[row sep={\\}]
        {
            \\
            0.193816802  0.04488823654667903  \\
            0.19381680264838852  0.05613182293510373  \\
            0.19381680323024608  0.06735480163046952  \\
            0.1938168036893129  0.07857263630941203  \\
            0.19381680398196213  0.0897926768834148  \\
            0.1938168040811733  0.10101834167588453  \\
            0.19381680397853998  0.11225110758252534  \\
            0.1938168036844555  0.12349163568811568  \\
            0.193816803226682  0.13474048468284108  \\
            0.1938168026475395  0.14599862705058791  \\
            0.193816802  0.15726777750621487  \\
        }
        ;
    \addplot[color={rgb,1:red,0.502;green,0.502;blue,0.502}, name path={68334e00-6024-480b-9134-a5166d1c9af4}, draw opacity={0.25}, line width={0.1}, solid, forget plot]
        table[row sep={\\}]
        {
            \\
            0.19996820216666666  0.04488243703899517  \\
            0.19996820294129872  0.056132848421925056  \\
            0.19996820363584517  0.0673633454997533  \\
            0.1999682041829581  0.07859055305782527  \\
            0.1999682045307765  0.08982210956312595  \\
            0.1999682046475313  0.1010613476062314  \\
            0.19996820452373618  0.11230959800557637  \\
            0.19996820417224237  0.12356747941597948  \\
            0.19996820362643988  0.13483567444425562  \\
            0.19996820293688403  0.1461154636342167  \\
            0.19996820216666666  0.15740915183117207  \\
        }
        ;
    \addplot[color={rgb,1:red,0.502;green,0.502;blue,0.502}, name path={a920762a-d336-4e1d-b66a-95ee861eb4f8}, draw opacity={0.25}, line width={0.1}, solid, forget plot]
        table[row sep={\\}]
        {
            \\
            0.20611960233333335  0.04487715711766979  \\
            0.20611960325761167  0.05613110596234895  \\
            0.20611960408550423  0.0673673598287346  \\
            0.2061196047364545  0.07860311632026916  \\
            0.20611960514897135  0.08984597375713471  \\
            0.20611960528587536  0.10109888172387062  \\
            0.20611960513660454  0.11236279307574953  \\
            0.20611960471705912  0.1236381149258848  \\
            0.2061196040673514  0.13492552158683233  \\
            0.20611960324777379  0.14622649016019462  \\
            0.20611960233333335  0.15754388480717682  \\
        }
        ;
    \addplot[color={rgb,1:red,0.502;green,0.502;blue,0.502}, name path={a82d04f0-67f4-4090-aa98-1eca4d8ef662}, draw opacity={0.25}, line width={0.1}, solid, forget plot]
        table[row sep={\\}]
        {
            \\
            0.2122710025  0.04487136962829655  \\
            0.21227100360196116  0.05612563287887272  \\
            0.21227100458783607  0.06736568162579662  \\
            0.21227100536131296  0.07860914849751283  \\
            0.21227100584967618  0.08986319860985523  \\
            0.21227100600963467  0.1011299937391038  \\
            0.21227100582962008  0.11240981908966702  \\
            0.21227100532931756  0.12370268190535638  \\
            0.21227100455687117  0.1350091095795613  \\
            0.21227100358408668  0.1463306100739647  \\
            0.2122710025  0.1576701304503246  \\
        }
        ;
    \addplot[color={rgb,1:red,0.502;green,0.502;blue,0.502}, name path={aa18d7ad-f348-4604-b748-3f4c99813c7d}, draw opacity={0.25}, line width={0.1}, solid, forget plot]
        table[row sep={\\}]
        {
            \\
            0.21842240266666665  0.04486309798935874  \\
            0.2184224039799177  0.056115209461446  \\
            0.21842240515313588  0.06735684052934798  \\
            0.21842240607120084  0.07860723787386087  \\
            0.2184224066483922  0.0898725857833963  \\
            0.218422406834599  0.10115369345270674  \\
            0.21842240661740944  0.11244982595007046  \\
            0.218422406021203  0.12376039434363861  \\
            0.21842240510371397  0.13508565582842222  \\
            0.21842240395035464  0.14642701504444167  \\
            0.21842240266666665  0.15778714679780784  \\
        }
        ;
    \addplot[color={rgb,1:red,0.502;green,0.502;blue,0.502}, name path={40837fe8-9786-49ee-a00f-4588a79151d8}, draw opacity={0.25}, line width={0.1}, solid, forget plot]
        table[row sep={\\}]
        {
            \\
            0.22457380283333334  0.04485360527005957  \\
            0.2245738043982078  0.05609862960554776  \\
            0.22457380579374378  0.06733896142861041  \\
            0.22457380688235992  0.07859567915993154  \\
            0.22457380756341977  0.08987279803444295  \\
            0.22457380777932734  0.10116896178434373  \\
            0.22457380751711667  0.11248200427707714  \\
            0.2245738068069738  0.12381055584075983  \\
            0.22457380571806682  0.13515451980748214  \\
            0.22457380435187183  0.14651516205877915  \\
            0.22457380283333334  0.15789500367297732  \\
        }
        ;
    \addplot[color={rgb,1:red,0.502;green,0.502;blue,0.502}, name path={677ee67e-fb43-4398-8e5f-25f93eb9b2c6}, draw opacity={0.25}, line width={0.1}, solid, forget plot]
        table[row sep={\\}]
        {
            \\
            0.230725203  0.0448476314166718  \\
            0.23072520486502654  0.0560742577875494  \\
            0.23072520652449496  0.06730951514369517  \\
            0.23072520781409608  0.07857239328873217  \\
            0.23072520861634768  0.08986235486443839  \\
            0.23072520886561346  0.10117477013910177  \\
            0.23072520854882148  0.11250560651005616  \\
            0.23072520770330537  0.1238525722232395  \\
            0.23072520641183372  0.1352151874130788  \\
            0.23072520479481992  0.14659461262109502  \\
            0.230725203  0.15799347431712235  \\
        }
        ;
    \addplot[color={rgb,1:red,0.502;green,0.502;blue,0.502}, name path={82768981-4a6c-46ca-b67d-d93e6c227c26}, draw opacity={0.25}, line width={0.1}, solid, forget plot]
        table[row sep={\\}]
        {
            \\
            0.23687660316666667  0.044848601431894536  \\
            0.236876605390478  0.05603861868336703  \\
            0.23687660736326796  0.06726491146826322  \\
            0.23687660888933887  0.07853484727372217  \\
            0.23687660983260314  0.08983964813226591  \\
            0.23687661011901207  0.10117011389147092  \\
            0.23687660973601576  0.11251997500326537  \\
            0.23687660872968597  0.12388596778461915  \\
            0.2368766071989195  0.13526726829449545  \\
            0.23687660528641738  0.1466649843626945  \\
            0.23687660316666667  0.1580818906263962  \\
        }
        ;
    \addplot[color={rgb,1:red,0.502;green,0.502;blue,0.502}, name path={1b755dd0-59ba-4d0b-8210-1371c1d9368e}, draw opacity={0.25}, line width={0.1}, solid, forget plot]
        table[row sep={\\}]
        {
            \\
            0.24302800333333333  0.04485107224020261  \\
            0.24302800598714014  0.055984546650524734  \\
            0.2430280083316136  0.0672000915347298  \\
            0.24302801013525505  0.07848003044705751  \\
            0.24302801124205642  0.08980299748430785  \\
            0.2430280115694266  0.10115406627813653  \\
            0.243028011106145  0.11252457917280535  \\
            0.2430280099088706  0.12391040769746414  \\
            0.24302800809555822  0.13531051547766182  \\
            0.24302800583509376  0.14672603705782658  \\
            0.24302800333333333  0.15815975337140403  \\
        }
        ;
    \addplot[color={rgb,1:red,0.502;green,0.502;blue,0.502}, name path={3136a9f1-3e60-4462-88ad-ed1b0ade82c3}, draw opacity={0.25}, line width={0.1}, solid, forget plot]
        table[row sep={\\}]
        {
            \\
            0.2491794035  0.04483645969082626  \\
            0.24917940667070462  0.05589987935954851  \\
            0.24917940945540965  0.06710837261316725  \\
            0.24917941158388188  0.07840456300083282  \\
            0.2491794128796673  0.08975076671054721  \\
            0.24917941325176104  0.10112585713526756  \\
            0.24917941269122246  0.11251906151488966  \\
            0.24917941126740398  0.12392572336658478  \\
            0.24917940912069386  0.13534484638930663  \\
            0.24917940645069037  0.14677775336237048  \\
            0.2491794035  0.15822728314697948  \\
        }
        ;
    \addplot[color={rgb,1:red,0.502;green,0.502;blue,0.502}, name path={b6387c91-1e0d-4702-b5df-9ab4c1731c4a}, draw opacity={0.25}, line width={0.1}, solid, forget plot]
        table[row sep={\\}]
        {
            \\
            0.25533080366666666  0.044774484794636006  \\
            0.25533080746060915  0.055767078407355675  \\
            0.25533081076544345  0.06698175664728288  \\
            0.2553308132727256  0.07830500208116842  \\
            0.25533081478615677  0.08968155388848503  \\
            0.2553308152066385  0.10108497604801062  \\
            0.2553308145285274  0.11250328981595437  \\
            0.2553308128362228  0.12393193634922521  \\
            0.25533081029642163  0.13537034567568784  \\
            0.2553308071446905  0.14682028588402926  \\
            0.25533080366666666  0.15828503040023142  \\
        }
        ;
    \addplot[color={rgb,1:red,0.502;green,0.502;blue,0.502}, name path={448d932f-5910-44f5-bf92-7e192fe972f8}, draw opacity={0.25}, line width={0.1}, solid, forget plot]
        table[row sep={\\}]
        {
            \\
            0.2614822038333333  0.044592370393489775  \\
            0.2614822083805179  0.055562800352296375  \\
            0.2614822122977724  0.06681186401686831  \\
            0.26148221524526294  0.07817837794098743  \\
            0.26148221700869273  0.0895944503248274  \\
            0.26148221748119715  0.10103129187374736  \\
            0.2614822166614041  0.11247741035986174  \\
            0.26148221465135507  0.12392927958575795  \\
            0.2614822116485037  0.13538726081860353  \\
            0.26148220793049315  0.1468538547638709  \\
            0.2614822038333333  0.15833308352046074  \\
        }
        ;
    \addplot[color={rgb,1:red,0.502;green,0.502;blue,0.502}, name path={2c48ce0c-f2a9-409e-8d72-c810913662b4}, draw opacity={0.25}, line width={0.1}, solid, forget plot]
        table[row sep={\\}]
        {
            \\
            0.267633604  0.04421530669424536  \\
            0.26763360945783743  0.055265286576094926  \\
            0.2676336140936322  0.06659199933255601  \\
            0.26763361755129433  0.0780229410851018  \\
            0.2676336196015953  0.0894893370174616  \\
            0.2676336201299862  0.10096517151955674  \\
            0.26763361914018513  0.11244189405263647  \\
            0.267633616754736  0.12391821512694683  \\
            0.2676336132069755  0.13539601336710996  \\
            0.26763360882374065  0.146878777620109  \\
            0.267633604  0.1583712647951522  \\
        }
        ;
    \addplot[color={rgb,1:red,0.502;green,0.502;blue,0.502}, name path={6783c59f-b682-4b65-bde9-8285fa0ade01}, draw opacity={0.25}, line width={0.1}, solid, forget plot]
        table[row sep={\\}]
        {
            \\
            0.2737850041666667  0.04362179231228574  \\
            0.2737850107226667  0.054863194210018736  \\
            0.27378501619896545  0.06631938676320463  \\
            0.27378502024718865  0.07783891230973548  \\
            0.2737850226271058  0.08936715393102769  \\
            0.27378502321600184  0.10088757302268427  \\
            0.2737850220232743  0.11239756478204808  \\
            0.27378501919516685  0.12389944276158213  \\
            0.2737850150068578  0.13539721850969472  \\
            0.27378500984270293  0.14689558885228318  \\
            0.2737850041666667  0.15839986466358472  \\
        }
        ;
    \addplot[color={rgb,1:red,0.502;green,0.502;blue,0.502}, name path={3c89ab1e-2669-4fa0-8adf-b3704b1ce81d}, draw opacity={0.25}, line width={0.1}, solid, forget plot]
        table[row sep={\\}]
        {
            \\
            0.27993640433333333  0.04280470972471024  \\
            0.2799364122072503  0.054354784921220145  \\
            0.27993641866383007  0.06599624378782942  \\
            0.27993642339614044  0.07762899205874955  \\
            0.27993642615630765  0.0892300705468901  \\
            0.2799364268119306  0.10080008660836114  \\
            0.2799364253784347  0.11234559906099814  \\
            0.27993642202944796  0.12387389111256711  \\
            0.27993641708899336  0.1353916860143239  \\
            0.27993641100872163  0.1469050910532755  \\
            0.27993640433333333  0.15841995001612175  \\
        }
        ;
    \addplot[color={rgb,1:red,0.502;green,0.502;blue,0.502}, name path={857743c6-760b-4c32-bb4e-080d6c2d3151}, draw opacity={0.25}, line width={0.1}, solid, forget plot]
        table[row sep={\\}]
        {
            \\
            0.2860878045  0.04173197487913953  \\
            0.28608781394460575  0.05374561692252722  \\
            0.28608782154156026  0.06563029646824577  \\
            0.28608782706858  0.07739859238088641  \\
            0.2860878302703292  0.08908151204073678  \\
            0.2860878310016841  0.10070490343691223  \\
            0.2860878292843385  0.11228748906818595  \\
            0.28608782532372357  0.12384268624056223  \\
            0.2860878195010295  0.13538039027131413  \\
            0.28608781234672537  0.14690827504726234  \\
            0.2860878045  0.15843280185958125  \\
        }
        ;
    \addplot[color={rgb,1:red,0.502;green,0.502;blue,0.502}, name path={e2ded9c9-8a1d-43ab-b6e5-6bdd685e94ba}, draw opacity={0.25}, line width={0.1}, solid, forget plot]
        table[row sep={\\}]
        {
            \\
            0.2922392046666667  0.040364235789079016  \\
            0.2922392159645786  0.05305301859198817  \\
            0.2922392248876448  0.06523561804303711  \\
            0.2922392313429909  0.07715580666129163  \\
            0.2922392350620107  0.08892601151667356  \\
            0.2922392358823359  0.10060470052744373  \\
            0.29223923383244643  0.11222496656167623  \\
            0.2922392291550817  0.12380709913175461  \\
            0.2922392222985777  0.13536442502857887  \\
            0.29223921388583923  0.14690623885046122  \\
            0.2922392046666667  0.15843959620544462  \\
        }
        ;
    \addplot[color={rgb,1:red,0.502;green,0.502;blue,0.502}, name path={993e17ec-266d-4a4a-a145-ab4ad8ed7010}, draw opacity={0.25}, line width={0.1}, solid, forget plot]
        table[row sep={\\}]
        {
            \\
            0.29839060483333335  0.038669523760712386  \\
            0.29839061828781116  0.05231414428199146  \\
            0.2983906287592233  0.06483332747820882  \\
            0.2983906363076548  0.07691097355749534  \\
            0.2983906406382771  0.08876885622063425  \\
            0.2983906415665778  0.1005024406188454  \\
            0.29839063912928604  0.11215989152137111  \\
            0.29839063361345675  0.12376847642707164  \\
            0.29839062554657736  0.13534495550240616  \\
            0.29839061566010217  0.146900166876214  \\
            0.29839060483333335  0.1584414626579145  \\
        }
        ;
    \addplot[color={rgb,1:red,0.502;green,0.502;blue,0.502}, name path={c64fb68f-89b3-4ff9-9e8f-98c0e06b812c}, draw opacity={0.25}, line width={0.1}, solid, forget plot]
        table[row sep={\\}]
        {
            \\
            0.304542005  0.036613858106809886  \\
            0.3045420209191786  0.051595405041978036  \\
            0.3045420332171931  0.06445101021750609  \\
            0.3045420420640752  0.0766755753846573  \\
            0.3045420471234639  0.08861550954242504  \\
            0.3045420481858039  0.10040110174599662  \\
            0.30454204529919815  0.11209411675340528  \\
            0.3045420388038831  0.12372815935402574  \\
            0.30454202932089514  0.13532316547991338  \\
            0.30454201770929595  0.14689135308812828  \\
            0.304542005  0.1584389348467872  \\
        }
        ;
    \addplot[color={rgb,1:red,0.502;green,0.502;blue,0.502}, name path={b3378eb0-d604-49d6-98c6-1b2d24f593c6}, draw opacity={0.25}, line width={0.1}, solid, forget plot]
        table[row sep={\\}]
        {
            \\
            0.30637899  0.03592799979999999  \\
            0.3063790067118232  0.051406599548768375  \\
            0.30637901962687536  0.0643463473441166  \\
            0.30637902891051905  0.0766090956582217  \\
            0.3063790342150497  0.08857138189245868  \\
            0.3063790353238422  0.10037153389437943  \\
            0.3063790322893496  0.11207466773029352  \\
            0.30637902546977697  0.12371603274379787  \\
            0.3063790155173971  0.13531642747556027  \\
            0.30637900333372864  0.14688851094714875  \\
            0.30637899  0.1584389348467872  \\
        }
        ;
    \addplot[color={rgb,1:red,0.502;green,0.502;blue,0.502}, name path={04b389c3-17bf-46e1-82cb-2ae3e5fabd8f}, draw opacity={0.25}, line width={0.1}, solid, forget plot]
        table[row sep={\\}]
        {
            \\
            0.3145490964  0.03592799979999999  \\
            0.31454911697933774  0.050842252216885195  \\
            0.3145491330389462  0.0639477319598452  \\
            0.31454914455114974  0.07633836320599678  \\
            0.3145491511093207  0.08838596984491395  \\
            0.31454915246455956  0.10024473922294365  \\
            0.3145491486937438  0.11198976551337624  \\
            0.3145491402472744  0.1236619888789181  \\
            0.3145491279358046  0.1352854868089334  \\
            0.31454911287516696  0.1468749285743424  \\
            0.3145490964  0.1584389348467872  \\
        }
        ;
    \addplot[color={rgb,1:red,0.502;green,0.502;blue,0.502}, name path={c9cdde0d-457b-42da-823a-00e3a175686a}, draw opacity={0.25}, line width={0.1}, solid, forget plot]
        table[row sep={\\}]
        {
            \\
            0.3227192028  0.03592799979999999  \\
            0.3227192280346334  0.050519563370960065  \\
            0.3227192479595764  0.06365532593896576  \\
            0.322719262253222  0.0761163200086327  \\
            0.32271927038915565  0.08822441201016228  \\
            0.32271927207158807  0.10012985818882085  \\
            0.3227192674084074  0.11191047652297306  \\
            0.32271925696575005  0.12361005525695089  \\
            0.32271924175138167  0.13525480603821718  \\
            0.3227192231467283  0.14686102249543315  \\
            0.3227192028  0.1584389348467872  \\
        }
        ;
    \addplot[color={rgb,1:red,0.502;green,0.502;blue,0.502}, name path={4c1cbfc0-e7a0-4a04-b2c0-0096fea1dad7}, draw opacity={0.25}, line width={0.1}, solid, forget plot]
        table[row sep={\\}]
        {
            \\
            0.33088930920000004  0.03592799979999999  \\
            0.33088934012977145  0.05032389511663432  \\
            0.3308893647732326  0.06344555413950083  \\
            0.3308893825008494  0.07594047385431368  \\
            0.33088939260291655  0.08808838882812306  \\
            0.3308893947096599  0.10002906445335612  \\
            0.330889388962727  0.111838736376796  \\
            0.3308893760706582  0.12356184175324728  \\
            0.33088935728515295  0.13522565408360918  \\
            0.3308893343162264  0.14684755481245004  \\
            0.33088930920000004  0.1584389348467872  \\
        }
        ;
    \addplot[color={rgb,1:red,0.502;green,0.502;blue,0.502}, name path={cc5df659-2cb9-413a-87d8-5e3a750481a7}, draw opacity={0.25}, line width={0.1}, solid, forget plot]
        table[row sep={\\}]
        {
            \\
            0.3390594156  0.03592799979999999  \\
            0.3390594535331789  0.050198985415511384  \\
            0.3390594839471249  0.06329508780318066  \\
            0.3390595058962311  0.07580368495588057  \\
            0.3390595184317064  0.08797648907387864  \\
            0.3390595210777388  0.0999427782695948  \\
            0.3390595140104026  0.11177546452623631  \\
            0.33905949811136643  0.12351830759697713  \\
            0.33905947493279903  0.1351988294755506  \\
            0.3390594465904846  0.14683499422035584  \\
            0.3390594156  0.1584389348467872  \\
        }
        ;
    \addplot[color={rgb,1:red,0.502;green,0.502;blue,0.502}, name path={c1f550f6-1092-4dbd-b75f-e3e610afb022}, draw opacity={0.25}, line width={0.1}, solid, forget plot]
        table[row sep={\\}]
        {
            \\
            0.347229522  0.03592799979999999  \\
            0.347229568558387  0.0501157010556521  \\
            0.34722960604725867  0.06318620669925075  \\
            0.34722963318103195  0.07569809149872803  \\
            0.34722964871705686  0.08788580478010745  \\
            0.3472296520389769  0.09987025383397206  \\
            0.347229643357913  0.11172078627973303  \\
            0.3472296237651819  0.12347987138476386  \\
            0.34722959518198687  0.13517475994558956  \\
            0.34722956022439017  0.146823603475183  \\
            0.347229522  0.1584389348467872  \\
        }
        ;
    \addplot[color={rgb,1:red,0.502;green,0.502;blue,0.502}, name path={38ed5571-cc71-45bb-a53c-f546caf2e636}, draw opacity={0.25}, line width={0.1}, solid, forget plot]
        table[row sep={\\}]
        {
            \\
            0.3553996284  0.03592799979999999  \\
            0.3553996855847916  0.05005814756146559  \\
            0.3553997317635693  0.06310646525569791  \\
            0.35539976526649114  0.075616723614403  \\
            0.3553997844959765  0.0878129941358577  \\
            0.3553997886575703  0.09981011168845659  \\
            0.35539977799939276  0.11167428840052289  \\
            0.355399753866825  0.12344654795511546  \\
            0.3553997186336654  0.13515359488510492  \\
            0.35539967553204227  0.14681349801125748  \\
            0.3553996284  0.1584389348467872  \\
        }
        ;
    \addplot[color={rgb,1:red,0.502;green,0.502;blue,0.502}, name path={f364a5a1-0b38-4861-ba07-a8d01c85a3f2}, draw opacity={0.25}, line width={0.1}, solid, forget plot]
        table[row sep={\\}]
        {
            \\
            0.3635697348  0.03592799979999999  \\
            0.3635698050772228  0.05001720842195701  \\
            0.3635698619412591  0.06304732859525827  \\
            0.3635699032723407  0.07555393145493997  \\
            0.36356992704499996  0.08775485321077865  \\
            0.36356993224432615  0.09976072362718605  \\
            0.3635699191595374  0.11163524379598788  \\
            0.3635698894446839  0.12341808044642985  \\
            0.3635698460282941  0.13513528773234879  \\
            0.3635697929005032  0.14680469016934244  \\
            0.3635697348  0.1584389348467872  \\
        }
        ;
    \addplot[color={rgb,1:red,0.502;green,0.502;blue,0.502}, name path={cde8446c-d92b-4e66-9ba6-81ee4fb00cde}, draw opacity={0.25}, line width={0.1}, solid, forget plot]
        table[row sep={\\}]
        {
            \\
            0.37173984120000003  0.03592799979999999  \\
            0.37173992760768426  0.049987405550129586  \\
            0.371739997618834  0.06300295281505029  \\
            0.3717400485754441  0.07550532423590893  \\
            0.37174007793497926  0.0877085635960768  \\
            0.3717400844129372  0.09972045244561493  \\
            0.3717400683464427  0.1116027812957342  \\
            0.3717400317654777  0.12339405088118224  \\
            0.37173997827820915  0.13511966502202966  \\
            0.37173991280679575  0.14679712389736138  \\
            0.37173984120000003  0.1584389348467872  \\
        }
        ;
    \addplot[color={rgb,1:red,0.502;green,0.502;blue,0.502}, name path={edddf818-5282-486a-bf46-b45be318bf4a}, draw opacity={0.25}, line width={0.1}, solid, forget plot]
        table[row sep={\\}]
        {
            \\
            0.3799099476  0.03592799979999999  \\
            0.3799100538813233  0.04996530544721607  \\
            0.3799101400745015  0.0629693016722311  \\
            0.3799102028699015  0.07546755580465006  \\
            0.3799102390987651  0.08767175970237416  \\
            0.3799102471492997  0.099687781838861  \\
            0.37991022741664443  0.11157600045120916  \\
            0.37991018238932317  0.12337396392155768  \\
            0.3799101165076361  0.1351064806361067  \\
            0.3799100358389651  0.1467907016420016  \\
            0.3799099476  0.1584389348467872  \\
        }
        ;
    \addplot[color={rgb,1:red,0.502;green,0.502;blue,0.502}, name path={e21fc0bd-597c-48ad-9aed-867030d54ca3}, draw opacity={0.25}, line width={0.1}, solid, forget plot]
        table[row sep={\\}]
        {
            \\
            0.388080054  0.03592799979999999  \\
            0.3880801847683727  0.049948674250106025  \\
            0.3880802908832295  0.06294355011247844  \\
            0.388080368241107  0.07543809536494703  \\
            0.3880804129144707  0.08764250885603442  \\
            0.388080422896663  0.09966137405082315  \\
            0.38808039865509203  0.1115540415055056  \\
            0.38808034323767204  0.12335730516822024  \\
            0.38808026210226687  0.1350954557794504  \\
            0.38808016272226664  0.1467853044195057  \\
            0.388080054  0.1584389348467872  \\
        }
        ;
    \addplot[color={rgb,1:red,0.502;green,0.502;blue,0.502}, name path={bdb8a1e5-e942-49c8-8d85-85f3fb9b6a32}, draw opacity={0.25}, line width={0.1}, solid, forget plot]
        table[row sep={\\}]
        {
            \\
            0.3962501604  0.03592799979999999  \\
            0.3962503213439879  0.04993601057118275  \\
            0.39625045198738607  0.06292369096447809  \\
            0.39625054725697945  0.07541503060303202  \\
            0.3962506023076106  0.08761925659644776  \\
            0.39625061465992667  0.0996400846401548  \\
            0.396250584873342  0.11153612285570005  \\
            0.3962505166771905  0.1233435784347396  \\
            0.3962504167708732  0.1350863065436754  \\
            0.3962502943518805  0.14678080605369023  \\
            0.3962501604  0.1584389348467872  \\
        }
        ;
    \addplot[color={rgb,1:red,0.502;green,0.502;blue,0.502}, name path={e3505ab6-19aa-46a9-a74c-b56f7d3df854}, draw opacity={0.25}, line width={0.1}, solid, forget plot]
        table[row sep={\\}]
        {
            \\
            0.4044202668  0.03592799979999999  \\
            0.4044204649384488  0.04992627670438774  \\
            0.4044206257847325  0.0629082765068913  \\
            0.40442074308041226  0.07539691264131396  \\
            0.4044208108760499  0.08760076370743183  \\
            0.40442082613295866  0.09962295424492405  \\
            0.40442078952990634  0.1115215568622359  \\
            0.40442070562344534  0.12333232717257825  \\
            0.4044205826222216  0.13507876151039622  \\
            0.40442043183406434  0.14677708266194575  \\
            0.4044202668  0.1584389348467872  \\
        }
        ;
    \addplot[color={rgb,1:red,0.502;green,0.502;blue,0.502}, name path={52f80ba1-e70d-445d-a333-94898050638f}, draw opacity={0.25}, line width={0.1}, solid, forget plot]
        table[row sep={\\}]
        {
            \\
            0.4125903732  0.03592799979999999  \\
            0.41259061720114876  0.04991873810880169  \\
            0.41259081523876096  0.06289624752843993  \\
            0.41259095960797343  0.07538263871295257  \\
            0.41259104304231786  0.08758604694108861  \\
            0.4125910618533038  0.09960918963998613  \\
            0.41259101687739697  0.11150975228942685  \\
            0.4125909136693054  0.12332314476826371  \\
            0.41259076226170527  0.13507257187209598  \\
            0.41259057653843373  0.14677401847132818  \\
            0.4125903732  0.1584389348467872  \\
        }
        ;
    \addplot[color={rgb,1:red,0.502;green,0.502;blue,0.502}, name path={8adb4671-701b-405f-bfe5-3c27800f4243}, draw opacity={0.25}, line width={0.1}, solid, forget plot]
        table[row sep={\\}]
        {
            \\
            0.4207604796  0.03592799979999999  \\
            0.4207607801833836  0.049912864219275314  \\
            0.42076102401820403  0.06288681890780116  \\
            0.42076120164109976  0.0753713649848897  \\
            0.42076130423833435  0.08757432784736127  \\
            0.4207613273889434  0.09959814115254509  \\
            0.42076127214181636  0.11150020917153226  \\
            0.4207611452443796  0.12331567752023932  \\
            0.4207609589138673  0.13506751625594868  \\
            0.4207607301653186  0.14677150894740648  \\
            0.4207604796  0.1584389348467872  \\
        }
        ;
    \addplot[color={rgb,1:red,0.502;green,0.502;blue,0.502}, name path={da3eb792-5934-4de1-81c5-441a20a1bf7a}, draw opacity={0.25}, line width={0.1}, solid, forget plot]
        table[row sep={\\}]
        {
            \\
            0.428930586  0.03592799979999999  \\
            0.42893095644763335  0.04990826521558875  \\
            0.42893125667534415  0.06287940190572144  \\
            0.4289314750975138  0.07536244221308494  \\
            0.4289316011277286  0.08756499052624503  \\
            0.4289316295615548  0.09958928019716523  \\
            0.4289315617399195  0.11149250990548899  \\
            0.42893140581376  0.12330962313150153  \\
            0.4289311765797538  0.13506340201414993  \\
            0.4289308948342452  0.14676946205934605  \\
            0.428930586  0.1584389348467872  \\
        }
        ;
    \addplot[color={rgb,1:red,0.502;green,0.502;blue,0.502}, name path={bceec131-b58e-4342-a869-e054fed86a34}, draw opacity={0.25}, line width={0.1}, solid, forget plot]
        table[row sep={\\}]
        {
            \\
            0.4371006924  0.03592799979999999  \\
            0.43710114921569976  0.04990465056326204  \\
            0.43710151887713694  0.06287355056068024  \\
            0.43710178727237936  0.07535536825558958  \\
            0.43710194187024826  0.0875575475904319  \\
            0.43710197670947615  0.09958217872865896  \\
            0.4371018935407276  0.11148630892654979  \\
            0.4371017021270834  0.12330472670796604  \\
            0.43710142024252935  0.13506006431355577  \\
            0.4371010732032426  0.1467677983278903  \\
            0.4371006924  0.1584389348467872  \\
        }
        ;
    \addplot[color={rgb,1:red,0.502;green,0.502;blue,0.502}, name path={ad0f599e-8562-4338-8d44-14f2507f9cb4}, draw opacity={0.25}, line width={0.1}, solid, forget plot]
        table[row sep={\\}]
        {
            \\
            0.4452707988  0.03592799979999999  \\
            0.4452713625764755  0.04990180113190044  \\
            0.4452718177100991  0.06286892412607183  \\
            0.4452721471607802  0.07534975289929793  \\
            0.4452723364304007  0.08755161311240367  \\
            0.44527237899040106  0.09957649130687644  \\
            0.4452722771764445  0.11148132234100594  \\
            0.44527204253287295  0.1233007755881061  \\
            0.4452716961422626  0.13505736397892873  \\
            0.44527126863525707  0.14676645013696812  \\
            0.4452707988  0.1584389348467872  \\
        }
        ;
    \addplot[color={rgb,1:red,0.502;green,0.502;blue,0.502}, name path={d522b895-6b44-4368-9d9b-5f4a03a6d336}, draw opacity={0.25}, line width={0.1}, solid, forget plot]
        table[row sep={\\}]
        {
            \\
            0.4534409052  0.03592799979999999  \\
            0.4534416017898208  0.04989955001406339  \\
            0.45344216209086075  0.06286526034495345  \\
            0.45344256585529447  0.07534529165925995  \\
            0.45344279692682216  0.08754688127810589  \\
            0.4534428487172092  0.09957193988220653  \\
            0.453442724405047  0.11147731825425801  \\
            0.4534424373788569  0.12329759384162746  \\
            0.4534420121533931  0.1350551847400687  \\
            0.4534414854402353  0.14676536064905302  \\
            0.4534409052  0.1584389348467872  \\
        }
        ;
    \addplot[color={rgb,1:red,0.502;green,0.502;blue,0.502}, name path={85546cd7-7375-4062-b0a0-857efa55437f}, draw opacity={0.25}, line width={0.1}, solid, forget plot]
        table[row sep={\\}]
        {
            \\
            0.4616110116  0.03592799979999999  \\
            0.4616118737533524  0.049897769060290524  \\
            0.4616125633309914  0.0628623561618278  \\
            0.4616130570344347  0.0753417461325479  \\
            0.46161333800683757  0.08754310960262762  \\
            0.46161340070814477  0.09956830112584683  \\
            0.4616132495192633  0.11147410813803929  \\
            0.46161289952734985  0.12329503693436912  \\
            0.4616123783195899  0.1350534303037881  \\
            0.4616117292452863  0.14676448255345573  \\
            0.4616110116  0.1584389348467872  \\
        }
        ;
    \addplot[color={rgb,1:red,0.502;green,0.502;blue,0.502}, name path={a2031982-e16c-4bd7-aba5-e38ec38e3a34}, draw opacity={0.25}, line width={0.1}, solid, forget plot]
        table[row sep={\\}]
        {
            \\
            0.469781118  0.03592799979999999  \\
            0.4697821877598345  0.049896359256787245  \\
            0.4697830359278678  0.06286005361862049  \\
            0.4697836375593605  0.07533892916690509  \\
            0.4697839772059938  0.0875401057510919  \\
            0.46978405261252476  0.09956539601100546  \\
            0.469783869779121  0.11147153934678937  \\
            0.46978344502730296  0.12329298682876325  \\
            0.4697828076354344  0.13505202150706594  \\
            0.46978200759368277  0.14676377679387123  \\
            0.469781118  0.1584389348467872  \\
        }
        ;
    \addplot[color={rgb,1:red,0.502;green,0.502;blue,0.502}, name path={3b51037e-1953-4aa7-9710-658e431c402b}, draw opacity={0.25}, line width={0.1}, solid, forget plot]
        table[row sep={\\}]
        {
            \\
            0.4779512244  0.03592799979999999  \\
            0.4779525568001732  0.04989524374163353  \\
            0.47795359867673637  0.06285822942069735  \\
            0.4779543281998854  0.07533669359232714  \\
            0.47795473520100135  0.08753771719333776  \\
            0.47795482514627424  0.09956308132165273  \\
            0.4779546058046166  0.11146948876243899  \\
            0.47795409400497524  0.12329134764296687  \\
            0.47795331721670903  0.13505089369454465  \\
            0.4779523309766128  0.14676321136241535  \\
            0.4779512244  0.1584389348467872  \\
        }
        ;
    \addplot[color={rgb,1:red,0.502;green,0.502;blue,0.502}, name path={76e655cf-6139-4619-b9ec-613694cea9e7}, draw opacity={0.25}, line width={0.1}, solid, forget plot]
        table[row sep={\\}]
        {
            \\
            0.4861213308  0.03592799979999999  \\
            0.4861229999453529  0.04989436266866223  \\
            0.48612427618904097  0.06285678714039036  \\
            0.4861251545321778  0.07533492361160196  \\
            0.48612563575046147  0.08753582308273485  \\
            0.4861257421556427  0.09956124277961106  \\
            0.4861254817658055  0.11146785748017712  \\
            0.4861248718771444  0.12329004190227692  \\
            0.48612393006187904  0.13504999438898  \\
            0.4861227147347506  0.14676276020703213  \\
            0.4861213308  0.1584389348467872  \\
        }
        ;
    \addplot[color={rgb,1:red,0.502;green,0.502;blue,0.502}, name path={c3437aa4-e56c-45f6-92bd-09da729a11a2}, draw opacity={0.25}, line width={0.1}, solid, forget plot]
        table[row sep={\\}]
        {
            \\
            0.4942914372  0.03592799979999999  \\
            0.49429354697824773  0.0498936693892024  \\
            0.49429510072870425  0.06285565134393827  \\
            0.49429614816331185  0.07533352819344553  \\
            0.4942967048348483  0.08753432788268865  \\
            0.494296830506958  0.09955978951408254  \\
            0.4942965249431467  0.1114665664195881  \\
            0.4942958111129259  0.12328900736500872  \\
            0.4942946776020239  0.13504928127644167  \\
            0.494293182827893  0.1467624022727971  \\
            0.4942914372  0.1584389348467872  \\
        }
        ;
    \addplot[color={rgb,1:red,0.502;green,0.502;blue,0.502}, name path={474dc2d1-d2d3-45c0-b007-32e301958bad}, draw opacity={0.25}, line width={0.1}, solid, forget plot]
        table[row sep={\\}]
        {
            \\
            0.5024615436  0.03592799979999999  \\
            0.5024642479774059  0.0498931275903585  \\
            0.5024661134932674  0.06285476314142364  \\
            0.5024673489413295  0.07533243598866415  \\
            0.5024679677106296  0.08753315637090663  \\
            0.5024681203045294  0.09955864963716243  \\
            0.5024677634546381  0.11146555273962629  \\
            0.5024669542191273  0.12328819437833258  \\
            0.5024656028282337  0.13504872050046546  \\
            0.5024637759106771  0.14676212068132216  \\
            0.5024615436  0.1584389348467872  \\
        }
        ;
    \addplot[color={rgb,1:red,0.502;green,0.502;blue,0.502}, name path={18ab6172-2082-409e-9262-e0d437a8711c}, draw opacity={0.25}, line width={0.1}, solid, forget plot]
        table[row sep={\\}]
        {
            \\
            0.51063165  0.03592799979999999  \\
            0.5106351944033974  0.049892709139502826  \\
            0.510637361592821  0.06285407680492099  \\
            0.5106388108383605  0.07533159141680935  \\
            0.5106394400997473  0.08753224973544467  \\
            0.510639648071418  0.09955776672796887  \\
            0.5106392184697081  0.11146476694217285  \\
            0.5106383603992285  0.123287563708417  \\
            0.5106367616621093  0.13504828524465456  \\
            0.5106345700757328  0.1467619020430869  \\
            0.51063165  0.1584389348467872  \\
        }
        ;
    \addplot[color={rgb,1:red,0.502;green,0.502;blue,0.502}, name path={b833bbef-9452-43c4-a540-a3d453ed6c85}, draw opacity={0.25}, line width={0.1}, solid, forget plot]
        table[row sep={\\}]
        {
            \\
            0.5188017564  0.03592799979999999  \\
            0.5188065683278625  0.04989239245934904  \\
            0.518808877237784  0.06285355720098837  \\
            0.518810621879253  0.07533095166289289  \\
            0.5188111013820477  0.08753156254240115  \\
            0.5188114735956995  0.09955709706379755  \\
            0.5188108788505589  0.11146417056132538  \\
            0.5188101253521777  0.12328708478744244  \\
            0.5188082115728961  0.1350479545769144  \\
            0.5188057239001983  0.1467617358927971  \\
            0.5188017564  0.1584389348467872  \\
        }
        ;
    \addplot[color={rgb,1:red,0.502;green,0.502;blue,0.502}, name path={4e772676-22e7-471b-910f-5c108ef40358}, draw opacity={0.25}, line width={0.1}, solid, forget plot]
        table[row sep={\\}]
        {
            \\
            0.5269718628  0.03592799979999999  \\
            0.5269787639157566  0.04989216130947543  \\
            0.5269805936959953  0.06285317785468095  \\
            0.5269829765002086  0.07533048439138054  \\
            0.5269828086779716  0.08753106040674549  \\
            0.5269837504807874  0.09955660746829151  \\
            0.5269826167802892  0.11146373435046013  \\
            0.526982450866389  0.12328673432366069  \\
            0.5269799439722115  0.13504771252740955  \\
            0.5269776053049885  0.14676161423624165  \\
            0.5269718628  0.1584389348467872  \\
        }
        ;
    \addplot[color={rgb,1:red,0.502;green,0.502;blue,0.502}, name path={2f10a5af-4ceb-431d-a181-95b42b26c426}, draw opacity={0.25}, line width={0.1}, solid, forget plot]
        table[row sep={\\}]
        {
            \\
            0.5351419692  0.03592799979999999  \\
            0.5351527026038095  0.049892003887225256  \\
            0.5351520492424573  0.06285291951215571  \\
            0.5351564383917821  0.07533016603620829  \\
            0.535154008691445  0.08753071823734165  \\
            0.5351569956825083  0.09955627367509137  \\
            0.5351538979259829  0.11146343689352065  \\
            0.5351559041316796  0.1232864952278606  \\
            0.5351515995595778  0.13504754737373442  \\
            0.5351511552596466  0.1467615311978387  \\
            0.5351419692  0.1584389348467872  \\
        }
        ;
    \addplot[color={rgb,1:red,0.502;green,0.502;blue,0.502}, name path={be1b0f73-eb08-417b-9c94-be84b39fa6c2}, draw opacity={0.25}, line width={0.1}, solid, forget plot]
        table[row sep={\\}]
        {
            \\
            0.5433120755999999  0.03592799979999999  \\
            0.5433306877795506  0.049891912191107786  \\
            0.5433213979586135  0.06285276909886658  \\
            0.543332881422732  0.07532998057230542  \\
            0.543322743829804  0.08753051895060487  \\
            0.5433330799713588  0.09955607913807567  \\
            0.5433227613075436  0.11146326357384015  \\
            0.5433323929047321  0.12328635582365197  \\
            0.5433213891694104  0.13504745110549474  \\
            0.5433289841335676  0.1467614827621099  \\
            0.5433120755999999  0.1584389348467872  \\
        }
        ;
    \addplot[color={rgb,1:red,0.502;green,0.502;blue,0.502}, name path={cadb7ee6-4a5c-4a17-b080-3dfd1e177609}, draw opacity={0.25}, line width={0.1}, solid, forget plot]
        table[row sep={\\}]
        {
            \\
            0.5514821820000001  0.03592799979999999  \\
            0.5515188154128429  0.049891881625735275  \\
            0.5514821820000002  0.06285271896110357  \\
            0.5515188154128428  0.07532991875100449  \\
            0.5514821820000001  0.08753045252169261  \\
            0.5515188154128428  0.09955601429240378  \\
            0.5514821820000001  0.11146320580061327  \\
            0.551518815412843  0.12328630935558241  \\
            0.5514821820000002  0.1350474190160815  \\
            0.551518815412843  0.14676146661686695  \\
            0.5514821820000002  0.1584389348467872  \\
        }
        ;
    \addplot[color={rgb,1:red,0.502;green,0.502;blue,0.502}, name path={b37cce2b-a741-46f1-b47b-1ebed49edd16}, draw opacity={1.0}, line width={0.75}, solid, forget plot]
        table[row sep={\\}]
        {
            \\
            0.12  0.04495252299071941  \\
            0.12615140016666665  0.04494248592934282  \\
            0.13230280033333333  0.04494776043555422  \\
            0.13845420049999999  0.044949764153980734  \\
            0.14460560066666667  0.044942278699579535  \\
            0.15075700083333332  0.04493204538203043  \\
            0.156908401  0.044923127246484104  \\
            0.16305980116666666  0.04491727692121097  \\
            0.16921120133333334  0.04491196172254526  \\
            0.1753626015  0.044906620646925216  \\
            0.18151400166666667  0.04490069966154803  \\
            0.18766540183333333  0.04489448945657971  \\
            0.193816802  0.04488823654667903  \\
            0.19996820216666666  0.04488243703899517  \\
            0.20611960233333335  0.04487715711766979  \\
            0.2122710025  0.04487136962829655  \\
            0.21842240266666665  0.04486309798935874  \\
            0.22457380283333334  0.04485360527005957  \\
            0.230725203  0.0448476314166718  \\
            0.23687660316666667  0.044848601431894536  \\
            0.24302800333333333  0.04485107224020261  \\
            0.2491794035  0.04483645969082626  \\
            0.25533080366666666  0.044774484794636006  \\
            0.2614822038333333  0.044592370393489775  \\
            0.267633604  0.04421530669424536  \\
            0.2737850041666667  0.04362179231228574  \\
            0.27993640433333333  0.04280470972471024  \\
            0.2860878045  0.04173197487913953  \\
            0.2922392046666667  0.040364235789079016  \\
            0.29839060483333335  0.038669523760712386  \\
            0.304542005  0.036613858106809886  \\
            0.30637899  0.03592799979999999  \\
            0.3145490964  0.03592799979999999  \\
            0.3227192028  0.03592799979999999  \\
            0.33088930920000004  0.03592799979999999  \\
            0.3390594156  0.03592799979999999  \\
            0.347229522  0.03592799979999999  \\
            0.3553996284  0.03592799979999999  \\
            0.3635697348  0.03592799979999999  \\
            0.37173984120000003  0.03592799979999999  \\
            0.3799099476  0.03592799979999999  \\
            0.388080054  0.03592799979999999  \\
            0.3962501604  0.03592799979999999  \\
            0.4044202668  0.03592799979999999  \\
            0.4125903732  0.03592799979999999  \\
            0.4207604796  0.03592799979999999  \\
            0.428930586  0.03592799979999999  \\
            0.4371006924  0.03592799979999999  \\
            0.4452707988  0.03592799979999999  \\
            0.4534409052  0.03592799979999999  \\
            0.4616110116  0.03592799979999999  \\
            0.469781118  0.03592799979999999  \\
            0.4779512244  0.03592799979999999  \\
            0.4861213308  0.03592799979999999  \\
            0.4942914372  0.03592799979999999  \\
            0.5024615436  0.03592799979999999  \\
            0.51063165  0.03592799979999999  \\
            0.5188017564  0.03592799979999999  \\
            0.5269718628  0.03592799979999999  \\
            0.5351419692  0.03592799979999999  \\
            0.5433120755999999  0.03592799979999999  \\
            0.5514821820000001  0.03592799979999999  \\
        }
        ;
    \addplot[color={rgb,1:red,0.502;green,0.502;blue,0.502}, name path={8cf05959-0002-4f13-b71e-e1fa1868f3e2}, draw opacity={1.0}, line width={0.75}, solid, forget plot]
        table[row sep={\\}]
        {
            \\
            0.12  0.05602935219164747  \\
            0.12615140019524057  0.0560404395104915  \\
            0.1323028003913925  0.05605296683605158  \\
            0.13845420058934013  0.05606501577684807  \\
            0.14460560079001664  0.05607536754340089  \\
            0.1507570009944323  0.056084531918666  \\
            0.15690840120370972  0.0560932491500544  \\
            0.16305980141911983  0.05610185389778746  \\
            0.16921120164212095  0.056110045676790836  \\
            0.17536260187440036  0.05611747133702889  \\
            0.1815140021179268  0.056123773948205766  \\
            0.18766540237501228  0.05612866310351651  \\
            0.19381680264838852  0.05613182293510373  \\
            0.19996820294129872  0.056132848421925056  \\
            0.20611960325761167  0.05613110596234895  \\
            0.21227100360196116  0.05612563287887272  \\
            0.2184224039799177  0.056115209461446  \\
            0.2245738043982078  0.05609862960554776  \\
            0.23072520486502654  0.0560742577875494  \\
            0.236876605390478  0.05603861868336703  \\
            0.24302800598714014  0.055984546650524734  \\
            0.24917940667070462  0.05589987935954851  \\
            0.25533080746060915  0.055767078407355675  \\
            0.2614822083805179  0.055562800352296375  \\
            0.26763360945783743  0.055265286576094926  \\
            0.2737850107226667  0.054863194210018736  \\
            0.2799364122072503  0.054354784921220145  \\
            0.28608781394460575  0.05374561692252722  \\
            0.2922392159645786  0.05305301859198817  \\
            0.29839061828781116  0.05231414428199146  \\
            0.3045420209191786  0.051595405041978036  \\
            0.3063790067118232  0.051406599548768375  \\
            0.31454911697933774  0.050842252216885195  \\
            0.3227192280346334  0.050519563370960065  \\
            0.33088934012977145  0.05032389511663432  \\
            0.3390594535331789  0.050198985415511384  \\
            0.347229568558387  0.0501157010556521  \\
            0.3553996855847916  0.05005814756146559  \\
            0.3635698050772228  0.05001720842195701  \\
            0.37173992760768426  0.049987405550129586  \\
            0.3799100538813233  0.04996530544721607  \\
            0.3880801847683727  0.049948674250106025  \\
            0.3962503213439879  0.04993601057118275  \\
            0.4044204649384488  0.04992627670438774  \\
            0.41259061720114876  0.04991873810880169  \\
            0.4207607801833836  0.049912864219275314  \\
            0.42893095644763335  0.04990826521558875  \\
            0.43710114921569976  0.04990465056326204  \\
            0.4452713625764755  0.04990180113190044  \\
            0.4534416017898208  0.04989955001406339  \\
            0.4616118737533524  0.049897769060290524  \\
            0.4697821877598345  0.049896359256787245  \\
            0.4779525568001732  0.04989524374163353  \\
            0.4861229999453529  0.04989436266866223  \\
            0.49429354697824773  0.0498936693892024  \\
            0.5024642479774059  0.0498931275903585  \\
            0.5106351944033974  0.049892709139502826  \\
            0.5188065683278625  0.04989239245934904  \\
            0.5269787639157566  0.04989216130947543  \\
            0.5351527026038095  0.049892003887225256  \\
            0.5433306877795506  0.049891912191107786  \\
            0.5515188154128429  0.049891881625735275  \\
        }
        ;
    \addplot[color={rgb,1:red,0.502;green,0.502;blue,0.502}, name path={a3fca1c9-e97d-45b7-bf1c-8ca56894c94d}, draw opacity={1.0}, line width={0.75}, solid, forget plot]
        table[row sep={\\}]
        {
            \\
            0.12  0.06710618139257553  \\
            0.1261514002210339  0.06713071695180639  \\
            0.13230280044378667  0.0671553292278602  \\
            0.13845420066992722  0.06717967633249355  \\
            0.1446056009012248  0.06720344097092065  \\
            0.1507570011396005  0.06722652798138998  \\
            0.15690840138718576  0.06724891750798433  \\
            0.16305980164638673  0.06727051141803732  \\
            0.16921120191995728  0.06729106515668723  \\
            0.1753626022110816  0.06731023233667703  \\
            0.1815140025234716  0.06732758926788507  \\
            0.18766540286148156  0.067342641873836  \\
            0.19381680323024608  0.06735480163046952  \\
            0.19996820363584517  0.0673633454997533  \\
            0.20611960408550423  0.0673673598287346  \\
            0.21227100458783607  0.06736568162579662  \\
            0.21842240515313588  0.06735684052934798  \\
            0.22457380579374378  0.06733896142861041  \\
            0.23072520652449496  0.06730951514369517  \\
            0.23687660736326796  0.06726491146826322  \\
            0.2430280083316136  0.0672000915347298  \\
            0.24917940945540965  0.06710837261316725  \\
            0.25533081076544345  0.06698175664728288  \\
            0.2614822122977724  0.06681186401686831  \\
            0.2676336140936322  0.06659199933255601  \\
            0.27378501619896545  0.06631938676320463  \\
            0.27993641866383007  0.06599624378782942  \\
            0.28608782154156026  0.06563029646824577  \\
            0.2922392248876448  0.06523561804303711  \\
            0.2983906287592233  0.06483332747820882  \\
            0.3045420332171931  0.06445101021750609  \\
            0.30637901962687536  0.0643463473441166  \\
            0.3145491330389462  0.0639477319598452  \\
            0.3227192479595764  0.06365532593896576  \\
            0.3308893647732326  0.06344555413950083  \\
            0.3390594839471249  0.06329508780318066  \\
            0.34722960604725867  0.06318620669925075  \\
            0.3553997317635693  0.06310646525569791  \\
            0.3635698619412591  0.06304732859525827  \\
            0.371739997618834  0.06300295281505029  \\
            0.3799101400745015  0.0629693016722311  \\
            0.3880802908832295  0.06294355011247844  \\
            0.39625045198738607  0.06292369096447809  \\
            0.4044206257847325  0.0629082765068913  \\
            0.41259081523876096  0.06289624752843993  \\
            0.42076102401820403  0.06288681890780116  \\
            0.42893125667534415  0.06287940190572144  \\
            0.43710151887713694  0.06287355056068024  \\
            0.4452718177100991  0.06286892412607183  \\
            0.45344216209086075  0.06286526034495345  \\
            0.4616125633309914  0.0628623561618278  \\
            0.4697830359278678  0.06286005361862049  \\
            0.47795359867673637  0.06285822942069735  \\
            0.48612427618904097  0.06285678714039036  \\
            0.49429510072870425  0.06285565134393827  \\
            0.5024661134932674  0.06285476314142364  \\
            0.510637361592821  0.06285407680492099  \\
            0.518808877237784  0.06285355720098837  \\
            0.5269805936959953  0.06285317785468095  \\
            0.5351520492424573  0.06285291951215571  \\
            0.5433213979586135  0.06285276909886658  \\
            0.5514821820000002  0.06285271896110357  \\
        }
        ;
    \addplot[color={rgb,1:red,0.502;green,0.502;blue,0.502}, name path={f05a5ea2-3766-4b5a-90a6-c9025e218578}, draw opacity={1.0}, line width={0.75}, solid, forget plot]
        table[row sep={\\}]
        {
            \\
            0.12  0.07818301059350359  \\
            0.126151400241577  0.07821950267194142  \\
            0.13230280048548837  0.0782559880131186  \\
            0.13845420073402734  0.07829231617435715  \\
            0.14460560098962408  0.07832831538807132  \\
            0.15075700125491626  0.07836381408209799  \\
            0.15690840153282662  0.07839861751686146  \\
            0.16305980182664898  0.07843247184616652  \\
            0.16921120214014587  0.07846504059891733  \\
            0.17536260247765992  0.07849590014122762  \\
            0.1815140028442434  0.07852453904258988  \\
            0.18766540324580985  0.07855035379718306  \\
            0.1938168036893129  0.07857263630941203  \\
            0.1999682041829581  0.07859055305782527  \\
            0.2061196047364545  0.07860311632026916  \\
            0.21227100536131296  0.07860914849751283  \\
            0.21842240607120084  0.07860723787386087  \\
            0.22457380688235992  0.07859567915993154  \\
            0.23072520781409608  0.07857239328873217  \\
            0.23687660888933887  0.07853484727372217  \\
            0.24302801013525505  0.07848003044705751  \\
            0.24917941158388188  0.07840456300083282  \\
            0.2553308132727256  0.07830500208116842  \\
            0.26148221524526294  0.07817837794098743  \\
            0.26763361755129433  0.0780229410851018  \\
            0.27378502024718865  0.07783891230973548  \\
            0.27993642339614044  0.07762899205874955  \\
            0.28608782706858  0.07739859238088641  \\
            0.2922392313429909  0.07715580666129163  \\
            0.2983906363076548  0.07691097355749534  \\
            0.3045420420640752  0.0766755753846573  \\
            0.30637902891051905  0.0766090956582217  \\
            0.31454914455114974  0.07633836320599678  \\
            0.322719262253222  0.0761163200086327  \\
            0.3308893825008494  0.07594047385431368  \\
            0.3390595058962311  0.07580368495588057  \\
            0.34722963318103195  0.07569809149872803  \\
            0.35539976526649114  0.075616723614403  \\
            0.3635699032723407  0.07555393145493997  \\
            0.3717400485754441  0.07550532423590893  \\
            0.3799102028699015  0.07546755580465006  \\
            0.388080368241107  0.07543809536494703  \\
            0.39625054725697945  0.07541503060303202  \\
            0.40442074308041226  0.07539691264131396  \\
            0.41259095960797343  0.07538263871295257  \\
            0.42076120164109976  0.0753713649848897  \\
            0.4289314750975138  0.07536244221308494  \\
            0.43710178727237936  0.07535536825558958  \\
            0.4452721471607802  0.07534975289929793  \\
            0.45344256585529447  0.07534529165925995  \\
            0.4616130570344347  0.0753417461325479  \\
            0.4697836375593605  0.07533892916690509  \\
            0.4779543281998854  0.07533669359232714  \\
            0.4861251545321778  0.07533492361160196  \\
            0.49429614816331185  0.07533352819344553  \\
            0.5024673489413295  0.07533243598866415  \\
            0.5106388108383605  0.07533159141680935  \\
            0.518810621879253  0.07533095166289289  \\
            0.5269829765002086  0.07533048439138054  \\
            0.5351564383917821  0.07533016603620829  \\
            0.543332881422732  0.07532998057230542  \\
            0.5515188154128428  0.07532991875100449  \\
        }
        ;
    \addplot[color={rgb,1:red,0.502;green,0.502;blue,0.502}, name path={a1998ca1-0793-4e63-9447-2dcbbb89b322}, draw opacity={1.0}, line width={0.75}, solid, forget plot]
        table[row sep={\\}]
        {
            \\
            0.12  0.08925983979443165  \\
            0.1261514002549012  0.08930768769610237  \\
            0.13230280051249943  0.08935560960449251  \\
            0.13845420077549261  0.08940353188882594  \\
            0.144605601046736  0.0894513380666458  \\
            0.15075700132932401  0.0894988566340506  \\
            0.15690840162667963  0.08954584775223029  \\
            0.16305980194265315  0.08959199140521185  \\
            0.1692112022816338  0.08963688056691509  \\
            0.1753626026486766  0.0896800198575199  \\
            0.181514003049649  0.08972082667202388  \\
            0.18766540349140065  0.08975863172261479  \\
            0.19381680398196213  0.0897926768834148  \\
            0.1999682045307765  0.08982210956312595  \\
            0.20611960514897135  0.08984597375713471  \\
            0.21227100584967618  0.08986319860985523  \\
            0.2184224066483922  0.0898725857833963  \\
            0.22457380756341977  0.08987279803444295  \\
            0.23072520861634768  0.08986235486443839  \\
            0.23687660983260314  0.08983964813226591  \\
            0.24302801124205642  0.08980299748430785  \\
            0.2491794128796673  0.08975076671054721  \\
            0.25533081478615677  0.08968155388848503  \\
            0.26148221700869273  0.0895944503248274  \\
            0.2676336196015953  0.0894893370174616  \\
            0.2737850226271058  0.08936715393102769  \\
            0.27993642615630765  0.0892300705468901  \\
            0.2860878302703292  0.08908151204073678  \\
            0.2922392350620107  0.08892601151667356  \\
            0.2983906406382771  0.08876885622063425  \\
            0.3045420471234639  0.08861550954242504  \\
            0.3063790342150497  0.08857138189245868  \\
            0.3145491511093207  0.08838596984491395  \\
            0.32271927038915565  0.08822441201016228  \\
            0.33088939260291655  0.08808838882812306  \\
            0.3390595184317064  0.08797648907387864  \\
            0.34722964871705686  0.08788580478010745  \\
            0.3553997844959765  0.0878129941358577  \\
            0.36356992704499996  0.08775485321077865  \\
            0.37174007793497926  0.0877085635960768  \\
            0.3799102390987651  0.08767175970237416  \\
            0.3880804129144707  0.08764250885603442  \\
            0.3962506023076106  0.08761925659644776  \\
            0.4044208108760499  0.08760076370743183  \\
            0.41259104304231786  0.08758604694108861  \\
            0.42076130423833435  0.08757432784736127  \\
            0.4289316011277286  0.08756499052624503  \\
            0.43710194187024826  0.0875575475904319  \\
            0.4452723364304007  0.08755161311240367  \\
            0.45344279692682216  0.08754688127810589  \\
            0.46161333800683757  0.08754310960262762  \\
            0.4697839772059938  0.0875401057510919  \\
            0.47795473520100135  0.08753771719333776  \\
            0.48612563575046147  0.08753582308273485  \\
            0.4942967048348483  0.08753432788268865  \\
            0.5024679677106296  0.08753315637090663  \\
            0.5106394400997473  0.08753224973544467  \\
            0.5188111013820477  0.08753156254240115  \\
            0.5269828086779716  0.08753106040674549  \\
            0.535154008691445  0.08753071823734165  \\
            0.543322743829804  0.08753051895060487  \\
            0.5514821820000001  0.08753045252169261  \\
        }
        ;
    \addplot[color={rgb,1:red,0.502;green,0.502;blue,0.502}, name path={34a410aa-ebc2-4dde-b0d0-cfed39befed0}, draw opacity={1.0}, line width={0.75}, solid, forget plot]
        table[row sep={\\}]
        {
            \\
            0.12  0.10033666899535972  \\
            0.12615140025972063  0.10039538250716509  \\
            0.13230280052221355  0.10045432517316684  \\
            0.13845420079032575  0.10051354088230255  \\
            0.14460560106706402  0.10057299701664563  \\
            0.15075700135567988  0.10063255991642192  \\
            0.15690840165976214  0.10069198275589758  \\
            0.1630598019833388  0.10075090393602434  \\
            0.16921120233099135  0.10080885315695454  \\
            0.1753626027079842  0.10086526204915033  \\
            0.1815140031204133  0.10091947637043401  \\
            0.18766540357537678  0.10097076746037408  \\
            0.1938168040811733  0.10101834167588453  \\
            0.1999682046475313  0.1010613476062314  \\
            0.20611960528587536  0.10109888172387062  \\
            0.21227100600963467  0.1011299937391038  \\
            0.218422406834599  0.10115369345270674  \\
            0.22457380777932734  0.10116896178434373  \\
            0.23072520886561346  0.10117477013910177  \\
            0.23687661011901207  0.10117011389147092  \\
            0.2430280115694266  0.10115406627813653  \\
            0.24917941325176104  0.10112585713526756  \\
            0.2553308152066385  0.10108497604801062  \\
            0.26148221748119715  0.10103129187374736  \\
            0.2676336201299862  0.10096517151955674  \\
            0.27378502321600184  0.10088757302268427  \\
            0.2799364268119306  0.10080008660836114  \\
            0.2860878310016841  0.10070490343691223  \\
            0.2922392358823359  0.10060470052744373  \\
            0.2983906415665778  0.1005024406188454  \\
            0.3045420481858039  0.10040110174599662  \\
            0.3063790353238422  0.10037153389437943  \\
            0.31454915246455956  0.10024473922294365  \\
            0.32271927207158807  0.10012985818882085  \\
            0.3308893947096599  0.10002906445335612  \\
            0.3390595210777388  0.0999427782695948  \\
            0.3472296520389769  0.09987025383397206  \\
            0.3553997886575703  0.09981011168845659  \\
            0.36356993224432615  0.09976072362718605  \\
            0.3717400844129372  0.09972045244561493  \\
            0.3799102471492997  0.099687781838861  \\
            0.388080422896663  0.09966137405082315  \\
            0.39625061465992667  0.0996400846401548  \\
            0.40442082613295866  0.09962295424492405  \\
            0.4125910618533038  0.09960918963998613  \\
            0.4207613273889434  0.09959814115254509  \\
            0.4289316295615548  0.09958928019716523  \\
            0.43710197670947615  0.09958217872865896  \\
            0.44527237899040106  0.09957649130687644  \\
            0.4534428487172092  0.09957193988220653  \\
            0.46161340070814477  0.09956830112584683  \\
            0.46978405261252476  0.09956539601100546  \\
            0.47795482514627424  0.09956308132165273  \\
            0.4861257421556427  0.09956124277961106  \\
            0.494296830506958  0.09955978951408254  \\
            0.5024681203045294  0.09955864963716243  \\
            0.510639648071418  0.09955776672796887  \\
            0.5188114735956995  0.09955709706379755  \\
            0.5269837504807874  0.09955660746829151  \\
            0.5351569956825083  0.09955627367509137  \\
            0.5433330799713588  0.09955607913807567  \\
            0.5515188154128428  0.09955601429240378  \\
        }
        ;
    \addplot[color={rgb,1:red,0.502;green,0.502;blue,0.502}, name path={fbedaecf-e85f-4fc8-a0e3-0296e2f08dd0}, draw opacity={1.0}, line width={0.75}, solid, forget plot]
        table[row sep={\\}]
        {
            \\
            0.12  0.11141349819628778  \\
            0.12615140025554467  0.11148236259103259  \\
            0.13230280051364623  0.11155172607354069  \\
            0.13845420077703385  0.11162185072557691  \\
            0.14460560104858047  0.11169285335570084  \\
            0.1507570013313835  0.11176466904790242  \\
            0.15690840162885206  0.1118370460812192  \\
            0.16305980194480232  0.11190956234905504  \\
            0.16921120228356445  0.11198165270245582  \\
            0.17536260265010248  0.11205263965240041  \\
            0.18151400305015145  0.11212176274436124  \\
            0.1876654034903738  0.11218820383875444  \\
            0.19381680397853998  0.11225110758252534  \\
            0.19996820452373618  0.11230959800557637  \\
            0.20611960513660454  0.11236279307574953  \\
            0.21227100582962008  0.11240981908966702  \\
            0.21842240661740944  0.11244982595007046  \\
            0.22457380751711667  0.11248200427707714  \\
            0.23072520854882148  0.11250560651005616  \\
            0.23687660973601576  0.11251997500326537  \\
            0.243028011106145  0.11252457917280535  \\
            0.24917941269122246  0.11251906151488966  \\
            0.2553308145285274  0.11250328981595437  \\
            0.2614822166614041  0.11247741035986174  \\
            0.26763361914018513  0.11244189405263647  \\
            0.2737850220232743  0.11239756478204808  \\
            0.2799364253784347  0.11234559906099814  \\
            0.2860878292843385  0.11228748906818595  \\
            0.29223923383244643  0.11222496656167623  \\
            0.29839063912928604  0.11215989152137111  \\
            0.30454204529919815  0.11209411675340528  \\
            0.3063790322893496  0.11207466773029352  \\
            0.3145491486937438  0.11198976551337624  \\
            0.3227192674084074  0.11191047652297306  \\
            0.330889388962727  0.111838736376796  \\
            0.3390595140104026  0.11177546452623631  \\
            0.347229643357913  0.11172078627973303  \\
            0.35539977799939276  0.11167428840052289  \\
            0.3635699191595374  0.11163524379598788  \\
            0.3717400683464427  0.1116027812957342  \\
            0.37991022741664443  0.11157600045120916  \\
            0.38808039865509203  0.1115540415055056  \\
            0.396250584873342  0.11153612285570005  \\
            0.40442078952990634  0.1115215568622359  \\
            0.41259101687739697  0.11150975228942685  \\
            0.42076127214181636  0.11150020917153226  \\
            0.4289315617399195  0.11149250990548899  \\
            0.4371018935407276  0.11148630892654979  \\
            0.4452722771764445  0.11148132234100594  \\
            0.453442724405047  0.11147731825425801  \\
            0.4616132495192633  0.11147410813803929  \\
            0.469783869779121  0.11147153934678937  \\
            0.4779546058046166  0.11146948876243899  \\
            0.4861254817658055  0.11146785748017712  \\
            0.4942965249431467  0.1114665664195881  \\
            0.5024677634546381  0.11146555273962629  \\
            0.5106392184697081  0.11146476694217285  \\
            0.5188108788505589  0.11146417056132538  \\
            0.5269826167802892  0.11146373435046013  \\
            0.5351538979259829  0.11146343689352065  \\
            0.5433227613075436  0.11146326357384015  \\
            0.5514821820000001  0.11146320580061327  \\
        }
        ;
    \addplot[color={rgb,1:red,0.502;green,0.502;blue,0.502}, name path={6f0edf5f-30d0-4bf7-aa84-823f6cef856d}, draw opacity={1.0}, line width={0.75}, solid, forget plot]
        table[row sep={\\}]
        {
            \\
            0.12  0.12249032739721583  \\
            0.12615140024271357  0.12256798715099464  \\
            0.13230280048750778  0.12264664646709793  \\
            0.13845420073673972  0.12272697691396277  \\
            0.14460560099287698  0.12280934155347033  \\
            0.1507570012585672  0.12289374982477055  \\
            0.15690840153671326  0.12297988993388287  \\
            0.16305980183055457  0.12306719467988293  \\
            0.16921120214375715  0.12315491172194172  \\
            0.17536260248051416  0.12324216719187772  \\
            0.1815140028456599  0.12332801937251056  \\
            0.18766540324479908  0.1234114987789757  \\
            0.1938168036844555  0.12349163568811568  \\
            0.19996820417224237  0.12356747941597948  \\
            0.20611960471705912  0.1236381149258848  \\
            0.21227100532931756  0.12370268190535638  \\
            0.218422406021203  0.12376039434363861  \\
            0.2245738068069738  0.12381055584075983  \\
            0.23072520770330537  0.1238525722232395  \\
            0.23687660872968597  0.12388596778461915  \\
            0.2430280099088706  0.12391040769746414  \\
            0.24917941126740398  0.12392572336658478  \\
            0.2553308128362228  0.12393193634922521  \\
            0.26148221465135507  0.12392927958575795  \\
            0.267633616754736  0.12391821512694683  \\
            0.27378501919516685  0.12389944276158213  \\
            0.27993642202944796  0.12387389111256711  \\
            0.28608782532372357  0.12384268624056223  \\
            0.2922392291550817  0.12380709913175461  \\
            0.29839063361345675  0.12376847642707164  \\
            0.3045420388038831  0.12372815935402574  \\
            0.30637902546977697  0.12371603274379787  \\
            0.3145491402472744  0.1236619888789181  \\
            0.32271925696575005  0.12361005525695089  \\
            0.3308893760706582  0.12356184175324728  \\
            0.33905949811136643  0.12351830759697713  \\
            0.3472296237651819  0.12347987138476386  \\
            0.355399753866825  0.12344654795511546  \\
            0.3635698894446839  0.12341808044642985  \\
            0.3717400317654777  0.12339405088118224  \\
            0.37991018238932317  0.12337396392155768  \\
            0.38808034323767204  0.12335730516822024  \\
            0.3962505166771905  0.1233435784347396  \\
            0.40442070562344534  0.12333232717257825  \\
            0.4125909136693054  0.12332314476826371  \\
            0.4207611452443796  0.12331567752023932  \\
            0.42893140581376  0.12330962313150153  \\
            0.4371017021270834  0.12330472670796604  \\
            0.44527204253287295  0.1233007755881061  \\
            0.4534424373788569  0.12329759384162746  \\
            0.46161289952734985  0.12329503693436912  \\
            0.46978344502730296  0.12329298682876325  \\
            0.47795409400497524  0.12329134764296687  \\
            0.4861248718771444  0.12329004190227692  \\
            0.4942958111129259  0.12328900736500872  \\
            0.5024669542191273  0.12328819437833258  \\
            0.5106383603992285  0.123287563708417  \\
            0.5188101253521777  0.12328708478744244  \\
            0.526982450866389  0.12328673432366069  \\
            0.5351559041316796  0.1232864952278606  \\
            0.5433323929047321  0.12328635582365197  \\
            0.551518815412843  0.12328630935558241  \\
        }
        ;
    \addplot[color={rgb,1:red,0.502;green,0.502;blue,0.502}, name path={026ef830-6c67-4a58-8e91-6e77c3ed252d}, draw opacity={1.0}, line width={0.75}, solid, forget plot]
        table[row sep={\\}]
        {
            \\
            0.12  0.1335671565981439  \\
            0.12615140022235488  0.1336508446344345  \\
            0.13230280044611692  0.13373657213266293  \\
            0.13845420067305042  0.13382579732752276  \\
            0.14460560090498  0.13391924135262664  \\
            0.15075700114384671  0.13401687233765402  \\
            0.15690840139176687  0.1341181109398657  \\
            0.1630598016510932  0.13422203132650048  \\
            0.1692112019244811  0.13432750486434455  \\
            0.17536260221496122  0.13443330414778515  \\
            0.18151400252602115  0.1345381905217773  \\
            0.18766540286169772  0.1346409642309278  \\
            0.193816803226682  0.13474048468284108  \\
            0.19996820362643988  0.13483567444425562  \\
            0.2061196040673514  0.13492552158683233  \\
            0.21227100455687117  0.1350091095795613  \\
            0.21842240510371397  0.13508565582842222  \\
            0.22457380571806682  0.13515451980748214  \\
            0.23072520641183372  0.1352151874130788  \\
            0.2368766071989195  0.13526726829449545  \\
            0.24302800809555822  0.13531051547766182  \\
            0.24917940912069386  0.13534484638930663  \\
            0.25533081029642163  0.13537034567568784  \\
            0.2614822116485037  0.13538726081860353  \\
            0.2676336132069755  0.13539601336710996  \\
            0.2737850150068578  0.13539721850969472  \\
            0.27993641708899336  0.1353916860143239  \\
            0.2860878195010295  0.13538039027131413  \\
            0.2922392222985777  0.13536442502857887  \\
            0.29839062554657736  0.13534495550240616  \\
            0.30454202932089514  0.13532316547991338  \\
            0.3063790155173971  0.13531642747556027  \\
            0.3145491279358046  0.1352854868089334  \\
            0.32271924175138167  0.13525480603821718  \\
            0.33088935728515295  0.13522565408360918  \\
            0.33905947493279903  0.1351988294755506  \\
            0.34722959518198687  0.13517475994558956  \\
            0.3553997186336654  0.13515359488510492  \\
            0.3635698460282941  0.13513528773234879  \\
            0.37173997827820915  0.13511966502202966  \\
            0.3799101165076361  0.1351064806361067  \\
            0.38808026210226687  0.1350954557794504  \\
            0.3962504167708732  0.1350863065436754  \\
            0.4044205826222216  0.13507876151039622  \\
            0.41259076226170527  0.13507257187209598  \\
            0.4207609589138673  0.13506751625594868  \\
            0.4289311765797538  0.13506340201414993  \\
            0.43710142024252935  0.13506006431355577  \\
            0.4452716961422626  0.13505736397892873  \\
            0.4534420121533931  0.1350551847400687  \\
            0.4616123783195899  0.1350534303037881  \\
            0.4697828076354344  0.13505202150706594  \\
            0.47795331721670903  0.13505089369454465  \\
            0.48612393006187904  0.13504999438898  \\
            0.4942946776020239  0.13504928127644167  \\
            0.5024656028282337  0.13504872050046546  \\
            0.5106367616621093  0.13504828524465456  \\
            0.5188082115728961  0.1350479545769144  \\
            0.5269799439722115  0.13504771252740955  \\
            0.5351515995595778  0.13504754737373442  \\
            0.5433213891694104  0.13504745110549474  \\
            0.5514821820000002  0.1350474190160815  \\
        }
        ;
    \addplot[color={rgb,1:red,0.502;green,0.502;blue,0.502}, name path={d1a01947-1cb9-44bb-b190-88259c1926ec}, draw opacity={1.0}, line width={0.75}, solid, forget plot]
        table[row sep={\\}]
        {
            \\
            0.12  0.14464398579907195  \\
            0.1261514001962611  0.1447277691363484  \\
            0.132302800393164  0.14481614363284223  \\
            0.1384542005917003  0.14491206495073028  \\
            0.14460560079285847  0.1450166051869055  \\
            0.15075700099767186  0.14512903043874306  \\
            0.15690840120725882  0.145247877297928  \\
            0.16305980142285748  0.1453714339695169  \\
            0.1692112016458617  0.14549789279093822  \\
            0.17536260187785932  0.14562542051196536  \\
            0.18151400212067537  0.1457523440446712  \\
            0.1876654023764194  0.1458771844962717  \\
            0.1938168026475395  0.14599862705058791  \\
            0.19996820293688403  0.1461154636342167  \\
            0.20611960324777379  0.14622649016019462  \\
            0.21227100358408668  0.1463306100739647  \\
            0.21842240395035464  0.14642701504444167  \\
            0.22457380435187183  0.14651516205877915  \\
            0.23072520479481992  0.14659461262109502  \\
            0.23687660528641738  0.1466649843626945  \\
            0.24302800583509376  0.14672603705782658  \\
            0.24917940645069037  0.14677775336237048  \\
            0.2553308071446905  0.14682028588402926  \\
            0.26148220793049315  0.1468538547638709  \\
            0.26763360882374065  0.146878777620109  \\
            0.27378500984270293  0.14689558885228318  \\
            0.27993641100872163  0.1469050910532755  \\
            0.28608781234672537  0.14690827504726234  \\
            0.29223921388583923  0.14690623885046122  \\
            0.29839061566010217  0.146900166876214  \\
            0.30454201770929595  0.14689135308812828  \\
            0.30637900333372864  0.14688851094714875  \\
            0.31454911287516696  0.1468749285743424  \\
            0.3227192231467283  0.14686102249543315  \\
            0.3308893343162264  0.14684755481245004  \\
            0.3390594465904846  0.14683499422035584  \\
            0.34722956022439017  0.146823603475183  \\
            0.35539967553204227  0.14681349801125748  \\
            0.3635697929005032  0.14680469016934244  \\
            0.37173991280679575  0.14679712389736138  \\
            0.3799100358389651  0.1467907016420016  \\
            0.38808016272226664  0.1467853044195057  \\
            0.3962502943518805  0.14678080605369023  \\
            0.40442043183406434  0.14677708266194575  \\
            0.41259057653843373  0.14677401847132818  \\
            0.4207607301653186  0.14677150894740648  \\
            0.4289308948342452  0.14676946205934605  \\
            0.4371010732032426  0.1467677983278903  \\
            0.44527126863525707  0.14676645013696812  \\
            0.4534414854402353  0.14676536064905302  \\
            0.4616117292452863  0.14676448255345573  \\
            0.46978200759368277  0.14676377679387123  \\
            0.4779523309766128  0.14676321136241535  \\
            0.4861227147347506  0.14676276020703213  \\
            0.494293182827893  0.1467624022727971  \\
            0.5024637759106771  0.14676212068132216  \\
            0.5106345700757328  0.1467619020430869  \\
            0.5188057239001983  0.1467617358927971  \\
            0.5269776053049885  0.14676161423624165  \\
            0.5351511552596466  0.1467615311978387  \\
            0.5433289841335676  0.1467614827621099  \\
            0.551518815412843  0.14676146661686695  \\
        }
        ;
    \addplot[color={rgb,1:red,0.502;green,0.502;blue,0.502}, name path={8aa5d03d-3754-4ec3-8721-d3822a0bb74d}, draw opacity={1.0}, line width={0.75}, solid, forget plot]
        table[row sep={\\}]
        {
            \\
            0.12  0.155720815  \\
            0.12615140016666665  0.15579077631842816  \\
            0.13230280033333333  0.1558732022935015  \\
            0.13845420049999999  0.1559729329828782  \\
            0.14460560066666667  0.15609102041875295  \\
            0.15075700083333332  0.15622265571003757  \\
            0.156908401  0.15636407383589124  \\
            0.16305980116666666  0.15651236616271635  \\
            0.16921120133333334  0.15666486336534555  \\
            0.1753626015  0.15681857836445  \\
            0.18151400166666667  0.15697133722994003  \\
            0.18766540183333333  0.15712150120938215  \\
            0.193816802  0.15726777750621487  \\
            0.19996820216666666  0.15740915183117207  \\
            0.20611960233333335  0.15754388480717682  \\
            0.2122710025  0.1576701304503246  \\
            0.21842240266666665  0.15778714679780784  \\
            0.22457380283333334  0.15789500367297732  \\
            0.230725203  0.15799347431712235  \\
            0.23687660316666667  0.1580818906263962  \\
            0.24302800333333333  0.15815975337140403  \\
            0.2491794035  0.15822728314697948  \\
            0.25533080366666666  0.15828503040023142  \\
            0.2614822038333333  0.15833308352046074  \\
            0.267633604  0.1583712647951522  \\
            0.2737850041666667  0.15839986466358472  \\
            0.27993640433333333  0.15841995001612175  \\
            0.2860878045  0.15843280185958125  \\
            0.2922392046666667  0.15843959620544462  \\
            0.29839060483333335  0.1584414626579145  \\
            0.304542005  0.1584389348467872  \\
            0.30637899  0.1584389348467872  \\
            0.3145490964  0.1584389348467872  \\
            0.3227192028  0.1584389348467872  \\
            0.33088930920000004  0.1584389348467872  \\
            0.3390594156  0.1584389348467872  \\
            0.347229522  0.1584389348467872  \\
            0.3553996284  0.1584389348467872  \\
            0.3635697348  0.1584389348467872  \\
            0.37173984120000003  0.1584389348467872  \\
            0.3799099476  0.1584389348467872  \\
            0.388080054  0.1584389348467872  \\
            0.3962501604  0.1584389348467872  \\
            0.4044202668  0.1584389348467872  \\
            0.4125903732  0.1584389348467872  \\
            0.4207604796  0.1584389348467872  \\
            0.428930586  0.1584389348467872  \\
            0.4371006924  0.1584389348467872  \\
            0.4452707988  0.1584389348467872  \\
            0.4534409052  0.1584389348467872  \\
            0.4616110116  0.1584389348467872  \\
            0.469781118  0.1584389348467872  \\
            0.4779512244  0.1584389348467872  \\
            0.4861213308  0.1584389348467872  \\
            0.4942914372  0.1584389348467872  \\
            0.5024615436  0.1584389348467872  \\
            0.51063165  0.1584389348467872  \\
            0.5188017564  0.1584389348467872  \\
            0.5269718628  0.1584389348467872  \\
            0.5351419692  0.1584389348467872  \\
            0.5433120755999999  0.1584389348467872  \\
            0.5514821820000002  0.1584389348467872  \\
        }
        ;
    \addplot[color={rgb,1:red,0.4118;green,0.6824;blue,0.3725}, name path={56add62f-64ae-481a-b331-a8f1e1b999ac}, draw opacity={1.0}, line width={2}, solid, forget plot]
        table[row sep={\\}]
        {
            \\
            0.12  0.04495252299071941  \\
            0.12  0.05602935219164747  \\
            0.12  0.06710618139257553  \\
            0.12  0.07818301059350359  \\
            0.12  0.08925983979443165  \\
            0.12  0.10033666899535972  \\
            0.12  0.11141349819628778  \\
            0.12  0.12249032739721583  \\
            0.12  0.1335671565981439  \\
            0.12  0.14464398579907195  \\
            0.12  0.155720815  \\
        }
        ;
    \addplot[color={rgb,1:red,0.0;green,0.3608;blue,0.6706}, name path={b28b339c-50a1-4eed-bfbb-20203a8780c9}, draw opacity={1.0}, line width={1.0}, solid, mark={*}, mark size={1.125 pt}, mark repeat={1}, mark options={color={rgb,1:red,0.0;green,0.0;blue,0.0}, draw opacity={0.0}, fill={rgb,1:red,0.0;green,0.3608;blue,0.6706}, fill opacity={1.0}, line width={0.75}, rotate={0}, solid}, forget plot]
        table[row sep={\\}]
        {
            \\
            0.304542005  0.1584389348467872  \\
            0.29839060483333335  0.1584414626579145  \\
            0.2922392046666667  0.15843959620544462  \\
            0.2860878045  0.15843280185958125  \\
            0.27993640433333333  0.15841995001612175  \\
            0.2737850041666667  0.15839986466358472  \\
            0.267633604  0.1583712647951522  \\
            0.2614822038333333  0.15833308352046074  \\
            0.25533080366666666  0.15828503040023142  \\
            0.2491794035  0.15822728314697948  \\
            0.24302800333333333  0.15815975337140403  \\
            0.23687660316666667  0.1580818906263962  \\
            0.230725203  0.15799347431712235  \\
            0.22457380283333334  0.15789500367297732  \\
            0.21842240266666665  0.15778714679780784  \\
            0.2122710025  0.1576701304503246  \\
            0.20611960233333335  0.15754388480717682  \\
            0.19996820216666666  0.15740915183117207  \\
            0.193816802  0.15726777750621487  \\
            0.18766540183333333  0.15712150120938215  \\
            0.18151400166666667  0.15697133722994003  \\
            0.1753626015  0.15681857836445  \\
            0.16921120133333334  0.15666486336534555  \\
            0.16305980116666666  0.15651236616271635  \\
            0.156908401  0.15636407383589124  \\
            0.15075700083333332  0.15622265571003757  \\
            0.14460560066666667  0.15609102041875295  \\
            0.13845420049999999  0.1559729329828782  \\
            0.13230280033333333  0.1558732022935015  \\
            0.12615140016666665  0.15579077631842816  \\
            0.12  0.155720815  \\
            0.11399279808861612  0.15565969080491326  \\
            0.10800206150011583  0.15560249443750745  \\
            0.10204421042707724  0.15555102467759901  \\
            0.09613557492516646  0.15551295342131385  \\
            0.09029235015358988  0.15549417513847946  \\
            0.08453055198528892  0.15550690575697393  \\
            0.07886597310854516  0.15555806588088175  \\
            0.07331413974031975  0.15565145553095425  \\
            0.06789026906997177  0.15579480039342583  \\
            0.06260922754999998  0.1559972892969931  \\
            0.05748549014812998  0.15626682407104356  \\
            0.052533100672433664  0.15660670998792758  \\
            0.04776563327822751  0.15702395996670845  \\
            0.04319615526225648  0.15752589111786994  \\
            0.038837191246141795  0.1581190016491485  \\
            0.03470068884726381  0.1588032721112729  \\
            0.030797985931173034  0.15958255369945537  \\
            0.02713977953528892  0.16046238560820839  \\
            0.023736096549063807  0.16144508306941582  \\
            0.020596266230975817  0.16253214292681187  \\
            0.01772889463767991  0.16372296488091606  \\
            0.01514184103540476  0.16501480690166814  \\
            0.012842196358250287  0.16640662505317158  \\
            0.010836263772430254  0.16789724603614412  \\
            0.009129541399731704  0.1694852415602899  \\
            0.007726707247545409  0.1711796485936844  \\
            0.006631606386772701  0.17300038865420833  \\
            0.005847240412753437  0.17482944254720592  \\
            0.005375759218101773  0.1769049528784241  \\
            0.0052184551  0.17818011312626458  \\
            0.005375759218101773  0.18111733553290799  \\
            0.005847240412753437  0.1832370262313411  \\
            0.006631606386772701  0.18538397929620315  \\
            0.007726707247545409  0.18749476787817393  \\
            0.009129541399731704  0.18959413853240056  \\
            0.010836263772430254  0.19167163675489338  \\
            0.012842196358250287  0.19371567907762274  \\
            0.01514184103540476  0.19571723114029266  \\
            0.01772889463767991  0.19766663307197876  \\
            0.020596266230975817  0.19955300101177723  \\
            0.023736096549063807  0.20136450910193648  \\
            0.02713977953528892  0.20309181071356452  \\
            0.030797985931173034  0.20472620584894421  \\
            0.03470068884726381  0.20625768286324309  \\
            0.038837191246141795  0.2076773483741656  \\
            0.04319615526225648  0.20897894650873994  \\
            0.04776563327822751  0.21015543159349476  \\
            0.052533100672433664  0.21119806292411006  \\
            0.05748549014812998  0.21210074637335335  \\
            0.06260922754999998  0.2128591458243588  \\
            0.06789026906997177  0.21346996882865465  \\
            0.07331413974031975  0.21393111499918402  \\
            0.07886597310854516  0.21424145596242564  \\
            0.08453055198528892  0.2144009116777412  \\
            0.09029235015358988  0.21441087192011882  \\
            0.09613557492516646  0.21427256561901806  \\
            0.10204421042707724  0.2139882937494529  \\
            0.10800206150011583  0.21356450119740195  \\
            0.11399279808861612  0.2130063830369111  \\
            0.12  0.21231528902120386  \\
            0.12615140016666665  0.21148163030815084  \\
            0.13230280033333333  0.21054232773485568  \\
            0.13845420049999999  0.20951351266887752  \\
            0.14460560066666667  0.20840624793627152  \\
            0.15075700083333332  0.2072267992995797  \\
            0.156908401  0.20597836002063233  \\
            0.16305980116666666  0.2046625102172418  \\
            0.16921120133333334  0.2032814554715009  \\
            0.1753626015  0.20183911842645766  \\
            0.18151400166666667  0.20033903152400717  \\
            0.18766540183333333  0.198782514933333  \\
            0.193816802  0.19717035779256822  \\
            0.19996820216666666  0.19550419230865274  \\
            0.20611960233333335  0.193785876944981  \\
            0.2122710025  0.1920164435257992  \\
            0.21842240266666665  0.1901965719398472  \\
            0.22457380283333334  0.18832735174399645  \\
            0.230725203  0.18640963464445282  \\
            0.23687660316666667  0.18444325695992825  \\
            0.24302800333333333  0.18242791302049308  \\
            0.2491794035  0.18036393406872206  \\
            0.25533080366666666  0.17825155523074446  \\
            0.2614822038333333  0.1760898628465513  \\
            0.267633604  0.17387797068935812  \\
            0.2737850041666667  0.17161626688836612  \\
            0.27993640433333333  0.16930459165100145  \\
            0.2860878045  0.1669411898638361  \\
            0.2922392046666667  0.16452422700875094  \\
            0.29839060483333335  0.16205211714309609  \\
            0.304542005  0.15952600512626458  \\
        }
        ;
    \addplot[color={rgb,1:red,0.7451;green,0.298;blue,0.302}, name path={16eca9c6-fe46-46c3-98a3-038192c88a0a}, draw opacity={1.0}, line width={1.0}, solid, mark={*}, mark size={1.125 pt}, mark repeat={1}, mark options={color={rgb,1:red,0.0;green,0.0;blue,0.0}, draw opacity={0.0}, fill={rgb,1:red,0.7451;green,0.298;blue,0.302}, fill opacity={1.0}, line width={0.75}, rotate={0}, solid}, forget plot]
        table[row sep={\\}]
        {
            \\
            0.0  0.0  \\
            0.00016445582945114  0.002740749981173598  \\
            0.0006573725558072052  0.005643307185909748  \\
            0.0014773991285834676  0.00831732970954065  \\
            0.0026222879119433174  0.010982863391086622  \\
            0.004088900845311803  0.013574999693898887  \\
            0.005873218044581576  0.01609783096746975  \\
            0.007970348820335791  0.01854798891045612  \\
            0.010374545082887895  0.020917835209454175  \\
            0.013079217097395852  0.023202469121967  \\
            0.016076951545867354  0.02539710667209484  \\
            0.019359531846549115  0.0274955317958163  \\
            0.022917960675006305  0.02949157705815014  \\
            0.026742484625163505  0.03138038128041586  \\
            0.03082262094271269  0.033157105979071975  \\
            0.035147186257614274  0.0348168854917705  \\
            0.03970432723693701  0.03635670057951697  \\
            0.0444815530740195  0.037773299047661385  \\
            0.04946576972490322  0.03906327587476941  \\
            0.054643315798196736  0.040224241354043955  \\
            0.059999999999999984  0.04125553477214516  \\
            0.06552114003125438  0.042158781258759444  \\
            0.07119160283090395  0.04293422532362003  \\
            0.07699584605456396  0.043581158174966446  \\
            0.0829179606750063  0.044101461786900796  \\
            0.0889417145876975  0.04450114439797895  \\
            0.09505059710186889  0.044786996687128246  \\
            0.10122786419517228  0.04495063424158973  \\
            0.10745658440788158  0.044998666961206364  \\
            0.11371968525084675  0.044983194126287505  \\
            0.12  0.04495252299071941  \\
            0.12615140016666665  0.04494248592934282  \\
            0.13230280033333333  0.04494776043555422  \\
            0.13845420049999999  0.044949764153980734  \\
            0.14460560066666667  0.044942278699579535  \\
            0.15075700083333332  0.04493204538203043  \\
            0.156908401  0.044923127246484104  \\
            0.16305980116666666  0.04491727692121097  \\
            0.16921120133333334  0.04491196172254526  \\
            0.1753626015  0.044906620646925216  \\
            0.18151400166666667  0.04490069966154803  \\
            0.18766540183333333  0.04489448945657971  \\
            0.193816802  0.04488823654667903  \\
            0.19996820216666666  0.04488243703899517  \\
            0.20611960233333335  0.04487715711766979  \\
            0.2122710025  0.04487136962829655  \\
            0.21842240266666665  0.04486309798935874  \\
            0.22457380283333334  0.04485360527005957  \\
            0.230725203  0.0448476314166718  \\
            0.23687660316666667  0.044848601431894536  \\
            0.24302800333333333  0.04485107224020261  \\
            0.2491794035  0.04483645969082626  \\
            0.25533080366666666  0.044774484794636006  \\
            0.2614822038333333  0.044592370393489775  \\
            0.267633604  0.04421530669424536  \\
            0.2737850041666667  0.04362179231228574  \\
            0.27993640433333333  0.04280470972471024  \\
            0.2860878045  0.04173197487913953  \\
            0.2922392046666667  0.040364235789079016  \\
            0.29839060483333335  0.038669523760712386  \\
            0.304542005  0.036613858106809886  \\
            0.30637899  0.03592799979999999  \\
        }
        ;
\end{axis}
\end{tikzpicture}
\hspace*{5em}
     \caption{Single rotor verification case geometry generated by DuctAPE. Duct geometry in \primary{blue}, center body geometry in \secondary{red}, rotor lifting line location in \tertiary{green}, and wake grid geometry in \gray{gray}.}
    \label{fig:singlerotorgeom}
\end{figure}

\begin{figure}[h!]
     \centering
% \tikzsetnextfilename{solvers/dfdc_single_rotor_geometry}
     \begin{subfigure}[t]{\textwidth}
         \centering
         \subcaptionbox{DFDC generated geometry.\label{fig:dfdcsinglerotorgeom}}{%
             % Recommended preamble:
% \usetikzlibrary{arrows.meta}
% \usetikzlibrary{backgrounds}
% \usepgfplotslibrary{patchplots}
% \usepgfplotslibrary{fillbetween}
% \pgfplotsset{%
%     layers/standard/.define layer set={%
%         background,axis background,axis grid,axis ticks,axis lines,axis tick labels,pre main,main,axis descriptions,axis foreground%
%     }{
%         grid style={/pgfplots/on layer=axis grid},%
%         tick style={/pgfplots/on layer=axis ticks},%
%         axis line style={/pgfplots/on layer=axis lines},%
%         label style={/pgfplots/on layer=axis descriptions},%
%         legend style={/pgfplots/on layer=axis descriptions},%
%         title style={/pgfplots/on layer=axis descriptions},%
%         colorbar style={/pgfplots/on layer=axis descriptions},%
%         ticklabel style={/pgfplots/on layer=axis tick labels},%
%         axis background@ style={/pgfplots/on layer=axis background},%
%         3d box foreground style={/pgfplots/on layer=axis foreground},%
%     },
% }

\begin{tikzpicture}[/tikz/background rectangle/.style={fill={rgb,1:red,0.0;green,0.0;blue,0.0}, fill opacity={0.0}, draw opacity={0.0}}, show background rectangle]
\begin{axis}[point meta max={nan}, point meta min={nan}, legend cell align={left}, legend columns={1}, title={}, title style={at={{(0.5,1)}}, anchor={south}, font={{\fontsize{14 pt}{18.2 pt}\selectfont}}, color={rgb,1:red,0.0;green,0.0;blue,0.0}, draw opacity={1.0}, rotate={0.0}, align={center}}, legend style={color={rgb,1:red,0.0;green,0.0;blue,0.0}, draw opacity={0.0}, line width={1}, solid, fill={rgb,1:red,0.0;green,0.0;blue,0.0}, fill opacity={0.0}, text opacity={1.0}, font={{\fontsize{8 pt}{10.4 pt}\selectfont}}, text={rgb,1:red,0.0;green,0.0;blue,0.0}, cells={anchor={center}}, at={(1.02, 1)}, anchor={north west}}, axis background/.style={fill={rgb,1:red,0.0;green,0.0;blue,0.0}, opacity={0.0}}, anchor={north west}, xshift={5.0mm}, yshift={-5.0mm}, width={131.304mm}, height={50.96mm}, scaled x ticks={false}, xlabel={$z~\mathrm{(m)}$}, x tick style={color={rgb,1:red,0.0;green,0.0;blue,0.0}, opacity={1.0}}, x tick label style={color={rgb,1:red,0.0;green,0.0;blue,0.0}, opacity={1.0}, rotate={0}}, xlabel style={at={(ticklabel cs:0.5)}, anchor=near ticklabel, at={{(ticklabel cs:0.5)}}, anchor={near ticklabel}, font={{\fontsize{11 pt}{14.3 pt}\selectfont}}, color={rgb,1:red,0.0;green,0.0;blue,0.0}, draw opacity={1.0}, rotate={0.0}}, xmajorgrids={true}, xmin={-0.01666598439}, xmax={0.57219879739}, xticklabels={{0.00,0.12,0.31,0.56}}, xtick={{0.0,0.120000012,0.30637899,0.555532813}}, xtick align={inside}, xticklabel style={font={{\fontsize{8 pt}{10.4 pt}\selectfont}}, color={rgb,1:red,0.0;green,0.0;blue,0.0}, draw opacity={1.0}, rotate={0.0}}, x grid style={color={rgb,1:red,0.0;green,0.0;blue,0.0}, draw opacity={0.1}, line width={0.5}, solid}, axis x line*={left}, x axis line style={color={rgb,1:red,0.0;green,0.0;blue,0.0}, draw opacity={1.0}, line width={1}, solid}, scaled y ticks={false}, ylabel={$r~\mathrm{(m)}$}, y tick style={color={rgb,1:red,0.0;green,0.0;blue,0.0}, opacity={1.0}}, y tick label style={color={rgb,1:red,0.0;green,0.0;blue,0.0}, opacity={1.0}, rotate={0}}, ylabel style={{rotate=-90}}, ymajorgrids={true}, ymin={-0.019809924434880805}, ymax={0.2342324034348808}, yticklabels={{0.00,0.04,0.16,0.18,0.21}}, ytick={{0.0,0.044952523,0.155720815,0.178180113,0.214422479}}, ytick align={inside}, yticklabel style={font={{\fontsize{8 pt}{10.4 pt}\selectfont}}, color={rgb,1:red,0.0;green,0.0;blue,0.0}, draw opacity={1.0}, rotate={0.0}}, y grid style={color={rgb,1:red,0.0;green,0.0;blue,0.0}, draw opacity={0.1}, line width={0.5}, solid}, axis y line*={left}, y axis line style={color={rgb,1:red,0.0;green,0.0;blue,0.0}, draw opacity={1.0}, line width={1}, solid}, colorbar={false}]
    \addplot[color={rgb,1:red,0.502;green,0.502;blue,0.502}, name path={1bab93e0-4c6a-429b-a8d5-f2ffbfb4e7ba}, draw opacity={0.25}, line width={0.1}, solid, forget plot]
        table[row sep={\\}]
        {
            \\
            0.119999997  0.044952523  \\
            0.119999997  0.0560293496  \\
            0.119999997  0.0671061799  \\
            0.120000005  0.0781830102  \\
            0.120000005  0.0892598331  \\
            0.120000005  0.100336671  \\
            0.120000005  0.111413494  \\
            0.120000005  0.122490317  \\
            0.120000012  0.133567154  \\
            0.120000012  0.144643977  \\
            0.120000012  0.155720815  \\
        }
        ;
    \addplot[color={rgb,1:red,0.502;green,0.502;blue,0.502}, name path={0151a7d3-bd93-419e-b221-ca540f2ec3d2}, draw opacity={0.25}, line width={0.1}, solid, forget plot]
        table[row sep={\\}]
        {
            \\
            0.126598373  0.0449425951  \\
            0.126558438  0.0560412109  \\
            0.126518264  0.0671321899  \\
            0.126477122  0.0782214478  \\
            0.12643443  0.0893098935  \\
            0.126389757  0.100397676  \\
            0.126343116  0.111484513  \\
            0.126295686  0.122569807  \\
            0.126252562  0.133652225  \\
            0.126234606  0.144728884  \\
            0.126339316  0.155793041  \\
        }
        ;
    \addplot[color={rgb,1:red,0.502;green,0.502;blue,0.502}, name path={6ce9280b-1758-4eb6-bd6f-5c029a2ff220}, draw opacity={0.25}, line width={0.1}, solid, forget plot]
        table[row sep={\\}]
        {
            \\
            0.133326128  0.0449483283  \\
            0.133245647  0.0560548678  \\
            0.133164659  0.0671587884  \\
            0.133081689  0.0782606304  \\
            0.132995501  0.0893610269  \\
            0.13290517  0.100460134  \\
            0.132810473  0.111557484  \\
            0.132712901  0.122651935  \\
            0.132618949  0.133741051  \\
            0.132551491  0.144819811  \\
            0.132587165  0.155877367  \\
        }
        ;
    \addplot[color={rgb,1:red,0.502;green,0.502;blue,0.502}, name path={e96166cc-5881-4c5c-b416-0199e76a77aa}, draw opacity={0.25}, line width={0.1}, solid, forget plot]
        table[row sep={\\}]
        {
            \\
            0.140088946  0.0449485965  \\
            0.139968172  0.0560677573  \\
            0.139846593  0.067185156  \\
            0.139721945  0.078299813  \\
            0.13959226  0.089412421  \\
            0.139456004  0.100523241  \\
            0.139312387  0.111631706  \\
            0.139162093  0.122736335  \\
            0.139009237  0.133834034  \\
            0.138866305  0.14491877  \\
            0.138760611  0.155978441  \\
        }
        ;
    \addplot[color={rgb,1:red,0.502;green,0.502;blue,0.502}, name path={7c05fabd-2bff-4a40-9acb-df32fa627490}, draw opacity={0.25}, line width={0.1}, solid, forget plot]
        table[row sep={\\}]
        {
            \\
            0.146876231  0.0449386574  \\
            0.146715805  0.0560786538  \\
            0.146554217  0.0672108829  \\
            0.146388352  0.0783387199  \\
            0.146215454  0.0894638523  \\
            0.146033287  0.100586854  \\
            0.145840228  0.111707218  \\
            0.145635501  0.12282332  \\
            0.14541921  0.133931905  \\
            0.145191148  0.145026967  \\
            0.144942939  0.156097993  \\
        }
        ;
    \addplot[color={rgb,1:red,0.502;green,0.502;blue,0.502}, name path={0e326861-6e2f-4257-8b27-9e9a8f0bceca}, draw opacity={0.25}, line width={0.1}, solid, forget plot]
        table[row sep={\\}]
        {
            \\
            0.153716758  0.0449272916  \\
            0.1535175  0.056088496  \\
            0.153316632  0.0672359988  \\
            0.153110132  0.078377299  \\
            0.152894422  0.0895153135  \\
            0.152666479  0.100651085  \\
            0.152423769  0.111784257  \\
            0.152164072  0.122913301  \\
            0.151883841  0.13403514  \\
            0.151572704  0.145144433  \\
            0.15119262  0.156232446  \\
        }
        ;
    \addplot[color={rgb,1:red,0.502;green,0.502;blue,0.502}, name path={4adc8110-dda0-42eb-9bf3-52b14fd4f082}, draw opacity={0.25}, line width={0.1}, solid, forget plot]
        table[row sep={\\}]
        {
            \\
            0.160557494  0.0449194387  \\
            0.160320878  0.0560980961  \\
            0.160082117  0.0672602504  \\
            0.159836203  0.0784149542  \\
            0.159578711  0.0895660743  \\
            0.159305796  0.100715086  \\
            0.15901418  0.111861899  \\
            0.15870063  0.123005226  \\
            0.15835996  0.134142399  \\
            0.157978609  0.145269051  \\
            0.157518655  0.156378597  \\
        }
        ;
    \addplot[color={rgb,1:red,0.502;green,0.502;blue,0.502}, name path={932dc358-2c43-4987-a0fd-01c097706aad}, draw opacity={0.25}, line width={0.1}, solid, forget plot]
        table[row sep={\\}]
        {
            \\
            0.167418554  0.0449135117  \\
            0.16714634  0.0561074205  \\
            0.166871279  0.0672834814  \\
            0.166587412  0.0784514323  \\
            0.166289359  0.0896158293  \\
            0.165972456  0.100778565  \\
            0.165632829  0.11193984  \\
            0.165267035  0.123098657  \\
            0.164870307  0.134252951  \\
            0.164431185  0.145399407  \\
            0.163917422  0.156533509  \\
        }
        ;
    \addplot[color={rgb,1:red,0.502;green,0.502;blue,0.502}, name path={506dd8ac-4587-446c-95cb-02c5feffca68}, draw opacity={0.25}, line width={0.1}, solid, forget plot]
        table[row sep={\\}]
        {
            \\
            0.174291193  0.0449075662  \\
            0.173985556  0.0561159588  \\
            0.173676267  0.0673052445  \\
            0.173356295  0.0784861669  \\
            0.17301926  0.0896639302  \\
            0.172659665  0.100840792  \\
            0.172273144  0.112017214  \\
            0.171856567  0.123192549  \\
            0.171407461  0.13436529  \\
            0.170921043  0.145533279  \\
            0.170381293  0.156694025  \\
        }
        ;
    \addplot[color={rgb,1:red,0.502;green,0.502;blue,0.502}, name path={a9ce3cd8-13d3-4c9d-9a50-21f0ca6492ee}, draw opacity={0.25}, line width={0.1}, solid, forget plot]
        table[row sep={\\}]
        {
            \\
            0.18116495  0.0449010506  \\
            0.180828556  0.0561232083  \\
            0.180487558  0.0673249513  \\
            0.18013376  0.0785184652  \\
            0.179759741  0.0897095799  \\
            0.179359019  0.100900881  \\
            0.178926647  0.112093031  \\
            0.178460047  0.123285651  \\
            0.177959904  0.134477779  \\
            0.177430943  0.145668238  \\
            0.176883996  0.156856567  \\
        }
        ;
    \addplot[color={rgb,1:red,0.502;green,0.502;blue,0.502}, name path={603bf71c-be4f-4c66-bc20-cb9396d52ef4}, draw opacity={0.25}, line width={0.1}, solid, forget plot]
        table[row sep={\\}]
        {
            \\
            0.188040003  0.0448941104  \\
            0.187675938  0.056128744  \\
            0.187306151  0.067341961  \\
            0.186921269  0.0785475522  \\
            0.186512634  0.0897519514  \\
            0.186072558  0.100957938  \\
            0.185595259  0.112166278  \\
            0.185078308  0.123376779  \\
            0.184525192  0.134588853  \\
            0.183949143  0.14580211  \\
            0.1833819  0.15701735  \\
        }
        ;
    \addplot[color={rgb,1:red,0.502;green,0.502;blue,0.502}, name path={cc39fee3-c414-4b3d-ac06-d32043ee1dc3}, draw opacity={0.25}, line width={0.1}, solid, forget plot]
        table[row sep={\\}]
        {
            \\
            0.194916084  0.0448871255  \\
            0.194527879  0.0561321341  \\
            0.194132686  0.0673554763  \\
            0.193719909  0.0785724893  \\
            0.193279505  0.0897900462  \\
            0.192802265  0.101010941  \\
            0.192280948  0.112235889  \\
            0.191712439  0.123464711  \\
            0.191101521  0.134697065  \\
            0.190466985  0.145933002  \\
            0.189850047  0.157173961  \\
        }
        ;
    \addplot[color={rgb,1:red,0.502;green,0.502;blue,0.502}, name path={57f25128-d887-4b93-b1f7-455f121ba2b5}, draw opacity={0.25}, line width={0.1}, solid, forget plot]
        table[row sep={\\}]
        {
            \\
            0.201786935  0.0448808186  \\
            0.201378569  0.0561328046  \\
            0.200961858  0.0673644841  \\
            0.200524941  0.0785921291  \\
            0.20005624  0.0898227021  \\
            0.199544623  0.101058744  \\
            0.198980644  0.112300679  \\
            0.198359087  0.123548187  \\
            0.197683975  0.134800941  \\
            0.196976677  0.146059319  \\
            0.196288183  0.157325283  \\
        }
        ;
    \addplot[color={rgb,1:red,0.502;green,0.502;blue,0.502}, name path={3ddfbee6-d4dc-4eb0-b487-cd9ec6fb6b3d}, draw opacity={0.25}, line width={0.1}, solid, forget plot]
        table[row sep={\\}]
        {
            \\
            0.208656073  0.0448750034  \\
            0.208231941  0.0561298095  \\
            0.207798049  0.0673677102  \\
            0.207341343  0.0786051229  \\
            0.206848547  0.0898486376  \\
            0.206306234  0.101100162  \\
            0.205701783  0.112359539  \\
            0.205026209  0.123626083  \\
            0.204279929  0.134899318  \\
            0.203482866  0.146179691  \\
            0.202688128  0.157469675  \\
        }
        ;
    \addplot[color={rgb,1:red,0.502;green,0.502;blue,0.502}, name path={31fce6ef-06fc-4ebd-8ad6-2e295b877716}, draw opacity={0.25}, line width={0.1}, solid, forget plot]
        table[row sep={\\}]
        {
            \\
            0.215509951  0.0448674187  \\
            0.215074837  0.0561216772  \\
            0.214628607  0.0673634559  \\
            0.21415709  0.0786097869  \\
            0.213645324  0.0898663402  \\
            0.213077247  0.101133838  \\
            0.212436169  0.112411208  \\
            0.211707339  0.123697169  \\
            0.210884154  0.134990916  \\
            0.209980592  0.146292716  \\
            0.209050387  0.157605141  \\
        }
        ;
    \addplot[color={rgb,1:red,0.502;green,0.502;blue,0.502}, name path={1f8a9dd5-5eea-425e-97c9-a9c35ce662f0}, draw opacity={0.25}, line width={0.1}, solid, forget plot]
        table[row sep={\\}]
        {
            \\
            0.222354814  0.0448568724  \\
            0.2219138  0.0561067387  \\
            0.221460417  0.067349568  \\
            0.220979676  0.078604117  \\
            0.22045505  0.0898741558  \\
            0.219867632  0.101158448  \\
            0.219195962  0.112454593  \\
            0.218417466  0.123760514  \\
            0.217514426  0.135074899  \\
            0.2164886  0.146397591  \\
            0.215387374  0.15773055  \\
        }
        ;
    \addplot[color={rgb,1:red,0.502;green,0.502;blue,0.502}, name path={b6fbf57c-6dd7-4f40-abae-de14c66ea1c2}, draw opacity={0.25}, line width={0.1}, solid, forget plot]
        table[row sep={\\}]
        {
            \\
            0.229183137  0.0448483042  \\
            0.228741482  0.0560832731  \\
            0.228286475  0.067323193  \\
            0.227802634  0.0785856098  \\
            0.227272227  0.0898701549  \\
            0.226673692  0.101172529  \\
            0.225980401  0.112488531  \\
            0.225160286  0.123815134  \\
            0.224179819  0.135150447  \\
            0.223020017  0.146493703  \\
            0.221715644  0.157846034  \\
        }
        ;
    \addplot[color={rgb,1:red,0.502;green,0.502;blue,0.502}, name path={7b9b882c-b13c-4773-b769-c5270cb8f6e1}, draw opacity={0.25}, line width={0.1}, solid, forget plot]
        table[row sep={\\}]
        {
            \\
            0.23595807  0.0448484048  \\
            0.235521033  0.056048099  \\
            0.235070005  0.0672804862  \\
            0.234589562  0.0785513818  \\
            0.234061286  0.089852348  \\
            0.233461618  0.101174623  \\
            0.232759148  0.112511829  \\
            0.231911585  0.123859957  \\
            0.230865002  0.135216579  \\
            0.229566857  0.146580324  \\
            0.228036374  0.157951623  \\
        }
        ;
    \addplot[color={rgb,1:red,0.502;green,0.502;blue,0.502}, name path={11867be1-ad60-42be-aeb1-b4f24c9405d0}, draw opacity={0.25}, line width={0.1}, solid, forget plot]
        table[row sep={\\}]
        {
            \\
            0.242728472  0.0448510461  \\
            0.242301211  0.055992946  \\
            0.241859689  0.0672152191  \\
            0.24138926  0.0784974843  \\
            0.240871623  0.0898183882  \\
            0.240282282  0.101163387  \\
            0.239586458  0.112523735  \\
            0.238732681  0.123894617  \\
            0.237644017  0.135273308  \\
            0.236213505  0.146657839  \\
            0.234346643  0.158046767  \\
        }
        ;
    \addplot[color={rgb,1:red,0.502;green,0.502;blue,0.502}, name path={a81d7027-0e85-406e-81b3-2eac8b00d157}, draw opacity={0.25}, line width={0.1}, solid, forget plot]
        table[row sep={\\}]
        {
            \\
            0.249308959  0.0448360331  \\
            0.248895854  0.0559058823  \\
            0.248468667  0.0671219453  \\
            0.248014197  0.0784218386  \\
            0.247515231  0.089767538  \\
            0.246947899  0.101138338  \\
            0.246276617  0.112523526  \\
            0.245444879  0.123917893  \\
            0.244357824  0.13531889  \\
            0.242850944  0.146724433  \\
            0.240648702  0.158130884  \\
        }
        ;
    \addplot[color={rgb,1:red,0.502;green,0.502;blue,0.502}, name path={cca1499c-428e-4c26-ab9a-6287642b9e8d}, draw opacity={0.25}, line width={0.1}, solid, forget plot]
        table[row sep={\\}]
        {
            \\
            0.255458832  0.0447726212  \\
            0.255062699  0.0557756722  \\
            0.254653007  0.0669992864  \\
            0.254218608  0.0783263519  \\
            0.253744245  0.0897022411  \\
            0.25320816  0.101101421  \\
            0.252577126  0.112512253  \\
            0.251795858  0.123929895  \\
            0.250764996  0.135352507  \\
            0.249291167  0.146778673  \\
            0.2469531  0.158204004  \\
        }
        ;
    \addplot[color={rgb,1:red,0.502;green,0.502;blue,0.502}, name path={4f5d18fd-aedd-462c-a7b2-f1efea78ffb0}, draw opacity={0.25}, line width={0.1}, solid, forget plot]
        table[row sep={\\}]
        {
            \\
            0.26105535  0.0446099378  \\
            0.260677278  0.0555957817  \\
            0.26028654  0.0668503419  \\
            0.259874225  0.0782157704  \\
            0.259427547  0.0896268114  \\
            0.25892812  0.101056084  \\
            0.258347571  0.112492524  \\
            0.25763768  0.123932414  \\
            0.256709009  0.135375157  \\
            0.255382955  0.146820739  \\
            0.253263652  0.158266708  \\
        }
        ;
    \addplot[color={rgb,1:red,0.502;green,0.502;blue,0.502}, name path={6394e1cc-7950-44e9-a922-30c164f777c0}, draw opacity={0.25}, line width={0.1}, solid, forget plot]
        table[row sep={\\}]
        {
            \\
            0.266296536  0.0443151407  \\
            0.265937179  0.0553589165  \\
            0.265566319  0.0666736141  \\
            0.265177429  0.0780902132  \\
            0.264760494  0.0895420015  \\
            0.2643013  0.101003513  \\
            0.26377812  0.112465963  \\
            0.26315406  0.123927556  \\
            0.262359738  0.135389298  \\
            0.261256844  0.14685294  \\
            0.2595779  0.15831925  \\
        }
        ;
    \addplot[color={rgb,1:red,0.502;green,0.502;blue,0.502}, name path={46618cb7-cb16-4071-ac41-5cfec562456f}, draw opacity={0.25}, line width={0.1}, solid, forget plot]
        table[row sep={\\}]
        {
            \\
            0.271419168  0.0438756235  \\
            0.271079302  0.0550550483  \\
            0.270729423  0.0664635822  \\
            0.270365387  0.0779465735  \\
            0.269980192  0.0894461349  \\
            0.26956436  0.100943066  \\
            0.269104183  0.112432793  \\
            0.268577129  0.123916283  \\
            0.267940551  0.13539654  \\
            0.267104477  0.14687711  \\
            0.265893519  0.158361465  \\
        }
        ;
    \addplot[color={rgb,1:red,0.502;green,0.502;blue,0.502}, name path={c73def6c-38f5-40a8-bce4-583deb2ea7b1}, draw opacity={0.25}, line width={0.1}, solid, forget plot]
        table[row sep={\\}]
        {
            \\
            0.276492447  0.043291498  \\
            0.276172608  0.0546808504  \\
            0.275844574  0.0662192404  \\
            0.275506437  0.0777847916  \\
            0.275154352  0.0893394351  \\
            0.27478376  0.100875117  \\
            0.274389625  0.112393506  \\
            0.273965776  0.123899221  \\
            0.27350086  0.135397568  \\
            0.272960097  0.146893889  \\
            0.272215903  0.158393458  \\
        }
        ;
    \addplot[color={rgb,1:red,0.502;green,0.502;blue,0.502}, name path={7d2e3c57-06bd-44de-a980-4c41ea5e8b53}, draw opacity={0.25}, line width={0.1}, solid, forget plot]
        table[row sep={\\}]
        {
            \\
            0.28147167  0.0425630659  \\
            0.281171709  0.0542436317  \\
            0.280865729  0.0659467727  \\
            0.280553728  0.0776092112  \\
            0.280234754  0.0892248452  \\
            0.279909104  0.100801654  \\
            0.279580146  0.11234951  \\
            0.279258162  0.123877406  \\
            0.278965652  0.135393143  \\
            0.278737038  0.146903813  \\
            0.278544694  0.158416122  \\
        }
        ;
    \addplot[color={rgb,1:red,0.502;green,0.502;blue,0.502}, name path={fde8c377-72e4-41b7-93f6-67dee79cbccd}, draw opacity={0.25}, line width={0.1}, solid, forget plot]
        table[row sep={\\}]
        {
            \\
            0.286317348  0.0416866243  \\
            0.286036581  0.0537559614  \\
            0.285752267  0.0656553805  \\
            0.285465807  0.077425532  \\
            0.28517884  0.0891059116  \\
            0.284895867  0.100725017  \\
            0.284627736  0.112302445  \\
            0.284399152  0.123852089  \\
            0.284265041  0.135384381  \\
            0.284342706  0.146907911  \\
            0.284838438  0.15843077  \\
        }
        ;
    \addplot[color={rgb,1:red,0.502;green,0.502;blue,0.502}, name path={a85e0d2f-bb83-4904-be89-a3d3632c831c}, draw opacity={0.25}, line width={0.1}, solid, forget plot]
        table[row sep={\\}]
        {
            \\
            0.291039765  0.0406553857  \\
            0.290777266  0.0532308333  \\
            0.290513963  0.0653534904  \\
            0.29025206  0.0772383511  \\
            0.289995372  0.0889852345  \\
            0.289751798  0.100646839  \\
            0.289537966  0.112253509  \\
            0.289389431  0.123824343  \\
            0.289386094  0.135372385  \\
            0.289723724  0.146907508  \\
            0.290954471  0.158438653  \\
        }
        ;
    \addplot[color={rgb,1:red,0.502;green,0.502;blue,0.502}, name path={36d86b8c-afed-4f70-bb2a-3a3b15d43a5e}, draw opacity={0.25}, line width={0.1}, solid, forget plot]
        table[row sep={\\}]
        {
            \\
            0.295656592  0.0394639932  \\
            0.295411348  0.0526852831  \\
            0.295168221  0.0650501922  \\
            0.294929653  0.0770520791  \\
            0.294701129  0.0888651237  \\
            0.294493079  0.100568503  \\
            0.294325948  0.112203702  \\
            0.29424122  0.123795047  \\
            0.29433012  0.13535814  \\
            0.294827372  0.146903947  \\
            0.296547413  0.158441409  \\
        }
        ;
    \addplot[color={rgb,1:red,0.502;green,0.502;blue,0.502}, name path={8b99cf19-8691-4c9d-a08f-9df48988547b}, draw opacity={0.25}, line width={0.1}, solid, forget plot]
        table[row sep={\\}]
        {
            \\
            0.300156772  0.0381152593  \\
            0.299927622  0.0521458276  \\
            0.299703658  0.0647572204  \\
            0.299486935  0.0768720955  \\
            0.299284071  0.0887483731  \\
            0.299107254  0.100491688  \\
            0.298978776  0.112154178  \\
            0.298940957  0.123765126  \\
            0.299079686  0.135342598  \\
            0.299590915  0.146898493  \\
            0.30104059  0.158440918  \\
        }
        ;
    \addplot[color={rgb,1:red,0.502;green,0.502;blue,0.502}, name path={6bf8ecb2-a0a4-484e-ac2c-246c41bc3558}, draw opacity={0.25}, line width={0.1}, solid, forget plot]
        table[row sep={\\}]
        {
            \\
            0.304466009  0.0366412401  \\
            0.304251671  0.0516529754  \\
            0.304045349  0.0644899309  \\
            0.303848445  0.0767055079  \\
            0.30366835  0.0886389539  \\
            0.303518087  0.100418866  \\
            0.303420186  0.112106599  \\
            0.303414196  0.123735793  \\
            0.303569436  0.135326713  \\
            0.303987443  0.14689216  \\
            0.304466009  0.158438995  \\
        }
        ;
    \addplot[color={rgb,1:red,0.502;green,0.502;blue,0.502}, name path={617be8f8-9411-433e-89ea-982c0f1e0a8f}, draw opacity={0.25}, line width={0.1}, solid, forget plot]
        table[row sep={\\}]
        {
            \\
            0.30637899  0.0359279998  \\
            0.30617103  0.0514550358  \\
            0.305972278  0.0643785745  \\
            0.305783778  0.0766345039  \\
            0.30561319  0.0885916799  \\
            0.305473655  0.100387089  \\
            0.305387467  0.112085633  \\
            0.305392295  0.123722702  \\
            0.305549711  0.135319471  \\
            0.30593279  0.146889165  \\
            0.30637899  0.158438995  \\
        }
        ;
    \addplot[color={rgb,1:red,0.502;green,0.502;blue,0.502}, name path={f3485236-6e39-4aa1-b6c6-55a98731969e}, draw opacity={0.25}, line width={0.1}, solid, forget plot]
        table[row sep={\\}]
        {
            \\
            0.309001654  0.0359279998  \\
            0.308802336  0.0512223616  \\
            0.308613658  0.0642347485  \\
            0.308436275  0.0765405819  \\
            0.308278024  0.0885283947  \\
            0.308152199  0.100344203  \\
            0.308080316  0.112057135  \\
            0.308097094  0.123704761  \\
            0.308253765  0.135309398  \\
            0.308596283  0.146884903  \\
            0.309001654  0.158438995  \\
        }
        ;
    \addplot[color={rgb,1:red,0.502;green,0.502;blue,0.502}, name path={8a8f89e7-4b30-45d8-8d75-2d33061e6272}, draw opacity={0.25}, line width={0.1}, solid, forget plot]
        table[row sep={\\}]
        {
            \\
            0.312035203  0.0359279998  \\
            0.31184572  0.0510053523  \\
            0.311668158  0.0640826449  \\
            0.311503053  0.0764375851  \\
            0.311358213  0.0884577706  \\
            0.311246753  0.100295797  \\
            0.311189145  0.112024672  \\
            0.311216205  0.123684108  \\
            0.311368674  0.135297656  \\
            0.311672181  0.146879822  \\
            0.312035203  0.158438995  \\
        }
        ;
    \addplot[color={rgb,1:red,0.502;green,0.502;blue,0.502}, name path={cd59a2ec-13ef-4d1a-bce7-3c7be77b041a}, draw opacity={0.25}, line width={0.1}, solid, forget plot]
        table[row sep={\\}]
        {
            \\
            0.315543979  0.0359279998  \\
            0.315365613  0.0508075356  \\
            0.315200329  0.0639262348  \\
            0.315048546  0.0763266906  \\
            0.314918011  0.0883799866  \\
            0.314821333  0.100241728  \\
            0.314777434  0.111987993  \\
            0.31481269  0.123660512  \\
            0.314957738  0.135284036  \\
            0.315223992  0.146873832  \\
            0.315543979  0.158438995  \\
        }
        ;
    \addplot[color={rgb,1:red,0.502;green,0.502;blue,0.502}, name path={bed2755e-abae-481e-9463-25991f0175ca}, draw opacity={0.25}, line width={0.1}, solid, forget plot]
        table[row sep={\\}]
        {
            \\
            0.319602489  0.0359279998  \\
            0.31943658  0.0506314039  \\
            0.319284648  0.0637700856  \\
            0.31914717  0.0762099251  \\
            0.319031596  0.0882957429  \\
            0.318949819  0.100182116  \\
            0.318918616  0.111947015  \\
            0.318959653  0.123633824  \\
            0.319094419  0.13526842  \\
            0.319325238  0.146866858  \\
            0.319602489  0.158438995  \\
        }
        ;
    \addplot[color={rgb,1:red,0.502;green,0.502;blue,0.502}, name path={f3c69f6c-1f5a-4e3b-bba4-8f0d8f3a5a12}, draw opacity={0.25}, line width={0.1}, solid, forget plot]
        table[row sep={\\}]
        {
            \\
            0.324296772  0.0359279998  \\
            0.324144721  0.0504782163  \\
            0.324007124  0.0636189654  \\
            0.323884755  0.0760901347  \\
            0.3237845  0.0882063583  \\
            0.323717207  0.10011749  \\
            0.323697239  0.111901879  \\
            0.323741436  0.123604015  \\
            0.323863685  0.135250732  \\
            0.324060977  0.146858841  \\
            0.324296772  0.158438995  \\
        }
        ;
    \addplot[color={rgb,1:red,0.502;green,0.502;blue,0.502}, name path={2da65b48-469c-4504-a352-cdeebc31602d}, draw opacity={0.25}, line width={0.1}, solid, forget plot]
        table[row sep={\\}]
        {
            \\
            0.329726517  0.0359279998  \\
            0.329589605  0.0503480025  \\
            0.329467267  0.0634773076  \\
            0.329360425  0.0759707838  \\
            0.329275489  0.0881138518  \\
            0.329221845  0.100048877  \\
            0.329211265  0.111853063  \\
            0.329256028  0.123571284  \\
            0.329364181  0.135231048  \\
            0.329530001  0.146849811  \\
            0.329726517  0.158438995  \\
        }
        ;
    \addplot[color={rgb,1:red,0.502;green,0.502;blue,0.502}, name path={d440e7fb-f43d-48ba-bc5e-adf5bfdff466}, draw opacity={0.25}, line width={0.1}, solid, forget plot]
        table[row sep={\\}]
        {
            \\
            0.33600688  0.0359279998  \\
            0.33588624  0.050239712  \\
            0.335779727  0.0633487701  \\
            0.335688621  0.0758556724  \\
            0.335618496  0.0880208537  \\
            0.33557725  0.0999778882  \\
            0.335573941  0.11180146  \\
            0.335616916  0.123536088  \\
            0.335709989  0.13520956  \\
            0.335846633  0.146839857  \\
            0.33600688  0.158438995  \\
        }
        ;
    \addplot[color={rgb,1:red,0.502;green,0.502;blue,0.502}, name path={d3649cfe-492c-43f1-8ef1-ce63841befb1}, draw opacity={0.25}, line width={0.1}, solid, forget plot]
        table[row sep={\\}]
        {
            \\
            0.343271166  0.0359279998  \\
            0.343167543  0.0501515158  \\
            0.343077153  0.0632358715  \\
            0.343001515  0.0757484511  \\
            0.342945307  0.0879304111  \\
            0.34291479  0.0999066234  \\
            0.342916548  0.111748412  \\
            0.342955828  0.123499222  \\
            0.343033612  0.135186732  \\
            0.343143642  0.146829158  \\
            0.343271166  0.158438995  \\
        }
        ;
    \addplot[color={rgb,1:red,0.502;green,0.502;blue,0.502}, name path={5f9cff80-dd24-4a16-abd0-11f1e90ee78d}, draw opacity={0.25}, line width={0.1}, solid, forget plot]
        table[row sep={\\}]
        {
            \\
            0.351673514  0.0359279998  \\
            0.351587176  0.0500811152  \\
            0.35151279  0.0631398857  \\
            0.351451844  0.075652197  \\
            0.351408213  0.0878456458  \\
            0.351386577  0.0998375863  \\
            0.351391405  0.111695699  \\
            0.351425529  0.123461857  \\
            0.351488471  0.135163233  \\
            0.351574689  0.146818027  \\
            0.351673514  0.158438995  \\
        }
        ;
    \addplot[color={rgb,1:red,0.502;green,0.502;blue,0.502}, name path={f61fa9ee-1ca0-487f-98c4-f243d9ef848d}, draw opacity={0.25}, line width={0.1}, solid, forget plot]
        table[row sep={\\}]
        {
            \\
            0.36139217  0.0359279998  \\
            0.36132282  0.0500260592  \\
            0.361263692  0.0630608946  \\
            0.361216277  0.0755690634  \\
            0.361183614  0.0877693817  \\
            0.361168951  0.0997733697  \\
            0.36117506  0.111645341  \\
            0.361203313  0.123425424  \\
            0.361252427  0.135139957  \\
            0.361317903  0.146806911  \\
            0.36139217  0.158438995  \\
        }
        ;
    \addplot[color={rgb,1:red,0.502;green,0.502;blue,0.502}, name path={18bbf5ac-347a-4027-919b-20c900950370}, draw opacity={0.25}, line width={0.1}, solid, forget plot]
        table[row sep={\\}]
        {
            \\
            0.372633398  0.0359279998  \\
            0.372579962  0.0499839634  \\
            0.372534901  0.0629980415  \\
            0.372499496  0.0755001307  \\
            0.372475952  0.0877037272  \\
            0.372466445  0.0997162983  \\
            0.372472584  0.111599416  \\
            0.372494727  0.123391509  \\
            0.372531503  0.135117978  \\
            0.372579485  0.146796286  \\
            0.372633398  0.158438995  \\
        }
        ;
    \addplot[color={rgb,1:red,0.502;green,0.502;blue,0.502}, name path={26451768-3c45-4824-9430-e12120592a3f}, draw opacity={0.25}, line width={0.1}, solid, forget plot]
        table[row sep={\\}]
        {
            \\
            0.385635734  0.0359279998  \\
            0.385596573  0.0499526002  \\
            0.385563821  0.0629497841  \\
            0.385538548  0.0754453912  \\
            0.385522306  0.0876498744  \\
            0.385516405  0.0996681154  \\
            0.38552174  0.111559674  \\
            0.385538161  0.12336158  \\
            0.385564446  0.135098293  \\
            0.385598153  0.146786705  \\
            0.385635734  0.158438995  \\
        }
        ;
    \addplot[color={rgb,1:red,0.502;green,0.502;blue,0.502}, name path={8a9aa7b4-f301-497f-8e90-08ae5650dca2}, draw opacity={0.25}, line width={0.1}, solid, forget plot]
        table[row sep={\\}]
        {
            \\
            0.400675058  0.0359279998  \\
            0.400647938  0.0499299727  \\
            0.400625437  0.0629141927  \\
            0.400608331  0.0754039288  \\
            0.400597632  0.0876079723  \\
            0.400594175  0.0996296778  \\
            0.400598288  0.111527286  \\
            0.400609732  0.123336755  \\
            0.400627553  0.135081753  \\
            0.400650084  0.146778584  \\
            0.400675058  0.158438995  \\
        }
        ;
    \addplot[color={rgb,1:red,0.502;green,0.502;blue,0.502}, name path={ac490a35-fc55-467f-a3e4-3ee45388391b}, draw opacity={0.25}, line width={0.1}, solid, forget plot]
        table[row sep={\\}]
        {
            \\
            0.418070495  0.0359279998  \\
            0.418052942  0.049914293  \\
            0.418038458  0.0628891364  \\
            0.41802755  0.0753741562  \\
            0.418020904  0.0875772387  \\
            0.418018937  0.0996009037  \\
            0.418021768  0.111502595  \\
            0.418029219  0.123317547  \\
            0.418040574  0.135068804  \\
            0.418054789  0.146772176  \\
            0.418070495  0.158438995  \\
        }
        ;
    \addplot[color={rgb,1:red,0.502;green,0.502;blue,0.502}, name path={809bddcb-2155-49b0-9008-2b1665b6ae1b}, draw opacity={0.25}, line width={0.1}, solid, forget plot]
        table[row sep={\\}]
        {
            \\
            0.438191146  0.0359279998  \\
            0.438180625  0.0499039702  \\
            0.438171983  0.0628724545  \\
            0.438165516  0.0753540471  \\
            0.438161641  0.0875561461  \\
            0.438160568  0.0995808542  \\
            0.438162357  0.111485146  \\
            0.438166827  0.123303808  \\
            0.438173562  0.135059461  \\
            0.438181937  0.146767527  \\
            0.438191146  0.158438995  \\
        }
        ;
    \addplot[color={rgb,1:red,0.502;green,0.502;blue,0.502}, name path={60b8b0b7-32e8-49f5-a210-6f0aafba7e66}, draw opacity={0.25}, line width={0.1}, solid, forget plot]
        table[row sep={\\}]
        {
            \\
            0.461463988  0.0359279998  \\
            0.461458236  0.0498975925  \\
            0.461453527  0.0628620759  \\
            0.461450011  0.0753414184  \\
            0.461447924  0.0875427574  \\
            0.461447388  0.0995679796  \\
            0.461448401  0.111473829  \\
            0.461450845  0.123294815  \\
            0.461454481  0.135053307  \\
            0.461459011  0.146764442  \\
            0.461463988  0.158438995  \\
        }
        ;
    \addplot[color={rgb,1:red,0.502;green,0.502;blue,0.502}, name path={0b07a53f-73ee-4960-bdf2-44405d211b95}, draw opacity={0.25}, line width={0.1}, solid, forget plot]
        table[row sep={\\}]
        {
            \\
            0.488382816  0.0359279998  \\
            0.488380045  0.0498939641  \\
            0.48837775  0.0628561452  \\
            0.488376051  0.0753341392  \\
            0.488375038  0.0875349864  \\
            0.48837477  0.099560447  \\
            0.488375247  0.11146716  \\
            0.488376439  0.123289488  \\
            0.488378197  0.135049641  \\
            0.488380402  0.146762609  \\
            0.488382816  0.158438995  \\
        }
        ;
    \addplot[color={rgb,1:red,0.502;green,0.502;blue,0.502}, name path={df4b1f03-a6e0-43a3-a713-d2d00abde52a}, draw opacity={0.25}, line width={0.1}, solid, forget plot]
        table[row sep={\\}]
        {
            \\
            0.519518852  0.0359279998  \\
            0.519517779  0.0498921089  \\
            0.519516885  0.0628531054  \\
            0.51951623  0.075330399  \\
            0.519515812  0.0875309706  \\
            0.519515753  0.0995565429  \\
            0.519515932  0.111463688  \\
            0.519516408  0.123286709  \\
            0.519517124  0.135047719  \\
            0.519517958  0.146761641  \\
            0.519518852  0.158438995  \\
        }
        ;
    \addplot[color={rgb,1:red,0.502;green,0.502;blue,0.502}, name path={3fd199db-477b-4214-a11c-86c977e3aa78}, draw opacity={0.25}, line width={0.1}, solid, forget plot]
        table[row sep={\\}]
        {
            \\
            0.555532813  0.0359279998  \\
            0.555532813  0.0498913489  \\
            0.555532813  0.0628518537  \\
            0.555532813  0.0753288642  \\
            0.555532813  0.0875293314  \\
            0.555532813  0.0995549411  \\
            0.555532813  0.111462265  \\
            0.555532813  0.123285569  \\
            0.555532813  0.135046944  \\
            0.555532813  0.146761253  \\
            0.555532813  0.158438995  \\
        }
        ;
    \addplot[color={rgb,1:red,0.502;green,0.502;blue,0.502}, name path={04b209fb-7451-4215-9f8d-a6e182ce6074}, draw opacity={1.0}, line width={0.75}, solid, forget plot]
        table[row sep={\\}]
        {
            \\
            0.119999997  0.0560293496  \\
            0.126558438  0.0560412109  \\
            0.133245647  0.0560548678  \\
            0.139968172  0.0560677573  \\
            0.146715805  0.0560786538  \\
            0.1535175  0.056088496  \\
            0.160320878  0.0560980961  \\
            0.16714634  0.0561074205  \\
            0.173985556  0.0561159588  \\
            0.180828556  0.0561232083  \\
            0.187675938  0.056128744  \\
            0.194527879  0.0561321341  \\
            0.201378569  0.0561328046  \\
            0.208231941  0.0561298095  \\
            0.215074837  0.0561216772  \\
            0.2219138  0.0561067387  \\
            0.228741482  0.0560832731  \\
            0.235521033  0.056048099  \\
            0.242301211  0.055992946  \\
            0.248895854  0.0559058823  \\
            0.255062699  0.0557756722  \\
            0.260677278  0.0555957817  \\
            0.265937179  0.0553589165  \\
            0.271079302  0.0550550483  \\
            0.276172608  0.0546808504  \\
            0.281171709  0.0542436317  \\
            0.286036581  0.0537559614  \\
            0.290777266  0.0532308333  \\
            0.295411348  0.0526852831  \\
            0.299927622  0.0521458276  \\
            0.304251671  0.0516529754  \\
            0.30617103  0.0514550358  \\
            0.308802336  0.0512223616  \\
            0.31184572  0.0510053523  \\
            0.315365613  0.0508075356  \\
            0.31943658  0.0506314039  \\
            0.324144721  0.0504782163  \\
            0.329589605  0.0503480025  \\
            0.33588624  0.050239712  \\
            0.343167543  0.0501515158  \\
            0.351587176  0.0500811152  \\
            0.36132282  0.0500260592  \\
            0.372579962  0.0499839634  \\
            0.385596573  0.0499526002  \\
            0.400647938  0.0499299727  \\
            0.418052942  0.049914293  \\
            0.438180625  0.0499039702  \\
            0.461458236  0.0498975925  \\
            0.488380045  0.0498939641  \\
            0.519517779  0.0498921089  \\
            0.555532813  0.0498913489  \\
        }
        ;
    \addplot[color={rgb,1:red,0.502;green,0.502;blue,0.502}, name path={b5909803-3b15-4b3c-b892-ac9acf28cc05}, draw opacity={1.0}, line width={0.75}, solid, forget plot]
        table[row sep={\\}]
        {
            \\
            0.119999997  0.0671061799  \\
            0.126518264  0.0671321899  \\
            0.133164659  0.0671587884  \\
            0.139846593  0.067185156  \\
            0.146554217  0.0672108829  \\
            0.153316632  0.0672359988  \\
            0.160082117  0.0672602504  \\
            0.166871279  0.0672834814  \\
            0.173676267  0.0673052445  \\
            0.180487558  0.0673249513  \\
            0.187306151  0.067341961  \\
            0.194132686  0.0673554763  \\
            0.200961858  0.0673644841  \\
            0.207798049  0.0673677102  \\
            0.214628607  0.0673634559  \\
            0.221460417  0.067349568  \\
            0.228286475  0.067323193  \\
            0.235070005  0.0672804862  \\
            0.241859689  0.0672152191  \\
            0.248468667  0.0671219453  \\
            0.254653007  0.0669992864  \\
            0.26028654  0.0668503419  \\
            0.265566319  0.0666736141  \\
            0.270729423  0.0664635822  \\
            0.275844574  0.0662192404  \\
            0.280865729  0.0659467727  \\
            0.285752267  0.0656553805  \\
            0.290513963  0.0653534904  \\
            0.295168221  0.0650501922  \\
            0.299703658  0.0647572204  \\
            0.304045349  0.0644899309  \\
            0.305972278  0.0643785745  \\
            0.308613658  0.0642347485  \\
            0.311668158  0.0640826449  \\
            0.315200329  0.0639262348  \\
            0.319284648  0.0637700856  \\
            0.324007124  0.0636189654  \\
            0.329467267  0.0634773076  \\
            0.335779727  0.0633487701  \\
            0.343077153  0.0632358715  \\
            0.35151279  0.0631398857  \\
            0.361263692  0.0630608946  \\
            0.372534901  0.0629980415  \\
            0.385563821  0.0629497841  \\
            0.400625437  0.0629141927  \\
            0.418038458  0.0628891364  \\
            0.438171983  0.0628724545  \\
            0.461453527  0.0628620759  \\
            0.48837775  0.0628561452  \\
            0.519516885  0.0628531054  \\
            0.555532813  0.0628518537  \\
        }
        ;
    \addplot[color={rgb,1:red,0.502;green,0.502;blue,0.502}, name path={294f8519-b1f1-4359-afef-95d26183ee13}, draw opacity={1.0}, line width={0.75}, solid, forget plot]
        table[row sep={\\}]
        {
            \\
            0.120000005  0.0781830102  \\
            0.126477122  0.0782214478  \\
            0.133081689  0.0782606304  \\
            0.139721945  0.078299813  \\
            0.146388352  0.0783387199  \\
            0.153110132  0.078377299  \\
            0.159836203  0.0784149542  \\
            0.166587412  0.0784514323  \\
            0.173356295  0.0784861669  \\
            0.18013376  0.0785184652  \\
            0.186921269  0.0785475522  \\
            0.193719909  0.0785724893  \\
            0.200524941  0.0785921291  \\
            0.207341343  0.0786051229  \\
            0.21415709  0.0786097869  \\
            0.220979676  0.078604117  \\
            0.227802634  0.0785856098  \\
            0.234589562  0.0785513818  \\
            0.24138926  0.0784974843  \\
            0.248014197  0.0784218386  \\
            0.254218608  0.0783263519  \\
            0.259874225  0.0782157704  \\
            0.265177429  0.0780902132  \\
            0.270365387  0.0779465735  \\
            0.275506437  0.0777847916  \\
            0.280553728  0.0776092112  \\
            0.285465807  0.077425532  \\
            0.29025206  0.0772383511  \\
            0.294929653  0.0770520791  \\
            0.299486935  0.0768720955  \\
            0.303848445  0.0767055079  \\
            0.305783778  0.0766345039  \\
            0.308436275  0.0765405819  \\
            0.311503053  0.0764375851  \\
            0.315048546  0.0763266906  \\
            0.31914717  0.0762099251  \\
            0.323884755  0.0760901347  \\
            0.329360425  0.0759707838  \\
            0.335688621  0.0758556724  \\
            0.343001515  0.0757484511  \\
            0.351451844  0.075652197  \\
            0.361216277  0.0755690634  \\
            0.372499496  0.0755001307  \\
            0.385538548  0.0754453912  \\
            0.400608331  0.0754039288  \\
            0.41802755  0.0753741562  \\
            0.438165516  0.0753540471  \\
            0.461450011  0.0753414184  \\
            0.488376051  0.0753341392  \\
            0.51951623  0.075330399  \\
            0.555532813  0.0753288642  \\
        }
        ;
    \addplot[color={rgb,1:red,0.502;green,0.502;blue,0.502}, name path={3771c237-7f99-4c8c-8929-114178023c4f}, draw opacity={1.0}, line width={0.75}, solid, forget plot]
        table[row sep={\\}]
        {
            \\
            0.120000005  0.0892598331  \\
            0.12643443  0.0893098935  \\
            0.132995501  0.0893610269  \\
            0.13959226  0.089412421  \\
            0.146215454  0.0894638523  \\
            0.152894422  0.0895153135  \\
            0.159578711  0.0895660743  \\
            0.166289359  0.0896158293  \\
            0.17301926  0.0896639302  \\
            0.179759741  0.0897095799  \\
            0.186512634  0.0897519514  \\
            0.193279505  0.0897900462  \\
            0.20005624  0.0898227021  \\
            0.206848547  0.0898486376  \\
            0.213645324  0.0898663402  \\
            0.22045505  0.0898741558  \\
            0.227272227  0.0898701549  \\
            0.234061286  0.089852348  \\
            0.240871623  0.0898183882  \\
            0.247515231  0.089767538  \\
            0.253744245  0.0897022411  \\
            0.259427547  0.0896268114  \\
            0.264760494  0.0895420015  \\
            0.269980192  0.0894461349  \\
            0.275154352  0.0893394351  \\
            0.280234754  0.0892248452  \\
            0.28517884  0.0891059116  \\
            0.289995372  0.0889852345  \\
            0.294701129  0.0888651237  \\
            0.299284071  0.0887483731  \\
            0.30366835  0.0886389539  \\
            0.30561319  0.0885916799  \\
            0.308278024  0.0885283947  \\
            0.311358213  0.0884577706  \\
            0.314918011  0.0883799866  \\
            0.319031596  0.0882957429  \\
            0.3237845  0.0882063583  \\
            0.329275489  0.0881138518  \\
            0.335618496  0.0880208537  \\
            0.342945307  0.0879304111  \\
            0.351408213  0.0878456458  \\
            0.361183614  0.0877693817  \\
            0.372475952  0.0877037272  \\
            0.385522306  0.0876498744  \\
            0.400597632  0.0876079723  \\
            0.418020904  0.0875772387  \\
            0.438161641  0.0875561461  \\
            0.461447924  0.0875427574  \\
            0.488375038  0.0875349864  \\
            0.519515812  0.0875309706  \\
            0.555532813  0.0875293314  \\
        }
        ;
    \addplot[color={rgb,1:red,0.502;green,0.502;blue,0.502}, name path={55508043-bd83-4e4e-9348-16d4013239b9}, draw opacity={1.0}, line width={0.75}, solid, forget plot]
        table[row sep={\\}]
        {
            \\
            0.120000005  0.100336671  \\
            0.126389757  0.100397676  \\
            0.13290517  0.100460134  \\
            0.139456004  0.100523241  \\
            0.146033287  0.100586854  \\
            0.152666479  0.100651085  \\
            0.159305796  0.100715086  \\
            0.165972456  0.100778565  \\
            0.172659665  0.100840792  \\
            0.179359019  0.100900881  \\
            0.186072558  0.100957938  \\
            0.192802265  0.101010941  \\
            0.199544623  0.101058744  \\
            0.206306234  0.101100162  \\
            0.213077247  0.101133838  \\
            0.219867632  0.101158448  \\
            0.226673692  0.101172529  \\
            0.233461618  0.101174623  \\
            0.240282282  0.101163387  \\
            0.246947899  0.101138338  \\
            0.25320816  0.101101421  \\
            0.25892812  0.101056084  \\
            0.2643013  0.101003513  \\
            0.26956436  0.100943066  \\
            0.27478376  0.100875117  \\
            0.279909104  0.100801654  \\
            0.284895867  0.100725017  \\
            0.289751798  0.100646839  \\
            0.294493079  0.100568503  \\
            0.299107254  0.100491688  \\
            0.303518087  0.100418866  \\
            0.305473655  0.100387089  \\
            0.308152199  0.100344203  \\
            0.311246753  0.100295797  \\
            0.314821333  0.100241728  \\
            0.318949819  0.100182116  \\
            0.323717207  0.10011749  \\
            0.329221845  0.100048877  \\
            0.33557725  0.0999778882  \\
            0.34291479  0.0999066234  \\
            0.351386577  0.0998375863  \\
            0.361168951  0.0997733697  \\
            0.372466445  0.0997162983  \\
            0.385516405  0.0996681154  \\
            0.400594175  0.0996296778  \\
            0.418018937  0.0996009037  \\
            0.438160568  0.0995808542  \\
            0.461447388  0.0995679796  \\
            0.48837477  0.099560447  \\
            0.519515753  0.0995565429  \\
            0.555532813  0.0995549411  \\
        }
        ;
    \addplot[color={rgb,1:red,0.502;green,0.502;blue,0.502}, name path={425e75a3-f6b9-4069-974a-17b3af2ef541}, draw opacity={1.0}, line width={0.75}, solid, forget plot]
        table[row sep={\\}]
        {
            \\
            0.120000005  0.111413494  \\
            0.126343116  0.111484513  \\
            0.132810473  0.111557484  \\
            0.139312387  0.111631706  \\
            0.145840228  0.111707218  \\
            0.152423769  0.111784257  \\
            0.15901418  0.111861899  \\
            0.165632829  0.11193984  \\
            0.172273144  0.112017214  \\
            0.178926647  0.112093031  \\
            0.185595259  0.112166278  \\
            0.192280948  0.112235889  \\
            0.198980644  0.112300679  \\
            0.205701783  0.112359539  \\
            0.212436169  0.112411208  \\
            0.219195962  0.112454593  \\
            0.225980401  0.112488531  \\
            0.232759148  0.112511829  \\
            0.239586458  0.112523735  \\
            0.246276617  0.112523526  \\
            0.252577126  0.112512253  \\
            0.258347571  0.112492524  \\
            0.26377812  0.112465963  \\
            0.269104183  0.112432793  \\
            0.274389625  0.112393506  \\
            0.279580146  0.11234951  \\
            0.284627736  0.112302445  \\
            0.289537966  0.112253509  \\
            0.294325948  0.112203702  \\
            0.298978776  0.112154178  \\
            0.303420186  0.112106599  \\
            0.305387467  0.112085633  \\
            0.308080316  0.112057135  \\
            0.311189145  0.112024672  \\
            0.314777434  0.111987993  \\
            0.318918616  0.111947015  \\
            0.323697239  0.111901879  \\
            0.329211265  0.111853063  \\
            0.335573941  0.11180146  \\
            0.342916548  0.111748412  \\
            0.351391405  0.111695699  \\
            0.36117506  0.111645341  \\
            0.372472584  0.111599416  \\
            0.38552174  0.111559674  \\
            0.400598288  0.111527286  \\
            0.418021768  0.111502595  \\
            0.438162357  0.111485146  \\
            0.461448401  0.111473829  \\
            0.488375247  0.11146716  \\
            0.519515932  0.111463688  \\
            0.555532813  0.111462265  \\
        }
        ;
    \addplot[color={rgb,1:red,0.502;green,0.502;blue,0.502}, name path={16ca0f0e-d152-49d1-af0e-889c2e791655}, draw opacity={1.0}, line width={0.75}, solid, forget plot]
        table[row sep={\\}]
        {
            \\
            0.120000005  0.122490317  \\
            0.126295686  0.122569807  \\
            0.132712901  0.122651935  \\
            0.139162093  0.122736335  \\
            0.145635501  0.12282332  \\
            0.152164072  0.122913301  \\
            0.15870063  0.123005226  \\
            0.165267035  0.123098657  \\
            0.171856567  0.123192549  \\
            0.178460047  0.123285651  \\
            0.185078308  0.123376779  \\
            0.191712439  0.123464711  \\
            0.198359087  0.123548187  \\
            0.205026209  0.123626083  \\
            0.211707339  0.123697169  \\
            0.218417466  0.123760514  \\
            0.225160286  0.123815134  \\
            0.231911585  0.123859957  \\
            0.238732681  0.123894617  \\
            0.245444879  0.123917893  \\
            0.251795858  0.123929895  \\
            0.25763768  0.123932414  \\
            0.26315406  0.123927556  \\
            0.268577129  0.123916283  \\
            0.273965776  0.123899221  \\
            0.279258162  0.123877406  \\
            0.284399152  0.123852089  \\
            0.289389431  0.123824343  \\
            0.29424122  0.123795047  \\
            0.298940957  0.123765126  \\
            0.303414196  0.123735793  \\
            0.305392295  0.123722702  \\
            0.308097094  0.123704761  \\
            0.311216205  0.123684108  \\
            0.31481269  0.123660512  \\
            0.318959653  0.123633824  \\
            0.323741436  0.123604015  \\
            0.329256028  0.123571284  \\
            0.335616916  0.123536088  \\
            0.342955828  0.123499222  \\
            0.351425529  0.123461857  \\
            0.361203313  0.123425424  \\
            0.372494727  0.123391509  \\
            0.385538161  0.12336158  \\
            0.400609732  0.123336755  \\
            0.418029219  0.123317547  \\
            0.438166827  0.123303808  \\
            0.461450845  0.123294815  \\
            0.488376439  0.123289488  \\
            0.519516408  0.123286709  \\
            0.555532813  0.123285569  \\
        }
        ;
    \addplot[color={rgb,1:red,0.502;green,0.502;blue,0.502}, name path={98fe6ec8-e466-4b28-9022-837d604a38af}, draw opacity={1.0}, line width={0.75}, solid, forget plot]
        table[row sep={\\}]
        {
            \\
            0.120000012  0.133567154  \\
            0.126252562  0.133652225  \\
            0.132618949  0.133741051  \\
            0.139009237  0.133834034  \\
            0.14541921  0.133931905  \\
            0.151883841  0.13403514  \\
            0.15835996  0.134142399  \\
            0.164870307  0.134252951  \\
            0.171407461  0.13436529  \\
            0.177959904  0.134477779  \\
            0.184525192  0.134588853  \\
            0.191101521  0.134697065  \\
            0.197683975  0.134800941  \\
            0.204279929  0.134899318  \\
            0.210884154  0.134990916  \\
            0.217514426  0.135074899  \\
            0.224179819  0.135150447  \\
            0.230865002  0.135216579  \\
            0.237644017  0.135273308  \\
            0.244357824  0.13531889  \\
            0.250764996  0.135352507  \\
            0.256709009  0.135375157  \\
            0.262359738  0.135389298  \\
            0.267940551  0.13539654  \\
            0.27350086  0.135397568  \\
            0.278965652  0.135393143  \\
            0.284265041  0.135384381  \\
            0.289386094  0.135372385  \\
            0.29433012  0.13535814  \\
            0.299079686  0.135342598  \\
            0.303569436  0.135326713  \\
            0.305549711  0.135319471  \\
            0.308253765  0.135309398  \\
            0.311368674  0.135297656  \\
            0.314957738  0.135284036  \\
            0.319094419  0.13526842  \\
            0.323863685  0.135250732  \\
            0.329364181  0.135231048  \\
            0.335709989  0.13520956  \\
            0.343033612  0.135186732  \\
            0.351488471  0.135163233  \\
            0.361252427  0.135139957  \\
            0.372531503  0.135117978  \\
            0.385564446  0.135098293  \\
            0.400627553  0.135081753  \\
            0.418040574  0.135068804  \\
            0.438173562  0.135059461  \\
            0.461454481  0.135053307  \\
            0.488378197  0.135049641  \\
            0.519517124  0.135047719  \\
            0.555532813  0.135046944  \\
        }
        ;
    \addplot[color={rgb,1:red,0.502;green,0.502;blue,0.502}, name path={586d507f-260c-4871-9981-f86b17eda546}, draw opacity={1.0}, line width={0.75}, solid, forget plot]
        table[row sep={\\}]
        {
            \\
            0.120000012  0.144643977  \\
            0.126234606  0.144728884  \\
            0.132551491  0.144819811  \\
            0.138866305  0.14491877  \\
            0.145191148  0.145026967  \\
            0.151572704  0.145144433  \\
            0.157978609  0.145269051  \\
            0.164431185  0.145399407  \\
            0.170921043  0.145533279  \\
            0.177430943  0.145668238  \\
            0.183949143  0.14580211  \\
            0.190466985  0.145933002  \\
            0.196976677  0.146059319  \\
            0.203482866  0.146179691  \\
            0.209980592  0.146292716  \\
            0.2164886  0.146397591  \\
            0.223020017  0.146493703  \\
            0.229566857  0.146580324  \\
            0.236213505  0.146657839  \\
            0.242850944  0.146724433  \\
            0.249291167  0.146778673  \\
            0.255382955  0.146820739  \\
            0.261256844  0.14685294  \\
            0.267104477  0.14687711  \\
            0.272960097  0.146893889  \\
            0.278737038  0.146903813  \\
            0.284342706  0.146907911  \\
            0.289723724  0.146907508  \\
            0.294827372  0.146903947  \\
            0.299590915  0.146898493  \\
            0.303987443  0.14689216  \\
            0.30593279  0.146889165  \\
            0.308596283  0.146884903  \\
            0.311672181  0.146879822  \\
            0.315223992  0.146873832  \\
            0.319325238  0.146866858  \\
            0.324060977  0.146858841  \\
            0.329530001  0.146849811  \\
            0.335846633  0.146839857  \\
            0.343143642  0.146829158  \\
            0.351574689  0.146818027  \\
            0.361317903  0.146806911  \\
            0.372579485  0.146796286  \\
            0.385598153  0.146786705  \\
            0.400650084  0.146778584  \\
            0.418054789  0.146772176  \\
            0.438181937  0.146767527  \\
            0.461459011  0.146764442  \\
            0.488380402  0.146762609  \\
            0.519517958  0.146761641  \\
            0.555532813  0.146761253  \\
        }
        ;
    \addplot[color={rgb,1:red,0.502;green,0.502;blue,0.502}, name path={91acbcd4-aa12-4e47-80e7-0e1278ec0493}, draw opacity={1.0}, line width={0.75}, solid, forget plot]
        table[row sep={\\}]
        {
            \\
            0.304466009  0.158438995  \\
            0.30637899  0.158438995  \\
            0.309001654  0.158438995  \\
            0.312035203  0.158438995  \\
            0.315543979  0.158438995  \\
            0.319602489  0.158438995  \\
            0.324296772  0.158438995  \\
            0.329726517  0.158438995  \\
            0.33600688  0.158438995  \\
            0.343271166  0.158438995  \\
            0.351673514  0.158438995  \\
            0.36139217  0.158438995  \\
            0.372633398  0.158438995  \\
            0.385635734  0.158438995  \\
            0.400675058  0.158438995  \\
            0.418070495  0.158438995  \\
            0.438191146  0.158438995  \\
            0.461463988  0.158438995  \\
            0.488382816  0.158438995  \\
            0.519518852  0.158438995  \\
            0.555532813  0.158438995  \\
        }
        ;
    \addplot[color={rgb,1:red,0.502;green,0.502;blue,0.502}, name path={4b730178-3359-4e90-9e26-6f3012485ce7}, draw opacity={1.0}, line width={0.75}, solid, forget plot]
        table[row sep={\\}]
        {
            \\
            0.30637899  0.0359279998  \\
            0.309001654  0.0359279998  \\
            0.312035203  0.0359279998  \\
            0.315543979  0.0359279998  \\
            0.319602489  0.0359279998  \\
            0.324296772  0.0359279998  \\
            0.329726517  0.0359279998  \\
            0.33600688  0.0359279998  \\
            0.343271166  0.0359279998  \\
            0.351673514  0.0359279998  \\
            0.36139217  0.0359279998  \\
            0.372633398  0.0359279998  \\
            0.385635734  0.0359279998  \\
            0.400675058  0.0359279998  \\
            0.418070495  0.0359279998  \\
            0.438191146  0.0359279998  \\
            0.461463988  0.0359279998  \\
            0.488382816  0.0359279998  \\
            0.519518852  0.0359279998  \\
            0.555532813  0.0359279998  \\
        }
        ;
    \addplot[color={rgb,1:red,0.4118;green,0.6824;blue,0.3725}, name path={15e1168b-2f5a-4564-ac06-d7b92cc3a91c}, draw opacity={1.0}, line width={2}, solid, forget plot]
        table[row sep={\\}]
        {
            \\
            0.119999997  0.044952523  \\
            0.119999997  0.0560293496  \\
            0.119999997  0.0671061799  \\
            0.120000005  0.0781830102  \\
            0.120000005  0.0892598331  \\
            0.120000005  0.100336671  \\
            0.120000005  0.111413494  \\
            0.120000005  0.122490317  \\
            0.120000012  0.133567154  \\
            0.120000012  0.144643977  \\
            0.120000012  0.155720815  \\
        }
        ;
    \addplot[color={rgb,1:red,0.0;green,0.3608;blue,0.6706}, name path={5b9a6b88-5d72-4172-b162-f0ca2bcb063b}, draw opacity={1.0}, line width={1.0}, solid, mark={*}, mark size={1.125 pt}, mark repeat={1}, mark options={color={rgb,1:red,0.0;green,0.0;blue,0.0}, draw opacity={0.0}, fill={rgb,1:red,0.0;green,0.3608;blue,0.6706}, fill opacity={1.0}, line width={0.75}, rotate={0}, solid}, forget plot]
        table[row sep={\\}]
        {
            \\
            0.304466009  0.158438995  \\
            0.30104059  0.158440918  \\
            0.296547413  0.158441409  \\
            0.290954471  0.158438653  \\
            0.284838438  0.15843077  \\
            0.278544694  0.158416122  \\
            0.272215903  0.158393458  \\
            0.265893519  0.158361465  \\
            0.2595779  0.15831925  \\
            0.253263652  0.158266708  \\
            0.2469531  0.158204004  \\
            0.240648702  0.158130884  \\
            0.234346643  0.158046767  \\
            0.228036374  0.157951623  \\
            0.221715644  0.157846034  \\
            0.215387374  0.15773055  \\
            0.209050387  0.157605141  \\
            0.202688128  0.157469675  \\
            0.196288183  0.157325283  \\
            0.189850047  0.157173961  \\
            0.1833819  0.15701735  \\
            0.176883996  0.156856567  \\
            0.170381293  0.156694025  \\
            0.163917422  0.156533509  \\
            0.157518655  0.156378597  \\
            0.15119262  0.156232446  \\
            0.144942939  0.156097993  \\
            0.138760611  0.155978441  \\
            0.132587165  0.155877367  \\
            0.126339316  0.155793041  \\
            0.120000012  0.155720815  \\
            0.111004442  0.155630499  \\
            0.102162637  0.155552015  \\
            0.0936912298  0.155501485  \\
            0.0855727866  0.155502617  \\
            0.0777727887  0.155572012  \\
            0.0703083277  0.155723751  \\
            0.0631919801  0.155972108  \\
            0.0564354546  0.156330809  \\
            0.0500534326  0.156811386  \\
            0.0440574624  0.157423645  \\
            0.0384658799  0.158174351  \\
            0.033303123  0.159064338  \\
            0.0285827313  0.160096079  \\
            0.024310058  0.161268204  \\
            0.02048482  0.162573695  \\
            0.0171024166  0.164010316  \\
            0.0141469585  0.165579274  \\
            0.0116014639  0.167288974  \\
            0.00946037937  0.169150949  \\
            0.00772702973  0.171179235  \\
            0.0064273621  0.17337504  \\
            0.0055845296  0.175722182  \\
            0.0052184551  0.178180113  \\
            0.00531059969  0.180689871  \\
            0.00582839875  0.183185413  \\
            0.00676491577  0.185668364  \\
            0.00811316539  0.188122496  \\
            0.00985984225  0.190536648  \\
            0.0120013645  0.19290553  \\
            0.0145334946  0.195220843  \\
            0.0174508858  0.197471961  \\
            0.0207423549  0.199641794  \\
            0.0243888609  0.201711833  \\
            0.028366942  0.203663021  \\
            0.0326463282  0.20547694  \\
            0.0372016616  0.20713985  \\
            0.0420092456  0.20864363  \\
            0.0470369235  0.209980547  \\
            0.052257929  0.211142972  \\
            0.0576559529  0.212128595  \\
            0.0632159039  0.212937504  \\
            0.0689225122  0.213569939  \\
            0.0747614354  0.214027002  \\
            0.0807205811  0.214310423  \\
            0.0867883936  0.214422479  \\
            0.0929517895  0.214365974  \\
            0.0992028862  0.21414265  \\
            0.105543934  0.213755563  \\
            0.111961305  0.213210344  \\
            0.118446901  0.212506652  \\
            0.125078455  0.211635858  \\
            0.131927684  0.210602611  \\
            0.138974875  0.209422499  \\
            0.146162391  0.208114341  \\
            0.153422281  0.206694216  \\
            0.160723969  0.20517005  \\
            0.168066114  0.203543484  \\
            0.175464198  0.201814786  \\
            0.182902128  0.199992642  \\
            0.190356091  0.198084086  \\
            0.197829321  0.196089581  \\
            0.205321759  0.194011703  \\
            0.212824583  0.191854686  \\
            0.220334753  0.189620689  \\
            0.227847159  0.187312916  \\
            0.235348269  0.184936389  \\
            0.242835328  0.182491779  \\
            0.2503106  0.179979146  \\
            0.25776273  0.177402914  \\
            0.265186906  0.174763754  \\
            0.272589475  0.172059685  \\
            0.279946595  0.16930072  \\
            0.287187666  0.166512892  \\
            0.294101626  0.16378139  \\
            0.300096214  0.161357403  \\
            0.304542005  0.159526005  \\
        }
        ;
    \addplot[color={rgb,1:red,0.7451;green,0.298;blue,0.302}, name path={8ce78fbc-9b8d-4d2b-846e-92b91f1885e0}, draw opacity={1.0}, line width={1.0}, solid, mark={*}, mark size={1.125 pt}, mark repeat={1}, mark options={color={rgb,1:red,0.0;green,0.0;blue,0.0}, draw opacity={0.0}, fill={rgb,1:red,0.7451;green,0.298;blue,0.302}, fill opacity={1.0}, line width={0.75}, rotate={0}, solid}, forget plot]
        table[row sep={\\}]
        {
            \\
            0.0  0.0  \\
            0.000221907947  0.00327736163  \\
            0.000816724729  0.00622903509  \\
            0.00176226569  0.00906565692  \\
            0.00305221137  0.0118118245  \\
            0.00468654046  0.0144791165  \\
            0.00665957527  0.0170694496  \\
            0.00896466151  0.0195757356  \\
            0.0115928082  0.0219898652  \\
            0.0145341456  0.0243067294  \\
            0.0177702773  0.0265152901  \\
            0.0212840736  0.0286078416  \\
            0.0250577405  0.0305783823  \\
            0.0290697441  0.0324218199  \\
            0.0333011299  0.0341345593  \\
            0.037733335  0.03571539  \\
            0.042347841  0.0371641666  \\
            0.0471230336  0.0384794995  \\
            0.0520442128  0.0396629795  \\
            0.0571017079  0.0407177247  \\
            0.0622883439  0.0416484587  \\
            0.0675890148  0.0424586125  \\
            0.0729805753  0.0431486107  \\
            0.0784593672  0.0437219962  \\
            0.0840299651  0.0441843681  \\
            0.0896970928  0.0445428453  \\
            0.0954227969  0.044800397  \\
            0.101146452  0.0449494347  \\
            0.107029006  0.0449990593  \\
            0.11332047  0.0449849516  \\
            0.119999997  0.044952523  \\
            0.126598373  0.0449425951  \\
            0.133326128  0.0449483283  \\
            0.140088946  0.0449485965  \\
            0.146876231  0.0449386574  \\
            0.153716758  0.0449272916  \\
            0.160557494  0.0449194387  \\
            0.167418554  0.0449135117  \\
            0.174291193  0.0449075662  \\
            0.18116495  0.0449010506  \\
            0.188040003  0.0448941104  \\
            0.194916084  0.0448871255  \\
            0.201786935  0.0448808186  \\
            0.208656073  0.0448750034  \\
            0.215509951  0.0448674187  \\
            0.222354814  0.0448568724  \\
            0.229183137  0.0448483042  \\
            0.23595807  0.0448484048  \\
            0.242728472  0.0448510461  \\
            0.249308959  0.0448360331  \\
            0.255458832  0.0447726212  \\
            0.26105535  0.0446099378  \\
            0.266296536  0.0443151407  \\
            0.271419168  0.0438756235  \\
            0.276492447  0.043291498  \\
            0.28147167  0.0425630659  \\
            0.286317348  0.0416866243  \\
            0.291039765  0.0406553857  \\
            0.295656592  0.0394639932  \\
            0.300156772  0.0381152593  \\
            0.304466009  0.0366412401  \\
            0.30637899  0.0359279998  \\
        }
        ;
\end{axis}
\end{tikzpicture}
\hspace*{5em}
     }\qquad
     \end{subfigure}

     \begin{subfigure}[t]{\textwidth}
         \centering
% \tikzsetnextfilename{solvers/ductape_single_rotor_geometry}
         \subcaptionbox{DuctAPE generated geometry.\label{fig:ductapesinglerotorgeom}}{%
             % Recommended preamble:
% \usetikzlibrary{arrows.meta}
% \usetikzlibrary{backgrounds}
% \usepgfplotslibrary{patchplots}
% \usepgfplotslibrary{fillbetween}
% \pgfplotsset{%
%     layers/standard/.define layer set={%
%         background,axis background,axis grid,axis ticks,axis lines,axis tick labels,pre main,main,axis descriptions,axis foreground%
%     }{
%         grid style={/pgfplots/on layer=axis grid},%
%         tick style={/pgfplots/on layer=axis ticks},%
%         axis line style={/pgfplots/on layer=axis lines},%
%         label style={/pgfplots/on layer=axis descriptions},%
%         legend style={/pgfplots/on layer=axis descriptions},%
%         title style={/pgfplots/on layer=axis descriptions},%
%         colorbar style={/pgfplots/on layer=axis descriptions},%
%         ticklabel style={/pgfplots/on layer=axis tick labels},%
%         axis background@ style={/pgfplots/on layer=axis background},%
%         3d box foreground style={/pgfplots/on layer=axis foreground},%
%     },
% }

\begin{tikzpicture}[/tikz/background rectangle/.style={fill={rgb,1:red,0.0;green,0.0;blue,0.0}, fill opacity={0.0}, draw opacity={0.0}}, show background rectangle]
\begin{axis}[point meta max={nan}, point meta min={nan}, legend cell align={left}, legend columns={1}, title={}, title style={at={{(0.5,1)}}, anchor={south}, font={{\fontsize{14 pt}{18.2 pt}\selectfont}}, color={rgb,1:red,0.0;green,0.0;blue,0.0}, draw opacity={1.0}, rotate={0.0}, align={center}}, legend style={color={rgb,1:red,0.0;green,0.0;blue,0.0}, draw opacity={0.0}, line width={1}, solid, fill={rgb,1:red,0.0;green,0.0;blue,0.0}, fill opacity={0.0}, text opacity={1.0}, font={{\fontsize{8 pt}{10.4 pt}\selectfont}}, text={rgb,1:red,0.0;green,0.0;blue,0.0}, cells={anchor={center}}, at={(1.02, 1)}, anchor={north west}}, axis background/.style={fill={rgb,1:red,0.0;green,0.0;blue,0.0}, opacity={0.0}}, anchor={north west}, xshift={5.0mm}, yshift={-5.0mm}, width={131.304mm}, height={50.96mm}, scaled x ticks={false}, xlabel={$z~\mathrm{(m)}$}, x tick style={color={rgb,1:red,0.0;green,0.0;blue,0.0}, opacity={1.0}}, x tick label style={color={rgb,1:red,0.0;green,0.0;blue,0.0}, opacity={1.0}, rotate={0}}, xlabel style={at={(ticklabel cs:0.5)}, anchor=near ticklabel, at={{(ticklabel cs:0.5)}}, anchor={near ticklabel}, font={{\fontsize{11 pt}{14.3 pt}\selectfont}}, color={rgb,1:red,0.0;green,0.0;blue,0.0}, draw opacity={1.0}, rotate={0.0}}, xmajorgrids={true}, xmin={-0.01654556446238531}, xmax={0.5680643798752283}, xticklabels={{0.00,0.12,0.31,0.55}}, xtick={{0.0,0.12,0.30637899,0.5514821820000002}}, xtick align={inside}, xticklabel style={font={{\fontsize{8 pt}{10.4 pt}\selectfont}}, color={rgb,1:red,0.0;green,0.0;blue,0.0}, draw opacity={1.0}, rotate={0.0}}, x grid style={color={rgb,1:red,0.0;green,0.0;blue,0.0}, draw opacity={0.1}, line width={0.5}, solid}, axis x line*={left}, x axis line style={color={rgb,1:red,0.0;green,0.0;blue,0.0}, draw opacity={1.0}, line width={1}, solid}, scaled y ticks={false}, ylabel={$r~\mathrm{(m)}$}, y tick style={color={rgb,1:red,0.0;green,0.0;blue,0.0}, opacity={1.0}}, y tick label style={color={rgb,1:red,0.0;green,0.0;blue,0.0}, opacity={1.0}, rotate={0}}, ylabel style={{rotate=-90}}, ymajorgrids={true}, ymin={-0.018897937641611184}, ymax={0.23330880956173}, yticklabels={{0.00,0.04,0.16,0.18,0.21}}, ytick={{0.0,0.044952523,0.155720815,0.17818011312626458,0.21441087192011882}}, ytick align={inside}, yticklabel style={font={{\fontsize{8 pt}{10.4 pt}\selectfont}}, color={rgb,1:red,0.0;green,0.0;blue,0.0}, draw opacity={1.0}, rotate={0.0}}, y grid style={color={rgb,1:red,0.0;green,0.0;blue,0.0}, draw opacity={0.1}, line width={0.5}, solid}, axis y line*={left}, y axis line style={color={rgb,1:red,0.0;green,0.0;blue,0.0}, draw opacity={1.0}, line width={1}, solid}, colorbar={false}]
    \addplot[color={rgb,1:red,0.502;green,0.502;blue,0.502}, name path={ad3ad9cd-3fec-45f0-b5e6-88e87c686960}, draw opacity={0.25}, line width={0.1}, solid, forget plot]
        table[row sep={\\}]
        {
            \\
            0.12  0.04495252299071941  \\
            0.12  0.05602935219164747  \\
            0.12  0.06710618139257553  \\
            0.12  0.07818301059350359  \\
            0.12  0.08925983979443165  \\
            0.12  0.10033666899535972  \\
            0.12  0.11141349819628778  \\
            0.12  0.12249032739721583  \\
            0.12  0.1335671565981439  \\
            0.12  0.14464398579907195  \\
            0.12  0.155720815  \\
        }
        ;
    \addplot[color={rgb,1:red,0.502;green,0.502;blue,0.502}, name path={68b01e25-813b-4b02-a6a7-910eb86f5f9e}, draw opacity={0.25}, line width={0.1}, solid, forget plot]
        table[row sep={\\}]
        {
            \\
            0.12615140016666665  0.04494248592934282  \\
            0.12615140019524057  0.0560404395104915  \\
            0.1261514002210339  0.06713071695180639  \\
            0.126151400241577  0.07821950267194142  \\
            0.1261514002549012  0.08930768769610237  \\
            0.12615140025972063  0.10039538250716509  \\
            0.12615140025554467  0.11148236259103259  \\
            0.12615140024271357  0.12256798715099464  \\
            0.12615140022235488  0.1336508446344345  \\
            0.1261514001962611  0.1447277691363484  \\
            0.12615140016666665  0.15579077631842816  \\
        }
        ;
    \addplot[color={rgb,1:red,0.502;green,0.502;blue,0.502}, name path={8118e2da-b4ae-4e8d-b030-3c5f03bbcb41}, draw opacity={0.25}, line width={0.1}, solid, forget plot]
        table[row sep={\\}]
        {
            \\
            0.13230280033333333  0.04494776043555422  \\
            0.1323028003913925  0.05605296683605158  \\
            0.13230280044378667  0.0671553292278602  \\
            0.13230280048548837  0.0782559880131186  \\
            0.13230280051249943  0.08935560960449251  \\
            0.13230280052221355  0.10045432517316684  \\
            0.13230280051364623  0.11155172607354069  \\
            0.13230280048750778  0.12264664646709793  \\
            0.13230280044611692  0.13373657213266293  \\
            0.132302800393164  0.14481614363284223  \\
            0.13230280033333333  0.1558732022935015  \\
        }
        ;
    \addplot[color={rgb,1:red,0.502;green,0.502;blue,0.502}, name path={96373f63-31b8-4647-9a2f-beaceca27e5c}, draw opacity={0.25}, line width={0.1}, solid, forget plot]
        table[row sep={\\}]
        {
            \\
            0.13845420049999999  0.044949764153980734  \\
            0.13845420058934013  0.05606501577684807  \\
            0.13845420066992722  0.06717967633249355  \\
            0.13845420073402734  0.07829231617435715  \\
            0.13845420077549261  0.08940353188882594  \\
            0.13845420079032575  0.10051354088230255  \\
            0.13845420077703385  0.11162185072557691  \\
            0.13845420073673972  0.12272697691396277  \\
            0.13845420067305042  0.13382579732752276  \\
            0.1384542005917003  0.14491206495073028  \\
            0.13845420049999999  0.1559729329828782  \\
        }
        ;
    \addplot[color={rgb,1:red,0.502;green,0.502;blue,0.502}, name path={3751e22c-a163-4492-98c5-d421a78bab84}, draw opacity={0.25}, line width={0.1}, solid, forget plot]
        table[row sep={\\}]
        {
            \\
            0.14460560066666667  0.044942278699579535  \\
            0.14460560079001664  0.05607536754340089  \\
            0.1446056009012248  0.06720344097092065  \\
            0.14460560098962408  0.07832831538807132  \\
            0.144605601046736  0.0894513380666458  \\
            0.14460560106706402  0.10057299701664563  \\
            0.14460560104858047  0.11169285335570084  \\
            0.14460560099287698  0.12280934155347033  \\
            0.14460560090498  0.13391924135262664  \\
            0.14460560079285847  0.1450166051869055  \\
            0.14460560066666667  0.15609102041875295  \\
        }
        ;
    \addplot[color={rgb,1:red,0.502;green,0.502;blue,0.502}, name path={3184714d-b5d8-45f1-a2fe-86396167fc4f}, draw opacity={0.25}, line width={0.1}, solid, forget plot]
        table[row sep={\\}]
        {
            \\
            0.15075700083333332  0.04493204538203043  \\
            0.1507570009944323  0.056084531918666  \\
            0.1507570011396005  0.06722652798138998  \\
            0.15075700125491626  0.07836381408209799  \\
            0.15075700132932401  0.0894988566340506  \\
            0.15075700135567988  0.10063255991642192  \\
            0.1507570013313835  0.11176466904790242  \\
            0.1507570012585672  0.12289374982477055  \\
            0.15075700114384671  0.13401687233765402  \\
            0.15075700099767186  0.14512903043874306  \\
            0.15075700083333332  0.15622265571003757  \\
        }
        ;
    \addplot[color={rgb,1:red,0.502;green,0.502;blue,0.502}, name path={47116a75-9c82-4992-b8f7-50dc81a3b762}, draw opacity={0.25}, line width={0.1}, solid, forget plot]
        table[row sep={\\}]
        {
            \\
            0.156908401  0.044923127246484104  \\
            0.15690840120370972  0.0560932491500544  \\
            0.15690840138718576  0.06724891750798433  \\
            0.15690840153282662  0.07839861751686146  \\
            0.15690840162667963  0.08954584775223029  \\
            0.15690840165976214  0.10069198275589758  \\
            0.15690840162885206  0.1118370460812192  \\
            0.15690840153671326  0.12297988993388287  \\
            0.15690840139176687  0.1341181109398657  \\
            0.15690840120725882  0.145247877297928  \\
            0.156908401  0.15636407383589124  \\
        }
        ;
    \addplot[color={rgb,1:red,0.502;green,0.502;blue,0.502}, name path={49c91940-de59-427d-84c8-75ae91518b16}, draw opacity={0.25}, line width={0.1}, solid, forget plot]
        table[row sep={\\}]
        {
            \\
            0.16305980116666666  0.04491727692121097  \\
            0.16305980141911983  0.05610185389778746  \\
            0.16305980164638673  0.06727051141803732  \\
            0.16305980182664898  0.07843247184616652  \\
            0.16305980194265315  0.08959199140521185  \\
            0.1630598019833388  0.10075090393602434  \\
            0.16305980194480232  0.11190956234905504  \\
            0.16305980183055457  0.12306719467988293  \\
            0.1630598016510932  0.13422203132650048  \\
            0.16305980142285748  0.1453714339695169  \\
            0.16305980116666666  0.15651236616271635  \\
        }
        ;
    \addplot[color={rgb,1:red,0.502;green,0.502;blue,0.502}, name path={6b0e2fcb-5ad8-461c-8a6f-51a89cddcb37}, draw opacity={0.25}, line width={0.1}, solid, forget plot]
        table[row sep={\\}]
        {
            \\
            0.16921120133333334  0.04491196172254526  \\
            0.16921120164212095  0.056110045676790836  \\
            0.16921120191995728  0.06729106515668723  \\
            0.16921120214014587  0.07846504059891733  \\
            0.1692112022816338  0.08963688056691509  \\
            0.16921120233099135  0.10080885315695454  \\
            0.16921120228356445  0.11198165270245582  \\
            0.16921120214375715  0.12315491172194172  \\
            0.1692112019244811  0.13432750486434455  \\
            0.1692112016458617  0.14549789279093822  \\
            0.16921120133333334  0.15666486336534555  \\
        }
        ;
    \addplot[color={rgb,1:red,0.502;green,0.502;blue,0.502}, name path={3703a70e-d83b-4223-a00d-7ae05d9cb1b3}, draw opacity={0.25}, line width={0.1}, solid, forget plot]
        table[row sep={\\}]
        {
            \\
            0.1753626015  0.044906620646925216  \\
            0.17536260187440036  0.05611747133702889  \\
            0.1753626022110816  0.06731023233667703  \\
            0.17536260247765992  0.07849590014122762  \\
            0.1753626026486766  0.0896800198575199  \\
            0.1753626027079842  0.10086526204915033  \\
            0.17536260265010248  0.11205263965240041  \\
            0.17536260248051416  0.12324216719187772  \\
            0.17536260221496122  0.13443330414778515  \\
            0.17536260187785932  0.14562542051196536  \\
            0.1753626015  0.15681857836445  \\
        }
        ;
    \addplot[color={rgb,1:red,0.502;green,0.502;blue,0.502}, name path={a49c7b81-dcad-43b7-a78b-09066fa71c87}, draw opacity={0.25}, line width={0.1}, solid, forget plot]
        table[row sep={\\}]
        {
            \\
            0.18151400166666667  0.04490069966154803  \\
            0.1815140021179268  0.056123773948205766  \\
            0.1815140025234716  0.06732758926788507  \\
            0.1815140028442434  0.07852453904258988  \\
            0.181514003049649  0.08972082667202388  \\
            0.1815140031204133  0.10091947637043401  \\
            0.18151400305015145  0.11212176274436124  \\
            0.1815140028456599  0.12332801937251056  \\
            0.18151400252602115  0.1345381905217773  \\
            0.18151400212067537  0.1457523440446712  \\
            0.18151400166666667  0.15697133722994003  \\
        }
        ;
    \addplot[color={rgb,1:red,0.502;green,0.502;blue,0.502}, name path={e1147850-e212-4918-ac80-e315a58a1a65}, draw opacity={0.25}, line width={0.1}, solid, forget plot]
        table[row sep={\\}]
        {
            \\
            0.18766540183333333  0.04489448945657971  \\
            0.18766540237501228  0.05612866310351651  \\
            0.18766540286148156  0.067342641873836  \\
            0.18766540324580985  0.07855035379718306  \\
            0.18766540349140065  0.08975863172261479  \\
            0.18766540357537678  0.10097076746037408  \\
            0.1876654034903738  0.11218820383875444  \\
            0.18766540324479908  0.1234114987789757  \\
            0.18766540286169772  0.1346409642309278  \\
            0.1876654023764194  0.1458771844962717  \\
            0.18766540183333333  0.15712150120938215  \\
        }
        ;
    \addplot[color={rgb,1:red,0.502;green,0.502;blue,0.502}, name path={4d544e49-f094-46ae-84c9-40633031beae}, draw opacity={0.25}, line width={0.1}, solid, forget plot]
        table[row sep={\\}]
        {
            \\
            0.193816802  0.04488823654667903  \\
            0.19381680264838852  0.05613182293510373  \\
            0.19381680323024608  0.06735480163046952  \\
            0.1938168036893129  0.07857263630941203  \\
            0.19381680398196213  0.0897926768834148  \\
            0.1938168040811733  0.10101834167588453  \\
            0.19381680397853998  0.11225110758252534  \\
            0.1938168036844555  0.12349163568811568  \\
            0.193816803226682  0.13474048468284108  \\
            0.1938168026475395  0.14599862705058791  \\
            0.193816802  0.15726777750621487  \\
        }
        ;
    \addplot[color={rgb,1:red,0.502;green,0.502;blue,0.502}, name path={68334e00-6024-480b-9134-a5166d1c9af4}, draw opacity={0.25}, line width={0.1}, solid, forget plot]
        table[row sep={\\}]
        {
            \\
            0.19996820216666666  0.04488243703899517  \\
            0.19996820294129872  0.056132848421925056  \\
            0.19996820363584517  0.0673633454997533  \\
            0.1999682041829581  0.07859055305782527  \\
            0.1999682045307765  0.08982210956312595  \\
            0.1999682046475313  0.1010613476062314  \\
            0.19996820452373618  0.11230959800557637  \\
            0.19996820417224237  0.12356747941597948  \\
            0.19996820362643988  0.13483567444425562  \\
            0.19996820293688403  0.1461154636342167  \\
            0.19996820216666666  0.15740915183117207  \\
        }
        ;
    \addplot[color={rgb,1:red,0.502;green,0.502;blue,0.502}, name path={a920762a-d336-4e1d-b66a-95ee861eb4f8}, draw opacity={0.25}, line width={0.1}, solid, forget plot]
        table[row sep={\\}]
        {
            \\
            0.20611960233333335  0.04487715711766979  \\
            0.20611960325761167  0.05613110596234895  \\
            0.20611960408550423  0.0673673598287346  \\
            0.2061196047364545  0.07860311632026916  \\
            0.20611960514897135  0.08984597375713471  \\
            0.20611960528587536  0.10109888172387062  \\
            0.20611960513660454  0.11236279307574953  \\
            0.20611960471705912  0.1236381149258848  \\
            0.2061196040673514  0.13492552158683233  \\
            0.20611960324777379  0.14622649016019462  \\
            0.20611960233333335  0.15754388480717682  \\
        }
        ;
    \addplot[color={rgb,1:red,0.502;green,0.502;blue,0.502}, name path={a82d04f0-67f4-4090-aa98-1eca4d8ef662}, draw opacity={0.25}, line width={0.1}, solid, forget plot]
        table[row sep={\\}]
        {
            \\
            0.2122710025  0.04487136962829655  \\
            0.21227100360196116  0.05612563287887272  \\
            0.21227100458783607  0.06736568162579662  \\
            0.21227100536131296  0.07860914849751283  \\
            0.21227100584967618  0.08986319860985523  \\
            0.21227100600963467  0.1011299937391038  \\
            0.21227100582962008  0.11240981908966702  \\
            0.21227100532931756  0.12370268190535638  \\
            0.21227100455687117  0.1350091095795613  \\
            0.21227100358408668  0.1463306100739647  \\
            0.2122710025  0.1576701304503246  \\
        }
        ;
    \addplot[color={rgb,1:red,0.502;green,0.502;blue,0.502}, name path={aa18d7ad-f348-4604-b748-3f4c99813c7d}, draw opacity={0.25}, line width={0.1}, solid, forget plot]
        table[row sep={\\}]
        {
            \\
            0.21842240266666665  0.04486309798935874  \\
            0.2184224039799177  0.056115209461446  \\
            0.21842240515313588  0.06735684052934798  \\
            0.21842240607120084  0.07860723787386087  \\
            0.2184224066483922  0.0898725857833963  \\
            0.218422406834599  0.10115369345270674  \\
            0.21842240661740944  0.11244982595007046  \\
            0.218422406021203  0.12376039434363861  \\
            0.21842240510371397  0.13508565582842222  \\
            0.21842240395035464  0.14642701504444167  \\
            0.21842240266666665  0.15778714679780784  \\
        }
        ;
    \addplot[color={rgb,1:red,0.502;green,0.502;blue,0.502}, name path={40837fe8-9786-49ee-a00f-4588a79151d8}, draw opacity={0.25}, line width={0.1}, solid, forget plot]
        table[row sep={\\}]
        {
            \\
            0.22457380283333334  0.04485360527005957  \\
            0.2245738043982078  0.05609862960554776  \\
            0.22457380579374378  0.06733896142861041  \\
            0.22457380688235992  0.07859567915993154  \\
            0.22457380756341977  0.08987279803444295  \\
            0.22457380777932734  0.10116896178434373  \\
            0.22457380751711667  0.11248200427707714  \\
            0.2245738068069738  0.12381055584075983  \\
            0.22457380571806682  0.13515451980748214  \\
            0.22457380435187183  0.14651516205877915  \\
            0.22457380283333334  0.15789500367297732  \\
        }
        ;
    \addplot[color={rgb,1:red,0.502;green,0.502;blue,0.502}, name path={677ee67e-fb43-4398-8e5f-25f93eb9b2c6}, draw opacity={0.25}, line width={0.1}, solid, forget plot]
        table[row sep={\\}]
        {
            \\
            0.230725203  0.0448476314166718  \\
            0.23072520486502654  0.0560742577875494  \\
            0.23072520652449496  0.06730951514369517  \\
            0.23072520781409608  0.07857239328873217  \\
            0.23072520861634768  0.08986235486443839  \\
            0.23072520886561346  0.10117477013910177  \\
            0.23072520854882148  0.11250560651005616  \\
            0.23072520770330537  0.1238525722232395  \\
            0.23072520641183372  0.1352151874130788  \\
            0.23072520479481992  0.14659461262109502  \\
            0.230725203  0.15799347431712235  \\
        }
        ;
    \addplot[color={rgb,1:red,0.502;green,0.502;blue,0.502}, name path={82768981-4a6c-46ca-b67d-d93e6c227c26}, draw opacity={0.25}, line width={0.1}, solid, forget plot]
        table[row sep={\\}]
        {
            \\
            0.23687660316666667  0.044848601431894536  \\
            0.236876605390478  0.05603861868336703  \\
            0.23687660736326796  0.06726491146826322  \\
            0.23687660888933887  0.07853484727372217  \\
            0.23687660983260314  0.08983964813226591  \\
            0.23687661011901207  0.10117011389147092  \\
            0.23687660973601576  0.11251997500326537  \\
            0.23687660872968597  0.12388596778461915  \\
            0.2368766071989195  0.13526726829449545  \\
            0.23687660528641738  0.1466649843626945  \\
            0.23687660316666667  0.1580818906263962  \\
        }
        ;
    \addplot[color={rgb,1:red,0.502;green,0.502;blue,0.502}, name path={1b755dd0-59ba-4d0b-8210-1371c1d9368e}, draw opacity={0.25}, line width={0.1}, solid, forget plot]
        table[row sep={\\}]
        {
            \\
            0.24302800333333333  0.04485107224020261  \\
            0.24302800598714014  0.055984546650524734  \\
            0.2430280083316136  0.0672000915347298  \\
            0.24302801013525505  0.07848003044705751  \\
            0.24302801124205642  0.08980299748430785  \\
            0.2430280115694266  0.10115406627813653  \\
            0.243028011106145  0.11252457917280535  \\
            0.2430280099088706  0.12391040769746414  \\
            0.24302800809555822  0.13531051547766182  \\
            0.24302800583509376  0.14672603705782658  \\
            0.24302800333333333  0.15815975337140403  \\
        }
        ;
    \addplot[color={rgb,1:red,0.502;green,0.502;blue,0.502}, name path={3136a9f1-3e60-4462-88ad-ed1b0ade82c3}, draw opacity={0.25}, line width={0.1}, solid, forget plot]
        table[row sep={\\}]
        {
            \\
            0.2491794035  0.04483645969082626  \\
            0.24917940667070462  0.05589987935954851  \\
            0.24917940945540965  0.06710837261316725  \\
            0.24917941158388188  0.07840456300083282  \\
            0.2491794128796673  0.08975076671054721  \\
            0.24917941325176104  0.10112585713526756  \\
            0.24917941269122246  0.11251906151488966  \\
            0.24917941126740398  0.12392572336658478  \\
            0.24917940912069386  0.13534484638930663  \\
            0.24917940645069037  0.14677775336237048  \\
            0.2491794035  0.15822728314697948  \\
        }
        ;
    \addplot[color={rgb,1:red,0.502;green,0.502;blue,0.502}, name path={b6387c91-1e0d-4702-b5df-9ab4c1731c4a}, draw opacity={0.25}, line width={0.1}, solid, forget plot]
        table[row sep={\\}]
        {
            \\
            0.25533080366666666  0.044774484794636006  \\
            0.25533080746060915  0.055767078407355675  \\
            0.25533081076544345  0.06698175664728288  \\
            0.2553308132727256  0.07830500208116842  \\
            0.25533081478615677  0.08968155388848503  \\
            0.2553308152066385  0.10108497604801062  \\
            0.2553308145285274  0.11250328981595437  \\
            0.2553308128362228  0.12393193634922521  \\
            0.25533081029642163  0.13537034567568784  \\
            0.2553308071446905  0.14682028588402926  \\
            0.25533080366666666  0.15828503040023142  \\
        }
        ;
    \addplot[color={rgb,1:red,0.502;green,0.502;blue,0.502}, name path={448d932f-5910-44f5-bf92-7e192fe972f8}, draw opacity={0.25}, line width={0.1}, solid, forget plot]
        table[row sep={\\}]
        {
            \\
            0.2614822038333333  0.044592370393489775  \\
            0.2614822083805179  0.055562800352296375  \\
            0.2614822122977724  0.06681186401686831  \\
            0.26148221524526294  0.07817837794098743  \\
            0.26148221700869273  0.0895944503248274  \\
            0.26148221748119715  0.10103129187374736  \\
            0.2614822166614041  0.11247741035986174  \\
            0.26148221465135507  0.12392927958575795  \\
            0.2614822116485037  0.13538726081860353  \\
            0.26148220793049315  0.1468538547638709  \\
            0.2614822038333333  0.15833308352046074  \\
        }
        ;
    \addplot[color={rgb,1:red,0.502;green,0.502;blue,0.502}, name path={2c48ce0c-f2a9-409e-8d72-c810913662b4}, draw opacity={0.25}, line width={0.1}, solid, forget plot]
        table[row sep={\\}]
        {
            \\
            0.267633604  0.04421530669424536  \\
            0.26763360945783743  0.055265286576094926  \\
            0.2676336140936322  0.06659199933255601  \\
            0.26763361755129433  0.0780229410851018  \\
            0.2676336196015953  0.0894893370174616  \\
            0.2676336201299862  0.10096517151955674  \\
            0.26763361914018513  0.11244189405263647  \\
            0.267633616754736  0.12391821512694683  \\
            0.2676336132069755  0.13539601336710996  \\
            0.26763360882374065  0.146878777620109  \\
            0.267633604  0.1583712647951522  \\
        }
        ;
    \addplot[color={rgb,1:red,0.502;green,0.502;blue,0.502}, name path={6783c59f-b682-4b65-bde9-8285fa0ade01}, draw opacity={0.25}, line width={0.1}, solid, forget plot]
        table[row sep={\\}]
        {
            \\
            0.2737850041666667  0.04362179231228574  \\
            0.2737850107226667  0.054863194210018736  \\
            0.27378501619896545  0.06631938676320463  \\
            0.27378502024718865  0.07783891230973548  \\
            0.2737850226271058  0.08936715393102769  \\
            0.27378502321600184  0.10088757302268427  \\
            0.2737850220232743  0.11239756478204808  \\
            0.27378501919516685  0.12389944276158213  \\
            0.2737850150068578  0.13539721850969472  \\
            0.27378500984270293  0.14689558885228318  \\
            0.2737850041666667  0.15839986466358472  \\
        }
        ;
    \addplot[color={rgb,1:red,0.502;green,0.502;blue,0.502}, name path={3c89ab1e-2669-4fa0-8adf-b3704b1ce81d}, draw opacity={0.25}, line width={0.1}, solid, forget plot]
        table[row sep={\\}]
        {
            \\
            0.27993640433333333  0.04280470972471024  \\
            0.2799364122072503  0.054354784921220145  \\
            0.27993641866383007  0.06599624378782942  \\
            0.27993642339614044  0.07762899205874955  \\
            0.27993642615630765  0.0892300705468901  \\
            0.2799364268119306  0.10080008660836114  \\
            0.2799364253784347  0.11234559906099814  \\
            0.27993642202944796  0.12387389111256711  \\
            0.27993641708899336  0.1353916860143239  \\
            0.27993641100872163  0.1469050910532755  \\
            0.27993640433333333  0.15841995001612175  \\
        }
        ;
    \addplot[color={rgb,1:red,0.502;green,0.502;blue,0.502}, name path={857743c6-760b-4c32-bb4e-080d6c2d3151}, draw opacity={0.25}, line width={0.1}, solid, forget plot]
        table[row sep={\\}]
        {
            \\
            0.2860878045  0.04173197487913953  \\
            0.28608781394460575  0.05374561692252722  \\
            0.28608782154156026  0.06563029646824577  \\
            0.28608782706858  0.07739859238088641  \\
            0.2860878302703292  0.08908151204073678  \\
            0.2860878310016841  0.10070490343691223  \\
            0.2860878292843385  0.11228748906818595  \\
            0.28608782532372357  0.12384268624056223  \\
            0.2860878195010295  0.13538039027131413  \\
            0.28608781234672537  0.14690827504726234  \\
            0.2860878045  0.15843280185958125  \\
        }
        ;
    \addplot[color={rgb,1:red,0.502;green,0.502;blue,0.502}, name path={e2ded9c9-8a1d-43ab-b6e5-6bdd685e94ba}, draw opacity={0.25}, line width={0.1}, solid, forget plot]
        table[row sep={\\}]
        {
            \\
            0.2922392046666667  0.040364235789079016  \\
            0.2922392159645786  0.05305301859198817  \\
            0.2922392248876448  0.06523561804303711  \\
            0.2922392313429909  0.07715580666129163  \\
            0.2922392350620107  0.08892601151667356  \\
            0.2922392358823359  0.10060470052744373  \\
            0.29223923383244643  0.11222496656167623  \\
            0.2922392291550817  0.12380709913175461  \\
            0.2922392222985777  0.13536442502857887  \\
            0.29223921388583923  0.14690623885046122  \\
            0.2922392046666667  0.15843959620544462  \\
        }
        ;
    \addplot[color={rgb,1:red,0.502;green,0.502;blue,0.502}, name path={993e17ec-266d-4a4a-a145-ab4ad8ed7010}, draw opacity={0.25}, line width={0.1}, solid, forget plot]
        table[row sep={\\}]
        {
            \\
            0.29839060483333335  0.038669523760712386  \\
            0.29839061828781116  0.05231414428199146  \\
            0.2983906287592233  0.06483332747820882  \\
            0.2983906363076548  0.07691097355749534  \\
            0.2983906406382771  0.08876885622063425  \\
            0.2983906415665778  0.1005024406188454  \\
            0.29839063912928604  0.11215989152137111  \\
            0.29839063361345675  0.12376847642707164  \\
            0.29839062554657736  0.13534495550240616  \\
            0.29839061566010217  0.146900166876214  \\
            0.29839060483333335  0.1584414626579145  \\
        }
        ;
    \addplot[color={rgb,1:red,0.502;green,0.502;blue,0.502}, name path={c64fb68f-89b3-4ff9-9e8f-98c0e06b812c}, draw opacity={0.25}, line width={0.1}, solid, forget plot]
        table[row sep={\\}]
        {
            \\
            0.304542005  0.036613858106809886  \\
            0.3045420209191786  0.051595405041978036  \\
            0.3045420332171931  0.06445101021750609  \\
            0.3045420420640752  0.0766755753846573  \\
            0.3045420471234639  0.08861550954242504  \\
            0.3045420481858039  0.10040110174599662  \\
            0.30454204529919815  0.11209411675340528  \\
            0.3045420388038831  0.12372815935402574  \\
            0.30454202932089514  0.13532316547991338  \\
            0.30454201770929595  0.14689135308812828  \\
            0.304542005  0.1584389348467872  \\
        }
        ;
    \addplot[color={rgb,1:red,0.502;green,0.502;blue,0.502}, name path={b3378eb0-d604-49d6-98c6-1b2d24f593c6}, draw opacity={0.25}, line width={0.1}, solid, forget plot]
        table[row sep={\\}]
        {
            \\
            0.30637899  0.03592799979999999  \\
            0.3063790067118232  0.051406599548768375  \\
            0.30637901962687536  0.0643463473441166  \\
            0.30637902891051905  0.0766090956582217  \\
            0.3063790342150497  0.08857138189245868  \\
            0.3063790353238422  0.10037153389437943  \\
            0.3063790322893496  0.11207466773029352  \\
            0.30637902546977697  0.12371603274379787  \\
            0.3063790155173971  0.13531642747556027  \\
            0.30637900333372864  0.14688851094714875  \\
            0.30637899  0.1584389348467872  \\
        }
        ;
    \addplot[color={rgb,1:red,0.502;green,0.502;blue,0.502}, name path={04b389c3-17bf-46e1-82cb-2ae3e5fabd8f}, draw opacity={0.25}, line width={0.1}, solid, forget plot]
        table[row sep={\\}]
        {
            \\
            0.3145490964  0.03592799979999999  \\
            0.31454911697933774  0.050842252216885195  \\
            0.3145491330389462  0.0639477319598452  \\
            0.31454914455114974  0.07633836320599678  \\
            0.3145491511093207  0.08838596984491395  \\
            0.31454915246455956  0.10024473922294365  \\
            0.3145491486937438  0.11198976551337624  \\
            0.3145491402472744  0.1236619888789181  \\
            0.3145491279358046  0.1352854868089334  \\
            0.31454911287516696  0.1468749285743424  \\
            0.3145490964  0.1584389348467872  \\
        }
        ;
    \addplot[color={rgb,1:red,0.502;green,0.502;blue,0.502}, name path={c9cdde0d-457b-42da-823a-00e3a175686a}, draw opacity={0.25}, line width={0.1}, solid, forget plot]
        table[row sep={\\}]
        {
            \\
            0.3227192028  0.03592799979999999  \\
            0.3227192280346334  0.050519563370960065  \\
            0.3227192479595764  0.06365532593896576  \\
            0.322719262253222  0.0761163200086327  \\
            0.32271927038915565  0.08822441201016228  \\
            0.32271927207158807  0.10012985818882085  \\
            0.3227192674084074  0.11191047652297306  \\
            0.32271925696575005  0.12361005525695089  \\
            0.32271924175138167  0.13525480603821718  \\
            0.3227192231467283  0.14686102249543315  \\
            0.3227192028  0.1584389348467872  \\
        }
        ;
    \addplot[color={rgb,1:red,0.502;green,0.502;blue,0.502}, name path={4c1cbfc0-e7a0-4a04-b2c0-0096fea1dad7}, draw opacity={0.25}, line width={0.1}, solid, forget plot]
        table[row sep={\\}]
        {
            \\
            0.33088930920000004  0.03592799979999999  \\
            0.33088934012977145  0.05032389511663432  \\
            0.3308893647732326  0.06344555413950083  \\
            0.3308893825008494  0.07594047385431368  \\
            0.33088939260291655  0.08808838882812306  \\
            0.3308893947096599  0.10002906445335612  \\
            0.330889388962727  0.111838736376796  \\
            0.3308893760706582  0.12356184175324728  \\
            0.33088935728515295  0.13522565408360918  \\
            0.3308893343162264  0.14684755481245004  \\
            0.33088930920000004  0.1584389348467872  \\
        }
        ;
    \addplot[color={rgb,1:red,0.502;green,0.502;blue,0.502}, name path={cc5df659-2cb9-413a-87d8-5e3a750481a7}, draw opacity={0.25}, line width={0.1}, solid, forget plot]
        table[row sep={\\}]
        {
            \\
            0.3390594156  0.03592799979999999  \\
            0.3390594535331789  0.050198985415511384  \\
            0.3390594839471249  0.06329508780318066  \\
            0.3390595058962311  0.07580368495588057  \\
            0.3390595184317064  0.08797648907387864  \\
            0.3390595210777388  0.0999427782695948  \\
            0.3390595140104026  0.11177546452623631  \\
            0.33905949811136643  0.12351830759697713  \\
            0.33905947493279903  0.1351988294755506  \\
            0.3390594465904846  0.14683499422035584  \\
            0.3390594156  0.1584389348467872  \\
        }
        ;
    \addplot[color={rgb,1:red,0.502;green,0.502;blue,0.502}, name path={c1f550f6-1092-4dbd-b75f-e3e610afb022}, draw opacity={0.25}, line width={0.1}, solid, forget plot]
        table[row sep={\\}]
        {
            \\
            0.347229522  0.03592799979999999  \\
            0.347229568558387  0.0501157010556521  \\
            0.34722960604725867  0.06318620669925075  \\
            0.34722963318103195  0.07569809149872803  \\
            0.34722964871705686  0.08788580478010745  \\
            0.3472296520389769  0.09987025383397206  \\
            0.347229643357913  0.11172078627973303  \\
            0.3472296237651819  0.12347987138476386  \\
            0.34722959518198687  0.13517475994558956  \\
            0.34722956022439017  0.146823603475183  \\
            0.347229522  0.1584389348467872  \\
        }
        ;
    \addplot[color={rgb,1:red,0.502;green,0.502;blue,0.502}, name path={38ed5571-cc71-45bb-a53c-f546caf2e636}, draw opacity={0.25}, line width={0.1}, solid, forget plot]
        table[row sep={\\}]
        {
            \\
            0.3553996284  0.03592799979999999  \\
            0.3553996855847916  0.05005814756146559  \\
            0.3553997317635693  0.06310646525569791  \\
            0.35539976526649114  0.075616723614403  \\
            0.3553997844959765  0.0878129941358577  \\
            0.3553997886575703  0.09981011168845659  \\
            0.35539977799939276  0.11167428840052289  \\
            0.355399753866825  0.12344654795511546  \\
            0.3553997186336654  0.13515359488510492  \\
            0.35539967553204227  0.14681349801125748  \\
            0.3553996284  0.1584389348467872  \\
        }
        ;
    \addplot[color={rgb,1:red,0.502;green,0.502;blue,0.502}, name path={f364a5a1-0b38-4861-ba07-a8d01c85a3f2}, draw opacity={0.25}, line width={0.1}, solid, forget plot]
        table[row sep={\\}]
        {
            \\
            0.3635697348  0.03592799979999999  \\
            0.3635698050772228  0.05001720842195701  \\
            0.3635698619412591  0.06304732859525827  \\
            0.3635699032723407  0.07555393145493997  \\
            0.36356992704499996  0.08775485321077865  \\
            0.36356993224432615  0.09976072362718605  \\
            0.3635699191595374  0.11163524379598788  \\
            0.3635698894446839  0.12341808044642985  \\
            0.3635698460282941  0.13513528773234879  \\
            0.3635697929005032  0.14680469016934244  \\
            0.3635697348  0.1584389348467872  \\
        }
        ;
    \addplot[color={rgb,1:red,0.502;green,0.502;blue,0.502}, name path={cde8446c-d92b-4e66-9ba6-81ee4fb00cde}, draw opacity={0.25}, line width={0.1}, solid, forget plot]
        table[row sep={\\}]
        {
            \\
            0.37173984120000003  0.03592799979999999  \\
            0.37173992760768426  0.049987405550129586  \\
            0.371739997618834  0.06300295281505029  \\
            0.3717400485754441  0.07550532423590893  \\
            0.37174007793497926  0.0877085635960768  \\
            0.3717400844129372  0.09972045244561493  \\
            0.3717400683464427  0.1116027812957342  \\
            0.3717400317654777  0.12339405088118224  \\
            0.37173997827820915  0.13511966502202966  \\
            0.37173991280679575  0.14679712389736138  \\
            0.37173984120000003  0.1584389348467872  \\
        }
        ;
    \addplot[color={rgb,1:red,0.502;green,0.502;blue,0.502}, name path={edddf818-5282-486a-bf46-b45be318bf4a}, draw opacity={0.25}, line width={0.1}, solid, forget plot]
        table[row sep={\\}]
        {
            \\
            0.3799099476  0.03592799979999999  \\
            0.3799100538813233  0.04996530544721607  \\
            0.3799101400745015  0.0629693016722311  \\
            0.3799102028699015  0.07546755580465006  \\
            0.3799102390987651  0.08767175970237416  \\
            0.3799102471492997  0.099687781838861  \\
            0.37991022741664443  0.11157600045120916  \\
            0.37991018238932317  0.12337396392155768  \\
            0.3799101165076361  0.1351064806361067  \\
            0.3799100358389651  0.1467907016420016  \\
            0.3799099476  0.1584389348467872  \\
        }
        ;
    \addplot[color={rgb,1:red,0.502;green,0.502;blue,0.502}, name path={e21fc0bd-597c-48ad-9aed-867030d54ca3}, draw opacity={0.25}, line width={0.1}, solid, forget plot]
        table[row sep={\\}]
        {
            \\
            0.388080054  0.03592799979999999  \\
            0.3880801847683727  0.049948674250106025  \\
            0.3880802908832295  0.06294355011247844  \\
            0.388080368241107  0.07543809536494703  \\
            0.3880804129144707  0.08764250885603442  \\
            0.388080422896663  0.09966137405082315  \\
            0.38808039865509203  0.1115540415055056  \\
            0.38808034323767204  0.12335730516822024  \\
            0.38808026210226687  0.1350954557794504  \\
            0.38808016272226664  0.1467853044195057  \\
            0.388080054  0.1584389348467872  \\
        }
        ;
    \addplot[color={rgb,1:red,0.502;green,0.502;blue,0.502}, name path={bdb8a1e5-e942-49c8-8d85-85f3fb9b6a32}, draw opacity={0.25}, line width={0.1}, solid, forget plot]
        table[row sep={\\}]
        {
            \\
            0.3962501604  0.03592799979999999  \\
            0.3962503213439879  0.04993601057118275  \\
            0.39625045198738607  0.06292369096447809  \\
            0.39625054725697945  0.07541503060303202  \\
            0.3962506023076106  0.08761925659644776  \\
            0.39625061465992667  0.0996400846401548  \\
            0.396250584873342  0.11153612285570005  \\
            0.3962505166771905  0.1233435784347396  \\
            0.3962504167708732  0.1350863065436754  \\
            0.3962502943518805  0.14678080605369023  \\
            0.3962501604  0.1584389348467872  \\
        }
        ;
    \addplot[color={rgb,1:red,0.502;green,0.502;blue,0.502}, name path={e3505ab6-19aa-46a9-a74c-b56f7d3df854}, draw opacity={0.25}, line width={0.1}, solid, forget plot]
        table[row sep={\\}]
        {
            \\
            0.4044202668  0.03592799979999999  \\
            0.4044204649384488  0.04992627670438774  \\
            0.4044206257847325  0.0629082765068913  \\
            0.40442074308041226  0.07539691264131396  \\
            0.4044208108760499  0.08760076370743183  \\
            0.40442082613295866  0.09962295424492405  \\
            0.40442078952990634  0.1115215568622359  \\
            0.40442070562344534  0.12333232717257825  \\
            0.4044205826222216  0.13507876151039622  \\
            0.40442043183406434  0.14677708266194575  \\
            0.4044202668  0.1584389348467872  \\
        }
        ;
    \addplot[color={rgb,1:red,0.502;green,0.502;blue,0.502}, name path={52f80ba1-e70d-445d-a333-94898050638f}, draw opacity={0.25}, line width={0.1}, solid, forget plot]
        table[row sep={\\}]
        {
            \\
            0.4125903732  0.03592799979999999  \\
            0.41259061720114876  0.04991873810880169  \\
            0.41259081523876096  0.06289624752843993  \\
            0.41259095960797343  0.07538263871295257  \\
            0.41259104304231786  0.08758604694108861  \\
            0.4125910618533038  0.09960918963998613  \\
            0.41259101687739697  0.11150975228942685  \\
            0.4125909136693054  0.12332314476826371  \\
            0.41259076226170527  0.13507257187209598  \\
            0.41259057653843373  0.14677401847132818  \\
            0.4125903732  0.1584389348467872  \\
        }
        ;
    \addplot[color={rgb,1:red,0.502;green,0.502;blue,0.502}, name path={8adb4671-701b-405f-bfe5-3c27800f4243}, draw opacity={0.25}, line width={0.1}, solid, forget plot]
        table[row sep={\\}]
        {
            \\
            0.4207604796  0.03592799979999999  \\
            0.4207607801833836  0.049912864219275314  \\
            0.42076102401820403  0.06288681890780116  \\
            0.42076120164109976  0.0753713649848897  \\
            0.42076130423833435  0.08757432784736127  \\
            0.4207613273889434  0.09959814115254509  \\
            0.42076127214181636  0.11150020917153226  \\
            0.4207611452443796  0.12331567752023932  \\
            0.4207609589138673  0.13506751625594868  \\
            0.4207607301653186  0.14677150894740648  \\
            0.4207604796  0.1584389348467872  \\
        }
        ;
    \addplot[color={rgb,1:red,0.502;green,0.502;blue,0.502}, name path={da3eb792-5934-4de1-81c5-441a20a1bf7a}, draw opacity={0.25}, line width={0.1}, solid, forget plot]
        table[row sep={\\}]
        {
            \\
            0.428930586  0.03592799979999999  \\
            0.42893095644763335  0.04990826521558875  \\
            0.42893125667534415  0.06287940190572144  \\
            0.4289314750975138  0.07536244221308494  \\
            0.4289316011277286  0.08756499052624503  \\
            0.4289316295615548  0.09958928019716523  \\
            0.4289315617399195  0.11149250990548899  \\
            0.42893140581376  0.12330962313150153  \\
            0.4289311765797538  0.13506340201414993  \\
            0.4289308948342452  0.14676946205934605  \\
            0.428930586  0.1584389348467872  \\
        }
        ;
    \addplot[color={rgb,1:red,0.502;green,0.502;blue,0.502}, name path={bceec131-b58e-4342-a869-e054fed86a34}, draw opacity={0.25}, line width={0.1}, solid, forget plot]
        table[row sep={\\}]
        {
            \\
            0.4371006924  0.03592799979999999  \\
            0.43710114921569976  0.04990465056326204  \\
            0.43710151887713694  0.06287355056068024  \\
            0.43710178727237936  0.07535536825558958  \\
            0.43710194187024826  0.0875575475904319  \\
            0.43710197670947615  0.09958217872865896  \\
            0.4371018935407276  0.11148630892654979  \\
            0.4371017021270834  0.12330472670796604  \\
            0.43710142024252935  0.13506006431355577  \\
            0.4371010732032426  0.1467677983278903  \\
            0.4371006924  0.1584389348467872  \\
        }
        ;
    \addplot[color={rgb,1:red,0.502;green,0.502;blue,0.502}, name path={ad0f599e-8562-4338-8d44-14f2507f9cb4}, draw opacity={0.25}, line width={0.1}, solid, forget plot]
        table[row sep={\\}]
        {
            \\
            0.4452707988  0.03592799979999999  \\
            0.4452713625764755  0.04990180113190044  \\
            0.4452718177100991  0.06286892412607183  \\
            0.4452721471607802  0.07534975289929793  \\
            0.4452723364304007  0.08755161311240367  \\
            0.44527237899040106  0.09957649130687644  \\
            0.4452722771764445  0.11148132234100594  \\
            0.44527204253287295  0.1233007755881061  \\
            0.4452716961422626  0.13505736397892873  \\
            0.44527126863525707  0.14676645013696812  \\
            0.4452707988  0.1584389348467872  \\
        }
        ;
    \addplot[color={rgb,1:red,0.502;green,0.502;blue,0.502}, name path={d522b895-6b44-4368-9d9b-5f4a03a6d336}, draw opacity={0.25}, line width={0.1}, solid, forget plot]
        table[row sep={\\}]
        {
            \\
            0.4534409052  0.03592799979999999  \\
            0.4534416017898208  0.04989955001406339  \\
            0.45344216209086075  0.06286526034495345  \\
            0.45344256585529447  0.07534529165925995  \\
            0.45344279692682216  0.08754688127810589  \\
            0.4534428487172092  0.09957193988220653  \\
            0.453442724405047  0.11147731825425801  \\
            0.4534424373788569  0.12329759384162746  \\
            0.4534420121533931  0.1350551847400687  \\
            0.4534414854402353  0.14676536064905302  \\
            0.4534409052  0.1584389348467872  \\
        }
        ;
    \addplot[color={rgb,1:red,0.502;green,0.502;blue,0.502}, name path={85546cd7-7375-4062-b0a0-857efa55437f}, draw opacity={0.25}, line width={0.1}, solid, forget plot]
        table[row sep={\\}]
        {
            \\
            0.4616110116  0.03592799979999999  \\
            0.4616118737533524  0.049897769060290524  \\
            0.4616125633309914  0.0628623561618278  \\
            0.4616130570344347  0.0753417461325479  \\
            0.46161333800683757  0.08754310960262762  \\
            0.46161340070814477  0.09956830112584683  \\
            0.4616132495192633  0.11147410813803929  \\
            0.46161289952734985  0.12329503693436912  \\
            0.4616123783195899  0.1350534303037881  \\
            0.4616117292452863  0.14676448255345573  \\
            0.4616110116  0.1584389348467872  \\
        }
        ;
    \addplot[color={rgb,1:red,0.502;green,0.502;blue,0.502}, name path={a2031982-e16c-4bd7-aba5-e38ec38e3a34}, draw opacity={0.25}, line width={0.1}, solid, forget plot]
        table[row sep={\\}]
        {
            \\
            0.469781118  0.03592799979999999  \\
            0.4697821877598345  0.049896359256787245  \\
            0.4697830359278678  0.06286005361862049  \\
            0.4697836375593605  0.07533892916690509  \\
            0.4697839772059938  0.0875401057510919  \\
            0.46978405261252476  0.09956539601100546  \\
            0.469783869779121  0.11147153934678937  \\
            0.46978344502730296  0.12329298682876325  \\
            0.4697828076354344  0.13505202150706594  \\
            0.46978200759368277  0.14676377679387123  \\
            0.469781118  0.1584389348467872  \\
        }
        ;
    \addplot[color={rgb,1:red,0.502;green,0.502;blue,0.502}, name path={3b51037e-1953-4aa7-9710-658e431c402b}, draw opacity={0.25}, line width={0.1}, solid, forget plot]
        table[row sep={\\}]
        {
            \\
            0.4779512244  0.03592799979999999  \\
            0.4779525568001732  0.04989524374163353  \\
            0.47795359867673637  0.06285822942069735  \\
            0.4779543281998854  0.07533669359232714  \\
            0.47795473520100135  0.08753771719333776  \\
            0.47795482514627424  0.09956308132165273  \\
            0.4779546058046166  0.11146948876243899  \\
            0.47795409400497524  0.12329134764296687  \\
            0.47795331721670903  0.13505089369454465  \\
            0.4779523309766128  0.14676321136241535  \\
            0.4779512244  0.1584389348467872  \\
        }
        ;
    \addplot[color={rgb,1:red,0.502;green,0.502;blue,0.502}, name path={76e655cf-6139-4619-b9ec-613694cea9e7}, draw opacity={0.25}, line width={0.1}, solid, forget plot]
        table[row sep={\\}]
        {
            \\
            0.4861213308  0.03592799979999999  \\
            0.4861229999453529  0.04989436266866223  \\
            0.48612427618904097  0.06285678714039036  \\
            0.4861251545321778  0.07533492361160196  \\
            0.48612563575046147  0.08753582308273485  \\
            0.4861257421556427  0.09956124277961106  \\
            0.4861254817658055  0.11146785748017712  \\
            0.4861248718771444  0.12329004190227692  \\
            0.48612393006187904  0.13504999438898  \\
            0.4861227147347506  0.14676276020703213  \\
            0.4861213308  0.1584389348467872  \\
        }
        ;
    \addplot[color={rgb,1:red,0.502;green,0.502;blue,0.502}, name path={c3437aa4-e56c-45f6-92bd-09da729a11a2}, draw opacity={0.25}, line width={0.1}, solid, forget plot]
        table[row sep={\\}]
        {
            \\
            0.4942914372  0.03592799979999999  \\
            0.49429354697824773  0.0498936693892024  \\
            0.49429510072870425  0.06285565134393827  \\
            0.49429614816331185  0.07533352819344553  \\
            0.4942967048348483  0.08753432788268865  \\
            0.494296830506958  0.09955978951408254  \\
            0.4942965249431467  0.1114665664195881  \\
            0.4942958111129259  0.12328900736500872  \\
            0.4942946776020239  0.13504928127644167  \\
            0.494293182827893  0.1467624022727971  \\
            0.4942914372  0.1584389348467872  \\
        }
        ;
    \addplot[color={rgb,1:red,0.502;green,0.502;blue,0.502}, name path={474dc2d1-d2d3-45c0-b007-32e301958bad}, draw opacity={0.25}, line width={0.1}, solid, forget plot]
        table[row sep={\\}]
        {
            \\
            0.5024615436  0.03592799979999999  \\
            0.5024642479774059  0.0498931275903585  \\
            0.5024661134932674  0.06285476314142364  \\
            0.5024673489413295  0.07533243598866415  \\
            0.5024679677106296  0.08753315637090663  \\
            0.5024681203045294  0.09955864963716243  \\
            0.5024677634546381  0.11146555273962629  \\
            0.5024669542191273  0.12328819437833258  \\
            0.5024656028282337  0.13504872050046546  \\
            0.5024637759106771  0.14676212068132216  \\
            0.5024615436  0.1584389348467872  \\
        }
        ;
    \addplot[color={rgb,1:red,0.502;green,0.502;blue,0.502}, name path={18ab6172-2082-409e-9262-e0d437a8711c}, draw opacity={0.25}, line width={0.1}, solid, forget plot]
        table[row sep={\\}]
        {
            \\
            0.51063165  0.03592799979999999  \\
            0.5106351944033974  0.049892709139502826  \\
            0.510637361592821  0.06285407680492099  \\
            0.5106388108383605  0.07533159141680935  \\
            0.5106394400997473  0.08753224973544467  \\
            0.510639648071418  0.09955776672796887  \\
            0.5106392184697081  0.11146476694217285  \\
            0.5106383603992285  0.123287563708417  \\
            0.5106367616621093  0.13504828524465456  \\
            0.5106345700757328  0.1467619020430869  \\
            0.51063165  0.1584389348467872  \\
        }
        ;
    \addplot[color={rgb,1:red,0.502;green,0.502;blue,0.502}, name path={b833bbef-9452-43c4-a540-a3d453ed6c85}, draw opacity={0.25}, line width={0.1}, solid, forget plot]
        table[row sep={\\}]
        {
            \\
            0.5188017564  0.03592799979999999  \\
            0.5188065683278625  0.04989239245934904  \\
            0.518808877237784  0.06285355720098837  \\
            0.518810621879253  0.07533095166289289  \\
            0.5188111013820477  0.08753156254240115  \\
            0.5188114735956995  0.09955709706379755  \\
            0.5188108788505589  0.11146417056132538  \\
            0.5188101253521777  0.12328708478744244  \\
            0.5188082115728961  0.1350479545769144  \\
            0.5188057239001983  0.1467617358927971  \\
            0.5188017564  0.1584389348467872  \\
        }
        ;
    \addplot[color={rgb,1:red,0.502;green,0.502;blue,0.502}, name path={4e772676-22e7-471b-910f-5c108ef40358}, draw opacity={0.25}, line width={0.1}, solid, forget plot]
        table[row sep={\\}]
        {
            \\
            0.5269718628  0.03592799979999999  \\
            0.5269787639157566  0.04989216130947543  \\
            0.5269805936959953  0.06285317785468095  \\
            0.5269829765002086  0.07533048439138054  \\
            0.5269828086779716  0.08753106040674549  \\
            0.5269837504807874  0.09955660746829151  \\
            0.5269826167802892  0.11146373435046013  \\
            0.526982450866389  0.12328673432366069  \\
            0.5269799439722115  0.13504771252740955  \\
            0.5269776053049885  0.14676161423624165  \\
            0.5269718628  0.1584389348467872  \\
        }
        ;
    \addplot[color={rgb,1:red,0.502;green,0.502;blue,0.502}, name path={2f10a5af-4ceb-431d-a181-95b42b26c426}, draw opacity={0.25}, line width={0.1}, solid, forget plot]
        table[row sep={\\}]
        {
            \\
            0.5351419692  0.03592799979999999  \\
            0.5351527026038095  0.049892003887225256  \\
            0.5351520492424573  0.06285291951215571  \\
            0.5351564383917821  0.07533016603620829  \\
            0.535154008691445  0.08753071823734165  \\
            0.5351569956825083  0.09955627367509137  \\
            0.5351538979259829  0.11146343689352065  \\
            0.5351559041316796  0.1232864952278606  \\
            0.5351515995595778  0.13504754737373442  \\
            0.5351511552596466  0.1467615311978387  \\
            0.5351419692  0.1584389348467872  \\
        }
        ;
    \addplot[color={rgb,1:red,0.502;green,0.502;blue,0.502}, name path={be1b0f73-eb08-417b-9c94-be84b39fa6c2}, draw opacity={0.25}, line width={0.1}, solid, forget plot]
        table[row sep={\\}]
        {
            \\
            0.5433120755999999  0.03592799979999999  \\
            0.5433306877795506  0.049891912191107786  \\
            0.5433213979586135  0.06285276909886658  \\
            0.543332881422732  0.07532998057230542  \\
            0.543322743829804  0.08753051895060487  \\
            0.5433330799713588  0.09955607913807567  \\
            0.5433227613075436  0.11146326357384015  \\
            0.5433323929047321  0.12328635582365197  \\
            0.5433213891694104  0.13504745110549474  \\
            0.5433289841335676  0.1467614827621099  \\
            0.5433120755999999  0.1584389348467872  \\
        }
        ;
    \addplot[color={rgb,1:red,0.502;green,0.502;blue,0.502}, name path={cadb7ee6-4a5c-4a17-b080-3dfd1e177609}, draw opacity={0.25}, line width={0.1}, solid, forget plot]
        table[row sep={\\}]
        {
            \\
            0.5514821820000001  0.03592799979999999  \\
            0.5515188154128429  0.049891881625735275  \\
            0.5514821820000002  0.06285271896110357  \\
            0.5515188154128428  0.07532991875100449  \\
            0.5514821820000001  0.08753045252169261  \\
            0.5515188154128428  0.09955601429240378  \\
            0.5514821820000001  0.11146320580061327  \\
            0.551518815412843  0.12328630935558241  \\
            0.5514821820000002  0.1350474190160815  \\
            0.551518815412843  0.14676146661686695  \\
            0.5514821820000002  0.1584389348467872  \\
        }
        ;
    \addplot[color={rgb,1:red,0.502;green,0.502;blue,0.502}, name path={b37cce2b-a741-46f1-b47b-1ebed49edd16}, draw opacity={1.0}, line width={0.75}, solid, forget plot]
        table[row sep={\\}]
        {
            \\
            0.12  0.04495252299071941  \\
            0.12615140016666665  0.04494248592934282  \\
            0.13230280033333333  0.04494776043555422  \\
            0.13845420049999999  0.044949764153980734  \\
            0.14460560066666667  0.044942278699579535  \\
            0.15075700083333332  0.04493204538203043  \\
            0.156908401  0.044923127246484104  \\
            0.16305980116666666  0.04491727692121097  \\
            0.16921120133333334  0.04491196172254526  \\
            0.1753626015  0.044906620646925216  \\
            0.18151400166666667  0.04490069966154803  \\
            0.18766540183333333  0.04489448945657971  \\
            0.193816802  0.04488823654667903  \\
            0.19996820216666666  0.04488243703899517  \\
            0.20611960233333335  0.04487715711766979  \\
            0.2122710025  0.04487136962829655  \\
            0.21842240266666665  0.04486309798935874  \\
            0.22457380283333334  0.04485360527005957  \\
            0.230725203  0.0448476314166718  \\
            0.23687660316666667  0.044848601431894536  \\
            0.24302800333333333  0.04485107224020261  \\
            0.2491794035  0.04483645969082626  \\
            0.25533080366666666  0.044774484794636006  \\
            0.2614822038333333  0.044592370393489775  \\
            0.267633604  0.04421530669424536  \\
            0.2737850041666667  0.04362179231228574  \\
            0.27993640433333333  0.04280470972471024  \\
            0.2860878045  0.04173197487913953  \\
            0.2922392046666667  0.040364235789079016  \\
            0.29839060483333335  0.038669523760712386  \\
            0.304542005  0.036613858106809886  \\
            0.30637899  0.03592799979999999  \\
            0.3145490964  0.03592799979999999  \\
            0.3227192028  0.03592799979999999  \\
            0.33088930920000004  0.03592799979999999  \\
            0.3390594156  0.03592799979999999  \\
            0.347229522  0.03592799979999999  \\
            0.3553996284  0.03592799979999999  \\
            0.3635697348  0.03592799979999999  \\
            0.37173984120000003  0.03592799979999999  \\
            0.3799099476  0.03592799979999999  \\
            0.388080054  0.03592799979999999  \\
            0.3962501604  0.03592799979999999  \\
            0.4044202668  0.03592799979999999  \\
            0.4125903732  0.03592799979999999  \\
            0.4207604796  0.03592799979999999  \\
            0.428930586  0.03592799979999999  \\
            0.4371006924  0.03592799979999999  \\
            0.4452707988  0.03592799979999999  \\
            0.4534409052  0.03592799979999999  \\
            0.4616110116  0.03592799979999999  \\
            0.469781118  0.03592799979999999  \\
            0.4779512244  0.03592799979999999  \\
            0.4861213308  0.03592799979999999  \\
            0.4942914372  0.03592799979999999  \\
            0.5024615436  0.03592799979999999  \\
            0.51063165  0.03592799979999999  \\
            0.5188017564  0.03592799979999999  \\
            0.5269718628  0.03592799979999999  \\
            0.5351419692  0.03592799979999999  \\
            0.5433120755999999  0.03592799979999999  \\
            0.5514821820000001  0.03592799979999999  \\
        }
        ;
    \addplot[color={rgb,1:red,0.502;green,0.502;blue,0.502}, name path={8cf05959-0002-4f13-b71e-e1fa1868f3e2}, draw opacity={1.0}, line width={0.75}, solid, forget plot]
        table[row sep={\\}]
        {
            \\
            0.12  0.05602935219164747  \\
            0.12615140019524057  0.0560404395104915  \\
            0.1323028003913925  0.05605296683605158  \\
            0.13845420058934013  0.05606501577684807  \\
            0.14460560079001664  0.05607536754340089  \\
            0.1507570009944323  0.056084531918666  \\
            0.15690840120370972  0.0560932491500544  \\
            0.16305980141911983  0.05610185389778746  \\
            0.16921120164212095  0.056110045676790836  \\
            0.17536260187440036  0.05611747133702889  \\
            0.1815140021179268  0.056123773948205766  \\
            0.18766540237501228  0.05612866310351651  \\
            0.19381680264838852  0.05613182293510373  \\
            0.19996820294129872  0.056132848421925056  \\
            0.20611960325761167  0.05613110596234895  \\
            0.21227100360196116  0.05612563287887272  \\
            0.2184224039799177  0.056115209461446  \\
            0.2245738043982078  0.05609862960554776  \\
            0.23072520486502654  0.0560742577875494  \\
            0.236876605390478  0.05603861868336703  \\
            0.24302800598714014  0.055984546650524734  \\
            0.24917940667070462  0.05589987935954851  \\
            0.25533080746060915  0.055767078407355675  \\
            0.2614822083805179  0.055562800352296375  \\
            0.26763360945783743  0.055265286576094926  \\
            0.2737850107226667  0.054863194210018736  \\
            0.2799364122072503  0.054354784921220145  \\
            0.28608781394460575  0.05374561692252722  \\
            0.2922392159645786  0.05305301859198817  \\
            0.29839061828781116  0.05231414428199146  \\
            0.3045420209191786  0.051595405041978036  \\
            0.3063790067118232  0.051406599548768375  \\
            0.31454911697933774  0.050842252216885195  \\
            0.3227192280346334  0.050519563370960065  \\
            0.33088934012977145  0.05032389511663432  \\
            0.3390594535331789  0.050198985415511384  \\
            0.347229568558387  0.0501157010556521  \\
            0.3553996855847916  0.05005814756146559  \\
            0.3635698050772228  0.05001720842195701  \\
            0.37173992760768426  0.049987405550129586  \\
            0.3799100538813233  0.04996530544721607  \\
            0.3880801847683727  0.049948674250106025  \\
            0.3962503213439879  0.04993601057118275  \\
            0.4044204649384488  0.04992627670438774  \\
            0.41259061720114876  0.04991873810880169  \\
            0.4207607801833836  0.049912864219275314  \\
            0.42893095644763335  0.04990826521558875  \\
            0.43710114921569976  0.04990465056326204  \\
            0.4452713625764755  0.04990180113190044  \\
            0.4534416017898208  0.04989955001406339  \\
            0.4616118737533524  0.049897769060290524  \\
            0.4697821877598345  0.049896359256787245  \\
            0.4779525568001732  0.04989524374163353  \\
            0.4861229999453529  0.04989436266866223  \\
            0.49429354697824773  0.0498936693892024  \\
            0.5024642479774059  0.0498931275903585  \\
            0.5106351944033974  0.049892709139502826  \\
            0.5188065683278625  0.04989239245934904  \\
            0.5269787639157566  0.04989216130947543  \\
            0.5351527026038095  0.049892003887225256  \\
            0.5433306877795506  0.049891912191107786  \\
            0.5515188154128429  0.049891881625735275  \\
        }
        ;
    \addplot[color={rgb,1:red,0.502;green,0.502;blue,0.502}, name path={a3fca1c9-e97d-45b7-bf1c-8ca56894c94d}, draw opacity={1.0}, line width={0.75}, solid, forget plot]
        table[row sep={\\}]
        {
            \\
            0.12  0.06710618139257553  \\
            0.1261514002210339  0.06713071695180639  \\
            0.13230280044378667  0.0671553292278602  \\
            0.13845420066992722  0.06717967633249355  \\
            0.1446056009012248  0.06720344097092065  \\
            0.1507570011396005  0.06722652798138998  \\
            0.15690840138718576  0.06724891750798433  \\
            0.16305980164638673  0.06727051141803732  \\
            0.16921120191995728  0.06729106515668723  \\
            0.1753626022110816  0.06731023233667703  \\
            0.1815140025234716  0.06732758926788507  \\
            0.18766540286148156  0.067342641873836  \\
            0.19381680323024608  0.06735480163046952  \\
            0.19996820363584517  0.0673633454997533  \\
            0.20611960408550423  0.0673673598287346  \\
            0.21227100458783607  0.06736568162579662  \\
            0.21842240515313588  0.06735684052934798  \\
            0.22457380579374378  0.06733896142861041  \\
            0.23072520652449496  0.06730951514369517  \\
            0.23687660736326796  0.06726491146826322  \\
            0.2430280083316136  0.0672000915347298  \\
            0.24917940945540965  0.06710837261316725  \\
            0.25533081076544345  0.06698175664728288  \\
            0.2614822122977724  0.06681186401686831  \\
            0.2676336140936322  0.06659199933255601  \\
            0.27378501619896545  0.06631938676320463  \\
            0.27993641866383007  0.06599624378782942  \\
            0.28608782154156026  0.06563029646824577  \\
            0.2922392248876448  0.06523561804303711  \\
            0.2983906287592233  0.06483332747820882  \\
            0.3045420332171931  0.06445101021750609  \\
            0.30637901962687536  0.0643463473441166  \\
            0.3145491330389462  0.0639477319598452  \\
            0.3227192479595764  0.06365532593896576  \\
            0.3308893647732326  0.06344555413950083  \\
            0.3390594839471249  0.06329508780318066  \\
            0.34722960604725867  0.06318620669925075  \\
            0.3553997317635693  0.06310646525569791  \\
            0.3635698619412591  0.06304732859525827  \\
            0.371739997618834  0.06300295281505029  \\
            0.3799101400745015  0.0629693016722311  \\
            0.3880802908832295  0.06294355011247844  \\
            0.39625045198738607  0.06292369096447809  \\
            0.4044206257847325  0.0629082765068913  \\
            0.41259081523876096  0.06289624752843993  \\
            0.42076102401820403  0.06288681890780116  \\
            0.42893125667534415  0.06287940190572144  \\
            0.43710151887713694  0.06287355056068024  \\
            0.4452718177100991  0.06286892412607183  \\
            0.45344216209086075  0.06286526034495345  \\
            0.4616125633309914  0.0628623561618278  \\
            0.4697830359278678  0.06286005361862049  \\
            0.47795359867673637  0.06285822942069735  \\
            0.48612427618904097  0.06285678714039036  \\
            0.49429510072870425  0.06285565134393827  \\
            0.5024661134932674  0.06285476314142364  \\
            0.510637361592821  0.06285407680492099  \\
            0.518808877237784  0.06285355720098837  \\
            0.5269805936959953  0.06285317785468095  \\
            0.5351520492424573  0.06285291951215571  \\
            0.5433213979586135  0.06285276909886658  \\
            0.5514821820000002  0.06285271896110357  \\
        }
        ;
    \addplot[color={rgb,1:red,0.502;green,0.502;blue,0.502}, name path={f05a5ea2-3766-4b5a-90a6-c9025e218578}, draw opacity={1.0}, line width={0.75}, solid, forget plot]
        table[row sep={\\}]
        {
            \\
            0.12  0.07818301059350359  \\
            0.126151400241577  0.07821950267194142  \\
            0.13230280048548837  0.0782559880131186  \\
            0.13845420073402734  0.07829231617435715  \\
            0.14460560098962408  0.07832831538807132  \\
            0.15075700125491626  0.07836381408209799  \\
            0.15690840153282662  0.07839861751686146  \\
            0.16305980182664898  0.07843247184616652  \\
            0.16921120214014587  0.07846504059891733  \\
            0.17536260247765992  0.07849590014122762  \\
            0.1815140028442434  0.07852453904258988  \\
            0.18766540324580985  0.07855035379718306  \\
            0.1938168036893129  0.07857263630941203  \\
            0.1999682041829581  0.07859055305782527  \\
            0.2061196047364545  0.07860311632026916  \\
            0.21227100536131296  0.07860914849751283  \\
            0.21842240607120084  0.07860723787386087  \\
            0.22457380688235992  0.07859567915993154  \\
            0.23072520781409608  0.07857239328873217  \\
            0.23687660888933887  0.07853484727372217  \\
            0.24302801013525505  0.07848003044705751  \\
            0.24917941158388188  0.07840456300083282  \\
            0.2553308132727256  0.07830500208116842  \\
            0.26148221524526294  0.07817837794098743  \\
            0.26763361755129433  0.0780229410851018  \\
            0.27378502024718865  0.07783891230973548  \\
            0.27993642339614044  0.07762899205874955  \\
            0.28608782706858  0.07739859238088641  \\
            0.2922392313429909  0.07715580666129163  \\
            0.2983906363076548  0.07691097355749534  \\
            0.3045420420640752  0.0766755753846573  \\
            0.30637902891051905  0.0766090956582217  \\
            0.31454914455114974  0.07633836320599678  \\
            0.322719262253222  0.0761163200086327  \\
            0.3308893825008494  0.07594047385431368  \\
            0.3390595058962311  0.07580368495588057  \\
            0.34722963318103195  0.07569809149872803  \\
            0.35539976526649114  0.075616723614403  \\
            0.3635699032723407  0.07555393145493997  \\
            0.3717400485754441  0.07550532423590893  \\
            0.3799102028699015  0.07546755580465006  \\
            0.388080368241107  0.07543809536494703  \\
            0.39625054725697945  0.07541503060303202  \\
            0.40442074308041226  0.07539691264131396  \\
            0.41259095960797343  0.07538263871295257  \\
            0.42076120164109976  0.0753713649848897  \\
            0.4289314750975138  0.07536244221308494  \\
            0.43710178727237936  0.07535536825558958  \\
            0.4452721471607802  0.07534975289929793  \\
            0.45344256585529447  0.07534529165925995  \\
            0.4616130570344347  0.0753417461325479  \\
            0.4697836375593605  0.07533892916690509  \\
            0.4779543281998854  0.07533669359232714  \\
            0.4861251545321778  0.07533492361160196  \\
            0.49429614816331185  0.07533352819344553  \\
            0.5024673489413295  0.07533243598866415  \\
            0.5106388108383605  0.07533159141680935  \\
            0.518810621879253  0.07533095166289289  \\
            0.5269829765002086  0.07533048439138054  \\
            0.5351564383917821  0.07533016603620829  \\
            0.543332881422732  0.07532998057230542  \\
            0.5515188154128428  0.07532991875100449  \\
        }
        ;
    \addplot[color={rgb,1:red,0.502;green,0.502;blue,0.502}, name path={a1998ca1-0793-4e63-9447-2dcbbb89b322}, draw opacity={1.0}, line width={0.75}, solid, forget plot]
        table[row sep={\\}]
        {
            \\
            0.12  0.08925983979443165  \\
            0.1261514002549012  0.08930768769610237  \\
            0.13230280051249943  0.08935560960449251  \\
            0.13845420077549261  0.08940353188882594  \\
            0.144605601046736  0.0894513380666458  \\
            0.15075700132932401  0.0894988566340506  \\
            0.15690840162667963  0.08954584775223029  \\
            0.16305980194265315  0.08959199140521185  \\
            0.1692112022816338  0.08963688056691509  \\
            0.1753626026486766  0.0896800198575199  \\
            0.181514003049649  0.08972082667202388  \\
            0.18766540349140065  0.08975863172261479  \\
            0.19381680398196213  0.0897926768834148  \\
            0.1999682045307765  0.08982210956312595  \\
            0.20611960514897135  0.08984597375713471  \\
            0.21227100584967618  0.08986319860985523  \\
            0.2184224066483922  0.0898725857833963  \\
            0.22457380756341977  0.08987279803444295  \\
            0.23072520861634768  0.08986235486443839  \\
            0.23687660983260314  0.08983964813226591  \\
            0.24302801124205642  0.08980299748430785  \\
            0.2491794128796673  0.08975076671054721  \\
            0.25533081478615677  0.08968155388848503  \\
            0.26148221700869273  0.0895944503248274  \\
            0.2676336196015953  0.0894893370174616  \\
            0.2737850226271058  0.08936715393102769  \\
            0.27993642615630765  0.0892300705468901  \\
            0.2860878302703292  0.08908151204073678  \\
            0.2922392350620107  0.08892601151667356  \\
            0.2983906406382771  0.08876885622063425  \\
            0.3045420471234639  0.08861550954242504  \\
            0.3063790342150497  0.08857138189245868  \\
            0.3145491511093207  0.08838596984491395  \\
            0.32271927038915565  0.08822441201016228  \\
            0.33088939260291655  0.08808838882812306  \\
            0.3390595184317064  0.08797648907387864  \\
            0.34722964871705686  0.08788580478010745  \\
            0.3553997844959765  0.0878129941358577  \\
            0.36356992704499996  0.08775485321077865  \\
            0.37174007793497926  0.0877085635960768  \\
            0.3799102390987651  0.08767175970237416  \\
            0.3880804129144707  0.08764250885603442  \\
            0.3962506023076106  0.08761925659644776  \\
            0.4044208108760499  0.08760076370743183  \\
            0.41259104304231786  0.08758604694108861  \\
            0.42076130423833435  0.08757432784736127  \\
            0.4289316011277286  0.08756499052624503  \\
            0.43710194187024826  0.0875575475904319  \\
            0.4452723364304007  0.08755161311240367  \\
            0.45344279692682216  0.08754688127810589  \\
            0.46161333800683757  0.08754310960262762  \\
            0.4697839772059938  0.0875401057510919  \\
            0.47795473520100135  0.08753771719333776  \\
            0.48612563575046147  0.08753582308273485  \\
            0.4942967048348483  0.08753432788268865  \\
            0.5024679677106296  0.08753315637090663  \\
            0.5106394400997473  0.08753224973544467  \\
            0.5188111013820477  0.08753156254240115  \\
            0.5269828086779716  0.08753106040674549  \\
            0.535154008691445  0.08753071823734165  \\
            0.543322743829804  0.08753051895060487  \\
            0.5514821820000001  0.08753045252169261  \\
        }
        ;
    \addplot[color={rgb,1:red,0.502;green,0.502;blue,0.502}, name path={34a410aa-ebc2-4dde-b0d0-cfed39befed0}, draw opacity={1.0}, line width={0.75}, solid, forget plot]
        table[row sep={\\}]
        {
            \\
            0.12  0.10033666899535972  \\
            0.12615140025972063  0.10039538250716509  \\
            0.13230280052221355  0.10045432517316684  \\
            0.13845420079032575  0.10051354088230255  \\
            0.14460560106706402  0.10057299701664563  \\
            0.15075700135567988  0.10063255991642192  \\
            0.15690840165976214  0.10069198275589758  \\
            0.1630598019833388  0.10075090393602434  \\
            0.16921120233099135  0.10080885315695454  \\
            0.1753626027079842  0.10086526204915033  \\
            0.1815140031204133  0.10091947637043401  \\
            0.18766540357537678  0.10097076746037408  \\
            0.1938168040811733  0.10101834167588453  \\
            0.1999682046475313  0.1010613476062314  \\
            0.20611960528587536  0.10109888172387062  \\
            0.21227100600963467  0.1011299937391038  \\
            0.218422406834599  0.10115369345270674  \\
            0.22457380777932734  0.10116896178434373  \\
            0.23072520886561346  0.10117477013910177  \\
            0.23687661011901207  0.10117011389147092  \\
            0.2430280115694266  0.10115406627813653  \\
            0.24917941325176104  0.10112585713526756  \\
            0.2553308152066385  0.10108497604801062  \\
            0.26148221748119715  0.10103129187374736  \\
            0.2676336201299862  0.10096517151955674  \\
            0.27378502321600184  0.10088757302268427  \\
            0.2799364268119306  0.10080008660836114  \\
            0.2860878310016841  0.10070490343691223  \\
            0.2922392358823359  0.10060470052744373  \\
            0.2983906415665778  0.1005024406188454  \\
            0.3045420481858039  0.10040110174599662  \\
            0.3063790353238422  0.10037153389437943  \\
            0.31454915246455956  0.10024473922294365  \\
            0.32271927207158807  0.10012985818882085  \\
            0.3308893947096599  0.10002906445335612  \\
            0.3390595210777388  0.0999427782695948  \\
            0.3472296520389769  0.09987025383397206  \\
            0.3553997886575703  0.09981011168845659  \\
            0.36356993224432615  0.09976072362718605  \\
            0.3717400844129372  0.09972045244561493  \\
            0.3799102471492997  0.099687781838861  \\
            0.388080422896663  0.09966137405082315  \\
            0.39625061465992667  0.0996400846401548  \\
            0.40442082613295866  0.09962295424492405  \\
            0.4125910618533038  0.09960918963998613  \\
            0.4207613273889434  0.09959814115254509  \\
            0.4289316295615548  0.09958928019716523  \\
            0.43710197670947615  0.09958217872865896  \\
            0.44527237899040106  0.09957649130687644  \\
            0.4534428487172092  0.09957193988220653  \\
            0.46161340070814477  0.09956830112584683  \\
            0.46978405261252476  0.09956539601100546  \\
            0.47795482514627424  0.09956308132165273  \\
            0.4861257421556427  0.09956124277961106  \\
            0.494296830506958  0.09955978951408254  \\
            0.5024681203045294  0.09955864963716243  \\
            0.510639648071418  0.09955776672796887  \\
            0.5188114735956995  0.09955709706379755  \\
            0.5269837504807874  0.09955660746829151  \\
            0.5351569956825083  0.09955627367509137  \\
            0.5433330799713588  0.09955607913807567  \\
            0.5515188154128428  0.09955601429240378  \\
        }
        ;
    \addplot[color={rgb,1:red,0.502;green,0.502;blue,0.502}, name path={fbedaecf-e85f-4fc8-a0e3-0296e2f08dd0}, draw opacity={1.0}, line width={0.75}, solid, forget plot]
        table[row sep={\\}]
        {
            \\
            0.12  0.11141349819628778  \\
            0.12615140025554467  0.11148236259103259  \\
            0.13230280051364623  0.11155172607354069  \\
            0.13845420077703385  0.11162185072557691  \\
            0.14460560104858047  0.11169285335570084  \\
            0.1507570013313835  0.11176466904790242  \\
            0.15690840162885206  0.1118370460812192  \\
            0.16305980194480232  0.11190956234905504  \\
            0.16921120228356445  0.11198165270245582  \\
            0.17536260265010248  0.11205263965240041  \\
            0.18151400305015145  0.11212176274436124  \\
            0.1876654034903738  0.11218820383875444  \\
            0.19381680397853998  0.11225110758252534  \\
            0.19996820452373618  0.11230959800557637  \\
            0.20611960513660454  0.11236279307574953  \\
            0.21227100582962008  0.11240981908966702  \\
            0.21842240661740944  0.11244982595007046  \\
            0.22457380751711667  0.11248200427707714  \\
            0.23072520854882148  0.11250560651005616  \\
            0.23687660973601576  0.11251997500326537  \\
            0.243028011106145  0.11252457917280535  \\
            0.24917941269122246  0.11251906151488966  \\
            0.2553308145285274  0.11250328981595437  \\
            0.2614822166614041  0.11247741035986174  \\
            0.26763361914018513  0.11244189405263647  \\
            0.2737850220232743  0.11239756478204808  \\
            0.2799364253784347  0.11234559906099814  \\
            0.2860878292843385  0.11228748906818595  \\
            0.29223923383244643  0.11222496656167623  \\
            0.29839063912928604  0.11215989152137111  \\
            0.30454204529919815  0.11209411675340528  \\
            0.3063790322893496  0.11207466773029352  \\
            0.3145491486937438  0.11198976551337624  \\
            0.3227192674084074  0.11191047652297306  \\
            0.330889388962727  0.111838736376796  \\
            0.3390595140104026  0.11177546452623631  \\
            0.347229643357913  0.11172078627973303  \\
            0.35539977799939276  0.11167428840052289  \\
            0.3635699191595374  0.11163524379598788  \\
            0.3717400683464427  0.1116027812957342  \\
            0.37991022741664443  0.11157600045120916  \\
            0.38808039865509203  0.1115540415055056  \\
            0.396250584873342  0.11153612285570005  \\
            0.40442078952990634  0.1115215568622359  \\
            0.41259101687739697  0.11150975228942685  \\
            0.42076127214181636  0.11150020917153226  \\
            0.4289315617399195  0.11149250990548899  \\
            0.4371018935407276  0.11148630892654979  \\
            0.4452722771764445  0.11148132234100594  \\
            0.453442724405047  0.11147731825425801  \\
            0.4616132495192633  0.11147410813803929  \\
            0.469783869779121  0.11147153934678937  \\
            0.4779546058046166  0.11146948876243899  \\
            0.4861254817658055  0.11146785748017712  \\
            0.4942965249431467  0.1114665664195881  \\
            0.5024677634546381  0.11146555273962629  \\
            0.5106392184697081  0.11146476694217285  \\
            0.5188108788505589  0.11146417056132538  \\
            0.5269826167802892  0.11146373435046013  \\
            0.5351538979259829  0.11146343689352065  \\
            0.5433227613075436  0.11146326357384015  \\
            0.5514821820000001  0.11146320580061327  \\
        }
        ;
    \addplot[color={rgb,1:red,0.502;green,0.502;blue,0.502}, name path={6f0edf5f-30d0-4bf7-aa84-823f6cef856d}, draw opacity={1.0}, line width={0.75}, solid, forget plot]
        table[row sep={\\}]
        {
            \\
            0.12  0.12249032739721583  \\
            0.12615140024271357  0.12256798715099464  \\
            0.13230280048750778  0.12264664646709793  \\
            0.13845420073673972  0.12272697691396277  \\
            0.14460560099287698  0.12280934155347033  \\
            0.1507570012585672  0.12289374982477055  \\
            0.15690840153671326  0.12297988993388287  \\
            0.16305980183055457  0.12306719467988293  \\
            0.16921120214375715  0.12315491172194172  \\
            0.17536260248051416  0.12324216719187772  \\
            0.1815140028456599  0.12332801937251056  \\
            0.18766540324479908  0.1234114987789757  \\
            0.1938168036844555  0.12349163568811568  \\
            0.19996820417224237  0.12356747941597948  \\
            0.20611960471705912  0.1236381149258848  \\
            0.21227100532931756  0.12370268190535638  \\
            0.218422406021203  0.12376039434363861  \\
            0.2245738068069738  0.12381055584075983  \\
            0.23072520770330537  0.1238525722232395  \\
            0.23687660872968597  0.12388596778461915  \\
            0.2430280099088706  0.12391040769746414  \\
            0.24917941126740398  0.12392572336658478  \\
            0.2553308128362228  0.12393193634922521  \\
            0.26148221465135507  0.12392927958575795  \\
            0.267633616754736  0.12391821512694683  \\
            0.27378501919516685  0.12389944276158213  \\
            0.27993642202944796  0.12387389111256711  \\
            0.28608782532372357  0.12384268624056223  \\
            0.2922392291550817  0.12380709913175461  \\
            0.29839063361345675  0.12376847642707164  \\
            0.3045420388038831  0.12372815935402574  \\
            0.30637902546977697  0.12371603274379787  \\
            0.3145491402472744  0.1236619888789181  \\
            0.32271925696575005  0.12361005525695089  \\
            0.3308893760706582  0.12356184175324728  \\
            0.33905949811136643  0.12351830759697713  \\
            0.3472296237651819  0.12347987138476386  \\
            0.355399753866825  0.12344654795511546  \\
            0.3635698894446839  0.12341808044642985  \\
            0.3717400317654777  0.12339405088118224  \\
            0.37991018238932317  0.12337396392155768  \\
            0.38808034323767204  0.12335730516822024  \\
            0.3962505166771905  0.1233435784347396  \\
            0.40442070562344534  0.12333232717257825  \\
            0.4125909136693054  0.12332314476826371  \\
            0.4207611452443796  0.12331567752023932  \\
            0.42893140581376  0.12330962313150153  \\
            0.4371017021270834  0.12330472670796604  \\
            0.44527204253287295  0.1233007755881061  \\
            0.4534424373788569  0.12329759384162746  \\
            0.46161289952734985  0.12329503693436912  \\
            0.46978344502730296  0.12329298682876325  \\
            0.47795409400497524  0.12329134764296687  \\
            0.4861248718771444  0.12329004190227692  \\
            0.4942958111129259  0.12328900736500872  \\
            0.5024669542191273  0.12328819437833258  \\
            0.5106383603992285  0.123287563708417  \\
            0.5188101253521777  0.12328708478744244  \\
            0.526982450866389  0.12328673432366069  \\
            0.5351559041316796  0.1232864952278606  \\
            0.5433323929047321  0.12328635582365197  \\
            0.551518815412843  0.12328630935558241  \\
        }
        ;
    \addplot[color={rgb,1:red,0.502;green,0.502;blue,0.502}, name path={026ef830-6c67-4a58-8e91-6e77c3ed252d}, draw opacity={1.0}, line width={0.75}, solid, forget plot]
        table[row sep={\\}]
        {
            \\
            0.12  0.1335671565981439  \\
            0.12615140022235488  0.1336508446344345  \\
            0.13230280044611692  0.13373657213266293  \\
            0.13845420067305042  0.13382579732752276  \\
            0.14460560090498  0.13391924135262664  \\
            0.15075700114384671  0.13401687233765402  \\
            0.15690840139176687  0.1341181109398657  \\
            0.1630598016510932  0.13422203132650048  \\
            0.1692112019244811  0.13432750486434455  \\
            0.17536260221496122  0.13443330414778515  \\
            0.18151400252602115  0.1345381905217773  \\
            0.18766540286169772  0.1346409642309278  \\
            0.193816803226682  0.13474048468284108  \\
            0.19996820362643988  0.13483567444425562  \\
            0.2061196040673514  0.13492552158683233  \\
            0.21227100455687117  0.1350091095795613  \\
            0.21842240510371397  0.13508565582842222  \\
            0.22457380571806682  0.13515451980748214  \\
            0.23072520641183372  0.1352151874130788  \\
            0.2368766071989195  0.13526726829449545  \\
            0.24302800809555822  0.13531051547766182  \\
            0.24917940912069386  0.13534484638930663  \\
            0.25533081029642163  0.13537034567568784  \\
            0.2614822116485037  0.13538726081860353  \\
            0.2676336132069755  0.13539601336710996  \\
            0.2737850150068578  0.13539721850969472  \\
            0.27993641708899336  0.1353916860143239  \\
            0.2860878195010295  0.13538039027131413  \\
            0.2922392222985777  0.13536442502857887  \\
            0.29839062554657736  0.13534495550240616  \\
            0.30454202932089514  0.13532316547991338  \\
            0.3063790155173971  0.13531642747556027  \\
            0.3145491279358046  0.1352854868089334  \\
            0.32271924175138167  0.13525480603821718  \\
            0.33088935728515295  0.13522565408360918  \\
            0.33905947493279903  0.1351988294755506  \\
            0.34722959518198687  0.13517475994558956  \\
            0.3553997186336654  0.13515359488510492  \\
            0.3635698460282941  0.13513528773234879  \\
            0.37173997827820915  0.13511966502202966  \\
            0.3799101165076361  0.1351064806361067  \\
            0.38808026210226687  0.1350954557794504  \\
            0.3962504167708732  0.1350863065436754  \\
            0.4044205826222216  0.13507876151039622  \\
            0.41259076226170527  0.13507257187209598  \\
            0.4207609589138673  0.13506751625594868  \\
            0.4289311765797538  0.13506340201414993  \\
            0.43710142024252935  0.13506006431355577  \\
            0.4452716961422626  0.13505736397892873  \\
            0.4534420121533931  0.1350551847400687  \\
            0.4616123783195899  0.1350534303037881  \\
            0.4697828076354344  0.13505202150706594  \\
            0.47795331721670903  0.13505089369454465  \\
            0.48612393006187904  0.13504999438898  \\
            0.4942946776020239  0.13504928127644167  \\
            0.5024656028282337  0.13504872050046546  \\
            0.5106367616621093  0.13504828524465456  \\
            0.5188082115728961  0.1350479545769144  \\
            0.5269799439722115  0.13504771252740955  \\
            0.5351515995595778  0.13504754737373442  \\
            0.5433213891694104  0.13504745110549474  \\
            0.5514821820000002  0.1350474190160815  \\
        }
        ;
    \addplot[color={rgb,1:red,0.502;green,0.502;blue,0.502}, name path={d1a01947-1cb9-44bb-b190-88259c1926ec}, draw opacity={1.0}, line width={0.75}, solid, forget plot]
        table[row sep={\\}]
        {
            \\
            0.12  0.14464398579907195  \\
            0.1261514001962611  0.1447277691363484  \\
            0.132302800393164  0.14481614363284223  \\
            0.1384542005917003  0.14491206495073028  \\
            0.14460560079285847  0.1450166051869055  \\
            0.15075700099767186  0.14512903043874306  \\
            0.15690840120725882  0.145247877297928  \\
            0.16305980142285748  0.1453714339695169  \\
            0.1692112016458617  0.14549789279093822  \\
            0.17536260187785932  0.14562542051196536  \\
            0.18151400212067537  0.1457523440446712  \\
            0.1876654023764194  0.1458771844962717  \\
            0.1938168026475395  0.14599862705058791  \\
            0.19996820293688403  0.1461154636342167  \\
            0.20611960324777379  0.14622649016019462  \\
            0.21227100358408668  0.1463306100739647  \\
            0.21842240395035464  0.14642701504444167  \\
            0.22457380435187183  0.14651516205877915  \\
            0.23072520479481992  0.14659461262109502  \\
            0.23687660528641738  0.1466649843626945  \\
            0.24302800583509376  0.14672603705782658  \\
            0.24917940645069037  0.14677775336237048  \\
            0.2553308071446905  0.14682028588402926  \\
            0.26148220793049315  0.1468538547638709  \\
            0.26763360882374065  0.146878777620109  \\
            0.27378500984270293  0.14689558885228318  \\
            0.27993641100872163  0.1469050910532755  \\
            0.28608781234672537  0.14690827504726234  \\
            0.29223921388583923  0.14690623885046122  \\
            0.29839061566010217  0.146900166876214  \\
            0.30454201770929595  0.14689135308812828  \\
            0.30637900333372864  0.14688851094714875  \\
            0.31454911287516696  0.1468749285743424  \\
            0.3227192231467283  0.14686102249543315  \\
            0.3308893343162264  0.14684755481245004  \\
            0.3390594465904846  0.14683499422035584  \\
            0.34722956022439017  0.146823603475183  \\
            0.35539967553204227  0.14681349801125748  \\
            0.3635697929005032  0.14680469016934244  \\
            0.37173991280679575  0.14679712389736138  \\
            0.3799100358389651  0.1467907016420016  \\
            0.38808016272226664  0.1467853044195057  \\
            0.3962502943518805  0.14678080605369023  \\
            0.40442043183406434  0.14677708266194575  \\
            0.41259057653843373  0.14677401847132818  \\
            0.4207607301653186  0.14677150894740648  \\
            0.4289308948342452  0.14676946205934605  \\
            0.4371010732032426  0.1467677983278903  \\
            0.44527126863525707  0.14676645013696812  \\
            0.4534414854402353  0.14676536064905302  \\
            0.4616117292452863  0.14676448255345573  \\
            0.46978200759368277  0.14676377679387123  \\
            0.4779523309766128  0.14676321136241535  \\
            0.4861227147347506  0.14676276020703213  \\
            0.494293182827893  0.1467624022727971  \\
            0.5024637759106771  0.14676212068132216  \\
            0.5106345700757328  0.1467619020430869  \\
            0.5188057239001983  0.1467617358927971  \\
            0.5269776053049885  0.14676161423624165  \\
            0.5351511552596466  0.1467615311978387  \\
            0.5433289841335676  0.1467614827621099  \\
            0.551518815412843  0.14676146661686695  \\
        }
        ;
    \addplot[color={rgb,1:red,0.502;green,0.502;blue,0.502}, name path={8aa5d03d-3754-4ec3-8721-d3822a0bb74d}, draw opacity={1.0}, line width={0.75}, solid, forget plot]
        table[row sep={\\}]
        {
            \\
            0.12  0.155720815  \\
            0.12615140016666665  0.15579077631842816  \\
            0.13230280033333333  0.1558732022935015  \\
            0.13845420049999999  0.1559729329828782  \\
            0.14460560066666667  0.15609102041875295  \\
            0.15075700083333332  0.15622265571003757  \\
            0.156908401  0.15636407383589124  \\
            0.16305980116666666  0.15651236616271635  \\
            0.16921120133333334  0.15666486336534555  \\
            0.1753626015  0.15681857836445  \\
            0.18151400166666667  0.15697133722994003  \\
            0.18766540183333333  0.15712150120938215  \\
            0.193816802  0.15726777750621487  \\
            0.19996820216666666  0.15740915183117207  \\
            0.20611960233333335  0.15754388480717682  \\
            0.2122710025  0.1576701304503246  \\
            0.21842240266666665  0.15778714679780784  \\
            0.22457380283333334  0.15789500367297732  \\
            0.230725203  0.15799347431712235  \\
            0.23687660316666667  0.1580818906263962  \\
            0.24302800333333333  0.15815975337140403  \\
            0.2491794035  0.15822728314697948  \\
            0.25533080366666666  0.15828503040023142  \\
            0.2614822038333333  0.15833308352046074  \\
            0.267633604  0.1583712647951522  \\
            0.2737850041666667  0.15839986466358472  \\
            0.27993640433333333  0.15841995001612175  \\
            0.2860878045  0.15843280185958125  \\
            0.2922392046666667  0.15843959620544462  \\
            0.29839060483333335  0.1584414626579145  \\
            0.304542005  0.1584389348467872  \\
            0.30637899  0.1584389348467872  \\
            0.3145490964  0.1584389348467872  \\
            0.3227192028  0.1584389348467872  \\
            0.33088930920000004  0.1584389348467872  \\
            0.3390594156  0.1584389348467872  \\
            0.347229522  0.1584389348467872  \\
            0.3553996284  0.1584389348467872  \\
            0.3635697348  0.1584389348467872  \\
            0.37173984120000003  0.1584389348467872  \\
            0.3799099476  0.1584389348467872  \\
            0.388080054  0.1584389348467872  \\
            0.3962501604  0.1584389348467872  \\
            0.4044202668  0.1584389348467872  \\
            0.4125903732  0.1584389348467872  \\
            0.4207604796  0.1584389348467872  \\
            0.428930586  0.1584389348467872  \\
            0.4371006924  0.1584389348467872  \\
            0.4452707988  0.1584389348467872  \\
            0.4534409052  0.1584389348467872  \\
            0.4616110116  0.1584389348467872  \\
            0.469781118  0.1584389348467872  \\
            0.4779512244  0.1584389348467872  \\
            0.4861213308  0.1584389348467872  \\
            0.4942914372  0.1584389348467872  \\
            0.5024615436  0.1584389348467872  \\
            0.51063165  0.1584389348467872  \\
            0.5188017564  0.1584389348467872  \\
            0.5269718628  0.1584389348467872  \\
            0.5351419692  0.1584389348467872  \\
            0.5433120755999999  0.1584389348467872  \\
            0.5514821820000002  0.1584389348467872  \\
        }
        ;
    \addplot[color={rgb,1:red,0.4118;green,0.6824;blue,0.3725}, name path={56add62f-64ae-481a-b331-a8f1e1b999ac}, draw opacity={1.0}, line width={2}, solid, forget plot]
        table[row sep={\\}]
        {
            \\
            0.12  0.04495252299071941  \\
            0.12  0.05602935219164747  \\
            0.12  0.06710618139257553  \\
            0.12  0.07818301059350359  \\
            0.12  0.08925983979443165  \\
            0.12  0.10033666899535972  \\
            0.12  0.11141349819628778  \\
            0.12  0.12249032739721583  \\
            0.12  0.1335671565981439  \\
            0.12  0.14464398579907195  \\
            0.12  0.155720815  \\
        }
        ;
    \addplot[color={rgb,1:red,0.0;green,0.3608;blue,0.6706}, name path={b28b339c-50a1-4eed-bfbb-20203a8780c9}, draw opacity={1.0}, line width={1.0}, solid, mark={*}, mark size={1.125 pt}, mark repeat={1}, mark options={color={rgb,1:red,0.0;green,0.0;blue,0.0}, draw opacity={0.0}, fill={rgb,1:red,0.0;green,0.3608;blue,0.6706}, fill opacity={1.0}, line width={0.75}, rotate={0}, solid}, forget plot]
        table[row sep={\\}]
        {
            \\
            0.304542005  0.1584389348467872  \\
            0.29839060483333335  0.1584414626579145  \\
            0.2922392046666667  0.15843959620544462  \\
            0.2860878045  0.15843280185958125  \\
            0.27993640433333333  0.15841995001612175  \\
            0.2737850041666667  0.15839986466358472  \\
            0.267633604  0.1583712647951522  \\
            0.2614822038333333  0.15833308352046074  \\
            0.25533080366666666  0.15828503040023142  \\
            0.2491794035  0.15822728314697948  \\
            0.24302800333333333  0.15815975337140403  \\
            0.23687660316666667  0.1580818906263962  \\
            0.230725203  0.15799347431712235  \\
            0.22457380283333334  0.15789500367297732  \\
            0.21842240266666665  0.15778714679780784  \\
            0.2122710025  0.1576701304503246  \\
            0.20611960233333335  0.15754388480717682  \\
            0.19996820216666666  0.15740915183117207  \\
            0.193816802  0.15726777750621487  \\
            0.18766540183333333  0.15712150120938215  \\
            0.18151400166666667  0.15697133722994003  \\
            0.1753626015  0.15681857836445  \\
            0.16921120133333334  0.15666486336534555  \\
            0.16305980116666666  0.15651236616271635  \\
            0.156908401  0.15636407383589124  \\
            0.15075700083333332  0.15622265571003757  \\
            0.14460560066666667  0.15609102041875295  \\
            0.13845420049999999  0.1559729329828782  \\
            0.13230280033333333  0.1558732022935015  \\
            0.12615140016666665  0.15579077631842816  \\
            0.12  0.155720815  \\
            0.11399279808861612  0.15565969080491326  \\
            0.10800206150011583  0.15560249443750745  \\
            0.10204421042707724  0.15555102467759901  \\
            0.09613557492516646  0.15551295342131385  \\
            0.09029235015358988  0.15549417513847946  \\
            0.08453055198528892  0.15550690575697393  \\
            0.07886597310854516  0.15555806588088175  \\
            0.07331413974031975  0.15565145553095425  \\
            0.06789026906997177  0.15579480039342583  \\
            0.06260922754999998  0.1559972892969931  \\
            0.05748549014812998  0.15626682407104356  \\
            0.052533100672433664  0.15660670998792758  \\
            0.04776563327822751  0.15702395996670845  \\
            0.04319615526225648  0.15752589111786994  \\
            0.038837191246141795  0.1581190016491485  \\
            0.03470068884726381  0.1588032721112729  \\
            0.030797985931173034  0.15958255369945537  \\
            0.02713977953528892  0.16046238560820839  \\
            0.023736096549063807  0.16144508306941582  \\
            0.020596266230975817  0.16253214292681187  \\
            0.01772889463767991  0.16372296488091606  \\
            0.01514184103540476  0.16501480690166814  \\
            0.012842196358250287  0.16640662505317158  \\
            0.010836263772430254  0.16789724603614412  \\
            0.009129541399731704  0.1694852415602899  \\
            0.007726707247545409  0.1711796485936844  \\
            0.006631606386772701  0.17300038865420833  \\
            0.005847240412753437  0.17482944254720592  \\
            0.005375759218101773  0.1769049528784241  \\
            0.0052184551  0.17818011312626458  \\
            0.005375759218101773  0.18111733553290799  \\
            0.005847240412753437  0.1832370262313411  \\
            0.006631606386772701  0.18538397929620315  \\
            0.007726707247545409  0.18749476787817393  \\
            0.009129541399731704  0.18959413853240056  \\
            0.010836263772430254  0.19167163675489338  \\
            0.012842196358250287  0.19371567907762274  \\
            0.01514184103540476  0.19571723114029266  \\
            0.01772889463767991  0.19766663307197876  \\
            0.020596266230975817  0.19955300101177723  \\
            0.023736096549063807  0.20136450910193648  \\
            0.02713977953528892  0.20309181071356452  \\
            0.030797985931173034  0.20472620584894421  \\
            0.03470068884726381  0.20625768286324309  \\
            0.038837191246141795  0.2076773483741656  \\
            0.04319615526225648  0.20897894650873994  \\
            0.04776563327822751  0.21015543159349476  \\
            0.052533100672433664  0.21119806292411006  \\
            0.05748549014812998  0.21210074637335335  \\
            0.06260922754999998  0.2128591458243588  \\
            0.06789026906997177  0.21346996882865465  \\
            0.07331413974031975  0.21393111499918402  \\
            0.07886597310854516  0.21424145596242564  \\
            0.08453055198528892  0.2144009116777412  \\
            0.09029235015358988  0.21441087192011882  \\
            0.09613557492516646  0.21427256561901806  \\
            0.10204421042707724  0.2139882937494529  \\
            0.10800206150011583  0.21356450119740195  \\
            0.11399279808861612  0.2130063830369111  \\
            0.12  0.21231528902120386  \\
            0.12615140016666665  0.21148163030815084  \\
            0.13230280033333333  0.21054232773485568  \\
            0.13845420049999999  0.20951351266887752  \\
            0.14460560066666667  0.20840624793627152  \\
            0.15075700083333332  0.2072267992995797  \\
            0.156908401  0.20597836002063233  \\
            0.16305980116666666  0.2046625102172418  \\
            0.16921120133333334  0.2032814554715009  \\
            0.1753626015  0.20183911842645766  \\
            0.18151400166666667  0.20033903152400717  \\
            0.18766540183333333  0.198782514933333  \\
            0.193816802  0.19717035779256822  \\
            0.19996820216666666  0.19550419230865274  \\
            0.20611960233333335  0.193785876944981  \\
            0.2122710025  0.1920164435257992  \\
            0.21842240266666665  0.1901965719398472  \\
            0.22457380283333334  0.18832735174399645  \\
            0.230725203  0.18640963464445282  \\
            0.23687660316666667  0.18444325695992825  \\
            0.24302800333333333  0.18242791302049308  \\
            0.2491794035  0.18036393406872206  \\
            0.25533080366666666  0.17825155523074446  \\
            0.2614822038333333  0.1760898628465513  \\
            0.267633604  0.17387797068935812  \\
            0.2737850041666667  0.17161626688836612  \\
            0.27993640433333333  0.16930459165100145  \\
            0.2860878045  0.1669411898638361  \\
            0.2922392046666667  0.16452422700875094  \\
            0.29839060483333335  0.16205211714309609  \\
            0.304542005  0.15952600512626458  \\
        }
        ;
    \addplot[color={rgb,1:red,0.7451;green,0.298;blue,0.302}, name path={16eca9c6-fe46-46c3-98a3-038192c88a0a}, draw opacity={1.0}, line width={1.0}, solid, mark={*}, mark size={1.125 pt}, mark repeat={1}, mark options={color={rgb,1:red,0.0;green,0.0;blue,0.0}, draw opacity={0.0}, fill={rgb,1:red,0.7451;green,0.298;blue,0.302}, fill opacity={1.0}, line width={0.75}, rotate={0}, solid}, forget plot]
        table[row sep={\\}]
        {
            \\
            0.0  0.0  \\
            0.00016445582945114  0.002740749981173598  \\
            0.0006573725558072052  0.005643307185909748  \\
            0.0014773991285834676  0.00831732970954065  \\
            0.0026222879119433174  0.010982863391086622  \\
            0.004088900845311803  0.013574999693898887  \\
            0.005873218044581576  0.01609783096746975  \\
            0.007970348820335791  0.01854798891045612  \\
            0.010374545082887895  0.020917835209454175  \\
            0.013079217097395852  0.023202469121967  \\
            0.016076951545867354  0.02539710667209484  \\
            0.019359531846549115  0.0274955317958163  \\
            0.022917960675006305  0.02949157705815014  \\
            0.026742484625163505  0.03138038128041586  \\
            0.03082262094271269  0.033157105979071975  \\
            0.035147186257614274  0.0348168854917705  \\
            0.03970432723693701  0.03635670057951697  \\
            0.0444815530740195  0.037773299047661385  \\
            0.04946576972490322  0.03906327587476941  \\
            0.054643315798196736  0.040224241354043955  \\
            0.059999999999999984  0.04125553477214516  \\
            0.06552114003125438  0.042158781258759444  \\
            0.07119160283090395  0.04293422532362003  \\
            0.07699584605456396  0.043581158174966446  \\
            0.0829179606750063  0.044101461786900796  \\
            0.0889417145876975  0.04450114439797895  \\
            0.09505059710186889  0.044786996687128246  \\
            0.10122786419517228  0.04495063424158973  \\
            0.10745658440788158  0.044998666961206364  \\
            0.11371968525084675  0.044983194126287505  \\
            0.12  0.04495252299071941  \\
            0.12615140016666665  0.04494248592934282  \\
            0.13230280033333333  0.04494776043555422  \\
            0.13845420049999999  0.044949764153980734  \\
            0.14460560066666667  0.044942278699579535  \\
            0.15075700083333332  0.04493204538203043  \\
            0.156908401  0.044923127246484104  \\
            0.16305980116666666  0.04491727692121097  \\
            0.16921120133333334  0.04491196172254526  \\
            0.1753626015  0.044906620646925216  \\
            0.18151400166666667  0.04490069966154803  \\
            0.18766540183333333  0.04489448945657971  \\
            0.193816802  0.04488823654667903  \\
            0.19996820216666666  0.04488243703899517  \\
            0.20611960233333335  0.04487715711766979  \\
            0.2122710025  0.04487136962829655  \\
            0.21842240266666665  0.04486309798935874  \\
            0.22457380283333334  0.04485360527005957  \\
            0.230725203  0.0448476314166718  \\
            0.23687660316666667  0.044848601431894536  \\
            0.24302800333333333  0.04485107224020261  \\
            0.2491794035  0.04483645969082626  \\
            0.25533080366666666  0.044774484794636006  \\
            0.2614822038333333  0.044592370393489775  \\
            0.267633604  0.04421530669424536  \\
            0.2737850041666667  0.04362179231228574  \\
            0.27993640433333333  0.04280470972471024  \\
            0.2860878045  0.04173197487913953  \\
            0.2922392046666667  0.040364235789079016  \\
            0.29839060483333335  0.038669523760712386  \\
            0.304542005  0.036613858106809886  \\
            0.30637899  0.03592799979999999  \\
        }
        ;
\end{axis}
\end{tikzpicture}
\hspace*{5em}
     }\qquad
     \end{subfigure}
     \caption{Single rotor verification case geometry as generated by DFDC and DuctAPE. Duct geometry in \primary{blue}, center body geometry in \secondary{red}, rotor lifting line location in \tertiary{green}, and wake grid geometry in \gray{gray}.}
    \label{fig:singlerotorgeom}
\end{figure}


We note here that DuctAPE also differs from DFDC in the geometry re-paneling approach.
%
The reason for a different approach to geometry generation in DuctAPE is so that the duct, center body, and wake can be paneled in such a way to avoid discontinuities in gradients if the relative position of the rotors, duct, and center body were to change in an optimization setting.
%
Comparing the subfigures of \cref{fig:singlerotorgeom} we see two major differences between the DFDC generated geometry (\cref{fig:dfdcsinglerotorgeom}) and the DuctAPE generated geometry (\cref{fig:ductapesinglerotorgeom}).
%
The DuctAPE geometry re-paneling approach aligns the duct, center body, and wake panels aft of the rotor and distributes them linearly.
%
We align the panels so that there is a consistent number of panels between discrete locations (such as rotor positions and body trailing edges) in the geometry, thereby avoiding discontinuities.
%
For example, the number of center body and duct panels ahead of and behind the rotor need to stay constant if the rotor position is selected as a design variable in an optimization.
%
Without the number of panels ahead of and behind the rotor staying constant, there would be discontinuities as the rotor passed over panels along the solid bodies.
%
 The second difference in geometries is that DuctAPE does not yet apply any expansion in the wake panel length aft of the duct exit. %TODO; should this be added?
%
 The main reason to apply expansion is to reduce the number of panels in the wake, and thereby reduce the size of the system being solved.
%
 % Not applying an expansion does not affect the solution, and is not yet implemented in DuctAPE (see \cref{sec:conclusions}), so we simply maintain a linear distribution of panels in the trailing wake.
%
 Also note that the number of panels in the wake needs to stay constant between each discrete location, even aft of the duct exit, in case the duct chord length is selected as a design variable in an optimization.
%
 One additional difference, not visible, is that the duct and center body geometries are defined counter-clockwise for DFDC and clockwise for DuctAPE, which simply led to some differences in sign (compared to the DFDC implementation) in the various induced velocity equations presented above.

As we are comparing the performance of solvers using the DFDC-like CSOR approach and our alternate approach, we verify here that both implementations match values from the original Fortran implementation of DFDC.
%
Scanning \cref{tab:hovercompsinglerotor,tab:cruisecompsinglerotor}, we see that the differences between DFDC and both approaches implemented in DuctAPE are less than 0.5\% for major output values for both a hover and a cruise case.
%
\Cref{fig:singlerotorcpcteta} shows comparisons of total thrust and power coefficients (\cref{fig:singlerotorcpct}) and total efficiency (\cref{fig:singlerotoreta}), across the range of advance ratios, showing excellent matching across the entire range.
%
Note that the results for both the DFDC-like solver and alternate DuctAPE solver yield identical plots, so we include only one here.

\begin{table}[h!]
    \caption{Comparison of solver outputs for hover case (\(J=0.0\)). Errors relative to DFDC.}
    \begin{subtable}[t]{0.35\textwidth}
        \begin{center}
                    \begin{tabular}{ r | c | c | c }
            Output Value & DFDC & CSOR & \% Error \\
            \hline
            Rotor Thrust (N) & 91.8 & 91.82 & 0.03 \\
            Body Thrust (N) & 106.45 & 106.96 & 0.48 \\
            Torque (N\(\cdot\)m) & 6.58 & 6.58 & 0.04 \\
        \end{tabular}

        \end{center}
    \end{subtable}

    \begin{subtable}[t]{0.45\textwidth}
        \begin{center}
                    \begin{tabular}{ r | c | c | c }
            Output Value & DFDC & Alternate & \% Error \\
            \hline
            Rotor Thrust (N) & 91.8 & 91.82 & 0.02 \\
            Body Thrust (N) & 106.45 & 106.95 & 0.47 \\
            Torque (N\(\cdot\)m) & 6.58 & 6.58 & 0.04 \\
        \end{tabular}

        \end{center}
    \end{subtable}
    \label{tab:hovercompsinglerotor}
\end{table}

\begin{table}[h!]
    \caption{Comparison of solver outputs for a cruise case (\(J=1.0\)). Errors relative to DFDC.}
    \begin{subtable}[t]{0.35\textwidth}
        \begin{center}
                    \begin{tabular}{ r | c | c | c }
            Output Value & DFDC & CSOR & \% Error \\
            \hline
            Rotor Thrust (N) & 70.0 & 70.19 & 0.27 \\
            Body Thrust (N) & 6.99 & 6.95 & -0.43 \\
            Torque (N\(\cdot\)m) & 5.5 & 5.52 & 0.3 \\
            Rotor Efficiency & 0.63 & 0.63 & 0.09 \\
            Total Efficiency & 0.69 & 0.69 & 0.02 \\
        \end{tabular}

        \end{center}
    \end{subtable}

    \begin{subtable}[t]{0.45\textwidth}
        \begin{center}
            \input{figures/single_rotor_J1_verification_table_DuctAPE.tex}
        \end{center}
    \end{subtable}
    \label{tab:cruisecompsinglerotor}
\end{table}

\begin{figure}[h!]
     \centering
     \begin{subfigure}[t]{0.45\textwidth}
         \centering
% \tikzsetnextfilename{solvers/single_rotor_cpct_comparison_CSOR}
        % Recommended preamble:
% \usetikzlibrary{arrows.meta}
% \usetikzlibrary{backgrounds}
% \usepgfplotslibrary{patchplots}
% \usepgfplotslibrary{fillbetween}
% \pgfplotsset{%
%     layers/standard/.define layer set={%
%         background,axis background,axis grid,axis ticks,axis lines,axis tick labels,pre main,main,axis descriptions,axis foreground%
%     }{
%         grid style={/pgfplots/on layer=axis grid},%
%         tick style={/pgfplots/on layer=axis ticks},%
%         axis line style={/pgfplots/on layer=axis lines},%
%         label style={/pgfplots/on layer=axis descriptions},%
%         legend style={/pgfplots/on layer=axis descriptions},%
%         title style={/pgfplots/on layer=axis descriptions},%
%         colorbar style={/pgfplots/on layer=axis descriptions},%
%         ticklabel style={/pgfplots/on layer=axis tick labels},%
%         axis background@ style={/pgfplots/on layer=axis background},%
%         3d box foreground style={/pgfplots/on layer=axis foreground},%
%     },
% }

\begin{tikzpicture}[/tikz/background rectangle/.style={fill={rgb,1:red,0.0;green,0.0;blue,0.0}, fill opacity={0.0}, draw opacity={0.0}}, show background rectangle]
\begin{axis}[point meta max={nan}, point meta min={nan}, legend cell align={left}, legend columns={1}, title={}, title style={at={{(0.5,1)}}, anchor={south}, font={{\fontsize{14 pt}{18.2 pt}\selectfont}}, color={rgb,1:red,0.0;green,0.0;blue,0.0}, draw opacity={1.0}, rotate={0.0}, align={center}}, legend style={color={rgb,1:red,0.0;green,0.0;blue,0.0}, draw opacity={0.0}, line width={1}, solid, fill={rgb,1:red,0.0;green,0.0;blue,0.0}, fill opacity={0.0}, text opacity={1.0}, font={{\fontsize{8 pt}{10.4 pt}\selectfont}}, text={rgb,1:red,0.0;green,0.0;blue,0.0}, cells={anchor={center}}, at={(1.02, 1)}, anchor={north west}}, axis background/.style={fill={rgb,1:red,0.0;green,0.0;blue,0.0}, opacity={0.0}}, anchor={north west}, xshift={5.0mm}, yshift={-5.0mm}, width={58.152mm}, height={38.768mm}, scaled x ticks={false}, xlabel={$\mathrm{Advance~Ratio}~\left(\frac{V_\infty}{nD}\right)$}, x tick style={color={rgb,1:red,0.0;green,0.0;blue,0.0}, opacity={1.0}}, x tick label style={color={rgb,1:red,0.0;green,0.0;blue,0.0}, opacity={1.0}, rotate={0}}, xlabel style={at={(ticklabel cs:0.5)}, anchor=near ticklabel, at={{(ticklabel cs:0.5)}}, anchor={near ticklabel}, font={{\fontsize{11 pt}{14.3 pt}\selectfont}}, color={rgb,1:red,0.0;green,0.0;blue,0.0}, draw opacity={1.0}, rotate={0.0}}, xmajorgrids={false}, xmin={-0.06000000000000005}, xmax={2.06}, xticklabels={{$0.0$,$0.5$,$1.0$,$1.5$,$2.0$}}, xtick={{0.0,0.5,1.0,1.5,2.0}}, xtick align={inside}, xticklabel style={font={{\fontsize{8 pt}{10.4 pt}\selectfont}}, color={rgb,1:red,0.0;green,0.0;blue,0.0}, draw opacity={1.0}, rotate={0.0}}, x grid style={color={rgb,1:red,0.0;green,0.0;blue,0.0}, draw opacity={0.1}, line width={0.5}, solid}, axis x line*={left}, x axis line style={color={rgb,1:red,0.0;green,0.0;blue,0.0}, draw opacity={1.0}, line width={1}, solid}, scaled y ticks={false}, ylabel={}, y tick style={color={rgb,1:red,0.0;green,0.0;blue,0.0}, opacity={1.0}}, y tick label style={color={rgb,1:red,0.0;green,0.0;blue,0.0}, opacity={1.0}, rotate={0}}, ylabel style={at={(ticklabel cs:0.5)}, anchor=near ticklabel, at={{(ticklabel cs:0.5)}}, anchor={near ticklabel}, font={{\fontsize{11 pt}{14.3 pt}\selectfont}}, color={rgb,1:red,0.0;green,0.0;blue,0.0}, draw opacity={1.0}, rotate={0.0}}, ymajorgrids={false}, ymin={-0.023334378635239017}, ymax={0.9983069998098729}, yticklabels={{$0.0$,$0.2$,$0.4$,$0.6$,$0.8$}}, ytick={{0.0,0.2,0.4,0.6000000000000001,0.8}}, ytick align={inside}, yticklabel style={font={{\fontsize{8 pt}{10.4 pt}\selectfont}}, color={rgb,1:red,0.0;green,0.0;blue,0.0}, draw opacity={1.0}, rotate={0.0}}, y grid style={color={rgb,1:red,0.0;green,0.0;blue,0.0}, draw opacity={0.1}, line width={0.5}, solid}, axis y line*={left}, y axis line style={color={rgb,1:red,0.0;green,0.0;blue,0.0}, draw opacity={1.0}, line width={1}, solid}, colorbar={false}]
    \addplot[color={rgb,1:red,0.0;green,0.3608;blue,0.6706}, name path={2cd8dbc9-072e-40a3-9058-02be8c252044}, draw opacity={1.0}, line width={2}, dashed, forget plot]
        table[row sep={\\}]
        {
            \\
            0.0  0.64763  \\
            0.1  0.64716  \\
            0.2  0.6448  \\
            0.3  0.64044  \\
            0.4  0.63401  \\
            0.5  0.62534  \\
            0.6  0.61428  \\
            0.7  0.6006  \\
            0.8  0.58411  \\
            0.9  0.56452  \\
            1.0  0.54158  \\
            1.1  0.51499  \\
            1.2  0.48446  \\
            1.3  0.44966  \\
            1.4  0.41031  \\
            1.5  0.36604  \\
            1.6  0.31654  \\
            1.7  0.26153  \\
            1.8  0.20061  \\
            1.9  0.13355  \\
            2.0  0.05993  \\
        }
        ;
    \addplot[color={rgb,1:red,0.0;green,0.3608;blue,0.6706}, name path={84ad8b86-7401-46a7-944f-903148769e13}, draw opacity={1.0}, line width={1.0}, solid, forget plot]
        table[row sep={\\}]
        {
            \\
            0.0  0.6476287159264367  \\
            0.1  0.647371654647963  \\
            0.2  0.6451967601427122  \\
            0.3  0.6410289875627373  \\
            0.4  0.634733091191257  \\
            0.5  0.6261957084095594  \\
            0.6  0.6152237993524527  \\
            0.7  0.6016475568072619  \\
            0.8  0.5851987856786138  \\
            0.9  0.5656912239188161  \\
            1.0  0.542763840922332  \\
            1.1  0.5162261680256004  \\
            1.2  0.4856810471236084  \\
            1.3  0.4507698669583917  \\
            1.4  0.4112445001146267  \\
            1.5  0.366778019697278  \\
            1.6  0.31714318825852233  \\
            1.7  0.2619153144006484  \\
            1.8  0.20081207639130014  \\
            1.9  0.1334344960580589  \\
            2.0  0.059482538174002134  \\
        }
        ;
    \addplot[color={rgb,1:red,0.7451;green,0.298;blue,0.302}, name path={bbc1084a-5e34-4ea9-b5c3-aa08f30e8334}, draw opacity={1.0}, line width={2}, dashed, forget plot]
        table[row sep={\\}]
        {
            \\
            0.0  0.96692  \\
            0.1  0.88394  \\
            0.2  0.80785  \\
            0.3  0.73801  \\
            0.4  0.67382  \\
            0.5  0.61468  \\
            0.6  0.56001  \\
            0.7  0.50925  \\
            0.8  0.46187  \\
            0.9  0.41738  \\
            1.0  0.37531  \\
            1.1  0.33522  \\
            1.2  0.2967  \\
            1.3  0.25937  \\
            1.4  0.2229  \\
            1.5  0.18694  \\
            1.6  0.15121  \\
            1.7  0.11547  \\
            1.8  0.07941  \\
            1.9  0.04287  \\
            2.0  0.00558  \\
        }
        ;
    \addplot[color={rgb,1:red,0.7451;green,0.298;blue,0.302}, name path={bceee90d-cf57-4883-8d32-12463597d281}, draw opacity={1.0}, line width={1.0}, solid, forget plot]
        table[row sep={\\}]
        {
            \\
            0.0  0.9693926211746339  \\
            0.1  0.886050766590429  \\
            0.2  0.8096645784047114  \\
            0.3  0.7396058808831841  \\
            0.4  0.6752194147096086  \\
            0.5  0.6159320016219962  \\
            0.6  0.5611321135172642  \\
            0.7  0.5102921432534201  \\
            0.8  0.46283933323260534  \\
            0.9  0.41832198129927645  \\
            1.0  0.3762063720769762  \\
            1.1  0.3361152863457382  \\
            1.2  0.2975871663012331  \\
            1.3  0.26021185781075223  \\
            1.4  0.2236799208927465  \\
            1.5  0.18766608187458816  \\
            1.6  0.15193143672519627  \\
            1.7  0.11614908894182809  \\
            1.8  0.08009197157348306  \\
            1.9  0.04348298870741282  \\
            2.0  0.006139971961726203  \\
        }
        ;
    \node[left, , color={rgb,1:red,0.7451;green,0.298;blue,0.302}, draw opacity={1.0}, rotate={0.0}, font={{\fontsize{8 pt}{10.4 pt}\selectfont}}]  at (axis cs:1.0,0.3) {$C_T$};
    \node[left, , color={rgb,1:red,0.0;green,0.3608;blue,0.6706}, draw opacity={1.0}, rotate={0.0}, font={{\fontsize{8 pt}{10.4 pt}\selectfont}}]  at (axis cs:1.0,0.7) {$C_P$};
\end{axis}
\end{tikzpicture}

        \caption{Power and thrust comparison.}
        \label{fig:singlerotorcpct}
     \end{subfigure}
\hfill
     \begin{subfigure}[t]{0.45\textwidth}
         \centering
% \tikzsetnextfilename{solvers/single_rotor_efficiency_comparison_CSOR}
         % Recommended preamble:
% \usetikzlibrary{arrows.meta}
% \usetikzlibrary{backgrounds}
% \usepgfplotslibrary{patchplots}
% \usepgfplotslibrary{fillbetween}
% \pgfplotsset{%
%     layers/standard/.define layer set={%
%         background,axis background,axis grid,axis ticks,axis lines,axis tick labels,pre main,main,axis descriptions,axis foreground%
%     }{
%         grid style={/pgfplots/on layer=axis grid},%
%         tick style={/pgfplots/on layer=axis ticks},%
%         axis line style={/pgfplots/on layer=axis lines},%
%         label style={/pgfplots/on layer=axis descriptions},%
%         legend style={/pgfplots/on layer=axis descriptions},%
%         title style={/pgfplots/on layer=axis descriptions},%
%         colorbar style={/pgfplots/on layer=axis descriptions},%
%         ticklabel style={/pgfplots/on layer=axis tick labels},%
%         axis background@ style={/pgfplots/on layer=axis background},%
%         3d box foreground style={/pgfplots/on layer=axis foreground},%
%     },
% }

\begin{tikzpicture}[/tikz/background rectangle/.style={fill={rgb,1:red,0.0;green,0.0;blue,0.0}, fill opacity={0.0}, draw opacity={0.0}}, show background rectangle]
\begin{axis}[point meta max={nan}, point meta min={nan}, legend cell align={left}, legend columns={1}, title={}, title style={at={{(0.5,1)}}, anchor={south}, font={{\fontsize{14 pt}{18.2 pt}\selectfont}}, color={rgb,1:red,0.0;green,0.0;blue,0.0}, draw opacity={1.0}, rotate={0.0}, align={center}}, legend style={color={rgb,1:red,0.0;green,0.0;blue,0.0}, draw opacity={0.0}, line width={1}, solid, fill={rgb,1:red,0.0;green,0.0;blue,0.0}, fill opacity={0.0}, text opacity={1.0}, font={{\fontsize{8 pt}{10.4 pt}\selectfont}}, text={rgb,1:red,0.0;green,0.0;blue,0.0}, cells={anchor={center}}, at={(1.02, 1)}, anchor={north west}}, axis background/.style={fill={rgb,1:red,0.0;green,0.0;blue,0.0}, opacity={0.0}}, anchor={north west}, xshift={5.0mm}, yshift={-5.0mm}, width={58.152mm}, height={38.768mm}, scaled x ticks={false}, xlabel={$\mathrm{Advance~Ratio}~\left(\frac{V_\infty}{nD}\right)$}, x tick style={color={rgb,1:red,0.0;green,0.0;blue,0.0}, opacity={1.0}}, x tick label style={color={rgb,1:red,0.0;green,0.0;blue,0.0}, opacity={1.0}, rotate={0}}, xlabel style={at={(ticklabel cs:0.5)}, anchor=near ticklabel, at={{(ticklabel cs:0.5)}}, anchor={near ticklabel}, font={{\fontsize{11 pt}{14.3 pt}\selectfont}}, color={rgb,1:red,0.0;green,0.0;blue,0.0}, draw opacity={1.0}, rotate={0.0}}, xmajorgrids={false}, xmin={-0.06000000000000005}, xmax={2.06}, xticklabels={{$0.0$,$0.5$,$1.0$,$1.5$,$2.0$}}, xtick={{0.0,0.5,1.0,1.5,2.0}}, xtick align={inside}, xticklabel style={font={{\fontsize{8 pt}{10.4 pt}\selectfont}}, color={rgb,1:red,0.0;green,0.0;blue,0.0}, draw opacity={1.0}, rotate={0.0}}, x grid style={color={rgb,1:red,0.0;green,0.0;blue,0.0}, draw opacity={0.1}, line width={0.5}, solid}, axis x line*={left}, x axis line style={color={rgb,1:red,0.0;green,0.0;blue,0.0}, draw opacity={1.0}, line width={1}, solid}, scaled y ticks={false}, ylabel={$\eta$}, y tick style={color={rgb,1:red,0.0;green,0.0;blue,0.0}, opacity={1.0}}, y tick label style={color={rgb,1:red,0.0;green,0.0;blue,0.0}, opacity={1.0}, rotate={0}}, ylabel style={{rotate=-90}}, ymajorgrids={false}, ymin={-0.02302475402241}, ymax={0.7905165547694095}, yticklabels={{$0.0$,$0.2$,$0.4$,$0.6$}}, ytick={{0.0,0.2,0.4,0.6000000000000001}}, ytick align={inside}, yticklabel style={font={{\fontsize{8 pt}{10.4 pt}\selectfont}}, color={rgb,1:red,0.0;green,0.0;blue,0.0}, draw opacity={1.0}, rotate={0.0}}, y grid style={color={rgb,1:red,0.0;green,0.0;blue,0.0}, draw opacity={0.1}, line width={0.5}, solid}, axis y line*={left}, y axis line style={color={rgb,1:red,0.0;green,0.0;blue,0.0}, draw opacity={1.0}, line width={1}, solid}, colorbar={false}]
    \addplot[color={rgb,1:red,0.7451;green,0.298;blue,0.302}, name path={00dc078f-ed87-49f9-9e7e-1aea920bf252}, draw opacity={1.0}, line width={2}, dashed, forget plot]
        table[row sep={\\}]
        {
            \\
            0.0  0.0  \\
            0.1  0.1366  \\
            0.2  0.2506  \\
            0.3  0.3457  \\
            0.4  0.4251  \\
            0.5  0.4915  \\
            0.6  0.547  \\
            0.7  0.5935  \\
            0.8  0.6326  \\
            0.9  0.6654  \\
            1.0  0.693  \\
            1.1  0.716  \\
            1.2  0.7349  \\
            1.3  0.7499  \\
            1.4  0.7606  \\
            1.5  0.7661  \\
            1.6  0.7643  \\
            1.7  0.7506  \\
            1.8  0.7126  \\
            1.9  0.61  \\
            2.0  0.1861  \\
        }
        ;
    \addplot[color={rgb,1:red,0.0;green,0.3608;blue,0.6706}, name path={6088c822-7a73-4822-8111-b5cbad1f4b72}, draw opacity={1.0}, line width={1.0}, solid, forget plot]
        table[row sep={\\}]
        {
            \\
            0.0  0.0  \\
            0.1  0.13686894695323948  \\
            0.2  0.25098222074941  \\
            0.3  0.34613374522823714  \\
            0.4  0.42551392015334677  \\
            0.5  0.4918047132472121  \\
            0.6  0.5472468205305563  \\
            0.7  0.5937105473725457  \\
            0.8  0.632727674164099  \\
            0.9  0.6655393742211919  \\
            1.0  0.6931308678147748  \\
            1.1  0.7162109127369477  \\
            1.2  0.7352656680271793  \\
            1.3  0.7504392816594427  \\
            1.4  0.7614737441170885  \\
            1.5  0.7674918007469995  \\
            1.6  0.7665001417661118  \\
            1.7  0.7538828023590322  \\
            1.8  0.7179127442084221  \\
            1.9  0.6191628175980557  \\
            2.0  0.20644619917748516  \\
        }
        ;
\end{axis}
\end{tikzpicture}

         \caption{Efficiency comparison.}
        \label{fig:singlerotoreta}
     \end{subfigure}
    \caption{Comparison of power and thrust coefficients and efficiency for DFDC (dashed) and the DuctAPE implementations (solid) across a range of advance ratios.}
    \label{fig:singlerotorcpcteta}
\end{figure}


\begin{figure}[h!]
     \centering
     \begin{subfigure}[t]{0.45\textwidth}
         \centering
% \tikzsetnextfilename{solvers/single_rotor_cpct_comparison_DuctAPE}
        % Recommended preamble:
% \usetikzlibrary{arrows.meta}
% \usetikzlibrary{backgrounds}
% \usepgfplotslibrary{patchplots}
% \usepgfplotslibrary{fillbetween}
% \pgfplotsset{%
%     layers/standard/.define layer set={%
%         background,axis background,axis grid,axis ticks,axis lines,axis tick labels,pre main,main,axis descriptions,axis foreground%
%     }{
%         grid style={/pgfplots/on layer=axis grid},%
%         tick style={/pgfplots/on layer=axis ticks},%
%         axis line style={/pgfplots/on layer=axis lines},%
%         label style={/pgfplots/on layer=axis descriptions},%
%         legend style={/pgfplots/on layer=axis descriptions},%
%         title style={/pgfplots/on layer=axis descriptions},%
%         colorbar style={/pgfplots/on layer=axis descriptions},%
%         ticklabel style={/pgfplots/on layer=axis tick labels},%
%         axis background@ style={/pgfplots/on layer=axis background},%
%         3d box foreground style={/pgfplots/on layer=axis foreground},%
%     },
% }

\begin{tikzpicture}[/tikz/background rectangle/.style={fill={rgb,1:red,0.0;green,0.0;blue,0.0}, fill opacity={0.0}, draw opacity={0.0}}, show background rectangle]
\begin{axis}[point meta max={nan}, point meta min={nan}, legend cell align={left}, legend columns={1}, title={}, title style={at={{(0.5,1)}}, anchor={south}, font={{\fontsize{14 pt}{18.2 pt}\selectfont}}, color={rgb,1:red,0.0;green,0.0;blue,0.0}, draw opacity={1.0}, rotate={0.0}, align={center}}, legend style={color={rgb,1:red,0.0;green,0.0;blue,0.0}, draw opacity={0.0}, line width={1}, solid, fill={rgb,1:red,0.0;green,0.0;blue,0.0}, fill opacity={0.0}, text opacity={1.0}, font={{\fontsize{8 pt}{10.4 pt}\selectfont}}, text={rgb,1:red,0.0;green,0.0;blue,0.0}, cells={anchor={center}}, at={(1.02, 1)}, anchor={north west}}, axis background/.style={fill={rgb,1:red,0.0;green,0.0;blue,0.0}, opacity={0.0}}, anchor={north west}, xshift={5.0mm}, yshift={-5.0mm}, width={58.152mm}, height={38.768mm}, scaled x ticks={false}, xlabel={$\mathrm{Advance~Ratio}~\left(\frac{V_\infty}{nD}\right)$}, x tick style={color={rgb,1:red,0.0;green,0.0;blue,0.0}, opacity={1.0}}, x tick label style={color={rgb,1:red,0.0;green,0.0;blue,0.0}, opacity={1.0}, rotate={0}}, xlabel style={at={(ticklabel cs:0.5)}, anchor=near ticklabel, at={{(ticklabel cs:0.5)}}, anchor={near ticklabel}, font={{\fontsize{11 pt}{14.3 pt}\selectfont}}, color={rgb,1:red,0.0;green,0.0;blue,0.0}, draw opacity={1.0}, rotate={0.0}}, xmajorgrids={false}, xmin={-0.06000000000000005}, xmax={2.06}, xticklabels={{$0.0$,$0.5$,$1.0$,$1.5$,$2.0$}}, xtick={{0.0,0.5,1.0,1.5,2.0}}, xtick align={inside}, xticklabel style={font={{\fontsize{8 pt}{10.4 pt}\selectfont}}, color={rgb,1:red,0.0;green,0.0;blue,0.0}, draw opacity={1.0}, rotate={0.0}}, x grid style={color={rgb,1:red,0.0;green,0.0;blue,0.0}, draw opacity={0.1}, line width={0.5}, solid}, axis x line*={left}, x axis line style={color={rgb,1:red,0.0;green,0.0;blue,0.0}, draw opacity={1.0}, line width={1}, solid}, scaled y ticks={false}, ylabel={}, y tick style={color={rgb,1:red,0.0;green,0.0;blue,0.0}, opacity={1.0}}, y tick label style={color={rgb,1:red,0.0;green,0.0;blue,0.0}, opacity={1.0}, rotate={0}}, ylabel style={at={(ticklabel cs:0.5)}, anchor=near ticklabel, at={{(ticklabel cs:0.5)}}, anchor={near ticklabel}, font={{\fontsize{11 pt}{14.3 pt}\selectfont}}, color={rgb,1:red,0.0;green,0.0;blue,0.0}, draw opacity={1.0}, rotate={0.0}}, ymajorgrids={false}, ymin={-0.02333312710793245}, ymax={0.9982640307056823}, yticklabels={{$0.0$,$0.2$,$0.4$,$0.6$,$0.8$}}, ytick={{0.0,0.2,0.4,0.6000000000000001,0.8}}, ytick align={inside}, yticklabel style={font={{\fontsize{8 pt}{10.4 pt}\selectfont}}, color={rgb,1:red,0.0;green,0.0;blue,0.0}, draw opacity={1.0}, rotate={0.0}}, y grid style={color={rgb,1:red,0.0;green,0.0;blue,0.0}, draw opacity={0.1}, line width={0.5}, solid}, axis y line*={left}, y axis line style={color={rgb,1:red,0.0;green,0.0;blue,0.0}, draw opacity={1.0}, line width={1}, solid}, colorbar={false}]
    \addplot[color={rgb,1:red,0.0;green,0.3608;blue,0.6706}, name path={5431e7a8-21dd-4c52-986c-5b8f7136f93b}, draw opacity={1.0}, line width={2}, dashed, forget plot]
        table[row sep={\\}]
        {
            \\
            0.0  0.64763  \\
            0.1  0.64716  \\
            0.2  0.6448  \\
            0.3  0.64044  \\
            0.4  0.63401  \\
            0.5  0.62534  \\
            0.6  0.61428  \\
            0.7  0.6006  \\
            0.8  0.58411  \\
            0.9  0.56452  \\
            1.0  0.54158  \\
            1.1  0.51499  \\
            1.2  0.48446  \\
            1.3  0.44966  \\
            1.4  0.41031  \\
            1.5  0.36604  \\
            1.6  0.31654  \\
            1.7  0.26153  \\
            1.8  0.20061  \\
            1.9  0.13355  \\
            2.0  0.05993  \\
        }
        ;
    \addplot[color={rgb,1:red,0.0;green,0.3608;blue,0.6706}, name path={9451e800-433f-4761-b194-b43981a0b2f7}, draw opacity={1.0}, line width={1.0}, solid, forget plot]
        table[row sep={\\}]
        {
            \\
            0.0  0.6476006934803172  \\
            0.1  0.6473639113933127  \\
            0.2  0.6451964053543922  \\
            0.3  0.6410232737381847  \\
            0.4  0.6347375855003032  \\
            0.5  0.6261995580747164  \\
            0.6  0.6152374983752142  \\
            0.7  0.6016498966512803  \\
            0.8  0.585208234983065  \\
            0.9  0.565660191565568  \\
            1.0  0.5427330158319916  \\
            1.1  0.5161369255543797  \\
            1.2  0.4855684352854993  \\
            1.3  0.45071356586707495  \\
            1.4  0.4112509102403113  \\
            1.5  0.3668545470038746  \\
            1.6  0.3171968062700479  \\
            1.7  0.2619509075965668  \\
            1.8  0.20079351077297322  \\
            1.9  0.13340724930972941  \\
            2.0  0.059483355514814915  \\
        }
        ;
    \addplot[color={rgb,1:red,0.7451;green,0.298;blue,0.302}, name path={3d38da6c-fc9c-4595-bb68-1ba0d27abcb6}, draw opacity={1.0}, line width={2}, dashed, forget plot]
        table[row sep={\\}]
        {
            \\
            0.0  0.96692  \\
            0.1  0.88394  \\
            0.2  0.80785  \\
            0.3  0.73801  \\
            0.4  0.67382  \\
            0.5  0.61468  \\
            0.6  0.56001  \\
            0.7  0.50925  \\
            0.8  0.46187  \\
            0.9  0.41738  \\
            1.0  0.37531  \\
            1.1  0.33522  \\
            1.2  0.2967  \\
            1.3  0.25937  \\
            1.4  0.2229  \\
            1.5  0.18694  \\
            1.6  0.15121  \\
            1.7  0.11547  \\
            1.8  0.07941  \\
            1.9  0.04287  \\
            2.0  0.00558  \\
        }
        ;
    \addplot[color={rgb,1:red,0.7451;green,0.298;blue,0.302}, name path={8e33f02b-d58d-4a00-b45d-e74817037597}, draw opacity={1.0}, line width={1.0}, solid, forget plot]
        table[row sep={\\}]
        {
            \\
            0.0  0.9693509035977498  \\
            0.1  0.8860318441412766  \\
            0.2  0.8096555611526135  \\
            0.3  0.7395906519727339  \\
            0.4  0.6752158810904156  \\
            0.5  0.6159258131332127  \\
            0.6  0.5611355675205223  \\
            0.7  0.5102847110084315  \\
            0.8  0.46284024810561747  \\
            0.9  0.418298699267034  \\
            1.0  0.37618731497340446  \\
            1.1  0.3360645256196932  \\
            1.2  0.2975197576810866  \\
            1.3  0.26017275495468734  \\
            1.4  0.2236725361390593  \\
            1.5  0.18769610478799587  \\
            1.6  0.151947011482736  \\
            1.7  0.11615385507052457  \\
            1.8  0.08006880197265562  \\
            1.9  0.043466200480338876  \\
            2.0  0.006141370893629546  \\
        }
        ;
    \node[left, , color={rgb,1:red,0.7451;green,0.298;blue,0.302}, draw opacity={1.0}, rotate={0.0}, font={{\fontsize{8 pt}{10.4 pt}\selectfont}}]  at (axis cs:1.0,0.3) {$C_T$};
    \node[left, , color={rgb,1:red,0.0;green,0.3608;blue,0.6706}, draw opacity={1.0}, rotate={0.0}, font={{\fontsize{8 pt}{10.4 pt}\selectfont}}]  at (axis cs:1.0,0.7) {$C_P$};
\end{axis}
\end{tikzpicture}

        \caption{Power and Thrust Comparison.}
        \label{fig:singlerotorcpct}
     \end{subfigure}
\hfill
     \begin{subfigure}[t]{0.45\textwidth}
         \centering
% \tikzsetnextfilename{solvers/single_rotor_efficiency_comparison_DuctAPE}
         % Recommended preamble:
% \usetikzlibrary{arrows.meta}
% \usetikzlibrary{backgrounds}
% \usepgfplotslibrary{patchplots}
% \usepgfplotslibrary{fillbetween}
% \pgfplotsset{%
%     layers/standard/.define layer set={%
%         background,axis background,axis grid,axis ticks,axis lines,axis tick labels,pre main,main,axis descriptions,axis foreground%
%     }{
%         grid style={/pgfplots/on layer=axis grid},%
%         tick style={/pgfplots/on layer=axis ticks},%
%         axis line style={/pgfplots/on layer=axis lines},%
%         label style={/pgfplots/on layer=axis descriptions},%
%         legend style={/pgfplots/on layer=axis descriptions},%
%         title style={/pgfplots/on layer=axis descriptions},%
%         colorbar style={/pgfplots/on layer=axis descriptions},%
%         ticklabel style={/pgfplots/on layer=axis tick labels},%
%         axis background@ style={/pgfplots/on layer=axis background},%
%         3d box foreground style={/pgfplots/on layer=axis foreground},%
%     },
% }

\begin{tikzpicture}[/tikz/background rectangle/.style={fill={rgb,1:red,0.0;green,0.0;blue,0.0}, fill opacity={0.0}, draw opacity={0.0}}, show background rectangle]
\begin{axis}[point meta max={nan}, point meta min={nan}, legend cell align={left}, legend columns={1}, title={}, title style={at={{(0.5,1)}}, anchor={south}, font={{\fontsize{14 pt}{18.2 pt}\selectfont}}, color={rgb,1:red,0.0;green,0.0;blue,0.0}, draw opacity={1.0}, rotate={0.0}, align={center}}, legend style={color={rgb,1:red,0.0;green,0.0;blue,0.0}, draw opacity={0.0}, line width={1}, solid, fill={rgb,1:red,0.0;green,0.0;blue,0.0}, fill opacity={0.0}, text opacity={1.0}, font={{\fontsize{8 pt}{10.4 pt}\selectfont}}, text={rgb,1:red,0.0;green,0.0;blue,0.0}, cells={anchor={center}}, at={(1.02, 1)}, anchor={north west}}, axis background/.style={fill={rgb,1:red,0.0;green,0.0;blue,0.0}, opacity={0.0}}, anchor={north west}, xshift={5.0mm}, yshift={-5.0mm}, width={58.152mm}, height={38.768mm}, scaled x ticks={false}, xlabel={$\mathrm{Advance~Ratio}~\left(\frac{V_\infty}{nD}\right)$}, x tick style={color={rgb,1:red,0.0;green,0.0;blue,0.0}, opacity={1.0}}, x tick label style={color={rgb,1:red,0.0;green,0.0;blue,0.0}, opacity={1.0}, rotate={0}}, xlabel style={at={(ticklabel cs:0.5)}, anchor=near ticklabel, at={{(ticklabel cs:0.5)}}, anchor={near ticklabel}, font={{\fontsize{11 pt}{14.3 pt}\selectfont}}, color={rgb,1:red,0.0;green,0.0;blue,0.0}, draw opacity={1.0}, rotate={0.0}}, xmajorgrids={false}, xmin={-0.06000000000000005}, xmax={2.06}, xticklabels={{$0.0$,$0.5$,$1.0$,$1.5$,$2.0$}}, xtick={{0.0,0.5,1.0,1.5,2.0}}, xtick align={inside}, xticklabel style={font={{\fontsize{8 pt}{10.4 pt}\selectfont}}, color={rgb,1:red,0.0;green,0.0;blue,0.0}, draw opacity={1.0}, rotate={0.0}}, x grid style={color={rgb,1:red,0.0;green,0.0;blue,0.0}, draw opacity={0.1}, line width={0.5}, solid}, axis x line*={left}, x axis line style={color={rgb,1:red,0.0;green,0.0;blue,0.0}, draw opacity={1.0}, line width={1}, solid}, scaled y ticks={false}, ylabel={$\eta$}, y tick style={color={rgb,1:red,0.0;green,0.0;blue,0.0}, opacity={1.0}}, y tick label style={color={rgb,1:red,0.0;green,0.0;blue,0.0}, opacity={1.0}, rotate={0}}, ylabel style={{rotate=-90}}, ymajorgrids={false}, ymin={-0.02302363371107574}, ymax={0.7904780907469331}, yticklabels={{$0.0$,$0.2$,$0.4$,$0.6$}}, ytick={{0.0,0.2,0.4,0.6000000000000001}}, ytick align={inside}, yticklabel style={font={{\fontsize{8 pt}{10.4 pt}\selectfont}}, color={rgb,1:red,0.0;green,0.0;blue,0.0}, draw opacity={1.0}, rotate={0.0}}, y grid style={color={rgb,1:red,0.0;green,0.0;blue,0.0}, draw opacity={0.1}, line width={0.5}, solid}, axis y line*={left}, y axis line style={color={rgb,1:red,0.0;green,0.0;blue,0.0}, draw opacity={1.0}, line width={1}, solid}, colorbar={false}]
    \addplot[color={rgb,1:red,0.7451;green,0.298;blue,0.302}, name path={c25a651e-505b-406a-b637-fc137ad46bdf}, draw opacity={1.0}, line width={2}, dashed, forget plot]
        table[row sep={\\}]
        {
            \\
            0.0  0.0  \\
            0.1  0.1366  \\
            0.2  0.2506  \\
            0.3  0.3457  \\
            0.4  0.4251  \\
            0.5  0.4915  \\
            0.6  0.547  \\
            0.7  0.5935  \\
            0.8  0.6326  \\
            0.9  0.6654  \\
            1.0  0.693  \\
            1.1  0.716  \\
            1.2  0.7349  \\
            1.3  0.7499  \\
            1.4  0.7606  \\
            1.5  0.7661  \\
            1.6  0.7643  \\
            1.7  0.7506  \\
            1.8  0.7126  \\
            1.9  0.61  \\
            2.0  0.1861  \\
        }
        ;
    \addplot[color={rgb,1:red,0.0;green,0.3608;blue,0.6706}, name path={2268dd9f-67d3-4604-8ea6-0652c4984ffa}, draw opacity={1.0}, line width={1.0}, solid, forget plot]
        table[row sep={\\}]
        {
            \\
            0.0  0.0  \\
            0.1  0.13686766107092413  \\
            0.2  0.25097956356650436  \\
            0.3  0.3461297033693075  \\
            0.4  0.4255086804467124  \\
            0.5  0.49179674848933885  \\
            0.6  0.5472380038626676  \\
            0.7  0.5936995912307734  \\
            0.8  0.6327187082307706  \\
            0.9  0.6655388428490688  \\
            1.0  0.693135121689109  \\
            1.1  0.716226566786713  \\
            1.2  0.7352696000665389  \\
            1.3  0.750420238162619  \\
            1.4  0.7614367355726976  \\
            1.5  0.7674544570358574  \\
            1.6  0.7664491368346241  \\
            1.7  0.7538112978176978  \\
            1.8  0.7177714209785072  \\
            1.9  0.6190501741093984  \\
            2.0  0.20649039854854792  \\
        }
        ;
\end{axis}
\end{tikzpicture}

         \caption{Efficiency comparison.}
        \label{fig:singlerotoreta}
     \end{subfigure}
    \caption{Comparison of power and thrust coefficients and efficiency for DFDC (dashed) and DuctAPE (solid) across a range of advance ratios.}
    \label{fig:singlerotorcpcteta}
\end{figure}

\subsection{Benchmarking Solver Implementations}
\label{ssec:benchmarking}

Now that we have shown that both solve approaches yield nearly identical results, we show a comparison in solver efficiency.
%
To do so, we benchmarked various solvers against the CSOR solver.
%
Included in our comparison are the following external solvers:
%
\begin{itemize}
    \item Fixed-point Solvers
        \begin{itemize}
        \item NLsolve's \scite{Mogensen_2020} Anderson accelerated fixed-point method \scite{Walker_2011}.
        \item SpeedMapping.jl\footnote{\url{https://github.com/NicolasL-S/SpeedMapping.jl}} which uses an alternating cyclic extrapolation algorithm \scite{Lepage_2021}.
        \item Fixedpoint.jl\footnote{\url{https://github.com/francescoalemanno/FixedPoint.jl}} which is a Nesterov accelerated fixed-point method.
        \end{itemize}
    \item Quasi-Newton Solvers
        \begin{itemize}
            \item The modified Powell Method \scite{Powell_1970} implemented in MINPACK's HYBRJ method,\footnote{\url{https://www.netlib.org/minpack/}} accessed through the Julia wrapper package, MINPACK.jl\footnote{\url{https://github.com/sglyon/MINPACK.jl}} which wraps a C++ re-write of MINPACK.
        \end{itemize}
    \item Newton Solvers
        \begin{itemize}
        \item NLsolve's Newton method using automatic differentiation for the Jacobian calculation and the Mor\'{e}-Thuente line search method \scite{More_1994} option available through the LineSearches.jl\footnote{\url{https://github.com/JuliaNLSolvers/LineSearches.jl}} package.
        \item The Newton-Raphson method implemented in the SimpleNonlinearSolve.jl package\footnote{\url{https://github.com/SciML/SimpleNonlinearSolve.jl}} \scite{pal2024}.
    \end{itemize}
\end{itemize}
%
\noindent Other than those specifically noted in this list, all solvers were set to use their default settings and given absolute convergence tolerances of 1e-12.


To perform the benchmarks, we used the same geometry and operating points used in \cref{ssec:verification}.
%
For each advance ratio, we used BenchmarkTools.jl\footnote{\url{https://github.com/JuliaCI/BenchmarkTools.jl}}, a benchmarking package in the Julia language, to run 200 samples, then calculated the median computational time across all samples and all advance ratios.
%
We allowed the fixed-point solvers an iteration limit of 1000, the quasi-Newton solvers an iteration limit of 100, and the Newton solvers an iteration limit of 25.
%
These iteration limits were sufficiently large that all solvers were able to converged on every analysis.
%
We also ran each advance ratio one additional time, saving the solve iteration counts and taking the mean number of iterations across the advance ratios in order to determine solver efficiency.
%
% Given the ducted fan design and iteration limits, all the solvers tested had a 100\% convergence rate, though convergence rates drop if iterations are limited further.

\renewcommand{\arraystretch}{1.5}
\begin{table}[h!]
    \caption{
    Comparison of benchmarked solver method median times and mean iterations.
    \primary{Blue} indicates fixed-point solvers, \secondary{red} indicates quasi-Newton solvers, and \tertiary{green} indicates Newton solvers.
    In all cases, except for CSOR Default, the solvers were given absolute convergence criteria of 1e-12.
    (Note that the SimpleNonlinearSolve.jl package does not have any iteration tracing functionality and so that information is missing from this table.)
    }
    \begin{center}
        
\begin{tabular}{ r | S[detect-weight,table-format=2.4] | S[detect-weight,table-format=3.3] }
Method                                 & {\thead{Median Time\\(seconds)}}                 & {\thead{Mean\\Iterations}}              \\
\hline
\makecell{\color{primary} CSOR Default} & \color{primary} 0.0042 & \color{primary} 15.571 \\
\makecell{\color{primary} CSOR Absolute} & \color{primary} 0.0183 & \color{primary} 76.476 \\
\makecell{\color{primary} NLSolve's Anderson Acceleration} & \color{primary} 0.0097 & \color{primary} 36.429 \\
\makecell{\color{primary} SpeedMapping.jl} & \color{primary} 0.0300 & \color{primary} 139.333 \\
\makecell{\color{primary} FixedPoint.jl (Nesterov Acceleration)} & \color{primary} 0.1399 & \color{primary} 592.286 \\
\makecell{\color{secondary} MINPACK's HYBRJ} & \color{secondary}3.0528 & \color{secondary}14.238 \\
\makecell{\color{tertiary} SimpleNonlinearSolve's Newton Raphson} & \color{tertiary} 10.7100 & \color{tertiary} \color{white} 0.0 \\
\makecell{\color{tertiary} NLSolve's Newton Method} & \color{tertiary} 22.0116 & \color{tertiary} 16.714 \\
\end{tabular}

    \end{center}
    \label{tab:benchmarkcomp}
\end{table}


\Cref{tab:benchmarkcomp} includes comparisons of the median solve times and mean number of iterations across all advance ratios for each of the benchmarked solvers.
%
From \cref{tab:benchmarkcomp}, we first see that the default DFDC-like CSOR solve approach with loose, relative tolerances was very fast and efficient.
%
We should expect this as the default convergence criteria is between 1e-3 and 2e-4, depending on the residual value. %(see \cref{eqn:convergencecrit}).
%
In contrast, all other solvers were given an absolute convergence tolerance of 1e-12, including the CSOR solver with absolute convergence criteria. % (see \cref{eqn:convergencecritabs}).
%
Therefore, for tight, absolute tolerances, an Anderson accelerated fixed-point solver may be considered in favor of the CSOR solver if speed is the absolute priority, though a much broader set of benchmarks would need to be run before making that a general recommendation.
%
In addition, selecting non-default options for the various solvers may lead to increases in speed or efficiency, again requiring a broader set of benchmarks before general recommendations can be made.
%
That being said, based on these results, further exploration is worth pursuing.


Another important result to notice here is the cost of computing the Jacobian of the residual.
%
Looking at the quasi- and full Newton methods, we see several orders of magnitude increase in time, despite the lower number of overall iterations.
%
As expected, the Jacobian-based methods are more efficient in iterations, but the cost to compute the Jacobian is so high that it outweighs any inherent efficiency of the method.

