% ------------- sidenotes ---------------
% create a new footnote series for sidenotes
%Note: alph here is lowercase letters. the alphalph package goes to aa, ab, ac, ... after z
\usepackage{alphalph}
\newcounter{snote}[chapter]
\newcommand\sidenote[1]{\stepcounter{snote}\textsuperscript{%
    \itshape\alphalph{\value{snote}}}\marginpar{\scriptsize\textsuperscript{%
    \scriptsize\itshape\alphalph{\value{snote}}} #1}}
% ---------------------------------------

% ------------------ tikz ------------
%%tikz
\RequirePackage{pgf,tikz}
\usepackage{pgfplots}
\pgfplotsset{compat=newest}
\usetikzlibrary{arrows.meta}
\usetikzlibrary{arrows}
\usetikzlibrary{shapes.geometric}
\usetikzlibrary{shapes.misc}
\usetikzlibrary{backgrounds}
\usetikzlibrary{bending}
\usetikzlibrary{shadows}
\usetikzlibrary{patterns}
\usetikzlibrary{patterns.meta}
\usetikzlibrary{intersections}
\usepgfplotslibrary{patchplots}
\usepgfplotslibrary{fillbetween}
\usepgfplotslibrary{groupplots}
\usepgfplotslibrary{polar}
\usepgfplotslibrary{smithchart}
\usepgfplotslibrary{statistics}
\usepgfplotslibrary{dateplot}
\usepgfplotslibrary{ternary}
\pgfplotsset{%
	layers/standard/.define layer set={%
		background,axis background,axis grid,axis ticks,axis lines,axis tick labels,pre main,main,axis descriptions,axis foreground%
	}{grid style= {/pgfplots/on layer=axis grid},%
		tick style= {/pgfplots/on layer=axis ticks},%
		axis line style= {/pgfplots/on layer=axis lines},%
		label style= {/pgfplots/on layer=axis descriptions},%
		legend style= {/pgfplots/on layer=axis descriptions},%
		title style= {/pgfplots/on layer=axis descriptions},%
		colorbar style= {/pgfplots/on layer=axis descriptions},%
		ticklabel style= {/pgfplots/on layer=axis tick labels},%
		axis background@ style={/pgfplots/on layer=axis background},%
		3d box foreground style={/pgfplots/on layer=axis foreground},%
	},
}
% new style at automates partial ellipse
\tikzset{
    partial ellipse/.style args={#1:#2:#3}{
        insert path={+ (#1:#3) arc (#1:#2:#3)}
    }
}

\RequirePackage{nicefrac}
\RequirePackage{cancel}


%\RequirePackage{marginfix}
%\let\evensidemargin\oddsidemargin
%\reversemarginpar
% ----------------------------------------



% -------- custom title page -------------
% This title page follows immediately after the title page required by BYU Grad Studies.
% This title page can be customized as the student desires, but should include the title,
% student name, department, degree, and the BYU Engineering logo.
\RequirePackage{pagecolor}
\RequirePackage{afterpage}
\newcommand{\mycustomtitlepage}{%
	\pagestyle{empty}
	\newgeometry{margin=.5in}
	\newpagecolor{navy}\afterpage{\restorepagecolor}
	\begin{Spacing}{2.2}
		\noindent\raggedleft
		{\color{white}\Huge\sffamily\itshape\@customtitle} \\
%		\vspace{\fill}
	\end{Spacing}
	\begin{Spacing}{1.4}
		{\color{white}\bfseries\Large
		\noindent
		\@author \\
		\@department \\
		\@degree} \\
		\vspace{\fill}
		\begin{figure}[htbp]
			\includegraphics[width=3.5in]{frontmatter/figures/college_logo}
		\end{figure}
	\end{Spacing}
\restoregeometry
}
% ----------------------------------------



% --------------- Part Page Format ---------------
\renewcommand*\part{%
	\cleardoublepage
	\thispagestyle{empty}
		\setlrmarginsandblock{1.75in}{1.75in}{*}
		\checkandfixthelayout
	\if@twocolumn
	\onecolumn
	\@tempswatrue
	\else
	\@tempswafalse
	\fi
	\null\vfil
	\secdef\@part\@spart
}
\newcommand{\resetpageformat}{%
    \setlrmarginsandblock{1.0in}{2.75in}{*}
    \checkandfixthelayout
    \setmarginnotes{0.125in}{2.0in}{\onelineskip}
}
\renewcommand*{\thepart}{\Roman{part}}
% \renewcommand*{\thepart}{}
\renewcommand*{\partnamefont}{\normalfont\LARGE\sffamily\bfseries\colorchaptitle}
\renewcommand*{\partnumfont}{\normalfont\LARGE\sffamily\bfseries\colorchaptitle}
\renewcommand*{\parttitlefont}{\normalfont\LARGE\sffamily\itshape\colorchaptitle}
% don't print the part number or word part
\renewcommand*{\printpartname}{}
\renewcommand*{\printpartnum}{}
% \renewcommand*{\cftpartfont}{\cftchapterfont}
% \renewcommand*{\cftpartpagefont}{\cftchapterpagefont}
\renewcommand{\cftpartpresnum}{Part }
\renewcommand{\cftpartnumwidth}{4em}


\newcommand{\sectionquote}{\@empty}
\newcommand*{\@printsectionquote}[1]{{\color{mediumgray}#1}}

\renewcommand*{\afterpartskip}{\vskip 20pt plus 0.7fil \@printsectionquote{\sectionquote}  \newpage}
% ----------------------------------------



% ---------------- Assumption Box --------------------
% Limits and Justification formats
\newcommand{\limit}[1]{\smallskip \noindent {\bfseries\color{primary}Limitations: }#1 \smallskip}
\newcommand{\why}[1]{\smallskip \noindent {\bfseries\color{primary}Justification: }#1}
\newcommand{\asm}[1]{\noindent \textit{\color{primary}#1}}


% Box frames
\RequirePackage[framemethod=TikZ]{mdframed}
\mdfsetup{}%skipabove=\topskip,skipbelow=\topskip}

\newcounter{assumption}[chapter]
\setcounter{assumption}{0}
\makeatletter
\newenvironment{assumption}%
{%
	\refstepcounter{assumption}
	\renewcommand{\label}[1]{\ltx@label{{##1}}}
	\mdfsetup{%
		frametitle={%
			\tikz[
				baseline=(current bounding box.east),
				outer sep=0pt,
				inner sep=5pt,
			]
			\node[
				anchor=east,
				rectangle,
				rounded corners=0.1cm,
				fill=primary,
				text=white,
                font=\normalfont\sffamily
			]
			{Assumption~\thechapter.\theassumption};
		}
	}
%
	\mdfsetup{
		roundcorner=5pt,
		innertopmargin=1pt,
		innerbottommargin=10pt,
		linecolor=primary,%
		linewidth=0.75pt,%
		topline=true,%
		frametitleaboveskip=\dimexpr-\ht\strutbox\relax,%
   }
 \goodbreak
   \begin{mdframed}[]\relax%
   }%
   {\end{mdframed}}
 \makeatother

\crefname{assumption}{assumption}{assumptions}
\Crefname{assumption}{Assumption}{Assumptions}
\labelformat{assumption}{\color{primary}\thechapter.#1}

% ----------------------------------------------------


% % ------- Other option for Assumptions --------------
% \RequirePackage{placeins}

% \newcounter{assumptioninternal}[chapter]

% \DeclareCaptionType{assumptioninternal}[Example]
% \DeclareCaptionType{tipinternal}[Tip]
% \DeclareCaptionType{algointernal}[Algorithm]
% \crefname{assumptioninternal}{assumption}{assumptions}  % name for cref
% \Crefname{assumptioninternal}{Assumption}{Assumptions}

% \RequirePackage{stackengine}

% \newenvironment{assumption}[1]
% {
% % \FloatBarrier
% %\refstepcounter{tipinternal}\refstepcounter{algointernal} % JM: don't increment the others
% \refstepcounter{assumptioninternal}
% \vspace{\baselineskip}
% \goodbreak
% \noindent\def\stackalignment{l}\stackon[-0.1pt]{\hspace*{-3px}
%   \colorbox{primary}{\textcolor{white}{\sffamily Assumption
%   \thechapter.\arabic{assumptioninternal}}} {\begin{minipage}[t]{3.1in}\sffamily #1\end{minipage}}
% }{
%   \hspace*{-3px}\textcolor{primary}{\rule{\linewidth}{3px}}
% }

% \medskip\small
% }
% {

% \noindent\textcolor{primary}{\rule{\linewidth}{2px}}

% \bigskip
% }

% ---------------- summary Box --------------------

\newcounter{summary}[chapter]
\setcounter{summary}{0}
\makeatletter
\newenvironment{summary}%
{%
	\refstepcounter{summary}
	\renewcommand{\label}[1]{\ltx@label{{##1}}}
	\mdfsetup{%
		frametitle={%
			\tikz[
				baseline=(current bounding box.east),
				outer sep=0pt,
				inner sep=5pt,
			]
			\node[
				anchor=east,
				rectangle,
				rounded corners=0.1cm,
				fill=byutan,
				text=black,
                font=\normalfont\sffamily
			]
			{Summary~\thechapter.\theassumption};
		}
	}
%
	\mdfsetup{
		roundcorner=5pt,
		innertopmargin=1pt,
		innerbottommargin=10pt,
		linecolor=byutan,%
		linewidth=0.75pt,%
		topline=true,%
		frametitleaboveskip=\dimexpr-\ht\strutbox\relax,%
	}
	\begin{mdframed}[]\relax%
	}%
	{\end{mdframed}}
\makeatother

\crefname{summary}{summary}{summaries}
\Crefname{Summary}{Summary}{Summaries}
\labelformat{summary}{\color{primary}\thechapter.#1}

% % ---------------- Summary Box --------------------
% \newcounter{summary}[chapter]
% \newenvironment{summary}[1]%
% {%
% 	\refstepcounter{summary}
% 	\ifstrempty{#1}%
% 	{%
% 		\mdfsetup{%
% 			frametitle={%
% 				\tikz[baseline=(current bounding box.east),outer sep=0pt]
% 				\node[anchor=east,rectangle,rounded corners=0.1cm,fill=byutan]
% 				{\strut \textcolor{white}{Summary~\thechapter.\thesummary} };}}
% 	}%
% 	{%
% 		\mdfsetup{%
% 			frametitle={%
% 				\tikz[baseline=(current bounding box.east),outer sep=0pt]
% 				\node[anchor=east,rectangle,rounded corners=0.1cm,fill=byutan]
% 				{\strut \color{white}{Summary~\thechapter.\thesummary: #1} };}}%
% 	}%
% 	\mdfsetup{
% 		roundcorner=5pt,
% 		innertopmargin=1pt,
% 		innerbottommargin=10pt,
% 		linecolor=byutan,%
% 		linewidth=0.5pt,
% 		topline=true,
% 		frametitleaboveskip=\dimexpr-\ht\strutbox\relax,}
% 	\begin{mdframed}[]\relax%
% 	}%
% 	{\end{mdframed}}

% \crefname{summary}{Summary}{Summaries}
% \Crefname{summary}{Summary}{Summaries}
% \labelformat{summary}{\thechapter.#1}

% ----------------------------------------------------



% -------------------- Equation Box --------------------

\usepackage{empheq}
\usepackage[most]{tcolorbox}

\newtcbox{\eqbox}[1][]{%
	nobeforeafter, math upper, tcbox raise base,
	enhanced, colframe=primary,
	colback=navy!3,
	boxrule=0.5pt,
	#1}

\newtcbox{\stepbox}[1][]{%
	nobeforeafter, math upper, tcbox raise base,
	enhanced, colframe=gray,
    standard jigsaw,
    opacityback=0,
	boxrule=0.5pt,
	#1}

% eqbox, but goes on the outside and you give it the environment to use, thus works with aligned as in
%\begin{eqboxed}{\eqbox}{align} blah blah \end{eqboxed}
\newenvironment{eqboxed}[2]{%
    \empheq[box={#1}]{#2}}{\endempheq}


% ---- equation number formatting -----

\makeatletter
\renewcommand\tagform@[1]{\maketag@@@{\color{primary}\ignorespaces(#1)\unskip\@@italiccorr}}
\makeatother

% ------------- Misc Cleveref Formatting ----------------
\crefname{figure}{figure}{figures}
\Crefname{figure}{Figure}{Figures}
\renewcommand{\thefigure}{\arabic{chapter}.\arabic{figure}}
\labelformat{figure}{\color{primary}\normalfont#1}


% try getting subfigs reference color
% \newcommand{\thesubfigure}{\thefigure(\alph{subfigure})}
\crefname{subfigure}{figure}{figures}
\Crefname{subfigure}{Figure}{Figures}
\labelformat{subfigure}{\color{primary}\normalfont\thefigure#1}

\crefname{section}{section}{sections}
\Crefname{section}{Section}{Sections}
\labelformat{section}{\color{primary}\sffamily#1}
\labelformat{subsection}{\color{primary}\sffamily#1}
\labelformat{subsubsection}{\color{primary}\sffamily#1}

\crefname{equation}{equation}{equations}
\Crefname{equation}{Equation}{Equations}
\makeatletter
\creflabelformat{equation}{%
  \textup{%
      \color{primary}
    (#2#1#3)%
  }%
}
\makeatother
% \labelformat{equation}{\color{primary}#1}

% -------------------------------------------------------

% ---------- Additional Colors -------------------------
% \definecolor{lightblue}{HTML}{64AFFA}
% % \definecolor{navyred}{HTML}{9b0000}
% % \definecolor{royalred}{HTML}{B82B00}
% % \definecolor{lightred}{HTML}{FA4B4B}

\definecolor{byutan}{HTML}{C5AF7D}
% \definecolor{plotsblue}{HTML}{002E5D}
% \definecolor{plotsred}{HTML}{9b0000}
% \definecolor{plotsgreen}{HTML}{A2E3A2}
% \definecolor{secondary}{HTML}{c05367}
% \definecolor{tertiary}{HTML}{8fa651}
\definecolor{secondary}{HTML}{be4c4d}
\definecolor{tertiary}{HTML}{69ae5f}
\definecolor{quaternary}{HTML}{a754a4}
\definecolor{quinary}{HTML}{be933d}
\definecolor{plotsgray}{HTML}{808080}


% -------------- easy coloring of things ----------------
\newcommand{\navy}[1]{{\color{navy}#1}}
\newcommand{\primary}[1]{{\color{primary}#1}}
\newcommand{\secondary}[1]{{\color{secondary}#1}}
\newcommand{\tertiary}[1]{{\color{tertiary}#1}}
\newcommand{\gray}[1]{{\color{gray}#1}}


% ------------- part page formatting help -------------
\newenvironment{fullwidth}{%
  \clearpage
    \setlrmarginsandblock{1.75in}{1.75in}{*}
  \checkandfixthelayout
}{%
  \clearpage
    \setlrmarginsandblock{1.75in}{1.75in}{*}
  \checkandfixthelayout
   }


% --------  Stuff from macros.tex ------- %

%%Potetially useful macros.
%\def\proof{\noindent{\it Proof: }}
%\def\QED{\mbox{\rule[0pt]{1.5ex}{1.5ex}}}
%\def\endproof{\hspace*{\fill}~\QED\par\endtrivlist\unskip}

%\newcommand{\abs}[1]{\left|#1\right|}
%\newcommand{\defeq}{\stackrel{\triangle}{=}}
%\newcommand{\re}{\mathbb{R}}
% real numbers
%\newcommand{\OMIT}[1]{{}}
% omit sections of text

%\newcommand{\superscript}[1]{\ensuremath{^\textrm{#1}}}
%\newcommand{\subscript}[1]{\ensuremath{_\textrm{#1}}}
%

%partial derivatives
\newcommand{\pd}[2]{\ensuremath{\frac{\partial #1}{\partial #2}}} % partial derivative
\newcommand{\pdd}[2]{\ensuremath{\frac{\partial^2 #1}{\partial #2^2}}} % partial derivative
\newcommand{\pdpd}[3]{\ensuremath{\frac{\partial^2 #1}{\partial #2 \partial #3}}} % partial derivative

%non-italic math letters
\renewcommand{\d}{\mathrm{d}}
\newcommand{\e}{\mathrm{e}}
%\newcommand{\i}{\mathrm{i}}

%decrease overset height
\makeatletter
\newcommand{\oset}[3][0ex]{%
	\mathrel{\mathop{#3}\limits^{
			\vbox to#1{\kern-0.5\ex@
				\hbox{$\scriptstyle#2$}\vss}}}}
\makeatother

%%Vector arrow over variable
%\newcommand{\vect}[1]{%
%	\oset{\rightharpoonup}{#1}}
\RequirePackage{bm}
\newcommand{\vect}[1]{\bm{#1}}

%expand fractions to full display size
\newcommand\ddfrac[2]{\frac{\displaystyle #1}{\displaystyle #2}}


%Norm
\newcommand{\norm}[1]{\left\lVert#1\right\rVert}

%Bezier word formatting
\newcommand{\beziername}{B\'{e}zier }

\let\underbrace\LaTeXunderbrace
\let\overbrace\LaTeXoverbrace


%Cauchy Integral
\def\Xint#1{\mathchoice
	{\XXint\displaystyle\textstyle{#1}}%
	{\XXint\textstyle\scriptstyle{#1}}%
	{\XXint\scriptstyle\scriptscriptstyle{#1}}%
	{\XXint\scriptscriptstyle\scriptscriptstyle{#1}}%
	\!\int}
\def\XXint#1#2#3{{\setbox0=\hbox{$#1{#2#3}{\int}$}
		\vcenter{\hbox{$#2#3$}}\kern-.5\wd0}}
\def\ddashint{\Xint=}
\def\dashint{\Xint-}


%Multi-row in matricies
\usepackage{multirow}


%-TODO notes
\definecolor{notesblue}{HTML}{6A7AB8}
\definecolor{notesorange}{HTML}{B8946A}
\definecolor{notesred}{HTML}{B86A6A}
\definecolor{notesgreen}{HTML}{72B86A}
\definecolor{notespurple}{HTML}{A36AB8}

\usepackage{xargs}
\usepackage[colorinlistoftodos,prependcaption,textsize=tiny]{todonotes}

\newcommandx{\question}[2][1=]{\todo[linecolor=notesblue,backgroundcolor=notesblue!25,bordercolor=notesblue,#1]{{\bfseries Question:} #2}}
\newcommandx{\change}[2][1=]{\todo[linecolor=notesorange,backgroundcolor=notesorange!25,bordercolor=notesorange,#1]{{\bfseries Change:} #2}}
\newcommandx{\note}[2][1=]{\todo[linecolor=notesgreen,backgroundcolor=notesgreen!25,bordercolor=notesgreen,#1]{{\bfseries NOTE:} #2}}
\newcommandx{\toadd}[2][1=]{\todo[linecolor=notespurple,backgroundcolor=notespurple!25,bordercolor=notespurple,#1]{{\bfseries Add:} #2}}
\newcommandx{\toremove}[2][1=]{\todo[linecolor=notesred,backgroundcolor=notesred!25,bordercolor=notesred,#1]{{\bfseries Remove:} #2}}
\newcommandx{\format}[2][1=]{\todo[linecolor=plotsgray,backgroundcolor=plotsgray!25,bordercolor=notesred,#1]{{\bfseries Format:} #2}}
\newcommandx{\thiswillnotshow}[2][1=]{\todo[disable,#1]{#2}}


%cutom title page
\usepackage{pdfpages}


%paper 1 packages

\RequirePackage{subcaption}
% \captionsetup[sub]{font=normalsize,labelfont={bf,sf}}
\captionsetup[subfigure]{subrefformat=simple,labelformat=simple}
    \renewcommand\thesubfigure{(\alph{subfigure})}
\newdimen\figrasterwd
\figrasterwd\textwidth
\usepackage{placeins}
\usepackage{overpic}


% add a command so you don't have to write noident all over the place
\newcommand{\where}{\noindent where }

%%%%%%%
% sub appendix stuff
\AtBeginEnvironment{subappendices}{%
\section*{Chapter~\thechapter~Appendices}
\addtocontents{toc}{\vspace{0.33em}}
\addcontentsline{toc}{section}{\numberline{}Chapter~\thechapter~Appendices}
\counterwithin{figure}{section}
\counterwithin{table}{section}
}

%%%%%%%%%%%%%%%
% index stuff from https://tex.stackexchange.com/questions/249128/makeindex-style
\RequirePackage{filecontents}
\begin{filecontents*}{\jobname.mst}
delim_0 "\\IndexDotfill "
delim_1 "\\IndexDotfill "
headings_flag 1
heading_prefix "  \\IndexHeading{"
heading_suffix "}\n"
\end{filecontents*}

\newcommand*{\IndexDotfill}{%
  \nobreak\hfill\ \nobreak
}
% \renewcommand*{\indexspace}{%
%   \par
%   \vspace{25pt plus 6pt minus 4pt}%
% }
\newcommand*{\IndexHeading}[1]{%
  \tikz\node[
    rounded corners=5pt,
    draw=primary,
    fill=navy!3,
    line width=1pt,
    inner sep=5pt,
    align=center,
    font=\sffamily\large,
    text=navy,
    minimum width=\linewidth-\pgflinewidth,
  ] {#1};%
  \nopagebreak
  \par
  \vspace{2mm}%
}

%\usepackage{imakeidx} % have to load before hyperref, so this is now in the .cls file
\makeindex


%%%%%%%%%%% set table of contents dept %%%%%%%%%%%%%%
\setcounter{tocdepth}{2}

%%%%% ------  add some breathing room between entries and brackets of matrices
\makeatletter
\renewenvironment{bmatrix}
{\left[\mkern3.5mu\env@matrix}
{\endmatrix\mkern3.5mu\right]}

\renewenvironment{vmatrix}
{\left\lvert\mkern5mu\env@matrix}
{\endmatrix\mkern5mu\right\rvert}
\makeatother

%%%%%%%%%%%% ------------  add room between figure numbers and descriptions in LoF -------
\renewcommand{\cftfigurenumwidth}{3em}

%%%%% ------ allow newlines inside table cells with the \makecell{} environment
\usepackage{makecell}


% fancy table alignment
\usepackage{array}
\newcommand{\PreserveBackslash}[1]{\let\temp=\\#1\let\\=\temp}
\newcolumntype{C}[1]{>{\PreserveBackslash\centering}m{#1}}
\newcolumntype{R}[1]{>{\PreserveBackslash\raggedleft}m{#1}}

\usepackage{multirow}
\usepackage{array, makecell}
 \renewcommand\theadfont{}
\renewcommand\theadalign{tc}
\renewcommand\cellalign{tr}
\setcellgapes{5pt}
\newcolumntype{/}{!{\color{plotsgray!50}\vrule width 0.001pt}}

% decimal aligned columns in table
\usepackage{siunitx}  % for 'S' column type
