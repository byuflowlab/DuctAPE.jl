\section{Transformation of Poisson Equations}
\label{app:poissontransform}

In order to interchange the dependent and independent variables of

\begin{subequations}
	\begin{align}
		\xi(z,r) &\equiv \xi_{zz} + \xi_{rr} = 0\\
		\label{eqn:nzr}
		\eta(z,r) &\equiv \eta_{zz} + \eta_{rr} = \frac{\psi_r}{r},
	\end{align}
\end{subequations}

\noindent where \(\eta = \psi =\) constant along streamlines (thus \(\eta\) coordinates correspond to the physical location of streamlines) and \(\xi\) is constant along radial lines and can be arbitrarily chosen,
we apply the derivative transformations:

\begin{subequations}
	\begin{align}
		\label{eqn:fztransform}
		f_z &= \frac{r_\eta f_\xi - r_\xi f_\eta}{J} \\
		\label{eqn:fytransform}
		f_r &= \frac{- z_\eta f_\xi + z_\xi f_\eta}{J},
	\end{align}
\end{subequations}

\where \(J = z_\xi r_\eta - z_\eta r_\xi\).

Let's look first at the \(\xi_{zz}\) term.
%
We will begin by applying \cref{eqn:fztransform}:

\begin{align}
	\xi_z &= \frac{r_\eta \xi_\xi - r_\xi \xi_\eta}{J} \\
	\xi_{zz} &= \frac{r_\eta \left(\frac{r_\eta \xi_\xi - r_\xi \xi_\eta}{J}\right)_\xi - r_\xi \left(\frac{r_\eta \xi_\xi - r_\xi \xi_\eta}{J}\right)_\eta}{J}.
\end{align}

\noindent Recognizing that \(\xi_\xi = 1\), and \(\xi_\eta = 0\) (by orthogonality), we can simplify.

\[
\xi_{zz} = \frac{r_\eta \left(\frac{r_\eta }{J}\right)_\xi - r_\xi \left(\frac{r_\eta}{J}\right)_\eta}{J}.
\]

\noindent Applying the quotient rule:

\[
\begin{aligned}
	\xi_{zz} &= \frac{r_\eta \left(\frac{r_{\eta\xi} J - r_\eta J_\xi}{J^2}\right) - r_\xi \left(\frac{r_{\eta\eta}J - r_\eta J_\eta}{J^2}\right)}{J} \\
	&=  \frac{r_\eta \left(r_{\eta\xi} J - r_\eta J_\xi\right) - r_\xi \left(r_{\eta\eta}J - r_\eta J_\eta\right)}{J^3}.
\end{aligned}
\]

\noindent Expanding:

\begin{equation}
	\label{eqn:xiz.zpause1}
	\xi_{zz} =  \frac{r_\eta r_{\eta\xi} J -  r_\eta^2 J_\xi -  r_\xi r_{\eta\eta}J + r_\xi r_\eta J_\eta}{J^3}.
\end{equation}

\noindent We'll leave \(\xi_{zz}\) here for now and follow the same procedure for \(\xi_{rr}\)---beginning by applying \cref{eqn:fytransform}:

\begin{align}
	\xi_r &= \frac{-z_\eta \xi_\xi + z_\xi \xi_\eta}{J} \\
	\xi_{rr} &= \frac{-z_\eta \left(\frac{-z_\eta \xi_\xi + z_\xi \xi_\eta}{J}\right)_\xi + z_\xi \left(\frac{-z_\eta \xi_\xi + z_\xi \xi_\eta}{J}\right)_\eta}{J}.
\end{align}

\noindent Again recognizing that \(\xi_\xi = 1\), and \(\xi_\eta = 0\) (by orthogonality), we can simplify:

\begin{equation}
	\xi_{rr} = \frac{-z_\eta \left(\frac{-z_\eta }{J}\right)_\xi + z_\xi \left(\frac{-z_\eta}{J}\right)_\eta}{J}.
\end{equation}

\noindent Applying the quotient rule:


\begin{equation}
	\xi_{rr} = \frac{-z_\eta \left(\frac{-z_{\eta\xi} J + z_\eta J_\xi}{J^2}\right) + z_\xi \left(\frac{-z_{\eta\eta}J + z_\eta J_\eta}{J^2}\right)}{J}.
\end{equation}


\noindent Expanding:

\begin{equation}
	\label{eqn:xirrpause1}
	\xi_{rr} =  \frac{z_\eta z_{\eta\xi} J - z_\eta^2 J_\xi - z_\xi z_{\eta\eta}J + z_\xi z_\eta J_\eta}{J^3}.
\end{equation}


Now that we have both \(\xi_{zz}\) and \(\xi_{rr}\), let us perform similar transformations for \(\eta_{zz}\) and \(\eta_{rr}\).
%
Let us begin with \cref{eqn:fztransform}:
\begin{align}
	\eta_z &= \frac{r_\eta \eta_\xi - r_\xi \eta_\eta}{J} \\
	\eta_{zz} &= \frac{r_\eta \left(\frac{r_\eta \eta_\xi - r_\xi \eta_\eta}{J}\right)_\xi - r_\xi \left(\frac{r_\eta \eta_\xi - r_\xi \eta_\eta}{J}\right)_\eta}{J}.
\end{align}

\noindent Here, \(\eta_\xi = 0\) (by orthogonality), and \(\eta_\eta = 1\).
%
So we simplify as follows
\begin{equation}
	\eta_{zz} = \frac{r_\eta \left(\frac{- r_\xi}{J}\right)_\xi - r_\xi \left(\frac{- r_\xi}{J}\right)_\eta}{J}.
\end{equation}

\noindent Applying the quotient rule:

\begin{equation}
	\eta_{zz} = \frac{r_\eta \left(\frac{- r_{\xi\xi}J + r_\xi J_\xi}{J^2}\right) - r_\xi \left(\frac{- r_{\xi\eta} J + r_\xi J_\eta}{J^2}\right)}{J}.
\end{equation}

\noindent Expanding:

\begin{equation}
	\eta_{zz} = \frac{-r_\eta r_{\xi\xi}J + r_\eta r_\xi J_\xi + r_\xi r_{\xi\eta} J - r_\xi^2 J_\eta}{J^3}.
\end{equation}

\noindent As we saw above, the expression for \(\eta_{rr}\) will be nearly identical to that for \(\eta_{zz}\)

\begin{align}
	\eta_r &= \frac{-z_\eta \eta_\xi + z_\xi \eta_\eta}{J} \\
	\eta_{rr} &= \frac{-z_\eta \left(\frac{-z_\eta \eta_\xi + z_\xi \eta_\eta}{J}\right)_\xi + z_\xi \left(\frac{-z_\eta \eta_\xi + z_\xi \eta_\eta}{J}\right)_\eta}{J}.
\end{align}

\noindent Again recognizing that \(\eta_\xi = 0\), and \(\eta_\eta = 1\), we simplify:

\begin{equation}
	\eta_{rr} = \frac{-z_\eta \left(\frac{z_\xi }{J}\right)_\xi + z_\xi \left(\frac{z_\xi}{J}\right)_\eta}{J}.
\end{equation}

\noindent Applying the quotient rule:

\begin{equation}
	\eta_{rr} = \frac{-z_\eta \left(\frac{z_{\xi\xi} J - z_\xi J_\xi}{J^2}\right) + z_\xi \left(\frac{z_{\xi\eta}J - z_\xi J_\eta}{J^2}\right)}{J}.
\end{equation}

\noindent Expanding:

\begin{equation}
	\label{eqn:etarrpause1}
	\eta_{rr} =  \frac{-z_\eta z_{\xi\xi} J + z_\eta z_\xi J_\xi + z_\xi z_{\xi\eta}J - z_\xi^2 J_\eta}{J^3}.
\end{equation}


Before putting everything together, we also need to transform the right hand side of \cref{eqn:nzr} using \cref{eqn:fytransform} as we have done, noting in this case that we only have a single, rather than a double, derivative.

\begin{equation}
	\frac{1}{r} \psi_r = \frac{-z_\eta \psi_\xi + z_\xi \psi_\eta}{r J}.
\end{equation}

\noindent Remembering that we have chosen \(\psi = \eta\), and making similar simplifications with the derivitives we have thus far (\(\eta_\eta = 1,~\eta_\xi=0\)), we are left with

\begin{equation}
	\frac{1}{r} \psi_r = \frac{z_\xi}{r J}.
\end{equation}

Let's now bring it all together in the Poisson equations to see where we are, multiplying everything by \(J^3\) to remove all the fractions.
%
For convenience and clarity, we'll also note that \((\cdot)_{\xi\eta} = (\cdot)_{\eta\xi}\) and put every instance in the \(\xi\eta\) order.

\begin{subequations}
	\begin{align}
        \label{eqn:poissontrans2a}
		\left[r_\eta r_{\xi\eta}J - r_\eta^2 J_\xi -  r_\xi r_{\eta\eta}J + r_\xi r_\eta J_\eta\right] + \left[z_\eta z_{\xi\eta}J - z_\eta^2 J_\xi -  z_\xi z_{\eta\eta}J + z_\xi z_\eta J_\eta \right] &= 0 \\
        \label{eqn:poissontrans2b}
		\left[-r_\eta r_{\xi\xi}J + r_\eta r_\xi J_\xi + r_\xi r_{\xi\eta}J - r_\xi^2 J_\eta\right] + \left[-z_\eta z_{\xi\xi}J + z_\eta z_\xi J_\xi + z_\xi z_{\xi\eta} J - z_\xi^2 J_\eta\right] &= \frac{z_\xi J^2}{r}.
	\end{align}
\end{subequations}

In order to get the final \(z(\xi,\eta)\) and \(r(\xi,\eta)\) relations, we'll first need to do some more expanding, specifically of the jacobian (and its derivatives, applying the product rule):

\begin{subequations}
	\begin{align}
		J &= z_\xi r_\eta - z_\eta r_\xi \\
		J_\xi &= z_{\xi\xi} r_\eta + z_\xi r_{\xi\eta} - z_{\xi\eta} r_\xi - z_\eta r_{\xi\xi} \\
		J_\eta &= z_{\xi\eta} r_\eta + z_\xi r_{\eta\eta} - z_{\eta\eta} r_\xi - z_\eta r_{\xi\eta}.
	\end{align}
\end{subequations}


Now we just need to expand everything out.
%
Let's start with the transformation of the \(\xi_{zz}\) term (first term on the left hand side of \cref{eqn:poissontrans2a}).
%
As we expand things out, we'll also rearrange terms to facilitate easier comparison.
\begin{equation}
\begin{aligned}
		\xi_{zz} =& r_\eta r_{\xi\eta}(z_\xi r_\eta - z_\eta r_\xi) \\
		&- r_\eta^2 (z_{\xi\xi} r_\eta + z_\xi r_{\xi\eta} - z_{\xi\eta} r_\xi - z_\eta r_{\xi\xi}) \\
		&-  r_\xi r_{\eta\eta}(z_\xi r_\eta - z_\eta r_\xi) \\
		&+ r_\xi r_\eta (z_{\xi\eta} r_\eta + z_\xi r_{\eta\eta} - z_{\eta\eta} r_\xi - z_\eta r_{\xi\eta}) \\
        %
		=& \cancel{z_\xi r_{\xi\eta} r_\eta^2} - z_\eta  r_\xi r_{\xi\eta} r_\eta   \\
		&- z_{\xi\xi} r_\eta^3 - \cancel{z_\xi r_{\xi\eta} r_\eta^2} + z_{\xi\eta} r_\xi r_\eta^2  + z_\eta r_{\xi\xi} r_\eta^2   \\
		&- \cancel{z_\xi r_\xi r_{\eta\eta}  r_\eta} +  z_\eta r_\xi^2 r_{\eta\eta} \\
		&+ z_{\xi\eta} r_\xi r_\eta^2 + \cancel{z_\xi r_\xi r_{\eta\eta}  r_\eta} - z_{\eta\eta} r_\xi^2 r_\eta - z_\eta r_\xi r_{\xi\eta}  r_\eta \\
        %
		=&- z_{\xi\xi} r_\eta^3 - z_{\eta\eta} r_\xi^2 r_\eta    \\
		& - 2 z_\eta  r_\xi r_{\xi\eta} r_\eta  + 2z_{\xi\eta} r_\xi r_\eta^2     \\
		& + z_\eta r_{\xi\xi} r_\eta^2 +  z_\eta r_\xi^2 r_{\eta\eta}. \\
\end{aligned}
\end{equation}
%
Now \(\xi_{rr}\) (second term in \cref{eqn:poissontrans2a}):
%
\begin{equation}
\begin{aligned}
		\xi_{rr} =& z_\eta z_{\xi\eta}(z_\xi r_\eta - z_\eta r_\xi) \\
		%
		&- z_\eta^2 (z_{\xi\xi} r_\eta + z_\xi r_{\xi\eta} - z_{\xi\eta} r_\xi - z_\eta r_{\xi\xi}) \\
		%
		&-  z_\xi z_{\eta\eta}(z_\xi r_\eta - z_\eta r_\xi) \\
		%
		&+ z_\xi z_\eta ( z_{\xi\eta} r_\eta + z_\xi r_{\eta\eta} - z_{\eta\eta} r_\xi - z_\eta r_{\xi\eta})\\
	%
		=& z_\xi z_{\xi\eta} z_\eta  r_\eta -  \cancel{z_{\xi\eta} z_\eta^2 r_\xi} \\
		%
		&- z_{\xi\xi} z_\eta^2  r_\eta - z_\xi z_\eta^2 r_{\xi\eta} + \cancel{z_{\xi\eta} z_\eta^2 r_\xi} + z_\eta^3 r_{\xi\xi} \\
		%
		&- z_\xi^2 z_{\eta\eta} r_\eta + \cancel{z_\xi z_{\eta\eta} z_\eta r_\xi} \\
		%
		&+ z_\xi z_{\xi\eta} z_\eta  r_\eta + z_\xi^2 z_\eta r_{\eta\eta} -  \cancel{z_\xi z_{\eta\eta} z_\eta r_\xi} - z_\xi z_\eta^2 r_{\xi\eta}\\
	%
		=& - z_{\xi\xi} z_\eta^2  r_\eta - z_\xi^2 z_{\eta\eta} r_\eta  \\
		%
		&+ 2z_\xi z_{\xi\eta} z_\eta  r_\eta - 2z_\xi z_\eta^2 r_{\xi\eta} \\
		%
		&+ z_\eta^3 r_{\xi\xi} + z_\xi^2 z_\eta r_{\eta\eta}.
\end{aligned}
\end{equation}
%
Next \(\eta_{zz}\) (first term in \cref{eqn:poissontrans2b}):
%
\begin{equation}
\begin{aligned}
		\eta_{zz} =& -r_\eta r_{\xi\xi}(z_\xi r_\eta - z_\eta r_\xi) \\
		&+ r_\eta r_\xi (z_{\xi\xi} r_\eta + z_\xi r_{\xi\eta} - z_{\xi\eta} r_\xi - z_\eta r_{\xi\xi}) \\
		&+ r_\xi r_{\xi\eta}(z_\xi r_\eta - z_\eta r_\xi) \\
		&- r_\xi^2 (z_{\xi\eta} r_\eta + z_\xi r_{\eta\eta} - z_{\eta\eta} r_\xi - z_\eta r_{\xi\eta})\\
	%
		=& - z_\xi r_{\xi\xi} r_\eta^2 + \cancel{z_\eta r_{\xi\xi} r_\xi r_\eta}  \\
		&+ z_{\xi\xi} r_\xi r_\eta^2 + z_\xi  r_\xi  r_{\xi\eta} r_\eta - z_{\xi\eta} r_\xi^2 r_\eta  - \cancel{z_\eta r_{\xi\xi} r_\xi r_\eta}  \\
		&+ z_\xi r_\xi r_{\xi\eta}  r_\eta - \cancel{z_\eta r_\xi^2  r_{\xi\eta}} \\
		&- z_{\xi\eta} r_\xi^2  r_\eta - z_\xi r_\xi^2  r_{\eta\eta} + z_{\eta\eta} r_\xi^3 + \cancel{z_\eta r_\xi^2  r_{\xi\eta}}\\
	%
		=& z_{\xi\xi} r_\xi r_\eta^2  + z_{\eta\eta} r_\xi^3\\
		& + 2z_\xi  r_\xi  r_{\xi\eta} r_\eta - 2z_{\xi\eta} r_\xi^2 r_\eta   \\
		& - z_\xi r_{\xi\xi} r_\eta^2 - z_\xi r_\xi^2  r_{\eta\eta}.
\end{aligned}
\end{equation}
%
Then \(\eta_{rr}\) (second term in \cref{eqn:poissontrans2b}):
%
\begin{equation}
\begin{aligned}
		\eta_{rr} =& -z_\eta z_{\xi\xi}(z_\xi r_\eta - z_\eta r_\xi) \\
		&+ z_\eta z_\xi (z_{\xi\xi} r_\eta + z_\xi r_{\xi\eta} - z_{\xi\eta} r_\xi - z_\eta r_{\xi\xi} ) \\
		&+ z_\xi z_{\xi\eta}(z_\xi r_\eta - z_\eta r_\xi) \\
		&- z_\xi^2 (z_{\xi\eta} r_\eta + z_\xi r_{\eta\eta} - z_{\eta\eta} r_\xi - z_\eta r_{\xi\eta})\\
	%
		=&- \cancel{z_{\xi\xi} z_\xi z_\eta  r_\eta} + z_{\xi\xi} z_\eta^2 r_\xi \\
		&+  \cancel{z_{\xi\xi} z_\xi z_\eta  r_\eta} + z_\xi^2 z_\eta  r_{\xi\eta} - z_\xi z_{\xi\eta} z_\eta r_\xi - z_\xi z_\eta^2 r_{\xi\xi}  \\
		&+ \cancel{z_\xi^2 z_{\xi\eta} r_\eta} - z_\xi z_{\xi\eta} z_\eta r_\xi \\
		&- \cancel{z_\xi^2 z_{\xi\eta} r_\eta} - z_\xi^3 r_{\eta\eta} + z_\xi^2 z_{\eta\eta} r_\xi + z_\xi^2 z_\eta r_{\xi\eta}\\
	%
		=& z_{\xi\xi} z_\eta^2 r_\xi + z_\xi^2 z_{\eta\eta} r_\xi  \\
		&+ 2z_\xi^2 z_\eta  r_{\xi\eta} - 2z_\xi z_{\xi\eta} z_\eta r_\xi  \\
		&- z_\xi z_\eta^2 r_{\xi\xi} - z_\xi^3 r_{\eta\eta}.
\end{aligned}
\end{equation}

Finally, we'll partially expand the right hand side term of \cref{eqn:poissontrans2b}:

\begin{equation}
\frac{z_\xi J^2}{r} = \frac{J}{r}z_\xi(z_\xi r_\eta - z_\eta r_\xi).
\end{equation}

Let's first look at the case where both parametric expressions are Laplace equations, that is to say, if the right hand side of \cref{eqn:nzr} was zero.
We can put our expanded expressions back together, gathering like terms.

\begin{subequations}
\begin{align}
	&\begin{split}
		\xi_{zz} + \xi_{rr} =&- z_{\xi\xi} ( z_\eta^2 + r_\eta^2)  r_\eta\\
		&- z_{\eta\eta} (z_\xi^2 + r_\xi^2)  r_\eta  \\
		%
		&+ 2z_{\xi\eta} (z_\xi z_\eta + r_\xi r_\eta)  r_\eta   \\
		&- 2r_{\xi\eta} (z_\xi z_\eta + r_\xi  r_\eta)  z_\eta \\
		%
		&+ r_{\xi\xi} ( z_\eta^2 + r_\eta^2)  z_\eta \\
		&+ r_{\eta\eta} (  z_\xi^2  + r_\xi^2) z_\eta  \\
	\end{split} \\[1em]
	&\begin{split}
		\eta_{zz} + \eta_{rr} =& z_{\xi\xi} (z_\eta^2 +  r_\eta^2)  r_\xi \\
		&+ z_{\eta\eta} (z_\xi^2 + r_\xi^2 )  r_\xi \\
		%
		&- 2z_{\xi\eta} (z_\xi z_\eta + r_\xi r_\eta ) r_\xi  \\
		&+ 2 r_{\xi\eta} ( z_\xi z_\eta +  r_\xi  r_\eta)  z_\xi \\
		%
		&- r_{\xi\xi} ( z_\eta^2 + r_\eta^2 ) z_\xi \\
		&- r_{\eta\eta} (z_\xi^2 + r_\xi^2 ) z_\xi.
	\end{split}
\end{align}
\end{subequations}


\noindent What we actually need from the above equations is \(z(\xi,\eta)\) and \(r(\xi,\eta)\), so we'll equate the two expressions and put the \(z\) terms together, and the \(r\) terms together.

\begin{subequations}
\begin{align}
	&\begin{split}
		0=& z_{\xi\xi} ( z_\eta^2 + r_\eta^2) (r_\xi - r_\eta)\\
		&- 2z_{\xi\eta} (z_\xi z_\eta + r_\xi r_\eta)  (r_\xi - r_\eta)   \\
		&+ z_{\eta\eta} (z_\xi^2 + r_\xi^2)  (r_\xi - r_\eta)
	\end{split} \\[1em]
	&\begin{split}
		0=& r_{\xi\xi} ( z_\eta^2 + r_\eta^2 )  (z_\xi - z_\eta) \\
		&- 2r_{\xi\eta} (z_\xi z_\eta + r_\xi  r_\eta) (z_\xi - z_\eta) \\
		&+ r_{\eta\eta} (z_\xi^2 + r_\xi^2 )  (z_\xi - z_\eta).
	\end{split}
\end{align}
\end{subequations}

Since both of the expressions equal zero, we can divide out the common terms and we are left with

\begin{subequations}
	\begin{align}
        z(\xi,\eta)	&\equiv \alpha z_{\xi\xi} - 2\beta z_{\xi\eta} + \gamma z_{\eta\eta} = 0 \\
        r(\xi,\eta)	&\equiv \alpha r_{\xi\xi} - 2\beta r_{\xi\eta} + \gamma r_{\eta\eta} = 0.
	\end{align}
\end{subequations}

\noindent where

\begin{subequations}
	\begin{align}
		\alpha &= z_\eta^2 + r_\eta^2 \\
		\beta &= z_\xi z_\eta + r_\xi r_\eta \\
		\gamma &= z_\xi^2 + r_\xi^2.
	\end{align}
\end{subequations}

Now in the case where the right hand side of \cref{eqn:nzr} is not zero, we need to apply some more considerations rather than simply equating things and dividing out terms.
%
Putting things together with the full Poisson equations we have

\begin{subequations}
\begin{align}
	&\begin{split}
		\xi_{zz} + \xi_{rr} =&- z_{\xi\xi} ( z_\eta^2 + r_\eta^2)  r_\eta\\
		&- z_{\eta\eta} (z_\xi^2 + r_\xi^2)  r_\eta  \\
		%
		&+ 2z_{\xi\eta} (z_\xi z_\eta + r_\xi r_\eta)  r_\eta   \\
		&- 2r_{\xi\eta} (z_\xi z_\eta + r_\xi  r_\eta)  z_\eta \\
		%
		&+ r_{\xi\xi} ( z_\eta^2 + r_\eta^2)  z_\eta \\
		&+ r_{\eta\eta} (  z_\xi^2  + r_\xi^2) z_\eta  \\
	\end{split} \\[1em]
    \label{eqn:etaetapoisson}
	&\begin{split}
		\eta_{zz} + \eta_{rr} - \frac{\eta_r}{r}=& z_{\xi\xi} (z_\eta^2 +  r_\eta^2)  r_\xi \\
		&+ z_{\eta\eta} (z_\xi^2 + r_\xi^2 )  r_\xi \\
		%
		&- 2z_{\xi\eta} (z_\xi z_\eta + r_\xi r_\eta ) r_\xi  \\
		&+ 2 r_{\xi\eta} ( z_\xi z_\eta +  r_\xi  r_\eta)  z_\xi \\
		%
		&- r_{\xi\xi} ( z_\eta^2 + r_\eta^2 ) z_\xi \\
		&- r_{\eta\eta} (z_\xi^2 + r_\xi^2 ) z_\xi \\
		%
		&- \frac{J}{r}z_\xi( r_\eta ) z_\xi\\
		&+ \frac{J}{r}z_\xi(z_\eta) r_\xi.
	\end{split}
\end{align}
\end{subequations}
%
To help combine things, we'll add and subtract the same expression from \cref{eqn:xixipoisson}.
%
\begin{equation}
    \label{eqn:xixipoisson}
    \begin{aligned}
		\xi_{zz} + \xi_{rr} =&- z_{\xi\xi} ( z_\eta^2 + r_\eta^2)  r_\eta\\
		&- z_{\eta\eta} (z_\xi^2 + r_\xi^2)  r_\eta  \\
		%
		&+ 2z_{\xi\eta} (z_\xi z_\eta + r_\xi r_\eta)  r_\eta   \\
		&- 2r_{\xi\eta} (z_\xi z_\eta + r_\xi  r_\eta)  z_\eta \\
		%
		&+ r_{\xi\xi} ( z_\eta^2 + r_\eta^2)  z_\eta \\
		&+ r_{\eta\eta} (  z_\xi^2  + r_\xi^2) z_\eta  \\
        & + \frac{J}{r}z_\xi r_\eta z_\eta \\
        & - \frac{J}{r}z_\xi r_\eta z_\eta.
    \end{aligned}
\end{equation}
%
Now adding \cref{eqn:xixipoisson} to \cref{eqn:etaetapoisson} gives
%
\begin{equation}
    \begin{aligned}
       0= &z_{\xi\xi} ( z_\eta^2 + r_\eta^2) (r_\xi - r_\eta)\\
		&- 2z_{\xi\eta} (z_\xi z_\eta + r_\xi r_\eta)  (r_\xi - r_\eta)   \\
        &+ z_{\eta\eta} (z_\xi^2 + r_\xi^2)  (r_\xi - r_\eta) \\
        &- \frac{J}{r}z_\xi (z_\eta) (r_\xi - r_\eta) \\
        &-r_{\xi\xi} ( z_\eta^2 + r_\eta^2 )  (z_\xi - z_\eta) \\
		&+ 2r_{\xi\eta} (z_\xi z_\eta + r_\xi  r_\eta) (z_\xi - z_\eta) \\
		&- r_{\eta\eta} (z_\xi^2 + r_\xi^2 )  (z_\xi - z_\eta) \\
        &+ \frac{J}{r}z_\xi (r_\eta) (z_\xi - z_\eta). \\
	\end{aligned}
\end{equation}
%
Separating out expressions for \(z\) and \(r\) gives
%
\begin{subequations}
\begin{align}
    &\begin{split}
       0= &z_{\xi\xi} ( z_\eta^2 + r_\eta^2) (r_\xi - r_\eta)\\
		&- 2z_{\xi\eta} (z_\xi z_\eta + r_\xi r_\eta)  (r_\xi - r_\eta)   \\
        &+ z_{\eta\eta} (z_\xi^2 + r_\xi^2)  (r_\xi - r_\eta) \\
        &- \frac{J}{r}z_\xi (z_\eta) (r_\xi - r_\eta) \\
   \end{split} \\[1em]
    &\begin{split}
        0=&-r_{\xi\xi} ( z_\eta^2 + r_\eta^2 )  (z_\xi - z_\eta) \\
		&+ 2r_{\xi\eta} (z_\xi z_\eta + r_\xi  r_\eta) (z_\xi - z_\eta) \\
		&- r_{\eta\eta} (z_\xi^2 + r_\xi^2 )  (z_\xi - z_\eta) \\
        &+ \frac{J}{r}z_\xi (r_\eta) (z_\xi - z_\eta). \\
    \end{split}
\end{align}
\end{subequations}

Dividing out common terms leaves

\begin{subequations}
\begin{align}
    &\begin{split}
       0= &z_{\xi\xi} ( z_\eta^2 + r_\eta^2) \\
		&- 2z_{\xi\eta} (z_\xi z_\eta + r_\xi r_\eta)     \\
        &+ z_{\eta\eta} (z_\xi^2 + r_\xi^2)   \\
        &- \frac{J}{r}z_\xi (z_\eta)  \\
   \end{split} \\[1em]
    &\begin{split}
       0= &-r_{\xi\xi} ( z_\eta^2 + r_\eta^2 )   \\
		&+ 2r_{\xi\eta} (z_\xi z_\eta + r_\xi  r_\eta)  \\
		&- r_{\eta\eta} (z_\xi^2 + r_\xi^2 )   \\
        &+ \frac{J}{r}z_\xi (r_\eta).  \\
    \end{split}
\end{align}
\end{subequations}

\noindent After final rearranging, we are left with

\begin{subequations}
    \begin{eqboxed}{\eqbox}{align}
        z(\xi,\eta)	&\equiv \alpha z_{\xi\xi} - 2\beta z_{\xi\eta} + \gamma z_{\eta\eta} = \frac{J}{r}z_\xi z_\eta \\
        r(\xi,\eta)	&\equiv \alpha r_{\xi\xi} - 2\beta r_{\xi\eta} + \gamma r_{\eta\eta} =  \frac{J}{r}z_\xi r_\eta,
	\end{eqboxed}
\end{subequations}

\noindent where again (repeated for convenience),

\begin{subequations}
	\begin{align}
		\alpha &= z_\eta^2 + r_\eta^2 \\
		\beta &= z_\xi z_\eta + r_\xi r_\eta \\
		\gamma &= z_\xi^2 + r_\xi^2 \\
        J &= z_\xi r_\eta - z_\eta r_\xi.
	\end{align}
\end{subequations}
