%%%%%%%%%%%%%%%%%%%%%%%%%%%%%%%%%%%%%%%%%%%%%%%%%%%%%%%%%%%%%%%%%

%                          WAKE VORTICITY

%%%%%%%%%%%%%%%%%%%%%%%%%%%%%%%%%%%%%%%%%%%%%%%%%%%%%%%%%%%%%%%%%
\section{A Smeared Vortex Sheet Model}
\label{sec:wakevorticity}

For a given position on a blade producing a circulation change,  \(\Delta \Gamma\), by conservation of circulation, a helical vortex filament of strength \(-\Delta \Gamma\) is shed into the flow.
%
In order to represent 3D vortex filaments in our axisymmetric reference frames, we will also make the approximation that they can be smeared into equivalent axisymmetric vortex sheets in the \(m\) and \(\theta\) directions.
%
\begin{assumption}

    \asm{Three-dimensional helical vortex filaments can be represented in a smeared axisymmetric model.}

    \limit{We are not caputuring the full 3D and unsteady effects of the wake.}

    \why{We will see that we can develop a model that works very well with the panel method formulation of the solid body aerodynamics.}

\end{assumption}
%
The smeared axisymmetric vortex sheets then have circulation to length ratios (densities) of \(\gamma_m\) and \(\gamma_\theta\) in their respective directions.
%
Because we are modeling the wake internal to the duct, we cannot guarantee a cylindrical wake, and therefore cannot simply model the wake with straight vortex cylinders.
%
Will will still use the concept of a wake cylinder, however to help us model discrete sections of the wake; so we continue with a description of how we smear a helical vortex filament into a cylindrical sheet.

\subsection{Starting with a Standard Wake Screw}
\label{ssec:wakescrew}

%As we continue in this section, we will use conservation of circulation and velocity jumps across axisymmetric vortex sheets to obtain an expression for the \(\gamma_m\) term.
%%
%We will then use a force-free wake requirement to obtain an expression for \(\gamma_\theta\).\sidenote{In practice, we will represent the wake influence in the \(\theta\) direction using a series of vortex panels, similar to how we model the duct and center body surfaces.
%%
%Rather than applying boundary conditions and solving a linear system, however, we will develop our expression for the vortex strength (circulation density just mentioned) in terms of a force-free wake requirement and some simplifying assumptions that will be discussed as we continue.}


We begin with a shed vortex sheet, the geometry of which we approximate by a left-handed helix such that the helical sheet is defined parametrically in terms of the variable \(\overline{t}\) as

% \begin{equation}
%     \begin{aligned}
%         x(\overline{t}) &= r \cos (-\overline{t}) \\
%         y(\overline{t}) &= r \sin (-\overline{t}) \\
%         z(\overline{t}) &=  \overline{t} \ell
%     \end{aligned}
% \end{equation}

% \noindent in Cartesian coordinates, and

\begin{equation}
    % \begin{aligned}
        r(\overline{t}) = r ,~~~~
        \theta(\overline{t}) = -\overline{t} ,~~~~
        z(\overline{t}) =  \overline{t} \ell,
    % \end{aligned}
\end{equation}

\noindent in polar coordinates; where \(\ell\) is the torsional parameter describing the distance traveled in the \(z\) direction relative to the angle traveled in \(\theta\):

 \begin{equation}
     \ell = \frac{h}{2\pi} = \frac{\d z}{-\d\theta},
 \end{equation}

 \where  \(h\) is the pitch of the helix, defined as the distance traveled in \(z\) for one rotation of the rotor blade, in other words, the distance traveled in \(z\) after traveling circumferentially \(2\pi r\)

 \begin{equation}
     h = 2\pi r \frac{\ell}{r} = 2\pi r  \frac{\d z}{-r\d \theta} = 2\pi  \frac{\d z}{-\d \theta}.
 \end{equation}

Given the polar coordinates, we can define the pitch angle of the helix such that the tangent of that angle is the ratio of the distance traveled in \(z\) to the distance traveled circumferentially

\begin{equation}
    \tan \phi = \frac{\d z}{-r\d \theta} = \frac{\ell}{r}.
\end{equation}

% It may be good to mention here that typically we see \(\ell\) defined in terms of \(h\) such that the torsional parameter is
%
% \begin{equation}
%     \ell = \frac{h}{2\pi}
% \end{equation}

\noindent From the pitch, we can obtain the apparent pitch, or the distance between the helix sheets created by consecutive blades by dividing the pitch by the number of blades, \(B\),

\begin{equation}
    h_B = \frac{2\pi}{B}  \frac{\d z}{-\d \theta}.
\end{equation}

If we now assume that:

\begin{assumption}

    \asm{Vortex filaments are shed parallel to the relative inflow velocity, \(\vect{W}\).}

    \limit{This is a simplified modeling approach that ignores the some of the flow turning of the blade.}

    \why{By using this lifting line approach rather than some other approach, such as a lifting surface, we (like many of our other assumptions) simplify the model, allowing for simpler implementation and faster computation.}

\end{assumption}

\noindent In other words, we assume that the local \(\d z\) is in the direction of \(\hat{\vect{e}}_m\), and likewise \(\d\theta\) in the direction of \(\hat{\vect{e}}_\theta\) as per \cref{fig:relativeframe}, we obtain the non-dimensional length in the \(m\) direction for defining the \(\gamma_\theta\) strength density

\begin{equation}
    h_B \approx \frac{2\pi}{B} \left(\frac{W_m}{-W_\theta}\right).
\end{equation}

\begin{figure}[htb]
     \centering
     \begin{subfigure}[t]{0.45\textwidth}
        \centering
        \begin{tikzpicture}[scale=1.0]

    \coordinate (s0) at (0.0,0.05);
    \coordinate (s1) at (0.0,1.25);
    \coordinate (s2) at (0.0,2.5);
    \coordinate (s3) at (0.0,3.75);
    \coordinate (ds) at (55:0.25cm);

    %airfoil
    \draw[shift={(s3)},plotsgray, pattern={Hatch[angle=80,distance=1.5pt,xshift=.1pt]}, pattern color=plotsgray] plot[smooth] file{rotor_wake_method/figures/wake-screw-airfoil.dat};

    %airfoil
    \draw[shift={(s2)},plotsgray, pattern={Hatch[angle=80,distance=1.5pt,xshift=.1pt]}, pattern color=plotsgray] plot[smooth] file{rotor_wake_method/figures/wake-screw-airfoil.dat};

    %airfoil
    \draw[shift={(s1)},plotsgray, pattern={Hatch[angle=80,distance=1.5pt,xshift=.1pt]}, pattern color=plotsgray] plot[smooth] file{rotor_wake_method/figures/wake-screw-airfoil.dat};

    %airfoil
    \draw[plotsgray, pattern={Hatch[angle=80,distance=1.5pt,xshift=.1pt]}, pattern color=plotsgray] plot[smooth] file{rotor_wake_method/figures/wake-screw-airfoil.dat};

    % \draw[-Stealth, secondary] ($(s0)+(ds)$) [partial ellipse =75:380:0.4 and 0.4] node[pos=0.5,left,shift={(-0.05,0.0)}] {\(\Gamma\)};

    \draw[-Stealth, primary,rotate around={-45:(55:1.75cm)}] (55:1.75cm) [partial ellipse =-70:255:0.3 and 0.1] node[pos=0.25,right,shift={(-0.05,0.0)}] {\(\Delta \Gamma\)};


    \draw[shift={($(s3)+(s0)+(ds)$)},primary,thick] (0,0) -- (55:1.25cm);
    \draw[shift={($(s2)+(s0)+(ds)$)},primary,thick] (0,0) -- (55:2.75cm);
    \draw[shift={($(s1)+(s0)+(ds)$)},primary,thick] (0,0) -- (55:3cm);
    \draw[shift={($(s0)+(ds)$)},primary,thick] (0,0) -- (55:1.525cm);
    \draw[shift={($(s0)+(ds)$)},primary,thick] (55:1.6cm) -- (55:3cm);

    \filldraw[shift={($(s3)+(s0)+(ds)$)},primary,thick] (0,0) circle (1pt);
    \filldraw[shift={($(s2)+(s0)+(ds)$)},primary,thick] (0,0) circle (1pt);
    \filldraw[shift={($(s1)+(s0)+(ds)$)},primary,thick] (0,0) circle (1pt);
    \filldraw[shift={($(s0)+(ds)$)},primary,thick] (0,0) circle (1pt);


    % Coordinate system parameters
    \coordinate (ow) at ($(s0)+(ds)+(55:4cm)$);
    \coordinate (wm) at ($(ow)+(0.574,0)$);
    \coordinate (wt) at ($(ow)+(0,0.819)$);
    \coordinate (wv) at ($(ow)+(55:1.0)$);

    \draw[densely dotted] (wm) -- (wv);
    \draw[densely dotted] (wt) -- (wv);
    \draw[shift={($(s0)+(ds)$)},primary,thick, dotted] (55:3cm) -- (ow);

    \draw[-Stealth] (ow) -- (wv);
    \node[anchor=south west,style={font=\tiny},shift={(-0.05,-0.05)}] at (wv) {$\vect{W}$};
    \draw[-Stealth] (ow) -- (wm);
    \node[anchor=west,style={font=\tiny}] at (wm) {$W_m$};
    \draw[-Stealth] (ow) -- (wt);
    \node[anchor=south,style={font=\tiny}] at (wt) {$-W_\theta$};

    % non-dim distances
    \draw[shift={($(s1)+(s0)+(ds)$)},{Stealth[length=4pt,width=3pt]}-{Stealth[length=4pt,width=3pt]},shorten >=2.5pt, shorten <=2.5pt] (0,0) -- node[midway,style={font=\tiny},left,shift={(0.1, 0.0)}] {\(\frac{2\pi}{B}\)} (0,1.25cm);

    \draw[shift={($(s2)+(s0)+(ds)+(55:1)$)},{Stealth[length=4pt,width=3pt]}-{Stealth[length=4pt,width=3pt]},shorten >=1pt, shorten <=1pt] (0,0) -- node[midway,style={font=\tiny},above,shift={(0.0,-0.075)}] {\(h_B\)} (0.875,0);


    % Coordinate system parameters
    \coordinate (csysO) at (2.5,1.25);
    \coordinate (et) at ($(csysO) +(0,-1)$);
    \coordinate (em) at ($(csysO) +(1,0)$);
    % z-axis
    \draw[-Stealth,] (csysO) -- (em);
    \node[anchor=south,outer sep=0] at (em) {$\hat{\vect{e}}_m$};
    % r-axis
    \draw[-Stealth,] (csysO) -- (et);
    \node[anchor=west,outer sep=0] at (et) {$\hat{\vect{e}}_\theta$};

\end{tikzpicture}

        \caption{Wake Screw Geometry.}
        \label{}
     \end{subfigure}
     \hfill
     \begin{subfigure}[t]{0.45\textwidth}
         \centering
        \begin{tikzpicture}[scale=1.0]

    \coordinate (s0) at (0.0,0.05);
    \coordinate (s1) at (0.0,1.25);
    \coordinate (s2) at (0.0,2.5);
    \coordinate (s3) at (0.0,3.75);
    \coordinate (ds) at (55:0.25cm);

    %airfoil
    \draw[shift={(s3)},plotsgray, pattern={Hatch[angle=80,distance=1.5pt,xshift=.1pt]}, pattern color=plotsgray] plot[smooth] file{rotor_wake_method/figures/wake-screw-airfoil.dat};

    %airfoil
    \draw[shift={(s2)},plotsgray, pattern={Hatch[angle=80,distance=1.5pt,xshift=.1pt]}, pattern color=plotsgray] plot[smooth] file{rotor_wake_method/figures/wake-screw-airfoil.dat};

    %airfoil
    \draw[shift={(s1)},plotsgray, pattern={Hatch[angle=80,distance=1.5pt,xshift=.1pt]}, pattern color=plotsgray] plot[smooth] file{rotor_wake_method/figures/wake-screw-airfoil.dat};

    %airfoil
    \draw[plotsgray, pattern={Hatch[angle=80,distance=1.5pt,xshift=.1pt]}, pattern color=plotsgray] plot[smooth] file{rotor_wake_method/figures/wake-screw-airfoil.dat};

    \filldraw[shift={($(s3)+(s0)+(ds)$)},thick] (0,0) circle (1pt);
    \filldraw[shift={($(s2)+(s0)+(ds)$)},thick] (0,0) circle (1pt);
    \filldraw[shift={($(s1)+(s0)+(ds)$)},thick] (0,0) circle (1pt);
    \filldraw[shift={($(s0)+(ds)$)},thick] (0,0) circle (1pt);



    \draw[shift={($(s3)+(s0)+(ds)$)},dashed] (0,0) -- (55:1.25cm);
    \draw[shift={($(s2)+(s0)+(ds)$)},dashed] (0,0) -- (55:2.75cm);
    \draw[shift={($(s1)+(s0)+(ds)$)},dashed] (0,0) -- (55:3.25cm);
    \draw[shift={($(s0)+(ds)$)},dashed] (0,0) -- (55:3.25cm);

    % gamma theta lines
    \coordinate (x1) at (1,0.0);
    \coordinate (x2) at ($(x1)+(0.4376,0.0)$);
    \coordinate (x3) at ($(x2)+(0.4376,0.0)$);

    \coordinate (gt1) at (0.0,0.0);
    \coordinate (gt2) at (0.0,1.025);
    \draw[primary,thick] ($(gt1)+(x1)$) -- ($(gt2)+(x1)$);
    \draw[primary,thick] ($(gt1)+(x2)$) -- ($(gt2)+(x2)$);
    \draw[primary,thick] ($(gt1)+(x3)$) -- ($(gt2)+(x3)$);

    \coordinate (gt1) at (0.0,1.075);
    \coordinate (gt2) at (0.0,5.0);
    \draw[primary,thick] ($(gt1)+(x1)$) -- ($(gt2)+(x1)$);
    \draw[primary,thick] ($(gt1)+(x2)$) -- ($(gt2)+(x2)$);
    \draw[primary,thick] ($(gt1)+(x3)$) -- ($(gt2)+(x3)$);

    % \draw[-Stealth, primary] ($(s0)+(ds)$) [partial ellipse =75:380:0.4 and 0.4] node[pos=0.5,left,shift={(-0.05,0.0)}] {\(\gamma_m\)};
    \draw[{Stealth[length=4pt,width=3pt]}-, primary] ($(gt1)+(x1)+(0.0,0.025)$) [partial ellipse =-160:82:0.2 and 0.05];
    \draw[primary] ($(gt1)+(x1)+(0.0,0.025)$) [partial ellipse =98:190:0.2 and 0.05];

    \draw[{Stealth[length=4pt,width=3pt]}-, primary] ($(gt1)+(x2)+(0.0,0.025)$) [partial ellipse =-160:82:0.2 and 0.05];
    \draw[primary] ($(gt1)+(x2)+(0.0,0.025)$) [partial ellipse =98:190:0.2 and 0.05];

    \draw[{Stealth[length=4pt,width=3pt]}-, primary] ($(gt1)+(x3)+(0.0,0.025)$) [partial ellipse =-160:82:0.2 and 0.05];
    \draw[primary] ($(gt1)+(x3)+(0.0,0.025)$) [partial ellipse =98:190:0.2 and 0.05];

    \node[primary] at ($(gt1)+(x3)+(0.5,0.025)$) {\(\gamma_\theta\)};

    % gamma m lines
    \coordinate (y1) at (0.75,2.75);
    \coordinate (y2) at ($(y1)+(0,0.625)$);
    \coordinate (y3) at ($(y2)+(0,0.625)$);
    \coordinate (gm1) at (0.0,0.0);
    \coordinate (gm2) at (2.025,0);
    \draw[secondary,thick] ($(gm1)+(y1)$) -- ($(gm2)+(y1)$);
    \draw[secondary,thick] ($(gm1)+(y2)$) -- ($(gm2)+(y2)$);
    \draw[secondary,thick] ($(gm1)+(y3)$) -- ($(gm2)+(y3)$);

    % gamma m lines
    \coordinate (gm1) at (2.075,0.0);
    \coordinate (gm2) at (3,0);
    \draw[secondary,thick] ($(gm1)+(y1)$) -- ($(gm2)+(y1)$);
    \draw[secondary,thick] ($(gm1)+(y2)$) -- ($(gm2)+(y2)$);
    \draw[secondary,thick] ($(gm1)+(y3)$) -- ($(gm2)+(y3)$);

    \draw[-{Stealth[length=4pt,width=3pt]}, secondary] ($(gm1)+(y1)+(0.025,0.0)$) [partial ellipse =7:240:0.05 and 0.2];
    \draw[secondary] ($(gm1)+(y1)+(0.025,0.0)$) [partial ellipse =270:353:0.05 and 0.2];

    \draw[-{Stealth[length=4pt,width=3pt]}, secondary] ($(gm1)+(y2)+(0.025,0.0)$) [partial ellipse =7:250:0.05 and 0.2];
    \draw[secondary] ($(gm1)+(y2)+(0.025,0.0)$) [partial ellipse =270:353:0.05 and 0.2];

    \draw[-{Stealth[length=4pt,width=3pt]}, secondary] ($(gm1)+(y3)+(0.025,0.0)$) [partial ellipse =7:250:0.05 and 0.2];
    \draw[secondary] ($(gm1)+(y3)+(0.025,0.0)$) [partial ellipse =270:353:0.05 and 0.2];

    \node[secondary,below] at ($(gm1)+(y1)-(0.0,0.1)$) {\(\gamma_m\)};

\end{tikzpicture}

        \caption{Axisymmetric Smeared Cylinder.}
        \label{}
     \end{subfigure}
     \caption[Smeared vortex sheets.]{2D vortex sheets are generated from ratios of circulation to lengths between vortex sheets.}
    \label{fig:smearscrew}
\end{figure}

\Cref{fig:smearscrew} shows graphically the wake screw non-dimensional geometry and orientation of the smeared vorticity.
%
To dimensionalize the lengths for a given smeared cylindrical surface, we multiply by the cylinder radius, \(r\), to obtain the dimensional length.
%
In addition, as we have defined our tangential vortices (see \cref{eqn:ringsourceinducedvelocity}) to be positive in the positive \(\theta\) direction (the negative \(\overline{t}\) direction), we need to apply an additional negative to ensure our vortices are oriented correctly in the context of the wake.
%
Thus

\begin{equation}
    \label{eqn:gammat1}
    \stepbox{
    \gamma_\theta = -\frac{-\Delta\Gamma}{h_B r} = -\Delta \Gamma \frac{B }{2 \pi r} \left(\frac{W_\theta}{W_m}\right).
}
\end{equation}

To obtain an expression for \(\gamma_m\) we look at the distance between blades in the \(\overline{t}\) direction, we know that the non-dimensional distance between the blade sections is the distance about \(\overline{t}\) divided by the number of blades (assuming even blade spacing), \(2\pi/B\).
%
For a given smeared cylinder of radius, \(r\), we multiply by \(r\) to obtain the dimensional distance, \(2\pi r/B\).
%
To keep the vortices oriented positively in our reference frame, we need to apply an additional negative.% as positive \(\theta\) is in the negative \(\overline{t}\) direction.
%
Applying this additional negative the meridional vortex strength density (strength per unit length), \(\gamma_m\), is

\begin{equation}
    \label{eqn:gammam1}
    \stepbox{
    \gamma_m = \Delta\Gamma \frac{B}{2 \pi r}.
}
\end{equation}


Our expression for \(\gamma_m\) is generally applicable for steady state conditions if we use the local circulation jumps across the wake at any given point.
%
Due to conservation of circulation, we know the circulation jumps anywhere downstream.
%
On the other hand, \(\gamma_\theta\) would only be generally applicable if we assumed that the \(\Omega r\) component of \(W_\theta\) (see \cref{eqn:relativevelocities}) was constant in the entire wake.
%
In actuality, we only know \(\Omega r\) right at the rotor lifting line, but not generally in the remainder of the wake.
%
We therefore want to develop a more general expression for \(\gamma_\theta\) based on requiring the wake to be force-free, or in other words, we demand static pressure continuity across the vortex sheets.
%
The somewhat lengthy derivation for this more general expression for \(\gamma_\theta\) comprises the rest of \cref{sec:wakevorticity}.

The following subsections will gradually assemble the pieces required to arrive at a final expression.
%
As a reminder, we will use blue boxes around equations to indicated expressions that are required for code implementation.
%
As previously noted, we will use gray boxes to indicate expressions that are not used in code implementation, but are referenced later (typically much later) in the derivation.


\subsection{Piece 1: Swirl/Circulation Relation}

The first piece we need is a relationship between the swirl velocity and the circulation.
%
The swirl velocity induced by upstream rotor blades, \(V_\theta\), can be determined by applying Stokes'\sidenote{\(\iint_\mathcal{S} \nabla \times \vect{V} \d^2 \mathcal{S} = \Gamma = \oint_{\delta\mathcal{S}} \vect{V} \d s\)} and Kelvin's\sidenote{\(\frac{D\Gamma}{Dt} = \pd{\Gamma}{t} + \vect{V} \boldsymbol{\cdot} \nabla \Gamma = 0\)} theorems.



%
If we define a control volume around a streamtube as shown in  \cref{fig:circulationsum}, where the first curve is taken about all upstream rotors along a streamline, and the second curve is taken about the axis of rotation, only in the \(r\)-\(\theta\) plane with radius such that the edge of the contour lies on the same streamline upon which the first curve lies (see the dotted line in \cref{fig:circulationsum}), we see by Kelvin's theorem (conservation of circulation), that the circulation due to the upstream rotors can be related to the tangential velocity downstream of the rotors through Stokes' theorem:

\begin{equation}
    \widetilde{\Gamma} = \oint_0^{2\pi} \vect{V} \cdot r \d \vect{\theta},
\end{equation}

\where \(\widetilde{\Gamma}\) is the net circulation contribution of all the blades of the upstream rotors:

\begin{equation}
    \label{eqn:gamma_tilde}
    \eqbox{
        \widetilde{\Gamma} = \sum_{i=1}^N B_i \Gamma_i.
    }
\end{equation}


\begin{figure}[h!]
    \centering
    \begin{tikzpicture}[scale=1]
    %Airfoil
    \draw[dash pattern=on 1cm off 0.1cm on 0.05cm off 0.1cm on 6.5cm off 0.1cm on 0.05cm off 0.1cm on 1cm] (-2,0) -- (7,0);
    \draw[ thick, plotsgray, pattern={Hatch[angle=35,distance=2pt,xshift=.1pt, line width=0.25pt]}, pattern color=plotsgray] plot[] file{rotor_wake_method/figures/scaled_dfdc_hub_coordinates.dat};

    \draw[] plot[] file{rotor_wake_method/figures/swirl-velocity-horseshoe1.dat};
    \draw[] plot[] file{rotor_wake_method/figures/swirl-velocity-horseshoe2.dat};
    \draw[] plot[] file{rotor_wake_method/figures/swirl-velocity-horseshoe3.dat};
    \draw[] plot[] file{rotor_wake_method/figures/swirl-velocity-horseshoe4.dat};
    \draw[] plot[] file{rotor_wake_method/figures/swirl-velocity-horseshoe5.dat};
    \draw[] plot[] file{rotor_wake_method/figures/swirl-velocity-horseshoe6.dat};
    \draw[] plot[] file{rotor_wake_method/figures/swirl-velocity-vert11.dat};
    \draw[] plot[] file{rotor_wake_method/figures/swirl-velocity-vert12.dat};
    \draw[] plot[] file{rotor_wake_method/figures/swirl-velocity-vert13.dat};
    \draw[] plot[] file{rotor_wake_method/figures/swirl-velocity-vert14.dat};
    \draw[] plot[] file{rotor_wake_method/figures/swirl-velocity-vert15.dat};
    \draw[] plot[] file{rotor_wake_method/figures/swirl-velocity-vert16.dat};
    \draw[] plot[] file{rotor_wake_method/figures/swirl-velocity-vert21.dat};
    \draw[] plot[] file{rotor_wake_method/figures/swirl-velocity-vert22.dat};
    \draw[] plot[] file{rotor_wake_method/figures/swirl-velocity-vert23.dat};
    \draw[] plot[] file{rotor_wake_method/figures/swirl-velocity-vert24.dat};
    \draw[] plot[] file{rotor_wake_method/figures/swirl-velocity-vert25.dat};
    \draw[] plot[] file{rotor_wake_method/figures/swirl-velocity-vert26.dat};
    \draw[secondary,thick, densely dotted] plot[] file{rotor_wake_method/figures/swirl-velocity-radialpos.dat};

    % Coordinate system parameters
    \coordinate (csysO) at (-1.5,3.0);
    \coordinate (er) at ($(csysO) +(0,1)$);
    \coordinate (zgap1) at ($(csysO) +(0.065,0)$);
    \coordinate (zgap2) at ($(csysO) +(0.135,0)$);
    \coordinate (ez) at ($(csysO) +(1,0)$);

    % z-axis
    \draw[] (csysO) -- (zgap1);
    \draw[-Stealth] (zgap2) -- (ez);
    \node[anchor=south,outer sep=0] at (ez) {$\hat{\vect{e}}_z$};

    % r-axis
    \draw[-Stealth,] (csysO) -- (er);
    \node[anchor=west,outer sep=0] at (er) {$\hat{\vect{e}}_r$};

    %theta direction
    \draw[-Stealth] ($(csysO) + (0.2,0)$) [partial ellipse =7:350:0.1 and 0.5];
    \node[anchor=north,outer sep=0,shift={(0.0,-0.5)}] at ($(csysO) + (0.2,0)$) {$\hat{\vect{e}}_\theta$};

    % bGamma contours
    \coordinate (bg1) at (1.35, 2.075);
    \draw[primary] (bg1) [partial ellipse =105:205:0.25 and 0.05];
    \draw[primary, -{Stealth[length=3pt,width=2pt]}] (bg1) [partial ellipse =70:-155:0.25 and 0.05];
    \draw[line width=0.2pt] ($(bg1)-(0.5,-0.2)$) to [out =0, in =180] ($(bg1)-(0.3,0)$);
    \node[left, primary, shift={(0.1,0.0)}, style={font=\tiny}] at ($(bg1)-(0.5,-0.2)$) {\(B_1\Gamma_1\)};

    \coordinate (bg2) at (2.6, 2.075);
    \draw[primary] (bg2) [partial ellipse =105:205:0.25 and 0.05];
    \draw[primary, -{Stealth[length=3pt,width=2pt]}] (bg2) [partial ellipse =70:-155:0.25 and 0.05];
    \draw[line width=0.2pt] ($(bg2)-(0.5,-0.4)$) to [out =270, in =180] ($(bg2)-(0.3,0)$);
    \node[above, primary, shift={(0.1,0.0)}, style={font=\tiny}] at ($(bg2)-(0.5,-0.4)$) {\(B_2\Gamma_2\)};


    % Gamma tilde contours
    \coordinate (g1) at (2.0, 1.5);
    \draw[primary, dashed] (g1) [partial ellipse =1:45:1 and 0.15];
    \draw[primary, dashed, -Stealth] (g1) [partial ellipse =60:90:1 and 0.15];
    \draw[primary, dashed] (g1) [partial ellipse =92:123:1 and 0.15];
    \draw[primary, dashed] (g1) [partial ellipse =140:360:1 and 0.15];

    \coordinate (g2) at (4.0, 0.0);
    \draw[primary,thick, densely dotted] (g2) [partial ellipse =-10:-1:0.1 and 1.5];
    \draw[primary,thick] (g2) [partial ellipse =23:28:0.1 and 1.5];
    \draw[primary,thick] (g2) [partial ellipse =32:38:0.1 and 1.5];
    \draw[primary,thick] (g2) [partial ellipse =43:47:0.1 and 1.5];
    \draw[primary,thick, densely dotted] (g2) [partial ellipse =180:190:0.1 and 1.5];
    \draw[primary,-Stealth,bend right, thick] (g2) [partial ellipse =180:48:0.1 and 1.5];

    \draw[line width=0.2pt] (4.0,1.55) to [out =90, in =200] (3.5,2) node[above right,shift={(-0.1, -0.1)}, primary] {\(\widetilde{\Gamma}\)};
    \draw[line width=0.2pt] (3.0,1.55) to [out =90, in =200] (3.5,2);

\end{tikzpicture}

    \caption[Conservation of circulation.]{Circulation is conserved between the dashed and solid contours, noting the red dotted line indicating the streamline on which the \(\widetilde{\Gamma}\) contours align. The integral over the contour about the axis of rotation yields \(V_\theta\) in terms of \(\widetilde{\Gamma}\).}
    \label{fig:circulationsum}
\end{figure}

\noindent Performing the integration for a give radial position and rearranging for \(V_\theta\)\sidenote{Note that the \(\theta\) component of \(V\) is the only component aligned with \(\d \vect{\theta}\) and is circumferentially constant due to our smearing approxmation. In addition the contour is a circle, so the integral is determined immediately.} gives

\begin{equation}
    \label{eqn:vtheta}
    \stepbox{
    V_\theta = C_\theta = \frac{\widetilde{\Gamma}}{2 \pi r},
}
\end{equation}

\where \(V_\theta\) in our smeared, axisymmetric model is the circumferentially averaged swirl velocity induced by upstream rotors

For the self-induced case, the contour is placed at the rotor plane.
%
This means that the rotor ``sees'' infinite trailing vortices from any upstream rotors, but only semi-infinite trailing vortices for itself.
%
Thus the rotor experiences the full swirl induced by upstream rotors, but only half of its own swirl contribution:

\begin{equation}
    \label{eqn:vthetaself}
    \eqbox{
        (V_\theta)_\mathrm{self} = \frac{1}{2 \pi r} \left( \widetilde{\Gamma} + \frac{1}{2} B \Gamma \right),
    }
\end{equation}

% \begin{equation}
%         (V_\theta)_\mathrm{self} = \frac{1}{2 \pi r} \left( \sum_{i=1}^N B_i \Gamma_i + \frac{1}{2} B \Gamma \right)
% \end{equation}


\where \(B \Gamma\) here is the number of blades and circulation of the rotor itself.


\subsection{Piece 2: Velocity Jumps}

The next piece of the puzzle relates velocity jumps to vorticities per unit length on wake sheets.
%
The smeared sheet strengths of \cref{eqn:gammam1,eqn:gammat1} can be defined in terms of velocity jumps across the sheets.\sidenote{Assuming here that the velocities in this subsection are the equivalent inviscid flow velocities resulting from replacing viscous rotor pressure drag effects with rotor source distributions in \cref{ssec:rotorsourcestrengths}.}
Starting with \cref{eqn:gammam1}, we can split the \(\Delta\Gamma\) into \(\Gamma_2 - \Gamma_1\) (taking \(\widetilde{\Gamma} = B\Gamma\) for the single rotor) for a given vortex sheet


\begin{figure}[h!]
    \centering
    \input{./rotor_wake_method/figures/velocity_jumps.tikz}
    \caption[Relationship between circulation and velocity.]{Vorticity per unit length can be related to velocity jump across axisymmetric vortex sheets.}
    \label{fig:velocityjump}
\end{figure}


\begin{equation}
    \begin{aligned}
        \gamma_m &= \frac{\Delta \widetilde{\Gamma}}{2 \pi r} \\
                 &= \frac{B(\Gamma_2-\Gamma_1)}{2 \pi r}.
    \end{aligned}
\end{equation}

\noindent Then using \cref{eqn:vtheta}

% \begin{equation}
%     \label{eqn:gamma2pre}
%     \begin{aligned}
%         C_{\theta_2} - C_{\theta_1} &= \frac{B\Gamma_2}{2 \pi r}  - \frac{B\Gamma_1}{2 \pi r} \\
%         &= \frac{B(\Gamma_2 - \Gamma_1)}{2 \pi r} ;
%     \end{aligned}
% \end{equation}

\begin{equation}
    \label{eqn:gamma2pre}
        C_{\theta_2} - C_{\theta_1} = \frac{B(\Gamma_2 - \Gamma_1)}{2 \pi r} ;
\end{equation}

\noindent which we can then substitute in to get the sheet strength in terms of the velocity jump:

\begin{equation}
    \label{eqn:gammam2}
    \stepbox{
    \gamma_m  = \frac{B(\Gamma_2-\Gamma_1)}{2 \pi r} = C_{\theta_2} - C_{\theta_1}.
}
\end{equation}

As it so happens, in general for inviscid flows, the jump in tangential velocity across a vortex sheet is equal to the sheet vorticity per unit length.
%
Therefore we can similarly equate \cref{eqn:gammat1} to a jump in the meridional velocities across the vortex sheet:

\begin{equation}
    \label{eqn:gammat2}
    \stepbox{
    \gamma_\theta  = -\frac{B(\Gamma_2 - \Gamma_1)}{2 \pi r} \frac{W_{\theta_\text{avg}}}{W_{m_\text{avg}}} = C_{m_1} - C_{m_2}.
}
\end{equation}

\noindent where, to obtain the relative velocity components on the sheet, we combine the blade relative velocities just to either side of the sheet into averages, \(W_\text{avg}\),  as

\begin{align}
    \label{eqn:wt}
    W_{\theta_\text{avg}} &\equiv \frac{1}{2} (W_{\theta_1} + W_{\theta_2}) = \frac{1}{2} (C_{\theta_1} + C_{\theta_2} - 2\Omega r)  \\
    \label{eqn:wm}
    W_{m_\text{avg}} &\equiv \frac{1}{2} (W_{m_1} + W_{m_2}) = \frac{1}{2} (C_{m_1} + C_{m_2}).
\end{align}


\noindent If we divide \cref{eqn:gammam2} by \cref{eqn:gammat2}, we get

% \begin{equation}
%     \label{eqn:wgam}
%     \begin{aligned}
%         \frac{\gamma_m}{\gamma_\theta} &= -\frac{W_{m_\text{avg}}}{W_{\theta_\text{avg}}} \\
%         \gamma_m W_{\theta_\text{avg}} &= -\gamma_\theta W_{m_\text{avg}} \\
%         W_{m_\text{avg}} \gamma_\theta + W_{\theta_\text{avg}} \gamma_m &= 0.
%     \end{aligned}
% \end{equation}

\begin{equation}
    \label{eqn:wgam}
        W_{m_\text{avg}} \gamma_\theta + W_{\theta_\text{avg}} \gamma_m = 0.
\end{equation}

\noindent Substituting in the average velocities from \cref{eqn:wt,eqn:wm} then gives

\begin{equation}
    \frac{1}{2}(C_{m_1} + C_{m_2}) \gamma_\theta + \frac{1}{2} (C_{\theta_1} + C_{\theta_2} - 2\Omega r)  \gamma_m = 0.
\end{equation}

\noindent Then applying the definitions of the vortex strengths from \cref{eqn:gammam2,eqn:gammat2} yields

\begin{equation}
    \frac{1}{2}(C_{m_1} + C_{m_2}) (C_{m_2} - C_{m_1})  + \frac{1}{2} (C_{\theta_1} + C_{\theta_2} - 2\Omega r)  (C_{\theta_2} - C_{\theta_1}) = 0.
\end{equation}

\noindent Simplifying

% \begin{equation}
    % \begin{aligned}
        % \frac{1}{2}(C_{m_1}^2 - C_{m_2}^2 +\cancel{C_{m_1}C_{m_2}}-\cancel{C_{m_1}C_{m_2}}) &= -\frac{1}{2} (C_{\theta_2}^2 - C_{\theta_1}^2 +\cancel{C_{\theta_1}C_{\theta_2}}-\cancel{C_{\theta_1}C_{\theta_2}}) - \Omega r(C_{\theta_2} - C_{\theta_1}) \\
        % \frac{1}{2} \left(C_{m_1}^2 - C_{m_2}^2 + C_{\theta_1}^2 - C_{\theta_2}^2 \right) &= -(C_{\theta_1} - C_{\theta_2}) \Omega r \\
        % \frac{1}{2} \left((C_{m_1}^2 + C_{\theta_1}^2) - \left(C_{m_2}^2 + C_{\theta_2}^2 \right)  \right) &= - (C_{\theta_1} - C_{\theta_2}) \Omega r \\
        % \frac{1}{2} \left(C_{1}^2 - C_{2}^2 \right) &= -(C_{\theta_1} - C_{\theta_2}) \Omega r
        % \frac{1}{2} \left(C_{1}^2 - C_{2}^2 \right) = -(C_{\theta_1} - C_{\theta_2}) \Omega r
    % \end{aligned}
% \end{equation}

\begin{equation}
    \frac{1}{2} \left(C_{1}^2 - C_{2}^2 \right) = -(C_{\theta_1} - C_{\theta_2}) \Omega r
\end{equation}

\where \(C^2 = C_m^2 + C_\theta^2\).
Then applying the definition in \cref{eqn:gamma2pre} (and multiplying both sides by -1),

\begin{equation}
    \label{eqn:vjumprel}
    \stepbox{
    \frac{1}{2} \left(C_{2}^2 - C_{1}^2 \right) =  -\frac{B(\Gamma_2-\Gamma_1)}{2 \pi} \Omega.
}
\end{equation}


% Referencing \cref{eqn:absolutevelocities,eqn:relativevelocities,eqn:vmwm}
% we see that the sheet strengths can also be defined in terms of \(W\) as

% \begin{align}
%     \gamma_m &= W_{\theta_1} - W_{\theta_2} \\
%     \gamma_\theta &= W_{m_2} - W_{m_1}
% \end{align}

% \noindent which we can follow a similar process for.
% Starting with \cref{eqn:wgam}, but substituting in the second terms of \cref{eqn:wm,eqn:wt} we have

% \begin{equation}
%     \begin{aligned}
%         \frac{1}{2}(W_{m_1} + W_{m_2}) \gamma_\theta &- \frac{1}{2} (W_{\theta_1} + W_{\theta_2} )  \gamma_m = 0\\
%         \frac{1}{2}(W_{m_1} + W_{m_2}) (v_{m_2} - v_{m_1})  &- \frac{1}{2} (W_{\theta_1} + W_{\theta_2} )  (v_{\theta_1} - v_{\theta_2}) = 0 \\
%         \frac{1}{2}(-W_{m_1}^2 + W_{m_2}^2 +\cancel{W_{m_1}W_{m_2}}-\cancel{W_{m_1}W_{m_2}}) &= \frac{1}{2} (W_{\theta_1}^2 - W_{\theta_2}^2 +\cancel{W_{\theta_1}W_{\theta_2}}-\cancel{W_{\theta_1}W_{\theta_2}})  \\
%         \frac{1}{2} \left(W_{m_2}^2 - W_{m_1}^2 + W_{\theta_2}^2 - W_{\theta_1}^2 \right) &= 0 \\
%         \frac{1}{2} \left(W_{m_2}^2 + W_{\theta_2}^2 - \left(W_{m_1}^2 + W_{\theta_1}^2 \right)  \right) &=  0. \\
%     \end{aligned}
% \end{equation}

% \noindent Thus

% \begin{equation}
%     \frac{1}{2} \left( W_2^2 - W_1^2 \right) = 0,
% \end{equation}

% \noindent or in other words,\question{what is the take away here?  it doesn't seem to be used anywhere else.  Does it justify our usage of average velocities?, but that seems self-fulfilling}

% \begin{equation}
%     \norm{W_2} = \norm{W_1}.
% \end{equation}



%As seen in \cref{fig:vortexsheets} the spacing between the \(\gamma_m\) vortex sheets, \(\ell_m\), is the local circumference, divided by the number of blades, in other words, the spacing between blades:
%
%\begin{equation}
%    \ell_m = \frac{2\pi r}{B}.
%\end{equation}
%%
%The circulation to length ratio defining the strength of \(\gamma_m\) is then
%
%\begin{equation}
%    \gamma_m = \frac{-\Delta\Gamma}{\ell_m}=-\frac{B\Delta \Gamma}{2 \pi r}.
%\end{equation}
%
%\noindent We can generalize this to any downstream location by using the total local circulation jump, \(\Delta \widetilde{\Gamma}\), where
%
%\begin{equation}
%    \label{eqn:gammatilde}
%    \widetilde{\Gamma} = \sum_{i=1}^{N} B_i \Gamma_i,
%\end{equation}
%
%\where \(N\) is the number of upstream rotors, \(B_i\) is the number of blades on the \(i\)th upstream rotor, and \(\Gamma_i\) is the local circulation along one blade on the \(i\)th upstream rotor.
%%
%Applying the total local circulation jump for all upstream rotors gives us the general form of \(\gamma_m\):
%
%\begin{equation}
%    \label{eqn:gammam1}
%    \gamma_m = -\frac{\Delta \widetilde{\Gamma}}{2 \pi r}. \\
%\end{equation}
%
%
%For \(\gamma_\theta\), the length between sheets can be found by relating what we know about the blade element frame velocity triangle to the lengths between vortex filaments such that
%
%\begin{equation}
%    \frac{-W_\theta}{W_m} = \frac{-\ell_m}{\ell_\theta}.
%\end{equation}
%
%\noindent Thus the length between the \(\gamma_\theta\) vortex sheets is
%
%\begin{equation}
%    \label{eqn:elltheta}
%    \ell_\theta = \frac{-2 \pi r}{B}\frac{W_m}{-W_\theta}.
%\end{equation}
%
%\noindent Now taking the circulation to length ratio to get the smeared \(\gamma_\theta\) strengths yields (noting the negatives in \cref{eqn:elltheta} cancel, leaving only the negative associated with the \(\Delta\Gamma\))
%
%\begin{equation}
%    \label{eqn:gammat1}
%    \gamma_\theta = \frac{-\Delta\Gamma}{\ell_\theta} = - \frac{B\Delta \Gamma}{2 \pi r} \frac{W_\theta}{W_m}.
%\end{equation}
%
%We have a problem with this definition for \(\gamma_\theta\), however.
%Although, \cref{eqn:gammat1} is true immediately behind a rotor, \textbf{it is not generally applicable further downstream} as \(\Omega r\) (and thus \(W_\theta\)) changes outside of the immediate influence of the rotor disk.
%%
%It will take a bit more development and consideration of pressure relations\sidenote{Specifically, setting the pressure jump across vortex sheets to zero, rather than assuming a constant \(\Omega r\).} to obtain the general expression for \(\gamma_\theta\).
%
%The following explanation is somewhat lengthy, therefore, by way of overview, we will first go through some pressure relations (\cref{ssec:pressure}), including total pressure and static pressure.
%%
%We will then look at several jump relations (\cref{ssec:diskjumps,ssec:vortexsheetjumps}), including the disk jumps of enthalpy and entropy and the vortex sheet jumps of velocity and pressure.
%%
%All these together will give us the pieces we need to assemble a general expression for the tangential vortex sheet strength (\cref{ssec:vortexsheetstrength}).



\subsection{Piece 3: Thermodynamic Pressure Relationships}
\label{ssec:pressure}

The largest piece in assembling a general expression for the tangential vortex sheet strength encompasses thermodynamic pressure relations for total and static pressures.

\paragraph{Total Pressure}


To determine the pressure relationships, we begin with the understanding that a rotor induces downstream changes in total enthalpy and entropy which are accompanied by changes in total pressure.
%
We can relate these changes in pressure, enthalpy, and entropy through the first and second laws of thermodynamics as follows.


The first law of thermodynamics expressed in terms of enthalpy and in differential form is:

\begin{equation}
    \label{eqn:firstlaw}
    \d q = \d h - v\d p_t
\end{equation}

\where \(q\) is specific heat, \(h\) is specific entropy, \(v\) is specific volume and \(p_t\) is total pressure.
%
The second law of thermodynamics, assuming an idealized (reversible) process, is expressed in differential form as:

\begin{equation}
    \label{eqn:secondlaw}
    T \d s = \d q
\end{equation}

\where \(T\) is total temperature, and \(s\) is specific entropy.
%
Plugging the second law (\cref{eqn:secondlaw}) into the first law (\cref{eqn:firstlaw}) gives:

\begin{equation}
    \label{eqn:tds2}
    T ds =  \d h - v\d p_t
\end{equation}

\noindent Which is a form of Gibb's equation in terms of enthalpy and is expression relating pressure, enthalpy, and entropy.
%
We will now, however, work from \cref{eqn:tds2} to arrive at a simpler and more useful expression for our application.
%
First, we'll isolate entropy on the left hand side for convenience.

\begin{equation}
    \label{eqn:tds2b}
    \d s = \frac{\d h}{T} - \frac{v \d p_t}{T}.
\end{equation}

\noindent Moving away from using enthalpy briefly, we will assume:

\begin{assumption}

    \asm{The fluid is a calorically perfect gas.}

    \limit{The specific heat capacity is constant.}

    \why{Our application is primarily at low Mach flows in electric ducted fans, for which air can reasonably be modeled as a calorically perfect gas.
    This allows us to obtain a simple relation between change in enthalpy, entropy, and pressure.}

\end{assumption}

\noindent In which case, we can relate enthalpy and temperature in both the following ways:

\begin{eqboxed}{\stepbox}{align}
\label{eqn:dhdef}
    \d h = c_p \d T \\
\label{eqn:hdef}
    h = c_p T,
\end{eqboxed}

\where \(c_p\) here is the specific heat.
%
Substituting \cref{eqn:dhdef}\sidenote{note that since the rest of the terms are still in differential form, we cannot directly use \cref{eqn:hdef} at this point.} into \cref{eqn:tds2b}, we have

\begin{equation}
   \d s = \frac{c_p \d T}{T} - \frac{v \d p_t}{T}.
\end{equation}

\noindent If we also apply the ideal gas law,

% \begin{equation}
%     \begin{aligned}
%         p_tv &= RT \\
%         v &= \frac{RT}{p_t}
%     \end{aligned}
% \end{equation}

\begin{equation}
        v = \frac{RT}{p_t},
\end{equation}

\noindent to the last term, we have

% \begin{equation}
%     \label{eqn:tds2c}
%     \begin{aligned}
%         \d s &= \frac{c_p \d T}{T} - \frac{R\cancel{T} \d p_t}{\cancel{T}p_t} \\
%         \d s &= c_p\frac{\d T}{T} - R\frac{\d p_t}{p_t}. \\
% \end{aligned}
% \end{equation}

\begin{equation}
    \label{eqn:tds2c}
        \d s = c_p\frac{\d T}{T} - R\frac{\d p_t}{p_t}.
\end{equation}

\noindent We then integrate \cref{eqn:tds2c} from the ambient to local conditions:

\begin{equation}
    \begin{aligned}
        \int_{s_\infty}^s \d s &= c_p \int_{T_\infty}^T \frac{\d T}{T} - R \int_{p_{t_\infty}}^{p_t} \frac{\d p_t}{p_t} \\
        % s\big|_{s_\infty}^s &= c_p \ln(T)\big|_{T_\infty}^T - R \ln(p_t)\big|_{p_{t_\infty}}^{p_t} \\
        % s-s_\infty &= c_p \left[\ln(T) - \ln(T_\infty)\right] - R \left[\ln(p_t) - \ln(p_{t_\infty})\right] \\
        s-s_\infty &= c_p \ln\left(\frac{T}{T_\infty}\right) - R \ln\left(\frac{p_t}{p_{t_\infty}}\right).
    \end{aligned}
\end{equation}

% \noindent We can then apply some algebra as follows:


% \begin{equation}
%     \begin{alignedat}{2}
%         s-s_\infty &= c_p \ln\left(\frac{T}{T_\infty}\right) - R \ln\left(\frac{p_t}{p_{t_\infty}}\right) \\
%         s-s_\infty &= \ln \left[ \left( \frac{T}{T_\infty} \right)^{c_p} \right]-  \ln \left[ \left( \frac{p_t}{p_{t_\infty}} \right)^R \right] & \text{bring multiples inside logarithms} \\
%         s-s_\infty &= \ln \left[ \frac{  \left( \frac{T}{T_\infty} \right)^{c_p} }{  \left( \frac{p_t}{p_{t_\infty}} \right)^R }\right] & \text{consolidate logarithm terms} \\
%         \frac{s-s_\infty}{R} &= \ln \left[ \frac{  \left( \frac{T}{T_\infty} \right)^{c_p} }{  \left( \frac{p_t}{p_{t_\infty}} \right)^R }\right]^{1/R} & \text{divide by } R \\
%         \frac{s-s_\infty}{R} &= \ln \left[ \frac{  \left( \frac{T}{T_\infty} \right)^\frac{c_p}{R} }{  \left( \frac{p_t}{p_{t_\infty}} \right) }\right] & \text{bring } R \text{ into logarithm} \\
%         \frac{s-s_\infty}{R} &= \ln \left[ \frac{  \left( \frac{T}{T_\infty} \right)^\frac{c_p}{c_p-c_v}}{  \left( \frac{p_t}{p_{t_\infty}} \right) }\right] & \text{apply specific heat relation}
%     \end{alignedat}
% \end{equation}

% \where we have used the specific heat relation: \(c_p = c_v + R\).
%
Next, we want to bring enthalpy back into the picture.
%
To do so, we now utilize \cref{eqn:hdef}, multiplying the temperatures by \(c_p\) to get back into terms of specific enthalpy

% \begin{equation}
%     \begin{aligned}
%         \frac{s-s_\infty}{R} &= \ln \left[ \frac{  \left( \frac{c_pT}{c_pT_\infty} \right)^\frac{c_p}{c_p-c_v}}{  \left( \frac{p_t}{p_{t_\infty}} \right) }\right] \\
%         \frac{s-s_\infty}{R} &= \ln \left[ \frac{  \left( \frac{h}{h_\infty} \right)^\frac{c_p}{c_p-c_v}}{  \left( \frac{p_t}{p_{t_\infty}} \right) }\right]
%     \end{aligned}
% \end{equation}

% \begin{equation}
%     \label{eqn:entropy1}
%     \begin{aligned}
%         s-s_\infty &= c_p \ln\left(\frac{c_pT}{c_pT_\infty}\right) - R \ln\left(\frac{p_t}{p_{t_\infty}}\right)\\
%         s-s_\infty &= c_p \ln\left(\frac{h}{h_\infty}\right) - R \ln\left(\frac{p_t}{p_{t_\infty}}\right).
%     \end{aligned}
% \end{equation}


\begin{equation}
    \label{eqn:entropy1}
        s-s_\infty = c_p \ln\left(\frac{h}{h_\infty}\right) - R \ln\left(\frac{p_t}{p_{t_\infty}}\right).
\end{equation}

% \noindent We will finish up our algrebra by getting things in terms of the specific heat ratio, \(\gamma = c_p/c_v\).

% \begin{equation}
%     \begin{aligned}
%         \frac{c_p}{c_p-c_v} &= \frac{1}{1-\frac{c_v}{c_p}} \\
%          &= \frac{1}{1-\gamma} \\
%          &= \frac{\gamma}{\gamma -1},
%     \end{aligned}
% \end{equation}

% \noindent substituting this in finally leaves us with

% \begin{equation}
%     \label{eqn:entropy1}
%     % \eqbox{
%     \frac{s- s_\infty}{R} = \ln \left[ \ddfrac{\left(\frac{h}{h_\infty}\right)^{\gamma/(\gamma-1)}}{\frac{p_t}{p_{t_\infty}}} \right].
% % }
% \end{equation}

% \where \(s-s_\infty\) is the change in entropy from the freestream to the point in question, \(h/h_\infty\) is the enthalpy ratio, \(p_t/p_{t_\infty}\) is the total pressure ratio, \(R\) is the universal gas constant, and \(\gamma\) here is the specific heat.

If we now define changes relative to the (far upstream) freestream values (\(\infty\) subscripts) as:

\begin{align}
    \widetilde{p_t} =& p_t - p_{t_\infty} \\
    \widetilde{h} =& h - h_\infty \\
    \widetilde{s} =& (s - s_\infty)/R,
\end{align}

\noindent then we can express \cref{eqn:entropy1} as\sidenote{Remembering that for \(x/y\), subtracting and adding \(1 = y/y\) gives \((x-y)/y + y/y =  (x-y)/y +1\)}

% \begin{equation}
%     \label{eqn:entropy2}
%     \widetilde{s} = \ln \left[ \ddfrac{\left( 1 + \frac{ \widetilde{h} }{ h_\infty} \right)^{ \gamma/(\gamma-1) } }{ 1+\frac{ \widetilde{p_t} }{ p_{t_\infty } } } \right].
% \end{equation}

\begin{equation}
    \label{eqn:entropy2}
    \widetilde{s} = \frac{c_p}{R} \ln\left(1+\frac{\widetilde{h}}{\widetilde{h}_\infty}\right) - \ln\left(1+\frac{\widetilde{p}_t}{\widetilde{p}_{t_\infty}}\right).
\end{equation}

% \noindent Rearranging:

% \begin{equation}
%     \label{eqn:entropy3}
%     1 + \frac{\widetilde{p_t}}{p_{t_\infty}} = \left( 1 + \frac{\widetilde{h}}{h_\infty}\right)^{\gamma/(\gamma-1)} e^{-\widetilde{s}}.
% \end{equation}

Now we will assume that

\begin{assumption}
    \label{asm:lowmach}

    \asm{The Mach number is sufficiently low such that}

    \begin{align}
        \frac{\widetilde{p_t}}{p_{t_\infty}}  &\ll  1 \\
        \frac{\widetilde{h}}{h_\infty} &\ll  1 \\
        \widetilde{s} &\ll  1,
    \end{align}

    \limit{We are limited to low mach number regimes.}

    \why{We can simplify the relationship between entropy, enthalpy, and pressure, again allowing for a simpler methodology and faster computation.}

\end{assumption}

\noindent With \cref{asm:lowmach} we can simplify \cref{eqn:entropy2} by noting that the Taylor series expansion for a logarithm is

\begin{equation}
    \ln(x) = (x-1) + \frac{1}{2}(x-1)^2 + \text{higher order terms},
\end{equation}

\noindent if \(x\approx1\).
%
Therefore, by \cref{asm:lowmach}, we can simplify \cref{eqn:entropy2} using the first term in Taylor series approximations of each of the logarithm terms.
%
If we then apply \cref{eqn:hdef} and the ideal gas law, we are left with

% \begin{equation}
%     \begin{alignedat}{2}
%         \widetilde{s} &= \frac{c_p}{R} \ln \left( 1 + \frac{ \widetilde{h} }{ h_\infty} \right) - \ln\left( 1+\frac{ \widetilde{p_t} }{ p_{t_\infty } }\right) \\
%         \widetilde{s} &\simeq \frac{c_p}{R} \frac{ \widetilde{h} }{ h_\infty} - \frac{ \widetilde{p_t} }{ p_{t_\infty} } & \text{apply Taylor series}\\
%         \widetilde{s} &\simeq \frac{\cancel{c_p}}{R} \frac{ \widetilde{h} }{ \cancel{c_p}T_\infty} - \frac{ \widetilde{p_t} }{ p_{t_\infty} } & \text{apply \cref{eqn:hdef}}\\
%         \widetilde{s} &\simeq \frac{\rho_\infty}{p_{t_\infty}}  \widetilde{h}  - \frac{ \widetilde{p_t} }{ p_{t_\infty} } & \text{apply ideal gas law} \\
%         p_{t_\infty} \widetilde{s} &\simeq \rho_\infty \widetilde{h}  -  \widetilde{p_t}.
%     \end{alignedat}
% \end{equation}
%
% \noindent Rearranging leaves us with

\begin{equation}
   \label{eqn:totalpressure1}
   \stepbox{
    \widetilde{p_t} \simeq \rho \left(\widetilde{h}-\widetilde{S} \right),
}
\end{equation}

\where

\begin{equation}
    \widetilde{S} \equiv \frac{p_{t_\infty}}{\rho_\infty} \widetilde{s},
\end{equation}


\where \(\rho\) is the air density, and for our steady, low Mach application, \(p_{t_\infty}/\rho_\infty\) is nearly constant, so we can convect \(\widetilde{S}\) downstream in place of \(\widetilde{s}\).
%
Therefore we end up seeing that the total pressure at any point in the rotor wake is the freestream total pressure plus any upstream work or losses:

\begin{equation}
    \label{eqn:totalpressure}
    p_t = p_{t_\infty} + \rho \left(\widetilde{h}-\widetilde{S} \right)
\end{equation}


\paragraph{Static Pressure}
\label{sssec:staticpressure}

The static pressure, \(p_s\), is the total pressure minus the dynamic pressure:

\begin{equation}
    \label{eqn:bernoulli}
    p_s = p_{t} - \frac{1}{2}\rho V_{visc}^2.
\end{equation}

\noindent Substituting in from \cref{eqn:totalpressure} gives us

\begin{equation}
    \label{eqn:staticpressurevisc}
    p_s = p_{t_\infty} - \frac{1}{2}\rho V_{visc}^2 + \rho\left(\widetilde{h}-\widetilde{S} \right),
\end{equation}

\where \(V_{visc}\) is the real viscous flow velocity.
%
Rather than finding the full viscous flow field, which (among other things) would require more costly wake treatment,
we can use the equivalent inviscid flow velocity, \(V_{inv}\),
through the addition of a source sheet at the drag elements in the flow (see \cref{asm:rotorsources}),
removing the need for trailing vortex sheets for drag elements.
%
% See \cref{fig:rvf_eif} for a visual representation of this concept.
%
Using the equivalent inviscid flow simply eliminates entropy from \cref{eqn:staticpressurevisc}

\begin{equation}
    \label{eqn:staticpressure}
    \stepbox{
    p_s = p_{t_\infty} - \frac{1}{2}\rho V_{inv}^2 + \rho \widetilde{h}.
}
\end{equation}


\subsection{Piece 4: Disk and Sheet Jumps}
\label{ssec:diskjumps}

The final piece we need includes the enthalpy jumps across rotor disks and pressure jumps across the wake sheets.

\paragraph{Enthalpy Jumps}

The specific work, \(w_c\), done by a rotor is related to a jump in enthalpy across the rotor.
%
As such, we can obtain \(\widetilde{h}\) as the accumulation of changes in enthalpy across upstream disks.

\begin{equation}
    \label{eqn:hjump}
    \eqbox{\widetilde{h} = \sum_{i=1}^N \Delta h_{\text{disk}_m}}
\end{equation}

\where the jump relation \(\Delta h_{\text{disk}}\) is defined according to the Euler turbine equation:

\begin{equation}
    \label{eqn:hjumprel}
    \Delta h_{\text{disk}} = w_c = \Omega \Delta(r C_\theta).
\end{equation}

\noindent We can relate the jump in enthalpy to the circulation by applying our lifting line assumption (\cref{asm:liftingline}),
which means that there is no radial deviation in flow across the blade, as well as substituting in for \(C_\theta\) from \cref{eqn:vtheta} (for a single disk).

% \begin{equation}
% \label{eqn:hjumprel}
%     \begin{aligned}
%     \Delta h_{\text{disk}} &= \Omega r C_\theta \\
%                            &= \Omega \cancel{r} \frac{B\Gamma}{2\pi \cancel{r}} \\
%                            &= \Omega \frac{B\Gamma}{2\pi}.
%     \end{aligned}
% \end{equation}


\begin{equation}
\label{eqn:hjumprel}
    \Delta h_{\text{disk}} = \Omega r C_\theta = \Omega \frac{B\Gamma}{2\pi}.
\end{equation}


% We can obtain \(\widetilde{S}\) as the accumulation of changes in entropy across upstream disks.

% \begin{equation}
%     \widetilde{S} = \sum_{i=1}^N \Delta S_{\text{disk}_m}
% \end{equation}

% \where the jump relation \(\Delta S_{\text{disk}}\) is defined as

% \begin{equation}
%     \label{eqn:hjumprel}
%     \Delta S_{\text{disk}} &= \frac{1}{2} V_m^2 C_f
% \end{equation}

% \where \(C_f\) is the friction coefficient of an equivalent screen representing the friction-producing disk.





\paragraph{Pressure Jumps}

Using \cref{eqn:staticpressure}, we see the jump in static pressure across a vortex sheet is

% \begin{align}
    % p_{s_2} - p_{s_1} &= -\frac{1}{2} \rho \left(V_{{visc}_2}^2 - V_{{visc}_1}^2 \right) + \rho \left( \widetilde{h}_2 - \widetilde{h}_1 - (\widetilde{S}_2 - \widetilde{S}_1)\right) \\
    % \shortintertext{for the real viscous flow case, or}
\begin{equation}
    \label{eqn:pressurejump}
    p_{s_2} - p_{s_1} = -\frac{1}{2} \rho \left(V_{{inv}_2}^2 - V_{{inv}_1}^2 \right) + \rho \left( \widetilde{h}_2 - \widetilde{h}_1 \right).
\end{equation}

% \noindent for the equivalent inviscid flow case.

\noindent If we substitute \cref{eqn:hjumprel} in for the enthalpy terms, and \cref{eqn:vjumprel} for the velocity terms in \cref{eqn:pressurejump}, we can simplify as follows

\begin{align}
    p_{s_2} - p_{s_1} &= -\cancel{\rho \frac{B(\Gamma_2-\Gamma_1)}{2 \pi} \Omega} + \cancel{\rho \frac{B(\Gamma_2-\Gamma_1)}{2 \pi} \Omega} \\
    p_{s_2} - p_{s_1} &= 0
\end{align}

\noindent which shows that there is no static pressure jump across the sheet, as we originally required to have a force-free wake. %and would be expected in reality.





\subsection{Putting the Pieces Together: The Tangential Vortex Sheet Strength}
\label{ssec:vortexsheetstrength}

As promised at the end of \cref{ssec:wakescrew}, we are finally posed to obtain a general expression for the tangential vortex sheet strength, \(\gamma_\theta\).
%
Just as a reminder, we've needed all this preparation because the tangential sheet strength at an arbitrary downstream location is not generally equal to the value just behind the rotor disk.
%
This is because we don't automatically know what the \(\Omega r\) portion of the tangential velocity is anywhere except right at the rotor disk.
%
Thus we have used the zero static pressure jump across the wake sheet as our condition for finding a general term for \(\gamma_\theta\).
%
We'll begin with \cref{eqn:pressurejump}, setting the pressure jump to zero, as is physical, and divide out the density (assumed to be constant in our low Mach case) to obtain

\begin{equation}
    \frac{1}{2} \left(C_2^2 - C_1^2 \right) = \widetilde{h}_2 - \widetilde{h}_1.
\end{equation}

\noindent Expanding out the left hand side gives

\begin{equation}
    C_{m_2}^2 - C_{m_1}^2 + C_{\theta_2}^2 - C_{\theta_1}^2 = 2 \left( \widetilde{h}_2 - \widetilde{h}_1 \right).
\end{equation}

\noindent Applying \cref{eqn:vtheta} for the \(C_\theta\) terms:

\begin{equation}
    \label{eqn:gent}
    C_{m_2}^2 - C_{m_1}^2 = -\left(\frac{1}{2 \pi r}\right)^2 \left(\widetilde{\Gamma}_2^2-\widetilde{\Gamma}_1^2\right) + 2 \left( \widetilde{h}_2 - \widetilde{h}_1 \right).
\end{equation}

To get the general expression for \(\gamma_\theta\), we have two options: if \(C_{m_2}\) is known, then using \cref{eqn:gammat2}

\begin{equation}
    \gamma_\theta = C_{m_1} - C_{m_2},
     \tag{\ref{eqn:gammat2}}
\end{equation}

\where from \cref{eqn:gent}

\begin{equation}
\label{eqn:vm1}
    C_{m_1}^2 = C_{m_2}^2 + \left(\frac{1}{2 \pi r}\right)^2 \left(\widetilde{\Gamma}_2^2-\widetilde{\Gamma}_1^2\right) - 2 \left( \widetilde{h}_2 - \widetilde{h}_1 \right),
\end{equation}

\noindent gives us our general expression.
We can march \cref{eqn:vm1} radially inward, starting with \(C_{m_2} = C_\infty\) just outside the outermost vortex sheet.
%
On the other hand, if \(C_{m_\text{avg}}\) is known instead, we can still use \cref{eqn:gent} to obtain \(\gamma_\theta\) as follows:

\begin{align}
    \label{eqn:vmavg}
    C_{m_\text{avg}} &= \frac{1}{2}\left(C_{m_1} + C_{m_2}\right) \\
    C_{m_2}^2 - C_{m_1}^2 &= \left(C_{m_1} + C_{m_2}\right)\left(C_{m_2} - C_{m_1}\right).
\end{align}

\noindent Substituting from \cref{eqn:gammat2}

\begin{equation}
    C_{m_2}^2 - C_{m_1}^2 = -\left(C_{m_1} + C_{m_2}\right) \gamma_\theta.
\end{equation}

\noindent Substituting from \cref{eqn:vmavg}

\begin{equation}
    C_{m_2}^2 - C_{m_1}^2 = -2C_{m_\text{avg}} \gamma_\theta.
\end{equation}

\noindent Rearranging for \(\gamma_\theta\) and substituting from \cref{eqn:gent}:

\begin{equation}
    \label{eqn:gamma_theta_general}
    \eqbox{
        \gamma_\theta = -\frac{1}{2 C_{m_\text{avg}}} \left(- \left(\frac{1}{2 \pi r}\right)^2 \left(\widetilde{\Gamma}_2^2-\widetilde{\Gamma}_1^2\right) + 2 \left( \widetilde{h}_2 - \widetilde{h}_1 \right) \right).
    }
\end{equation}


% For initialization purposes, we can approximate the average meridional velocity at the rotors from momentum theory as
%
%\begin{equation}
%    \label{eqn:Vm}
%    C_{m_\text{avg}} = v_z + V_\infty,
%\end{equation}
%
%\where \(v_z\) is the axially induced velocity of the rotor found from conservation of momentum:
%
%\begin{equation}
%    \begin{aligned}
%    \rho V_\infty^2 A &= \rho (V_\infty - v_z)^2 + T \\
%    \rho A [V_\infty^2 - (V_\infty - v_z)^2] &= T \\
%    \rho A (V_\infty^2 - V_\infty^2 + V_\infty v_z + v_z^2) &= T \\
%    \rho A ( V_\infty v_z + v_z^2) &= T \\
%    V_\infty v_z + v_z^2 &= \frac{T}{\rho A} \\
%    v_z^2 + V_\infty v_z - \frac{T}{\rho A} &= 0
%    \end{aligned}
%\end{equation}
%
%\noindent then using the quadratic formula, we arrive at
%
%\begin{equation}
%    \label{eqn:vx}
%    v_z = \left( V_\infty^2  +  \frac{2T}{\rho A_d} \right)^{1/2} - \frac{V_\infty}{2},
%\end{equation}
%
%\where \(\rho\) is the freestream air density and \(A_d=\pi R_\text{tip}^2\) is the disk area. We can get the total thrust, \(T\), for a rotor from a sum of the weighted averaged contributions across the blade panels
%
%\begin{equation}
%T = \frac{\rho A_d \Omega}{2 \pi} \frac{\sum_{i=1}^{nw}    B\overline{\Gamma}_i \Delta A_i}{A_a}
%\end{equation}
%
%\where \(\overline{\Gamma}\) is the average circulation of the blade elements at the edges of the ith blade panel, the wake sheet begin shed from the center thereof (thus the number of wake sheets, \(nw \), is one fewer than the number of blade elements).  \(\Delta A\) is the annular area of the blade section between those blade elements, and \(A_a\) is the total annular area of the rotor.
%
%For rotors behind others, \(V_\infty\) in \cref{eqn:Vm,eqn:vx} is increased by the accumulation of \(v_z\) of upstream rotors.
%
%After initialization, we can simply use averages of the induced velocities calculated from
%
%\begin{equation}
%    \label{eqn:vm_induced}
%    \eqbox{
%    v_{m}^{P} = A^{PB} \gamma^{B} + A^{PW} \gamma_\theta^{W} + B^{PR} \sigma^{R},
%    }
%\end{equation}
%
%\where \(v_m^P\) are the meridional induced velocities at the points of interest (for example, at the rotor plane(s)), respectively, \(A^{PB}\) is the vortex coefficient matrix for the duct-hub system on the points of interest,  \(A^{PW}\) is the vortex coefficient matrix of the wake influencing the points of interest, and \(B^{PR}\) is the source coefficient matrix for the rotors influencing the points of interest.
%For the rotor source strengths, \(\sigma^R\), we take the average of the strengths found at the blade element locations using \cref{eqn:rotorsourcestrengths} in order to obtain the strengths at the source panel control points.

