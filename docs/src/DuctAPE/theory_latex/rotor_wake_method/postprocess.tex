\section{Rotor Performance}
\label{sec:rotorperformance}

\subsection{Blade Loading}

Rotor performance calculation begins with determining the blade element aerodynamic loads.
%
To obtain the loads in the axial and tangential direction, we start with the lift and drag coefficients for the blade elements, calculated as explained in \cref{ssec:bladeelementmodel}.
%
The lift and drag coefficients are then rotated into the axial and tangential directions using the inflow angle, \(\beta_1\):

\begin{align}
    c_z &= c_\ell \cos(\beta_1) - c_d \sin(\beta_1) \\
    c_\theta &= c_\ell \sin(\beta_1) + c_d \cos(\beta_1),
\end{align}

\where \(c_z\) is the force coefficient in the axial direction, and \(c_\theta\) is the force coefficient in the tangential direction.
%
We then multiply by the chord length to scale the force and dimensionalize to obtain the forces per unit length:\sidenote{
It is these forces that are used in an aerostructural analysis and optimization setting.}

\begin{align}
    f_n &= \frac{1}{2} \rho_\infty W^2 c c_z \\
    f_t &= \frac{1}{2} \rho_\infty W^2 c c_\theta.
\end{align}

\noindent We can then integrate these forces per unit length across the blade and multiply by the number of blades to obtain the full rotor thrust, \(T_\text{rot}\), and torque, \(Q\), on the rotor.

\begin{equation}
    T_\text{rot} = B \int_{R_\text{hub}}^{R_\text{tip}} f_n \d r
\end{equation}

\begin{equation}
    Q = B \int_{R_\text{hub}}^{R_\text{tip}} f_t r \d r
\end{equation}

\noindent Power is related to torque by the rotation rate, \(\Omega\), and is therefore immediately found as well:

\begin{equation}
    P = Q\Omega.
\end{equation}

It is common to express the rotor thrust, torque and power as non-dimensional coefficients.
%
We use the propeller convention here.
%
The thrust coefficient, \(C_T\), is

\begin{equation}
    C_T = \frac{T}{\rho_\infty n^2 D^4},
\end{equation}

\where \(n=\Omega/2\pi\) is the rotation rate in revolutions per second and \(D=2R_\text{tip}\) is the rotor tip diameter.
%
The torque coefficient, \(C_Q\), is

\begin{equation}
    C_Q = \frac{Q}{\rho_\infty n^2 D^5},
\end{equation}

\noindent and the power coefficient, \(C_P\), is

\begin{equation}
    C_P = C_Q \Omega
\end{equation}


\subsection{Efficiency}

The rotor efficiency is the ratio of the thrust to power multiplied by the freestream velocity.
%
\begin{equation}
    \eta_\text{rot} = \frac{T_\text{rot}}{P} V_\infty.
\end{equation}
%
To obtain the total system efficiency, we simply add the body thrust to the rotor thrust.

\begin{equation}
    \eta_\text{tot} = \frac{T_\text{tot}}{P} V_\infty.
\end{equation}

\where

\begin{equation}
    T_\text{tot} = T_\text{rot}+T_\text{bod}
\end{equation}

%\subsection{Induced Efficiency}
%%todo: is this needed for anything? if so, need to separate the rotated cn and ct out to just lift or drag contributions, the inv subscripts are from the lift only contributions to thrust and power
%\begin{equation}
%    \eta_\text{inv} = \frac{T_\text{inv} + T_\text{bod}}{P_\text{inv}}
%\end{equation}

\noindent The ideal efficiency is useful for comparing the actual efficiency with the theoretical potential and is defined as

\begin{equation}
    \eta_\text{ideal} = \frac{2}{1 + \left[\frac{1 + T_\text{tot}}{\frac{1}{2}\rho_\infty V_\infty^2 \pi R_\text{ref}^2}\right]^{1/2}}.
\end{equation}

