%%%%%%%%%%%%%%%%%%%%%%%%%%%%%%%%%%%%%%%%%%%%%%%%%%%%%%%%%%%%%%%%%

%                          COUPLING

%%%%%%%%%%%%%%%%%%%%%%%%%%%%%%%%%%%%%%%%%%%%%%%%%%%%%%%%%%%%%%%%%
\section{Coupling the Rotor-Wake and Body Aerodynamics}

Coupling the body and rotor-wake aerodynamics is, for the most part, a straightforward process, with only a few particular concerns.
%
Essentially, the coupling takes place through mutually induced velocities.
%
Specifically, the solid bodies, rotor(s), and wake(s), all induce velocities one themselves and each other.
%
Accounting for these induced velocities is nearly the totality of the coupling methodology.


For the solid bodies, we subtract the induced velocities normal to the panels to the right hand side of the linear system, just as we do for the freestream velocities.
%
In addition, at the wake-body interfaces, or in other words, for the wake elements shed from the rotor hub and tip (assuming the rotor tip ``touches'' the duct wall), the wake panels lie directly on the boundary being solved by the linear system.
%
This requires the use of the separation of singularity integration when applying the induced velocity of the wake panels onto the body panels on which they lie.
%
In order to make this consistent, the geometry is defined from the outset to have the body and wake panel nodes and control points be coincident in order that the separation of singularity method in which we assume the position of the singular portion, functions properly.
%
In addition, since the wake panels lie on the boundary being solved, they will also induce a jump in tangential velocity across the boundary which must be accounted for the post-processing step to calculate the body surface velocity, pressure, and thereby thrust/drag.
%
Since the rotor tip and hub wakes interact with the body surfaces, additional consideration is also required for considering the strengths to define along the wake elements inside the duct.
%
In reality, this is a complex interaction that we cannot properly capture given the inviscid methodologies of DuctAPE.
%
As an approximation, we apply a linear interpolation from the body trailing edge, which gets the full rotor wake strength at that point, to the intersection of the rotor plane and wall, which we set to zero just ahead of the rotor.
%
[look into what happens experimentally/in reality/in cfd.  is the wake vorticity suppressed by the body and then pops back up outside the duct?]



For the rotor(s) we simply the induced axial and radial velocities from the bodies and wakes to the relative axial and radial velocities used in determining the inflow angles and magnitudes across the blades.
%
Similarly for the wake, we combine all the induced velocities together to obtain the average meridional velocities required to compute the wake node strengths.

