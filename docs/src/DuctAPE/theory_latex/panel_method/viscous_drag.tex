\section{Beyond the Inviscid Assumption: An Integral Boundary Layer Model for Viscous Drag}

Since panel methods are developed on an assumption of inviscid flow, they are inherently unable to model viscous drag effects on the boundaries they model.
Fortunately, we can combine inviscid panel methods with additional boundary layer models in order to capture viscous drag along the body surfaces.
To do so, we will need to assume that:

\begin{assumption}{}
    \label{asm:large_re}

    \asm{The Reynolds number of the inviscid flow is large.}

    \limit{We can't model very slow flows.}

    \why{For all the applications in this work, we will have large Reynolds numbers.}

\end{assumption}

\noindent By \cref{asm:large_re} we can split the flow into an approximate invisicd outer flow and a viscous layer close to the body surface (or boundary of the boundary element method), which we call the boundary layer.
The typical approach to modeling boundary layers in conjunction with panel methods is to use three models: a laminar boundary layer model, a turbulent boundary layer model, and a model for transition from laminar to turbulent.
For our purposes here, we will assume:

\begin{assumption}{}
    \asm{The boundary layer can be modeled as fully turbulent.}

    \limit{We are over-predicting the viscous drag as we are not modeling any laminar region in the boundary layer.}

    \why{It is beyond the scope of this work to develop a continuous boundary layer transition model, and our over-prediction should be relatively small.}

\end{assumption}

\noindent We will also assume the following:

% Assumptions:
% - adiabatic wall conditions
% - Prandtl number = 1 ("no weird wall temps" says Youngren)
% - Isentropic flow
% - viscosity is evaluated by Sutherland's law

\begin{assumption}{}
    \label{asm:adiabatic_walls}

    \asm{There is no heat transfer across the body surfaces; in other words, an adiabatic wall condition.}

    \limit{We will not be modeling thermal effects on the boundary layer, which may be somewhat important especially for the center body which houses the motor.}

    \why{We are not including a motor/heat model in this work anyway, and due to the presence of the rotor, boundary layer transition would likely occur before thermal effects influence the boundary layer.}

\end{assumption}

\begin{assumption}{}
    \label{asm:isentropic_flow}

    \asm{The flow is isentropic.}

    \limit{Again, we won't be including any heat transfer, and no energy transformations due to friction or viscosity.}

    \why{This is a suitable simplifying assumption for the level of fidelity of the models in this work, especially since we don't have any thermal models.}

\end{assumption}


In modeling the boundary layer, we will want to think in terms of several idealized thicknesses.
We first define the boundary layer thickness to be the length from the body surface to the point normal to the surface where the velocity equals the inviscid velocity.
We will call this the boundary layer thickness, \(\delta\).
The idealized thickness with equivalent mass flow is called the displacement thickness, \(\delta_1\) (many use the symbol \(\delta^*\)).
Mathematically, the displacement thickness is defined at a specific location along the surface as

\begin{equation}
    \delta_1 = \int_0^\infty \left(1-\frac{\rho u}{\rho_e U_e}\right)dy,
\end{equation}

\where \(\rho u\) is the density and velocity inside the boundary layer,
\(\rho_e U_e\) is the density and velocity at the edge of the boundary layer,
and \(y\) is in the direction normal to the surface.
Another idealized thickness with equivalent momentum flow rate is called the momentum thickness, \(\delta_2\) (many use the symbol \(\theta\)).
Mathematically, the momentum thickness is defined as

\begin{equation}
    \delta_2 = \int_0^\infty \frac{\rho u}{\rho_e U_e} \left(1-\frac{u}{U_e}\right) dy.
\end{equation}

% \noindent The last idealized thickness we will consider is the energy thickness, \(\delta_3\), which has an equivalent energy flow rate.
% Mathematically, the energy thickness is defined as

% \begin{equation}
%     \delta_3 = \int_0^\infty \frac{\rho u}{\rho_e U_e} \left[1-\left(\frac{u}{U_e}\right)^2\right] dy
% \end{equation}

Based on these idealized thicknesses, we can define shape parameters that will determine the ``health'' of the boundary layer:

% Based on these idealized thicknesses, we can define several shape parameters that will determine the ``health'' of the boundary layer.
% The shape parameters are

\begin{align}
    H_1 &= \frac{\delta-\delta_1}{\delta_2}; \\
    H_{12} &= \frac{\delta_1}{\delta_2}.
\end{align}


% \begin{align}
%     H_1 &= \frac{\delta-\delta_1}{\delta_2}; \\
%     H_{12} &= \frac{\delta_1}{\delta_2}; \\
%     \overline{H}_{12} &= \frac{\int_0^\infty \frac{\rho}{\rho_e}\left(1-\frac{u}{U_e}\right)}{\delta_2}; \\
%     H_{32} &= \frac{\delta_3}{\delta_2};
% \end{align}

% \where \(\overline{H}_{12}\) is a compressible analogue to \(H_{12}\).
% We also define an entrainment coefficient, \(C_E\):

% \begin{equation}
%     \begin{aligned}
%         C_E &= \frac{1}{r\rho_eU_e}\frac{d}{ds}\left(r\int_0^\delta \rho U dy\right)\\
%             &= \frac{1}{r\rho_eU_e}\frac{d}{ds}\left(r \rho_e U_e H_1 \delta_2\right)
%     \end{aligned}
% \end{equation}

% \where \(r\) is the body radius in axisymmetric flows (set to 1 for planar a model),
% and
% \(s\) is the direction along the surface.

Viscous effects are very difficult to model analytically, so most boundary layer models are semi-empirical in nature.
For this work, we will be using a well-known, widely used, semi-empirical method for our boundary layer model.

% \section{A Laminar Boundary Layer Model}

% % Laminar model taken more or less directly from Eppler's panel method code (2 equation momentum/energy method correlations for H23-H12) (compare about line 1247 in blcalc.f and about line 1284 in eppler's polar.f90 code)
% To model the laminar portion of the boundary layer, we take Eppler and Somers' approach\scite{Eppler_1980}, which is an integral method using momentum and energy equations.
% The two integral equations we will use are

% \begin{align}
%     \frac{d \delta_2}{ds} &= C_f + \frac{v_{\hat{n}}}{U_e} - \frac{2+H_{12}}{U_e}\frac{dU_e}{ds}\delta_2 && \mathrm{(momentum)}\\
%     \frac{d \delta_3}{ds} &= C_\mathcal{D} + \frac{v_{\hat{n}}}{U_e} - 3\frac{dU_e}{ds}\delta_3 && \mathrm{(energy)}
% \end{align}

% \where \(U_e\) is the boundary layer edge velocity we assume to be the surface velocity determined from the panel method,
% \(x\) is the length along the surface boundary from the stagnation point,
% \(v_{\hat{n}}\) is any velocity normal to the surface, for example suction (negative) or blowing (positive),
% and the remaining terms are defined as follows:
% The skin-friction coefficient, \(C_f\) is approximated as

% \begin{equation}
%     C_f = \frac{\epsilon^*}{Re_\ell \frac{U_e}{U_\infty}\frac{\delta_2}{\ell}}
% \end{equation}

% \where \(\ell\) is the reference length (e.g. surface chord length), \(U_\infty\) is the reference (freestream) velocity, \(Re_\ell\) is the reference Reynolds number

% \begin{equation}
%     Re_\ell = \frac{\rho_\infty U_\infty \ell}{\mu_\infty},
% \end{equation}

% \noindent and

% \begin{equation}
%     \epsilon^* =
%     \begin{cases}
%         \begin{aligned}
%             2.512589-1.686095H_{12}+&0.391541H^2_{12}-0.031729H^3_{12} \\
%                                     & \mathrm{for}~~1.51509 \leq H_{32} < 1.57258, \\
%             1.372391-4.226253H_{32} +& 2.221687H^2_{32} \\
%                                      & \mathrm{for}~~ H_{32}\geq1.57258.
%     \end{aligned}
%     \end{cases}
% \end{equation}

% The dissipation factor, \(C_\mathcal{D}\) is approximated as

% \begin{equation}
%     C_\mathcal{D} = \frac{2D^*}{Re_\ell \frac{U_e}{U_\infty}\frac{\delta_2}{\ell}}
% \end{equation}

% \where

% \begin{equation}
%     D^* = 7.853976 - 10.260551H_{32} + 3.418898H^2_{32}
% \end{equation}

% The shape factor, \(H_{12}\) is approximated as

% \begin{equation}
%     H_{12} =
%     \begin{cases}
%         \begin{aligned}
%             4.&02922 - \left(583.60182 - 724.55916H_{32} \right.\\
%             +&\left. 227.1822H^2_{32}\right) \left(H_{32}-1.51509\right)^{1/2}\\
%              &~~~~~~~~~~~~~~~~\mathrm{for}~~1.51509 \leq H_{32} < 1.57258\\
%             79.&870845 - 89.582142H_{32} + 25.715786H^2_{32} \\
%                &~~~~~~~~~~~~~~~~~~~~~~~~~~~~~~~~~~~\mathrm{for}~~H_{32} \geq 1.57258
%         \end{aligned}
%     \end{cases}
% \end{equation}

% \where \(H_{32} = 1.51509\) is the laminar separation limit (where \(\pd{u}{y(0)} = 0\)).


% \subsection{Solving the laminar boundary layer}

% We can use a standard ODE solver to determine the solution to the laminar boundary layer.
% For initial values for \(\delta_2\) and \(\delta_3\) we can set the first step from a stagnation point, such as on the duct or a round-nosed center body, to be

% \begin{align}
%     \delta_{2_{\Delta x}} &= 0.29004\ell \left(\frac{U_\infty \Delta x}{Re_\ell U_{e_{\Delta x}} \ell}\right)^{1/2}, \\
%     \delta_3 &= 1.61998 \delta_2;
% \end{align}

% \where the subscript \(\Delta x\) indicates the value at the end of the first step.
% For a sharp edge, in the case of a sharp center body nose cone, we can use the Blasius solution (for flat plates) to initialize the thickness values:

% \begin{align}
%     \delta_{2_{\Delta_x}} &= 0.66411\ell \left(\frac{U_\infty \Delta x}{Re_\ell U_{e_{\Delta x}} \ell}\right)^{1/2}, \\
%     \delta_3 &= 1.57258 \delta_2.
% \end{align}



\subsection{Head's Turbulent Boundary Layer Model}

For the turbulent boundary layer, we use Head's well-known entrainment method\scite{Head_1958}.
%
The governing equations of the method are:

\begin{subequations}
\begin{eqboxed}{\eqbox}{align}
    \label{eqn:heads_turbulent_bl}
    \frac{\d H_1}{\d s} &= \frac{0.0306}{\delta_2}\left(H_1-3\right)^{-0.6169} - \frac{\d U_e}{\d s}\frac{H_1}{U_e} - \frac{\d \delta_2}{\d s}\frac{H_1}{\delta_2} \\
    \frac{\d \delta_2}{\d s} &= \frac{c_f}{2} - \frac{\d U_e}{\d s}\frac{\delta_2}{U_e}\left(H_{12}+2\right)
\end{eqboxed}
\end{subequations}

\where \(H_{12}\) is a function of \(H_1\) such that

\begin{equation}
    \eqbox{
    H_{12} = \begin{cases}
        0.86 \left(H_1 - 3.3\right)^{-0.777} + 1.1 & H_1 \geq 5.3 \\
        1.1538 \left(H_1 - 3.3\right)^{-0.326} + 0.6778 & H_1 < 5.3.
    \end{cases}
}
\end{equation}

\noindent Also, \(c_f\) is defined to be

\begin{equation}
    \eqbox{
    c_f = 0.246 \times 10^{-0.678 H_{12}} Re_{\delta_2}^{-0.268}
}
\end{equation}

\where \(Re_{\delta_2}\) is the Reynolds number based on local edge velocity and momentum thickness.






% % Turbulent flow comes from Green's lag entrainment formulation (3 equations)


% For the turbulent boundary layer, we use the compressible lag-entrainment turbulent boundary layer model presented by \citeauthor{Green_1977}\scite{Green_1977}.
% In their report, they discuss the details of where various constants for this semi-empirical model come from and their reasoning for choosing the values they do.
% The following equations are taken directly from their summary of their method.
% We begin with the three differential equations comprising the model:

% \begin{subequations}
% \label{eqn:turbulent_boundary_system}
% \begin{align}
%     \label{eqn:greens_turbulent_bl}
%     \frac{d(r\delta_2)}{ds} =& \frac{r C_f}{2} - \left(H_{12}+2-M_e^2\right)\frac{\left(r\delta_2\right)}{U_e}\frac{dU_e}{ds} ; \\
%     \label{eqn:greens_turbulent_bl2}
%     \frac{d\overline{H}_{12}}{ds} =& \frac{1}{\delta_2}\frac{d \overline{H}_{12}}{dH_1} \left[C_E-H_1\left(\frac{C_f}{2} - \left[H_{12}+1\right]\frac{\delta_2}{U_e}\frac{dU_e}{ds}\right)\right] ; \\
%     \label{eqn:greens_turbulent_bl3}
%     \begin{split}
%         \frac{dC_E}{ds} =& \frac{F}{\delta_2}\left[\frac{2.8}{H_{12}+H_1} \left(\left[C_\tau\right]^{1/2}_{EQ_0} - \lambda C_\tau^{1/2} \right) + \left(\frac{\delta_2}{U_e} \frac{dU_e}{ds}\right)_{EQ} \right. \\
%            &- \left.\frac{\delta_2}{U_e}\frac{dU_e}{ds}\left(1+0.075M_e^2 \left[\frac{1+0.2M_e^2}{1+0.1M_e^2}\right]\right)\right].
%     \end{split}
% \end{align}
% \end{subequations}

% \where \cref{eqn:greens_turbulent_bl} is the momentum equation (note that the \(r\delta_2\) terms are grouped in the differential in order to avoid the need for computing \(dr/ds\) in the flat plate case), \cref{eqn:greens_turbulent_bl2} is the entrainment equation, and \cref{eqn:greens_turbulent_bl3} is the lag equation.
% The subscript \(EQ_0\) indicates equilibrium flow features unchanged by secondary influences,
% the subscript \(EQ\) indicates equilibrium flow features susceptible to secondary influences,
% % \(r\) is the body radius in axisymmetric flows (set to 1 for planar model),
% % \(x\) is the arc length position along the surface,
% and
% \(C_f\) is the skin friction coefficient.
% % \(\delta_2\) is the momentum thickness,
% % \(H_1\) is the mass-flow shape parameter (\((\delta-\delta^*)/\delta_2\)),
% % \(\overline{H}_{12}\) is the compressible shape parameter,
% % \(C_E\) is the entrainment coefficient,
% % and
% % \(U_e\) is the boundary layer edge velocity.
% The mathematical definitions/approximates for these various terms are somewhat lengthy and comprise the remainder of this section.


% \paragraph{For \(C_f\):}

% Note that separation occurs when \(C_f=0\) as this is when the shear stress is zero which is the physical condition resulting in separation.

% \begin{equation}
%     C_f = C_{f_0} \left[0.9\left(\frac{\overline{H}_{12}}{\overline{H}_{{12}_0}}-0.4\right)^{-1} - 0.5 \right];
% \end{equation}

% \where we determine \(\overline{H}_{{12}_0}\) from

% \begin{equation}
%     \label{eqn:H12naught}
%     \overline{H}_{{12}_0} = \left(1-6.55\left[\frac{C_{f_0}}{2} \left(1+0.04M_e^2\right)\right]^{1/2}\right)^{-1},
% \end{equation}

% \noindent and \(C_{f_0}\) from

% \begin{equation}
%     C_{f_0} = \frac{1}{F_c}\left(\frac{0.01013}{\log_{10}\left(F_R Re_{\delta_2}\right) - 1.02} - 0.00075\right);
% \end{equation}

% \where

% \begin{equation}
%     F_c = \left(1+0.2M_e^2\right)^{1/2},
% \end{equation}

% \begin{equation}
%     F_R = 1+0.056M_e^2,
% \end{equation}

% \noindent and

% \begin{equation}
%     Re_{\delta_2} = \frac{\rho_e U_e \delta_2}{\mu_e}.
% \end{equation}

% The values for \(M_e\), \(U_e\), and \(\rho_e\) come from the local surface pressure, free stream, and isentropic gas dynamics relations (note \cref{asm:isentropic_flow}); and \(\mu_e\) comes from Sutherland's law:
% % isentropic relations: https://www.grc.nasa.gov/www/k-12/airplane/isentrop.html
% We first assume that

% \begin{assumption}{}

%     \asm{The boundary layer edge velocity is approximately the velocity surface velocity obtained by the inviscid panel method: \[U_e \approx V_\mathrm{tan}.\]}

%     \limit{Without some sort of iterative method coupling the boundary layer model and panel method, the inviscid velocities found by the panel method will be slightly inaccurate.}

%     \why{As the boundary layer is so small, this approximation should be fine for our purposes.}

% \end{assumption}

% To get the local Mach and Reynolds numbers, we will start by taking in the altitude as a user input to the model.
% With the altitude, we will use a standard atmosphere approximation to get the freestream static temperature, pressure, and density.
% With the freestream static temperature and density we can get a reference speed of sound using isentropic\sidenote{By \cref{asm:isentropic_flow}.} relations and assuming the specific heat of air is \(\gamma_\mathrm{air} = 1.4\).

% \begin{equation}
%     a_\infty = \left(\frac{\gamma_\mathrm{air} p_{s_\infty}}{\rho_\infty}\right)^{1/2}.
% \end{equation}

% \noindent From the speed of sound and the surface velocities, we can find the Mach number distribution at the edge of the boundary layer.

% \begin{equation}
%     M_e = \frac{U_e}{a_\infty}.
% \end{equation}

% To obtain the local Reynolds number we need to get a local density and viscosity also from isentropic relations.
% We start with the stagnation pressure and temperature in the freestream and assume they stay constant:

% \begin{equation}
%     p_0 = p_{s_\infty} \left(1 + \frac{\gamma_\mathrm{air}-1}{2}M_\infty^2\right)^{\frac{\gamma_\mathrm{air}}{\gamma_\mathrm{air}-1}},
% \end{equation}

% \noindent and

% \begin{equation}
%     T_0 = T_{s_\infty} \left(1 + \frac{\gamma_\mathrm{air}-1}{2}M_\infty^2\right).
% \end{equation}

% \noindent Then we can get the local static pressures from

% \begin{equation}
%      p_e = p_0 \left(1 + \frac{\gamma_\mathrm{air}-1}{2}M_e^2\right)^{\frac{-\gamma_\mathrm{air}}{\gamma_\mathrm{air}-1}},
% \end{equation}

% \noindent and the local static temperatures from

% \begin{equation}
%     T_e = \frac{T_0}{1 + \frac{\gamma_\mathrm{air}-1}{2}M_e^2}.
% \end{equation}

% With the local static pressure and Mach number, we can determine density from the dynamic pressure definition:

% \begin{equation}
%     \begin{aligned}
%         q = \frac{1}{2} \rho_e U_e^2 &= \frac{1}{2} \gamma_\mathrm{air} p_e M_e^2 \\
%         \rho_e &= \frac{\gamma_\mathrm{air} p_e }{a_\infty^2}.
%     \end{aligned}
% \end{equation}

% With the local static temperature, we can determine the viscosity using Sutherland's law:

% \begin{equation}
%     \mu_e = \mu_\mathrm{sl} \left(\frac{T_e}{T_\mathrm{sl}}\right)^{3/2} \frac{T_\mathrm{sl} + S_\mathrm{air}}{T_e+S_\mathrm{air}};
% \end{equation}

% \where the ``sl'' subscript indicates values at sea level, and \(S_\mathrm{air}\) is the Sutherland temperature which we take to be 110.4 Kelvin for air.

% \paragraph{For \(H_{12}\):}

% \begin{equation}
%     H_{12} = \left(\overline{H}_{12}+1\right)\left(1+\frac{\left(Pr\right)^{1/3}M_e^2}{5}\right)-1,
% \end{equation}

% \where we approximate the temperature recovery factor as \(\left(Pr\right)^{1/3}\), in which \(Pr\) is the Prandtl number.
% In practice, we assume the Prandtl number is unity as recommended by Green\scite{Green_1977}.

% \paragraph{For \(H_1\) and \(\frac{d\overline{H}_{12}}{dH_1}\):}

% \begin{equation}
%     H_1 = 3.15+\frac{1.72}{\overline{H}_{12}-1} - 0.01\left(\overline{H}_{12}-1\right)^2.
% \end{equation}

% \begin{equation}
%     \frac{d\overline{H}_{12}}{dH_1} = -\frac{\left(\overline{H}_{12}-1\right)^2}{1.72+0.02\left(\overline{H}_{12}-1\right)^3}.
% \end{equation}


% \paragraph{For \(C_\tau\) and \(F\):}

% \begin{equation}
%     C_\tau = \left(0.024C_E + 1.2C_E^2+0.32C_{f_0}\right) \left(1+0.1M_e^2\right).
% \end{equation}

% \begin{equation}
%     F = \frac{0.02C_E + C_E^2 + 0.8\frac{C_{f_0}}{3}}{0.01+C_E}.
% \end{equation}


% \paragraph{For secondary influences:}

% Secondary influences are contained in the \(\lambda\) term, where \(\lambda = \lambda_1 \lambda_2 \lambda_3 ...\).
% In this case, we include three secondary influences: longitudinal curvature, lateral strain, and dilation as provided by Green\scite{Green_1977}.

% For longitudinal curvature, we have

% \begin{equation}
%     \lambda_1 = 1+\beta \left(1+\frac{M_e^2}{5}\right) Ri
% \end{equation}

% \where

% \begin{equation}
%     \beta =
%     \begin{cases}
%         7 &\mathrm{for} ~~ Ri>0\\
%         4.5 &\mathrm{for} ~~ Ri<0
%     \end{cases}
% \end{equation}

% \noindent and \(Ri\) is the Richardson number defined as

% \begin{equation}
%     Ri = \frac{2}{3} \frac{\delta_2}{R}\left(H_{12}+H_1\right)\left(\frac{H_1}{\overline{H}_{12}}+0.3\right),
% \end{equation}

% \where we take \(R\) to be the radius of longitudinal curvature (convex positive).

% For lateral strain we have

% \begin{equation}
%     \lambda_2 = 1-\frac{7}{3}\left(\frac{H_1}{\overline{H}_{12}}+0.3\right)\left(H_{12}+H_1\right)\frac{\delta_2}{r}\frac{dr}{ds}.
% \end{equation}

% For dilation we have

% \begin{equation}
%     \lambda_3 = 1+\frac{7}{3}M_e^2\left(\frac{H_1}{\overline{H}_{12}}+1\right)\left(H_{12}+H_1\right)\frac{\delta_2}{U_e}\frac{dU_e}{ds}.
% \end{equation}

% Note that for these secondary influences, Green suggests including them as user options and also only applying them when \(0.4 > \lambda > 2.5\).


% \paragraph{For equilibrium quantities:}

% In \cref{eqn:greens_turbulent_bl3} we have the equilibrium quantities \(\left(C_\tau\right)_{EQ_0}\) and \(\left(\frac{\delta_2}{U_e}\frac{dU_e}{ds}\right)_{EQ}\) defined as follows:

% \begin{equation}
%     \left(C_\tau\right)_{EQ_0} = \left[0.24\left(C_E\right)_{EQ_0} +1.2\left(C_E\right)_{EQ_0}^2 +0.32C_{f_0}\right]\left(1+0.1M_e^2\right),
% \end{equation}

% \where

% \begin{equation}
%     \left(C_E\right)_{EQ_0} = H_1 \left[\frac{C_f}{2}-\left(H_{12}+1\right)\left(\frac{\delta_2}{U_e}\frac{dU_e}{ds}\right)_{EQ_0}\right],
% \end{equation}

% \where

% \begin{equation}
%     \label{eqn:deq0}
%     \left(\frac{\delta_2}{U_e}\frac{dU_e}{ds}\right)_{EQ_0} = \frac{1.25}{H_{12}}\left[\frac{C_f}{2} - \left(\frac{\overline{H}_{12}-1}{6.432 \overline{H}_{12}}\right)^2 \left(1+0.04M_e^2\right)^{-1}\right].
% \end{equation}

% \noindent And

% \begin{equation}
%     \left(\frac{\delta_2}{U_e}\frac{dU_e}{ds}\right)_{EQ} = \frac{\frac{C_f}{2} - \frac{\left(C_E\right)_{EQ}}{H_1}}{H_{12}+1},
% \end{equation}

% \where

% \begin{equation}
%     \label{eqn:ceeq}
%     \left(C_E\right)_{EQ} = \left(\frac{C}{1.2}+0.0001\right)^{1/2} - 0.01,
% \end{equation}

% \where

% \begin{equation}
%     C = \left(C_\tau\right)_{EQ_0} \left(1+0.1M_e^2\right)^{-1} \lambda^{-2} - 0.32 C_{f_0}.
% \end{equation}

\subsection{Solving the turbulent boundary layer}

We can use a standard ODE solver to determine the solution of the turbulent boundary layer system of equations, for example a 2nd order Runge-Kutta method.
%
We determine initial conditions by starting with the momentum thickness value from the Schlichting empirical fit for a flat plate:

\begin{equation}
    \eqbox{
    \delta_2 = \frac{0.036 s}{Re_s^{1/5}}
}
\end{equation}

\noindent We initialize \(H_{12}\) to the value for a turbulent flat plate, 1.28, which gives us an initial \(H_1\) of 10.6.
%
Separation takes place as \(H_{12}\) becomes large, so a typical cutoff value for separation is 3, at which point the ODE solve is terminated.

% \noindent Using the momentum thickness, we can get an estimate for \(\overline{H}_{12_0}\) from \cref{eqn:H12naught} if we assume a flat plate velocity profile at the start of the turbulent section.
% % Another option for initializing \(\overline{H}_{12_0}\) is to assume local equilibrium and apply a root finder to \cref{eqn:deq0}.
% Once \(\overline{H}_{12_0}\) is determined, \(C_E\) can be initialized from \cref{eqn:ceeq}, letting \(\lambda=1\).

% \section{A Boundary Layer Transition Model}

% % transition is from an $e^n$ method
% % Somewhere an amplification rate equation from Drela is used (about line 1274 in blcalc.f)
% To model the transition from a laminar to turbulent boundary layer, we use the \(e^n\) method which is based on empirical fits to stability envelopes of Tollmien-Schlichting waves in the boundary layer.\todo{revisit this explanation of the en method after you read the paper on in}
% In slightly more detail, the \(e^n\) model sets a threshold on the amplification of instabilities in the boundary layer and estimates transition based on when the maximum amplitude of the instabilities reaches the threshold.
% Values near, but less than 10 (say, 7-9) are common for clean configurations in open air, while smaller values (say, 3-5) are more common for wind tunnel conditions where more turbulence is likely to be present in the freestream.

% An empirical approximation of the \(e^n\) method is provided by Drela and Giles\scite{Drela_1987}, where we approximate \(n\) with \(\widetilde{n}\) as

% \begin{equation}
%     \widetilde{n} \approx \pd{\widetilde{n}}{Re_{\delta_2}} \left(Re_{\delta_2} - Re_{\delta_{2_\mathrm{crit}}}\right);
% \end{equation}

% \where

% \begin{equation}
% \pd{\widetilde{n}}{Re_{\delta_2}}  = 0.01\left[\left(2.4H_{12} -3.7 +2.5\tanh\left[1.5H_{12}-4.65\right]\right)^2+0.25\right]^{1/2}
% \end{equation}

% \noindent and

% \begin{equation}
%     Re_{\delta_{2_\mathrm{crit}}} = 10^{\left(\frac{1.415}{H_{12}-1}-0.489\right)\tanh\left(\frac{20}{H_{12}-1}-12.9\right)+\frac{3.295}{H_{12}-1}+0.44}.
% \end{equation}

% \noindent Transition can occur when \(Re_{\delta_{2}} > Re_{\delta_{2_\mathrm{crit}}}\).

% Though more recent approximations are used in new versions XFoil, presumably because they are better either for accuracy or computation.  They are:

% \begin{equation}
%     \pd{\widetilde{n}}{Re_{\delta_2}}  = 0.028\left(H_{12}-1\right) - 0.0345 e^{-\left(\frac{3.87}{H_{12}-1}-2.52\right)^2}
% \end{equation}

% \begin{equation}
%     Re_{\delta_{2_\mathrm{crit}}} = 10^{0.7*\tanh\left(\frac{14}{H_{12}-1}-9.24\right)+2.492\left(\frac{1}{H_{12}-1}\right)^{0.43}+0.62}
% \end{equation}

% \where the last constant could be chosen to be anything between 0.62 and 0.7 (or so it has varied in different versions of XFoil).


% Transition can be tripped manually with the application of rough strips applied to the physical bodies, and we can model those by simply manually setting the transition point.
% In addition, for ducted rotor applications, we assume that if transition has not occurred in front of the rotor, the rotor trips turbulence.


\subsection{Viscous Drag}
% Profile drag is computed by applying a compressible version of the Squire/Young formula using values of BL parameters from the trailing edge extrapolated to unity H12 in wake (from Eppler).

\subsubsection{Squire-Young Formula}

The Squire-Young formula\scite{Squire_1937} is a well-known method for approximating the 2D viscous drag of airfoils.
They show that the airfoil viscous drag coefficient can be determined from the momentum thickness of the wake far downstream of the airfoil trailing edge as

\begin{equation}
\label{eqn:sqcd}
    c_d = \frac{2 \delta_{2_\mathrm{wake}}}{c}.
\end{equation}

\noindent They then derive an expression for the downstream wake momentum thickness based on values at the airfoil trailing edge:

\begin{equation}
\label{eqn:sqd2w}
    \delta_{2_\mathrm{wake}} = \delta_{2_{TE}} \left(\frac{U_{e_{TE}}}{U_\infty}\right)^{\frac{5+H_{12_{TE}}}{2}}.
\end{equation}

\noindent Putting \cref{eqn:sqcd,eqn:sqd2w} together, we have the Squire-Young formula:

\begin{equation}
\label{eqn:squire-young}
\eqbox{
    c_d = \frac{2 \delta_{2_{TE}}}{c} \left(\frac{U_{e_{TE}}}{U_\infty}\right)^{\frac{5+H_{12_{TE}}}{2}}.
}
\end{equation}

% Eppler\scite{Eppler_1980} notes that the Squire-Young formula over-predicts the drag coefficient for cases where the boundary layer is near separation.

% \begin{equation}
% \label{eqn:squire-young_eppler-mod}
%     c_d = \frac{2 \delta_{2_{TE}}}{c} \left(\frac{U_{e_{TE}}}{U_\infty}\right)^{\frac{5+H^*_{12_{TE}}}{2}};
% \end{equation}

% \where

% \begin{equation}
%     H^*_{12_{TE}} =
%     \begin{cases}
%         H_{12_{TE}}  & \mathrm{for}~~ H_{12_{TE}} \leq 2.5 \\
%         2.5  & \mathrm{for}~~ H_{12_{TE}} > 2.5.
%     \end{cases}
% \end{equation}

\noindent Note that \cref{eqn:squire-young} applies to just one side of the airfoil at a time, so we add the values associated with the top and bottom together to obtain the complete drag estimate.

% \subsubsection{Separation Treatment}
%
% In order to allow, but discourage separation in an optimization context, we apply additional drag if separation occurs before a user-defined allowance.
% We apply additional drag from a linear distribution beginning at the start of the boundary layer and ending at the user-defined allowance point, with a maximum value at the start of the boundary layer being another user-defined option, and tapering to zero additional drag at the selected allowance point.
% Thus, if separation occurs early, a large drag penalty is applied; and if separation occurs very near the trailing edge, little to no penalty is applied.
% The user can set the separation distance allowance and maximum penalty as desired to further discourage or allow the optimizer to select designs with some separation.
% Note that this treatment is used for optimization purposes and does not represent any sort of physically accurate modeling of post-separation drag.

\subsubsection{Total Drag Force}

To obtain the total, dimensional drag force on the duct, we apply the definition of drag coefficient to get the dimensional drag per unit length:

\begin{equation}
    \eqbox{
    D' = \frac{1}{2} \rho V_\infty^2 c \left(c_{d_\text{upper}} + c_{d_\text{lower}}\right)
}
\end{equation}

\where \(c\) is the chord length of the duct.
%
We then integrate the drag per unit length about the circumference of the duct, using the duct exit diameter as the characteristic length

\begin{equation}
    \eqbox{
    D = D' \pi d_\text{exit}
}
\end{equation}

% \subsection{Separating skin friction and pressure drag}

% The Squire-Young formula approximates the combination of the skin friction and pressure drag.
% If we would like to separate these, we need to have another method for determining either the skin friction drag or the pressure drag.
% Fortunately, part of our boundary layer calculations includes the calculation of the skin friction coefficient, which we can integrate across the surface to obtain the total skin friction drag.

% \subsection{Post-separation treatment}
% integrates separated boundary layer using a nameless empirical relation (TODO: look up methods for this to see if there is a typical/well known one being used)
% (see line 1057 in blcalc.f for equations used.)


% \section{Boundary Layer Solution}
% ode solver in dfdc is a 2nd order Runge-Kutta scheme
% change in H23 (for laminar) and H12 (for turbulent) are used as error measures.


\subsection{A Simplified Drag Model for Center Bodies}

Since integral boundary layer methods are developed on the assumption of 2-D bodies, those assumptions break down when we consider relatively narrow bodies of revolution such as the center body of a ducted fun.
%
Therefore, we will use a different model for such cases.
%
For the center body, we instead utilize a method similar to those used to approximate fuselage drag on aircraft.
%
We take the drag coefficient for the center body (\(C_D\)) to be

\begin{equation}
    \begin{aligned}
        C_D &= \frac{D}{0.5 \rho V_\infty^2 S_\text{ref}}, \\
            &= C_f f_\text{form} \frac{S_\text{wet}}{S_\text{ref}};
    \end{aligned}
\end{equation}

\where \(\rho_\infty\) is the freestream density, \(V_\infty\) is the freestream velocity, \(C_f\) is the flat-plate skin-friction coefficient, \(f_\text{form}\) is a form factor correction, \(S_\text{wet}\) is the wetted area of the center body, and \(S_\text{ref}\) is a reference area.
%
Setting the two expressions equal, we solve for drag as

\begin{equation}
    \eqbox{
    D = 0.5 \rho_\infty V_\infty^2 C_f f_\text{form} S_\text{wet}.
}
\end{equation}

\noindent We use the form factor expression from \nolink{\citeauthor{Shevell_1983}} \scite{Shevell_1983} based on fineness-ratio (\(l/d\)):

\begin{equation}
    \eqbox{
    f_\text{form} = 1 + \frac{2.8}{\left(l/d\right)^{1.5}} + \frac{3.8}{\left(l/d\right)^3}.
}
\end{equation}

\noindent We take \(C_f\) to be the skin friction coefficient from Schlichting for a flat plate of length equal to the length of the center body and the velocity at the body trailing edge:

\begin{equation}
    \eqbox{
    C_f = \frac{0.074}{Re_l^{0.2}}.
}
\end{equation}
