%---------------------------------#
%    Assemble and Solve System    #
%---------------------------------#
\section{Assembling and Solving the Linear System}

To find the strengths of each vortex node that result in a vortex distribution inducing a flow field matching our prescribed body geometry, we need to assemble a system composed of \cref{eqn:neumann2} for each panel.
%
Note, however, that currently our expression for \(\vect{K}\) is indexed according to panel, and contains information about more than one panel node, which we need to remedy in order to get expressions for the individual strengths at each node.
%
This is precisely why we separated out the node influences in the previous subsection.
%
Thus each node has a component of influences associated with each panel to which it is an edge point.
%
For the \(j\)th node then, we can add the contributions due to the \((j-1)\)th and \(j\)th panels for which it is an edge point.
%
This allows us to assemble the influence coefficient matrix based on a node-control point scheme rather than a panel-control point scheme:

\begin{equation}
\vect{G}_{ij} =
    \begin{cases}
        IC_{ij}        \hfill & \text{for }j=1,N+1 \\
        IC_{ij} + IC_{i(j-1)} & \text{for }2 \leq j \leq N \\
           % \hfill IC_{i(j-1)} & \text{for }j = N+1. \\
    \end{cases}
\end{equation}

\where \(\vect{G}\) is the \(N \times N+1\) matrix whose elements, \(\vect{G}_{ij}\), are the influence coefficients of the \(j\)th node (\(N+1\) total) on the \(i\)th control point (\(N\) total); and the influence components, \(IC\), are defined in \cref{eqn:nominalic,eqn:panelselfic} for the nominal and self-induced cases, respectively.
%
Since \(\vect{G}\) is not square, as it has one more unknown than boundary conditions, we cannot solve the system directly as is.
%
Fortunately, we also require an additional condition to make things work.

%---------------------------------#
%         Kutta Condition         #
%---------------------------------#
\subsection{The Kutta Condition}

One of the shortcomings of using potential flow theory is that by itself, it lacks inherent mechanisms for ensuring the flow leaves the surface of lifting bodies at the correct location and in the correct direction.
%
One solution to this problem is known as the Kutta condition, which can be stated in several equivalent ways.
%
However it may be stated, the Kutta condition requires the flow over a lifting body with a sharp trailing edge to leave the body at the trailing edge in a manner roughly tangent to the trailing edge.
%
Therefore we can artificially enforce conditions that are observed in real, viscous flows at relatively low angles of attack.

Just as there are several equivalent ways to state the Kutta condition, there are several ways that the Kutta condition may be implemented.
%
One method is to require zero circulation at the trailing edge.
%
We can enforce this by setting the strengths of the first and last panel nodes to be equal\sidenote{Note that for a sharp trailing edge, where the nodes are coincident, they really should be equal anyway since they occupy the same point in space.} and opposite such that

\begin{equation}
    \eqbox{
    \gamma_1 + \gamma_N = 0.
}
\end{equation}
%
In order to make our system square, we simply add the Kutta condition as the \(N+1\)th equation.

By itself, this version of the Kutta condition can lead to spurious spikes in surface velocity near the trailing edge.
%
In order increase the numerical robustness of the panel method, we apply an additional, indirect Kutta condition used in the panel method of the Ducted Fan Design Code.\scite{DFDC}
%
It entails placing an additional control point just inside the interior of the duct trailing edge and define an associated unit normal oriented such that the unit normal is effectivelyin the direction of the bisection angle of the trailing edge panels.
%
We also place an additional control point inside the center body if it has a blunt trailing edge.\sidenote{We will discuss shortly the case where the center body has a sharp trailing edge.}

 We apply the same boundary condition on these control point as the other control points in that we set the normal velocity induced by the freestream to be equal and opposite to the tangential velocity induced by the body boundaries on the control point.

\begin{equation}
    \eqbox{
    \sum_{j=1}^{N+1} \gamma_j\vect{G}^\multimapinv_{kj} = - \vect{V}_\infty \cdot \hat{\vect{n}}_k.
}
\end{equation}

\where the elements of \(\vect{G}^\multimapinv\) are the expressions defined in \cref{eqn:nominalic}.

Upon the addition of this equation, however, we find ourselves with insufficient unknowns (one for each body being modeled).
%
To remedy this insufficiency, we simply apply a dummy strength, \(\tau_k\), for the \(k\)th body and set all of its associated influence coefficients, \(\mathcal{T}\), to 1 for the panels of the body it is applied to and zero elsewhere (including itself).

\begin{equation}
    \mathcal{T}_{ik} =
        \begin{cases}
            1 & \text{if}~i=k\\
            0 & \text{otherwise}
        \end{cases}
\end{equation}

We mentioned placing the additional control point just inside the trailing edge.
%
This is done (rather than right at the middle of the trailing edge gap between the trailing edge nodes) to avoid numerical issues if the trailing edge is indeed sharp.
%
We specifically place the node along the line bisecting the trailing edge angle and passing through the point halfway between the trailing edge nodes.
%
The position is calculated as follows

\begin{subequations}
    \begin{align}
        z_{cp} = \overline{z}_{TE} - \epsilon \overline{\Delta s}_{TE} \frac{z_\text{diff}}{s_\text{diff}} \\
        r_{cp} = \overline{r}_{TE} - \epsilon \overline{\Delta s}_{TE} \frac{r_\text{diff}}{s_\text{diff}} \\
        \hat{n}_{z_{cp}} = \frac{z_\text{diff}}{s_\text{diff}} \\
        \hat{n}_{r_{cp}} = \frac{r_\text{diff}}{s_\text{diff}}
    \end{align}
\end{subequations}

\where

\begin{align}
    \epsilon &= 0.05 \\
    \overline{z}_{TE} &= \frac{z_1+z_{N+1}}{2}\\
    \overline{r}_{TE} &= \frac{r_1+r_{N+1}}{2}\\
    \overline{\Delta s}_{TE} &= \frac{\Delta s_1+\Delta s_{N}}{2}\\
    z_\text{diff} &= \Delta z_{N} - \Delta z_{1} \\
    r_\text{diff} &= \Delta r_{N} - \Delta r_{1} \\
    s_\text{diff} &= \left[ z_\text{diff}^2 + r_\text{diff}^2\right]^{1/2}
\end{align}

\where the \(\Delta (\cdot)\) lengths are calculated in the clockwise direction as before, and \(\epsilon\) is chosen for generally good numerical behavior.

\begin{figure}
    \centering
    \input{panel_method/figures/internal_controlpoint_placement.tikz}
    \caption{Geometric explanation of internal control point placement.}
    \label{fig:pseudocplocation}
\end{figure}

\subsection{Additional Considerations for Open Bodies}

The Kutta condition we have applied assumes that the trailing edge is both sharp and thin.
%
This approximation tends to be relatively good for a large variety of geometries, and is well behaved numerically, but eventually breaks down.
%
Specifically in the case of blunt trailing edges, when the trailing edge panel nodes are not coincident, the flow field can tend to flow into the inside of the body through the open trailing edge.
%
To prevent this, we will add a trailing edge panel with distribution strengths determined from the adjacent panels, similar to the method used by XFOIL\scite{Xfoil,mfoil} for blunt trailing edges.

For any trailing edge panel, we will set a vortex and source distribution along the panel based on its orientation to the adjacent panels and the distribution strengths at the shared node locations:

\begin{align}
    \gamma_{TE_j} &= \left(\hat{\vect{n}}_{TE_j} \cdot \hat{\vect{n}}_{\text{adj}_j}\right)\gamma_{\text{adj}_j} \\ %- \left|\hat{\vect{n}}_{TE} \times \hat{\vect{n}}_{\text{adj}_j}\right|\sigma_{\text{adj}_j} \\
    \sigma_{TE_j} &= -\left|\hat{\vect{n}}_{TE_j} \times \hat{\vect{n}}_{\text{adj}_j}\right|\gamma_{\text{adj}_j}. %+ \left(\hat{\vect{n}}_{TE} \cdot \hat{\vect{n}}_{\text{adj}_j}\right)\sigma_{\text{adj}_j}
\end{align}

\where the ``adj'' subscript indicates the adjacent panel.
%
Based on these definitions of strength distributions across the trailing edge panels, we can take the unit strengths (relative to the unknown distribution strengths on the shared nodes) to be

\begin{align}
        % \pd{\gamma_{TE_j}}{\sigma} &= - \left|\hat{\vect{n}}_{TE} \times \hat{\vect{n}}_{\text{adj}_j}\right| \\
        \hat{\gamma}_{TE_j} &= \hat{\vect{n}}_{TE_j} \cdot \hat{\vect{n}}_{\text{adj}_j} \\
        % \pd{\sigma_{TE_j}}{\sigma} &=  \left(\hat{\vect{n}}_{TE} \cdot \hat{\vect{n}}_{\text{adj}_j}\right) \\
        \hat{\sigma}_{TE_j} &= -\left|\hat{\vect{n}}_{TE_j} \times \hat{\vect{n}}_{\text{adj}_j}\right|.
\end{align}

\noindent The vortex term here enforces smooth streamlines off the upper and lower surfaces, despite the trailing edge gap.
%
The source term enforces the no flow through condition at the trailing edge.

For trailing edge panels which have a node on the axis of rotation, for example, in the case of a center body with a blunt trailing edge, we set the strength (\(\gamma_{TE_j}\)) and derivative (\(\partial \gamma_{TE_j}/\partial \gamma\)) of the vortex distribution at the axis to zero.
%
Since we do not have an adjacent panel on the axis side of such a trailing edge panel, we will simply use the same adjacent panel to calculate values for both source nodes of the trailing edge panel. %\sidenote{We are effectively defining a constant strength source panel in this case.}

To add the trailing edge panels to the linear system we do not want to add any more equations, because we have defined the trailing edge panel strengths according to unknowns we already have in the system.
%
As such, we simply need to augment the influence coefficients for the panels adjacent to the trailing edge panels, since all the trailing edge panel information comes directly from those adjacent panels.
%
For each panel with a node bordering a trailing edge panel, we add the following to the unit induced velocity on every control point

\begin{equation}
    \hat{\vect{V}}^\gamma_{i{TE}_j} \stackrel{+}{=} \hat{\vect{V}}^\gamma_{iTE_j}\hat{\gamma}_{TE_j} + \hat{\vect{V}}^\sigma_{iTE_j}\hat{\sigma}_{TE_j}.
\end{equation}
%
In other words, we add the unit induced velocity associated with the trailing edge node to the panel sharing that node scaled by how aligned the trailing edge and adjacent panel are.
%
As an example, if the duct had a blunt trailing edge, we would define a trailing edge panel spanning the gap from the first to the last node in the airfoil geoemetry.
%
We would then define the strengths and changes in strength relative to the first and last panels of the geometry (those at the trailing edge).
%
Finally, we would augment the unit induced velocity due to the first and last nodes by the above expressions for the trailing edge gap panel we defined.
%
We can apply this to the velocity directly, or we can simply add the velocities dotted with the control point normal vectors to the influence coefficient matrix after the fact.

\begin{equation}
    \label{eqn:teaicaugment}
    \eqbox{
    \vect{G}_{ij} \stackrel{+}{=}  \left[\hat{\vect{V}}^\gamma_{iTE_j}\hat{\gamma}_{TE_j} + \hat{\vect{V}}^\sigma_{iTE_j}\hat{\sigma}_{TE_j}\right]\cdot\hat{\vect{n}}_i,
}
\end{equation}

\where the \(j\)th components of unit induced velocity, \(\hat{\vect{V}}\), are calculated from \cref{eqn:nominalic}.\sidenote{Exchanging the vortex ring induced unit velocities for those induced by source rings for the source terms in \cref{eqn:teaicaugment}.}



%-----------------------------------------#
%            Prescribed Nodes             #
%-----------------------------------------#
\subsection{Additional Considerations for Nodes on the Axis of Revolution}

As we have already discussed, annular airfoils with non-zero cambered cross-sections require the addition of a Kutta condition.
%
Bodies of revolution do not require such a condition in an axisymmetric scheme, but rather have other unique features to consider.
%
Specifically, bodies of revolution will have a panel node on the axis of revolution (at the leading edge).
%
As we can see in the definition of unit induced velocity (\cref{eqn:ringvortexinducedvelocity}), if an influence point lies on the axis, that is if \(r_o = 0\), then the induced velocity becomes infinite.
%
In reality, the induced velocity from such a point is zero.
%
Therefore in our system, we will need to prescribe the strengths of panel nodes on the axis of rotation to be zero strength.
%
In order to achieve this, we take an approach similar to applying the Kutta condition: we simply add the equation

\begin{equation}
    \eqbox{
    \gamma_{LE}^{cb} = 0
}
\end{equation}
%
to the system, where \(\gamma_{LE}^{cb}\) is the prescribed node strength for the center body leading edge.
%
This additional equation also solves the issue of the matrix not being square due to there still being \(N+1\) nodes and only \(N\) panels for a body of revolution.
%
If the center body trailing edge is sharp, then we have an additional node on the axis of rotation and also need to prescribe its strength to zero.
%
As it turns out, we do not actually need the additional internal control point for bodies of revolution, but it doesn't hurt us to have it implemented.
%
In the case of a closed trailing edge, we will effectively remove the internal control point and substitute its equation with an equation prescribing the trailing edge node strength to be zero like the leading edge node:

\begin{equation}
    \eqbox{
    \gamma_{TE}^{cb} = 0.
}
\end{equation}

\noindent Since we still have an additional equation, we will keep the dummy variable in place simply to keep the system square.
%
Note that we could have instead removed equations from the system rather than adding equations, but it turns out to be more convenient for consistent book-keeping purposes to put things together as we have described.


%---------------------------------#
%              SOLVE              #
%---------------------------------#
\subsection{Solving the linear system}

To avoid confusion, we will let \(\vect{G}^*\) represent the influence matrix augmented by the Kutta condition, additional trailing edge control point equations, and any prescribed node equations.
%
Because the overall coupled solver in DuctAPE will need to solve the linear system for the panel method many times, it is advantageous to do as much precomputation as possible for the panel method.
%
The first thing that we will note is that the body geometry will not change throughout the coupled solve.
%
This means that the influence matrix \(\vect{G}^*\) can be fully precomputed and stored.
%
Due to this fact, we can also speed up the multiple linear solves by performing a Lower-Upper (LU) decomposition of \(\vect{G}^*\) such that

\begin{equation}
    \vect{G}^* = \vect{LU}
\end{equation}

\where \(\vect{L}\) and \(\vect{U}\) are the lower and upper triangular matrices of the LU decomposition.
%
By precomputing the LU decomposition, we can speed up the solution process of the linear system, which can now be expressed as

\begin{equation}
    \vect{LU}\vect{\gamma} = \vect{b}
\end{equation}

\where \(\vect{b}=\left(-\vect{V}_\text{ext} \cdot \hat{n}\right)\).
%
We can solve this system through the forward and backward substitution in two steps:
\begin{enumerate}
    \item Solve \(\vect{L}\vect{y} = \vect{b}\) for \(\vect{y}\).
    \item Solve \(\vect{U}\vect{\gamma} = \vect{y}\) for \(\vect{\gamma}\).
\end{enumerate}
%
Although this is a two-step process, it ends up being numerically more efficient than a more direct system solve method, and again has the benefit of being able to be precomputed and used repeatedly.


