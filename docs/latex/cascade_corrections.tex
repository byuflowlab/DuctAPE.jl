\section{Airfoil Polar Corrections for Ducted Rotors}

When airfoil/cascade aerodynamic data is unavailable for each of the rotor blade sections, it may be possible to use airfoil section data for subsonic flow regimes (and perhaps even mildly super sonic regimes) without terrible inaccuracy if we apply some corrections to the airfoil polar.
%
Though the accuracy of solutions does break down quickly for high subsonic, transonic, and supersonic cases.
%
If supersonic airfoil or cascade data is an absolute necessity (e.g. for fully super sonic data), then the following corrections should not be used, and proper section polars should be generated through appropriate means.
%
The following subsections cover the airfoil data corrections and adjustments available in DuctAPE as well as addition adjustments made to the implementation of each as required for suitability in gradient-based optimization.
%
Specifically, we discuss the nominal correction methodology and then go over implementation details required for removing any discontinuities\sidenote{As we do not necessarily know a priori where in the design space an optimizer will search, we need to ensure that the correction models are continuous over the entire design space when using gradient-based optimization.} present in the nominal formulation.


\subsection{Stall Cutoffs}
Before any actual corrections are applied, we need to make an important adjustment to the nominal airfoil data.
%
Especially if the airfoil data provided includes information in the post stall regime, we see that it is possible to obtain the same lift coefficient at two different angles of attack.
%
This feature of the airfoil data can make it difficult for the DuctAPE solver to converge, since a blade element method is the foundation of the rotor and wake models.
%
To remove the possibility of multiple solutions for the lift, we effectively cut off the airfoil data post stall and assign our own, slightly positive lift slope above the maximum lift coefficient and below the minimum lift coefficient.
%
We keep the nominal data between the minimum and maximum lift coefficients and smoothly blend that data into the prescribed lift slopes for the rest of the possible range of angles of attack.
%
We apply a similar procedure to the drag data, but use the cutoff angles of attack from the lift curve.
%
\Cref{fig:stall-cutoff} shows an example of our stall cutoff adjustment to lift and drag data.

\begin{figure}[htb]
     \centering
     \begin{subfigure}[t]{0.45\textwidth}
         \centering
        % Recommended preamble:
% \usetikzlibrary{arrows.meta}
% \usetikzlibrary{backgrounds}
% \usepgfplotslibrary{patchplots}
% \usepgfplotslibrary{fillbetween}
% \pgfplotsset{%
%     layers/standard/.define layer set={%
%         background,axis background,axis grid,axis ticks,axis lines,axis tick labels,pre main,main,axis descriptions,axis foreground%
%     }{
%         grid style={/pgfplots/on layer=axis grid},%
%         tick style={/pgfplots/on layer=axis ticks},%
%         axis line style={/pgfplots/on layer=axis lines},%
%         label style={/pgfplots/on layer=axis descriptions},%
%         legend style={/pgfplots/on layer=axis descriptions},%
%         title style={/pgfplots/on layer=axis descriptions},%
%         colorbar style={/pgfplots/on layer=axis descriptions},%
%         ticklabel style={/pgfplots/on layer=axis tick labels},%
%         axis background@ style={/pgfplots/on layer=axis background},%
%         3d box foreground style={/pgfplots/on layer=axis foreground},%
%     },
% }

\begin{tikzpicture}[/tikz/background rectangle/.style={fill={rgb,1:red,1.0;green,1.0;blue,1.0}, fill opacity={1.0}, draw opacity={1.0}}, show background rectangle]
\begin{axis}[point meta max={nan}, point meta min={nan}, legend cell align={left}, legend columns={1}, title={}, title style={at={{(0.5,1)}}, anchor={south}, font={{\fontsize{14 pt}{18.2 pt}\selectfont}}, color={rgb,1:red,0.0;green,0.0;blue,0.0}, draw opacity={1.0}, rotate={0.0}, align={center}}, legend style={color={rgb,1:red,0.0;green,0.0;blue,0.0}, draw opacity={0.0}, line width={1}, solid, fill={rgb,1:red,0.0;green,0.0;blue,0.0}, fill opacity={0.0}, text opacity={1.0}, font={{\fontsize{8 pt}{10.4 pt}\selectfont}}, text={rgb,1:red,0.0;green,0.0;blue,0.0}, cells={anchor={center}}, at={(1.02, 1)}, anchor={north west}}, axis background/.style={fill={rgb,1:red,0.0;green,0.0;blue,0.0}, opacity={0.0}}, anchor={north west}, xshift={0.0mm}, yshift={-0.0mm}, width={45.8mm}, height={50.8mm}, scaled x ticks={false}, xlabel={Angle of Attack (radians)}, x tick style={color={rgb,1:red,0.0;green,0.0;blue,0.0}, opacity={1.0}}, x tick label style={color={rgb,1:red,0.0;green,0.0;blue,0.0}, opacity={1.0}, rotate={0}}, xlabel style={at={(ticklabel cs:0.5)}, anchor=near ticklabel, at={{(ticklabel cs:0.5)}}, anchor={near ticklabel}, font={{\fontsize{11 pt}{14.3 pt}\selectfont}}, color={rgb,1:red,0.0;green,0.0;blue,0.0}, draw opacity={1.0}, rotate={0.0}}, xmajorgrids={false}, xmin={-0.7853981633974483}, xmax={0.7853981633974483}, xticklabels={{$-0.7$,$0.0$,$0.7$}}, xtick={{-0.7,0.0,0.7}}, xtick align={inside}, xticklabel style={font={{\fontsize{8 pt}{10.4 pt}\selectfont}}, color={rgb,1:red,0.0;green,0.0;blue,0.0}, draw opacity={1.0}, rotate={0.0}}, x grid style={color={rgb,1:red,0.0;green,0.0;blue,0.0}, draw opacity={0.1}, line width={0.5}, solid}, axis x line*={left}, x axis line style={color={rgb,1:red,0.0;green,0.0;blue,0.0}, draw opacity={1.0}, line width={1}, solid}, scaled y ticks={false}, ylabel={$c_\ell$}, y tick style={color={rgb,1:red,0.0;green,0.0;blue,0.0}, opacity={1.0}}, y tick label style={color={rgb,1:red,0.0;green,0.0;blue,0.0}, opacity={1.0}, rotate={0}}, ylabel style={{rotate=-90}}, ymajorgrids={false}, ymin={-1.5176967357511295}, ymax={2.0405709888787014}, yticklabels={{$-1$,$0$,$1$,$2$}}, ytick={{-1.0,0.0,1.0,2.0}}, ytick align={inside}, yticklabel style={font={{\fontsize{8 pt}{10.4 pt}\selectfont}}, color={rgb,1:red,0.0;green,0.0;blue,0.0}, draw opacity={1.0}, rotate={0.0}}, y grid style={color={rgb,1:red,0.0;green,0.0;blue,0.0}, draw opacity={0.1}, line width={0.5}, solid}, axis y line*={left}, y axis line style={color={rgb,1:red,0.0;green,0.0;blue,0.0}, draw opacity={1.0}, line width={1}, solid}, colorbar={false}]
    \addplot[color={rgb,1:red,0.0;green,0.3608;blue,0.6706}, name path={9171459e-a13d-49e5-92f5-52e4d505738f}, draw opacity={1.0}, line width={1.0}, solid, forget plot]
        table[row sep={\\}]
        {
            \\
            -0.3490658503988659  -0.5201441118382882  \\
            -0.3316125578789226  -0.3009904795535917  \\
            -0.3141592653589793  -0.2779784053141576  \\
            -0.2792526803190927  -0.8135825414572085  \\
            -0.2617993877991494  -1.1290117188523527  \\
            -0.24434609527920614  -1.042343558618539  \\
            -0.22689280275926285  -0.9311329870526581  \\
            -0.20943951023931953  -0.8182553907504921  \\
            -0.19198621771937624  -0.7003629922793199  \\
            -0.17453292519943295  -0.5891841174576591  \\
            -0.15707963267948966  -0.48953335928919767  \\
            -0.13962634015954636  -0.3847910179920036  \\
            -0.12217304763960307  -0.28163008275874873  \\
            -0.10471975511965977  -0.17689779064109243  \\
            -0.08726646259971647  -0.07238560327726853  \\
            -0.06981317007977318  0.03280051003689001  \\
            -0.05235987755982988  0.13735032221985904  \\
            -0.03490658503988659  0.24218615431637724  \\
            -0.017453292519943295  0.346774152000716  \\
            0.0  0.4509307006002609  \\
            0.017453292519943295  0.5529658741865668  \\
            0.03490658503988659  0.6422769913003485  \\
            0.05235987755982988  0.8167511813706088  \\
            0.06981317007977318  0.9318767158290138  \\
            0.08726646259971647  1.0230377139261682  \\
            0.10471975511965977  1.1089781698218313  \\
            0.12217304763960307  1.186140905228946  \\
            0.13962634015954636  1.2500405013185019  \\
            0.15707963267948966  1.3052698796338766  \\
            0.17453292519943295  1.3699631833331694  \\
            0.19198621771937624  1.434119447651737  \\
            0.20943951023931953  1.4925656461899794  \\
            0.22689280275926285  1.5451063002185872  \\
            0.24434609527920614  1.5906938590534738  \\
            0.2617993877991494  1.6222318346659939  \\
            0.2792526803190927  1.6483353638284044  \\
            0.29670597283903605  1.6516017692844258  \\
            0.3141592653589793  1.6571219597359075  \\
            0.3316125578789226  1.6460403304035667  \\
            0.3490658503988659  1.6181595598688847  \\
            0.3665191429188092  1.566194508085643  \\
            0.3839724354387525  1.5270802607938798  \\
            0.40142572795869574  1.4855879466186925  \\
            0.41887902047863906  1.443112490399634  \\
            0.4363323129985824  1.4019277874118095  \\
        }
        ;
    \addplot[color={rgb,1:red,0.7529;green,0.3255;blue,0.4039}, name path={52538775-2d0b-4763-898f-8f7bf95f81c3}, draw opacity={1.0}, line width={2}, dashed, forget plot]
        table[row sep={\\}]
        {
            \\
            -3.141592653589793  -1.4169910454314172  \\
            -3.12413936106985  -1.415245716179423  \\
            -3.1066860685499065  -1.4135003869274285  \\
            -3.0892327760299634  -1.4117550576754343  \\
            -3.07177948351002  -1.4100097284234399  \\
            -3.0543261909900767  -1.4082643991714456  \\
            -3.036872898470133  -1.4065190699194512  \\
            -3.01941960595019  -1.404773740667457  \\
            -3.0019663134302466  -1.4030284114154625  \\
            -2.9845130209103035  -1.4012830821634683  \\
            -2.9670597283903604  -1.3995377529114739  \\
            -2.949606435870417  -1.3977924236594796  \\
            -2.9321531433504737  -1.3960470944074852  \\
            -2.91469985083053  -1.394301765155491  \\
            -2.897246558310587  -1.3925564359034965  \\
            -2.8797932657906435  -1.3908111066515023  \\
            -2.8623399732707004  -1.389065777399508  \\
            -2.844886680750757  -1.3873204481475137  \\
            -2.827433388230814  -1.3855751188955192  \\
            -2.8099800957108707  -1.383829789643525  \\
            -2.792526803190927  -1.3820844603915305  \\
            -2.775073510670984  -1.3803391311395363  \\
            -2.7576202181510405  -1.3785938018875417  \\
            -2.7401669256310974  -1.3768484726355474  \\
            -2.722713633111154  -1.3751031433835532  \\
            -2.705260340591211  -1.3733578141315588  \\
            -2.6878070480712672  -1.3716124848795646  \\
            -2.670353755551324  -1.36986715562757  \\
            -2.652900463031381  -1.368121826375576  \\
            -2.6354471705114375  -1.3663764971235814  \\
            -2.6179938779914944  -1.3646311678715872  \\
            -2.600540585471551  -1.3628858386195928  \\
            -2.5830872929516078  -1.3611405093675986  \\
            -2.5656340004316642  -1.3593951801156041  \\
            -2.548180707911721  -1.35764985086361  \\
            -2.5307274153917776  -1.3559045216116155  \\
            -2.5132741228718345  -1.3541591923596212  \\
            -2.4958208303518914  -1.352413863107627  \\
            -2.478367537831948  -1.3506685338556326  \\
            -2.4609142453120048  -1.3489232046036383  \\
            -2.443460952792061  -1.347177875351644  \\
            -2.426007660272118  -1.3454325460996497  \\
            -2.4085543677521746  -1.3436872168476552  \\
            -2.3911010752322315  -1.341941887595661  \\
            -2.373647782712288  -1.3401965583436666  \\
            -2.356194490192345  -1.3384512290916724  \\
            -2.3387411976724017  -1.336705899839678  \\
            -2.321287905152458  -1.3349605705876837  \\
            -2.303834612632515  -1.3332152413356895  \\
            -2.2863813201125716  -1.331469912083695  \\
            -2.2689280275926285  -1.3297245828317008  \\
            -2.251474735072685  -1.3279792535797064  \\
            -2.234021442552742  -1.326233924327712  \\
            -2.2165681500327983  -1.3244885950757177  \\
            -2.199114857512855  -1.3227432658237233  \\
            -2.181661564992912  -1.320997936571729  \\
            -2.1642082724729685  -1.3192526073197346  \\
            -2.1467549799530254  -1.3175072780677404  \\
            -2.129301687433082  -1.315761948815746  \\
            -2.111848394913139  -1.3140166195637517  \\
            -2.0943951023931953  -1.3122712903117575  \\
            -2.076941809873252  -1.310525961059763  \\
            -2.0594885173533086  -1.3087806318077688  \\
            -2.0420352248333655  -1.3070353025557744  \\
            -2.0245819323134224  -1.3052899733037802  \\
            -2.007128639793479  -1.3035446440517857  \\
            -1.9896753472735356  -1.3017993147997915  \\
            -1.9722220547535922  -1.300053985547797  \\
            -1.9547687622336491  -1.2983086562958028  \\
            -1.9373154697137058  -1.2965633270438084  \\
            -1.9198621771937625  -1.2948179977918142  \\
            -1.9024088846738192  -1.2930726685398197  \\
            -1.8849555921538759  -1.2913273392878255  \\
            -1.8675022996339325  -1.289582010035831  \\
            -1.8500490071139892  -1.2878366807838366  \\
            -1.832595714594046  -1.2860913515318422  \\
            -1.8151424220741026  -1.2843460222798482  \\
            -1.7976891295541595  -1.2826006930278537  \\
            -1.7802358370342162  -1.2808553637758595  \\
            -1.7627825445142729  -1.279110034523865  \\
            -1.7453292519943295  -1.2773647052718706  \\
            -1.7278759594743862  -1.2756193760198764  \\
            -1.710422666954443  -1.273874046767882  \\
            -1.6929693744344996  -1.2721287175158877  \\
            -1.6755160819145563  -1.2703833882638933  \\
            -1.658062789394613  -1.268638059011899  \\
            -1.6406094968746698  -1.2668927297599046  \\
            -1.6231562043547265  -1.2651474005079104  \\
            -1.6057029118347832  -1.263402071255916  \\
            -1.5882496193148399  -1.2616567420039217  \\
            -1.5707963267948966  -1.2599114127519273  \\
            -1.5533430342749532  -1.258166083499933  \\
            -1.53588974175501  -1.2564207542479386  \\
            -1.5184364492350666  -1.2546754249959444  \\
            -1.5009831567151233  -1.25293009574395  \\
            -1.4835298641951802  -1.2511847664919558  \\
            -1.4660765716752369  -1.2494394372399615  \\
            -1.4486232791552935  -1.247694107987967  \\
            -1.4311699866353502  -1.2459487787359729  \\
            -1.413716694115407  -1.2442034494839784  \\
            -1.3962634015954636  -1.2424581202319842  \\
            -1.3788101090755203  -1.2407127909799898  \\
            -1.361356816555577  -1.2389674617279955  \\
            -1.3439035240356336  -1.237222132476001  \\
            -1.3264502315156905  -1.2354768032240069  \\
            -1.3089969389957472  -1.2337314739720124  \\
            -1.2915436464758039  -1.2319861447200182  \\
            -1.2740903539558606  -1.2302408154680238  \\
            -1.2566370614359172  -1.2284954862160296  \\
            -1.239183768915974  -1.226750156964035  \\
            -1.2217304763960306  -1.225004827712041  \\
            -1.2042771838760873  -1.2232594984600467  \\
            -1.186823891356144  -1.2215141692080522  \\
            -1.1693705988362009  -1.219768839956058  \\
            -1.1519173063162575  -1.2180235107040636  \\
            -1.1344640137963142  -1.2162781814520693  \\
            -1.117010721276371  -1.214532852200075  \\
            -1.0995574287564276  -1.2127875229480807  \\
            -1.0821041362364843  -1.2110421936960862  \\
            -1.064650843716541  -1.209296864444092  \\
            -1.0471975511965976  -1.2075515351920976  \\
            -1.0297442586766543  -1.2058062059401033  \\
            -1.0122909661567112  -1.204060876688109  \\
            -0.9948376736367678  -1.2023155474361147  \\
            -0.9773843811168246  -1.2005702181841202  \\
            -0.9599310885968813  -1.198824888932126  \\
            -0.9424777960769379  -1.1970795596801316  \\
            -0.9250245035569946  -1.1953342304281374  \\
            -0.9075712110370513  -1.1935889011761431  \\
            -0.8901179185171081  -1.1918435719241487  \\
            -0.8726646259971648  -1.1900982426721545  \\
            -0.8552113334772214  -1.18835291342016  \\
            -0.8377580409572781  -1.1866075841681658  \\
            -0.8203047484373349  -1.1848622549161714  \\
            -0.8028514559173916  -1.1831169256641771  \\
            -0.7853981633974483  -1.1813715964121825  \\
            -0.767944870877505  -1.1796262671601883  \\
            -0.7504915783575616  -1.177880937908194  \\
            -0.7330382858376184  -1.1761356086561996  \\
            -0.7155849933176751  -1.1743902794042052  \\
            -0.6981317007977318  -1.172644950152211  \\
            -0.6806784082777885  -1.1708996209002165  \\
            -0.6632251157578453  -1.1691542916482223  \\
            -0.6457718232379019  -1.1674089623962278  \\
            -0.6283185307179586  -1.1656636331442336  \\
            -0.6108652381980153  -1.1639183038922394  \\
            -0.593411945678072  -1.162172974640245  \\
            -0.5759586531581288  -1.1604276453882505  \\
            -0.5585053606381855  -1.1586823161362563  \\
            -0.5410520681182421  -1.156936986884262  \\
            -0.5235987755982988  -1.1551916576322676  \\
            -0.5061454830783556  -1.1534463283802732  \\
            -0.4886921905584123  -1.1517009991282787  \\
            -0.47123889803846897  -1.1499556698762843  \\
            -0.45378560551852565  -1.14821034062429  \\
            -0.4363323129985824  -1.1464650113722956  \\
            -0.41887902047863906  -1.1447196821203012  \\
            -0.4014257279586958  -1.1429743528683032  \\
            -0.3839724354387525  -1.1412290236162885  \\
            -0.3665191429188092  -1.1394836943642543  \\
            -0.3490658503988659  -1.1377383651121984  \\
            -0.33161255787892263  -1.1359930358601178  \\
            -0.3141592653589793  -1.1342477066080103  \\
            -0.296705972839036  -1.1325023773558738  \\
            -0.2792526803190927  -1.1307570481037057  \\
            -0.2617993877991494  -1.1290117188515032  \\
            -0.24434609527920614  -1.0673703297654489  \\
            -0.22689280275926282  -0.9600263024640482  \\
            -0.20943951023931953  -0.8390275063155772  \\
            -0.19198621771937624  -0.7128355753988755  \\
            -0.17453292519943295  -0.5958631517799181  \\
            -0.15707963267948966  -0.4928629392470688  \\
            -0.13962634015954636  -0.3864150493039548  \\
            -0.12217304763960307  -0.28240367531486954  \\
            -0.10471975511965977  -0.17726116393609845  \\
            -0.08726646259971647  -0.07255413228679569  \\
            -0.06981317007977318  0.03272308144716237  \\
            -0.05235987755982988  0.13731507045520258  \\
            -0.03490658503988659  0.24217023426102238  \\
            -0.017453292519943295  0.34676705675582137  \\
            2.524354896707238e-29  0.45092766780971344  \\
            0.017453292519943295  0.5529648339671895  \\
            0.03490658503988659  0.6422772150452551  \\
            0.05235987755982988  0.8167525748675211  \\
            0.06981317007977318  0.9318800530445993  \\
            0.08726646259971647  1.0230448926166291  \\
            0.10471975511965977  1.108993075002632  \\
            0.12217304763960307  1.1861715124428007  \\
            0.13962634015954636  1.2501036864275183  \\
            0.15707963267948966  1.3054003193012897  \\
            0.17453292519943295  1.3702167676527894  \\
            0.19198621771937624  1.4345870979114317  \\
            0.20943951023931953  1.4933812755086455  \\
            0.22689280275926282  1.5464052450423527  \\
            0.24434609527920614  1.592452247119527  \\
            0.2617993877991494  1.6242479988388934  \\
            0.2792526803190927  1.649122537580502  \\
            0.296705972839036  1.6527142214500519  \\
            0.3141592653589793  1.6571219597354827  \\
            0.33161255787892263  1.6588672889876588  \\
            0.3490658503988659  1.6606126182397793  \\
            0.3665191429188092  1.6623579474918544  \\
            0.3839724354387525  1.6641032767438921  \\
            0.4014257279586958  1.6658486059959028  \\
            0.41887902047863906  1.6675939352478955  \\
            0.4363323129985824  1.6693392644998795  \\
            0.45378560551852565  1.671084593751864  \\
            0.47123889803846897  1.6728299230038564  \\
            0.4886921905584123  1.6745752522558508  \\
            0.5061454830783556  1.676320581507845  \\
            0.5235987755982988  1.6780659107598395  \\
            0.5410520681182421  1.6798112400118337  \\
            0.5585053606381855  1.6815565692638281  \\
            0.5759586531581288  1.6833018985158223  \\
            0.593411945678072  1.6850472277678168  \\
            0.6108652381980153  1.686792557019811  \\
            0.6283185307179586  1.6885378862718055  \\
            0.6457718232379019  1.6902832155237997  \\
            0.6632251157578453  1.6920285447757941  \\
            0.6806784082777885  1.6937738740277883  \\
            0.6981317007977318  1.6955192032797828  \\
            0.7155849933176751  1.697264532531777  \\
            0.7330382858376184  1.6990098617837714  \\
            0.7504915783575616  1.7007551910357657  \\
            0.767944870877505  1.70250052028776  \\
            0.7853981633974483  1.7042458495397543  \\
            0.8028514559173916  1.7059911787917488  \\
            0.8203047484373349  1.707736508043743  \\
            0.8377580409572781  1.7094818372957374  \\
            0.8552113334772214  1.7112271665477317  \\
            0.8726646259971648  1.712972495799726  \\
            0.8901179185171081  1.7147178250517203  \\
            0.9075712110370513  1.7164631543037148  \\
            0.9250245035569946  1.718208483555709  \\
            0.9424777960769379  1.7199538128077034  \\
            0.9599310885968813  1.7216991420596977  \\
            0.9773843811168246  1.723444471311692  \\
            0.9948376736367678  1.7251898005636863  \\
            1.0122909661567112  1.7269351298156808  \\
            1.0297442586766543  1.728680459067675  \\
            1.0471975511965976  1.7304257883196692  \\
            1.064650843716541  1.7321711175716636  \\
            1.0821041362364843  1.7339164468236579  \\
            1.0995574287564276  1.7356617760756523  \\
            1.117010721276371  1.7374071053276465  \\
            1.1344640137963142  1.739152434579641  \\
            1.1519173063162575  1.7408977638316352  \\
            1.1693705988362009  1.7426430930836296  \\
            1.186823891356144  1.7443884223356239  \\
            1.2042771838760873  1.7461337515876183  \\
            1.2217304763960306  1.7478790808396125  \\
            1.239183768915974  1.749624410091607  \\
            1.2566370614359172  1.7513697393436012  \\
            1.2740903539558606  1.7531150685955956  \\
            1.2915436464758039  1.7548603978475898  \\
            1.3089969389957472  1.7566057270995843  \\
            1.3264502315156905  1.7583510563515785  \\
            1.3439035240356336  1.760096385603573  \\
            1.361356816555577  1.7618417148555672  \\
            1.3788101090755203  1.7635870441075616  \\
            1.3962634015954636  1.7653323733595558  \\
            1.413716694115407  1.7670777026115503  \\
            1.4311699866353502  1.7688230318635445  \\
            1.4486232791552935  1.770568361115539  \\
            1.4660765716752369  1.7723136903675332  \\
            1.4835298641951802  1.7740590196195276  \\
            1.5009831567151233  1.7758043488715218  \\
            1.5184364492350666  1.7775496781235163  \\
            1.53588974175501  1.7792950073755105  \\
            1.5533430342749532  1.781040336627505  \\
            1.5707963267948966  1.7827856658794992  \\
            1.5882496193148399  1.7845309951314936  \\
            1.6057029118347832  1.7862763243834878  \\
            1.6231562043547265  1.7880216536354823  \\
            1.6406094968746698  1.7897669828874765  \\
            1.658062789394613  1.7915123121394707  \\
            1.6755160819145563  1.7932576413914652  \\
            1.6929693744344996  1.7950029706434596  \\
            1.710422666954443  1.7967482998954538  \\
            1.7278759594743862  1.798493629147448  \\
            1.7453292519943295  1.8002389583994425  \\
            1.7627825445142729  1.801984287651437  \\
            1.7802358370342162  1.8037296169034311  \\
            1.7976891295541595  1.8054749461554254  \\
            1.8151424220741026  1.8072202754074198  \\
            1.832595714594046  1.808965604659414  \\
            1.8500490071139892  1.8107109339114085  \\
            1.8675022996339325  1.8124562631634027  \\
            1.8849555921538759  1.8142015924153971  \\
            1.9024088846738192  1.8159469216673914  \\
            1.9198621771937625  1.8176922509193858  \\
            1.9373154697137058  1.81943758017138  \\
            1.9547687622336491  1.8211829094233745  \\
            1.9722220547535922  1.8229282386753687  \\
            1.9896753472735356  1.8246735679273631  \\
            2.007128639793479  1.8264188971793573  \\
            2.0245819323134224  1.8281642264313518  \\
            2.0420352248333655  1.829909555683346  \\
            2.0594885173533086  1.8316548849353405  \\
            2.076941809873252  1.8334002141873347  \\
            2.0943951023931953  1.8351455434393291  \\
            2.111848394913139  1.8368908726913233  \\
            2.129301687433082  1.8386362019433178  \\
            2.1467549799530254  1.840381531195312  \\
            2.1642082724729685  1.8421268604473064  \\
            2.181661564992912  1.8438721896993007  \\
            2.199114857512855  1.845617518951295  \\
            2.2165681500327983  1.8473628482032893  \\
            2.234021442552742  1.8491081774552838  \\
            2.251474735072685  1.850853506707278  \\
            2.2689280275926285  1.8525988359592724  \\
            2.2863813201125716  1.8543441652112667  \\
            2.303834612632515  1.856089494463261  \\
            2.321287905152458  1.8578348237152553  \\
            2.3387411976724017  1.8595801529672498  \\
            2.356194490192345  1.861325482219244  \\
            2.373647782712288  1.8630708114712382  \\
            2.3911010752322315  1.8648161407232327  \\
            2.4085543677521746  1.8665614699752269  \\
            2.426007660272118  1.8683067992272213  \\
            2.443460952792061  1.8700521284792155  \\
            2.4609142453120048  1.87179745773121  \\
            2.478367537831948  1.8735427869832042  \\
            2.4958208303518914  1.8752881162351986  \\
            2.5132741228718345  1.8770334454871929  \\
            2.5307274153917776  1.8787787747391873  \\
            2.548180707911721  1.8805241039911815  \\
            2.5656340004316642  1.882269433243176  \\
            2.5830872929516078  1.8840147624951702  \\
            2.600540585471551  1.8857600917471646  \\
            2.6179938779914944  1.8875054209991589  \\
            2.6354471705114375  1.8892507502511533  \\
            2.652900463031381  1.8909960795031475  \\
            2.670353755551324  1.892741408755142  \\
            2.6878070480712672  1.8944867380071362  \\
            2.705260340591211  1.8962320672591306  \\
            2.722713633111154  1.8979773965111248  \\
            2.7401669256310974  1.8997227257631193  \\
            2.7576202181510405  1.9014680550151135  \\
            2.775073510670984  1.903213384267108  \\
            2.792526803190927  1.9049587135191022  \\
            2.8099800957108707  1.9067040427710966  \\
            2.827433388230814  1.9084493720230908  \\
            2.844886680750757  1.9101947012750853  \\
            2.8623399732707004  1.9119400305270795  \\
            2.8797932657906435  1.913685359779074  \\
            2.897246558310587  1.9154306890310682  \\
            2.91469985083053  1.9171760182830626  \\
            2.9321531433504737  1.9189213475350568  \\
            2.949606435870417  1.9206666767870513  \\
            2.9670597283903604  1.9224120060390455  \\
            2.9845130209103035  1.92415733529104  \\
            3.0019663134302466  1.9259026645430342  \\
            3.01941960595019  1.9276479937950286  \\
            3.036872898470133  1.9293933230470228  \\
            3.0543261909900767  1.9311386522990173  \\
            3.07177948351002  1.9328839815510115  \\
            3.0892327760299634  1.934629310803006  \\
            3.1066860685499065  1.9363746400550002  \\
            3.12413936106985  1.9381199693069946  \\
            3.141592653589793  1.9398652985589888  \\
        }
        ;
    \node[, above, color={rgb,1:red,0.0;green,0.3608;blue,0.6706}, draw opacity={1.0}, rotate={0.0}, font={{\fontsize{8 pt}{10.4 pt}\selectfont}}]  at (axis cs:0.4,0) {Nominal};
    \node[right, , color={rgb,1:red,0.7529;green,0.3255;blue,0.4039}, draw opacity={1.0}, rotate={0.0}, font={{\fontsize{8 pt}{10.4 pt}\selectfont}}]  at (axis cs:-0.7,1.9) {Stall Limited};
\end{axis}
\end{tikzpicture}

        \caption{Lift data is overwritten and extended outside the minimum and maximum lift coefficient values.}
        \label{fig:liftstallcutoff}
     \end{subfigure}
     \hfill
     \begin{subfigure}[t]{0.45\textwidth}
         \centering
         % Recommended preamble:
% \usetikzlibrary{arrows.meta}
% \usetikzlibrary{backgrounds}
% \usepgfplotslibrary{patchplots}
% \usepgfplotslibrary{fillbetween}
% \pgfplotsset{%
%     layers/standard/.define layer set={%
%         background,axis background,axis grid,axis ticks,axis lines,axis tick labels,pre main,main,axis descriptions,axis foreground%
%     }{
%         grid style={/pgfplots/on layer=axis grid},%
%         tick style={/pgfplots/on layer=axis ticks},%
%         axis line style={/pgfplots/on layer=axis lines},%
%         label style={/pgfplots/on layer=axis descriptions},%
%         legend style={/pgfplots/on layer=axis descriptions},%
%         title style={/pgfplots/on layer=axis descriptions},%
%         colorbar style={/pgfplots/on layer=axis descriptions},%
%         ticklabel style={/pgfplots/on layer=axis tick labels},%
%         axis background@ style={/pgfplots/on layer=axis background},%
%         3d box foreground style={/pgfplots/on layer=axis foreground},%
%     },
% }

\begin{tikzpicture}[/tikz/background rectangle/.style={fill={rgb,1:red,1.0;green,1.0;blue,1.0}, fill opacity={1.0}, draw opacity={1.0}}, show background rectangle]
\begin{axis}[point meta max={nan}, point meta min={nan}, legend cell align={left}, legend columns={1}, title={}, title style={at={{(0.5,1)}}, anchor={south}, font={{\fontsize{14 pt}{18.2 pt}\selectfont}}, color={rgb,1:red,0.0;green,0.0;blue,0.0}, draw opacity={1.0}, rotate={0.0}, align={center}}, legend style={color={rgb,1:red,0.0;green,0.0;blue,0.0}, draw opacity={0.0}, line width={1}, solid, fill={rgb,1:red,0.0;green,0.0;blue,0.0}, fill opacity={0.0}, text opacity={1.0}, font={{\fontsize{8 pt}{10.4 pt}\selectfont}}, text={rgb,1:red,0.0;green,0.0;blue,0.0}, cells={anchor={center}}, at={(1.02, 1)}, anchor={north west}}, axis background/.style={fill={rgb,1:red,0.0;green,0.0;blue,0.0}, opacity={0.0}}, anchor={north west}, xshift={0.0mm}, yshift={-0.0mm}, width={45.8mm}, height={50.8mm}, scaled x ticks={false}, xlabel={Angle of Attack (radians)}, x tick style={color={rgb,1:red,0.0;green,0.0;blue,0.0}, opacity={1.0}}, x tick label style={color={rgb,1:red,0.0;green,0.0;blue,0.0}, opacity={1.0}, rotate={0}}, xlabel style={at={(ticklabel cs:0.5)}, anchor=near ticklabel, at={{(ticklabel cs:0.5)}}, anchor={near ticklabel}, font={{\fontsize{11 pt}{14.3 pt}\selectfont}}, color={rgb,1:red,0.0;green,0.0;blue,0.0}, draw opacity={1.0}, rotate={0.0}}, xmajorgrids={false}, xmin={-0.7853981633974483}, xmax={0.7853981633974483}, xticklabels={{$-0.7$,$0.0$,$0.7$}}, xtick={{-0.7,0.0,0.7}}, xtick align={inside}, xticklabel style={font={{\fontsize{8 pt}{10.4 pt}\selectfont}}, color={rgb,1:red,0.0;green,0.0;blue,0.0}, draw opacity={1.0}, rotate={0.0}}, x grid style={color={rgb,1:red,0.0;green,0.0;blue,0.0}, draw opacity={0.1}, line width={0.5}, solid}, axis x line*={left}, x axis line style={color={rgb,1:red,0.0;green,0.0;blue,0.0}, draw opacity={1.0}, line width={1}, solid}, scaled y ticks={false}, ylabel={$c_d$}, y tick style={color={rgb,1:red,0.0;green,0.0;blue,0.0}, opacity={1.0}}, y tick label style={color={rgb,1:red,0.0;green,0.0;blue,0.0}, opacity={1.0}, rotate={0}}, ylabel style={{rotate=-90}}, ymajorgrids={false}, ymin={-0.004726450363115031}, ymax={0.3501599290647303}, yticklabels={{$0.0$,$0.1$,$0.2$,$0.3$}}, ytick={{0.0,0.1,0.2,0.30000000000000004}}, ytick align={inside}, yticklabel style={font={{\fontsize{8 pt}{10.4 pt}\selectfont}}, color={rgb,1:red,0.0;green,0.0;blue,0.0}, draw opacity={1.0}, rotate={0.0}}, y grid style={color={rgb,1:red,0.0;green,0.0;blue,0.0}, draw opacity={0.1}, line width={0.5}, solid}, axis y line*={left}, y axis line style={color={rgb,1:red,0.0;green,0.0;blue,0.0}, draw opacity={1.0}, line width={1}, solid}, colorbar={false}]
    \addplot[color={rgb,1:red,0.0;green,0.3608;blue,0.6706}, name path={bdd58c6d-19f0-4a35-8c87-e4223043d1e4}, draw opacity={1.0}, line width={1.0}, solid, forget plot]
        table[row sep={\\}]
        {
            \\
            -0.3490658503988659  0.22253838132616477  \\
            -0.3316125578789226  0.16796282769950013  \\
            -0.3141592653589793  0.1596778159277443  \\
            -0.2792526803190927  0.08422465361046082  \\
            -0.2617993877991494  0.025664080343649625  \\
            -0.24434609527920614  0.020321706448338677  \\
            -0.22689280275926285  0.017362599181849295  \\
            -0.20943951023931953  0.015129582907198956  \\
            -0.19198621771937624  0.01264039494333479  \\
            -0.17453292519943295  0.011265337773330403  \\
            -0.15707963267948966  0.009916094661964555  \\
            -0.13962634015954636  0.00913068830622313  \\
            -0.12217304763960307  0.008279670274612487  \\
            -0.10471975511965977  0.007660311583124659  \\
            -0.08726646259971647  0.007183773068856983  \\
            -0.06981317007977318  0.006832662122038555  \\
            -0.05235987755982988  0.006599951987190126  \\
            -0.03490658503988659  0.0064221777662872076  \\
            -0.017453292519943295  0.006329332347483174  \\
            0.0  0.00629638407416882  \\
            0.017453292519943295  0.0060679012451152325  \\
            0.03490658503988659  0.005317503771635288  \\
            0.05235987755982988  0.005633927747874414  \\
            0.06981317007977318  0.006056196387976169  \\
            0.08726646259971647  0.006576603305621797  \\
            0.10471975511965977  0.007431334942248487  \\
            0.12217304763960307  0.008755888667524615  \\
            0.13962634015954636  0.010572317026226193  \\
            0.15707963267948966  0.012306674271232844  \\
            0.17453292519943295  0.013993677931318668  \\
            0.19198621771937624  0.01591991787576938  \\
            0.20943951023931953  0.018326453917654036  \\
            0.22689280275926285  0.021350809116202588  \\
            0.24434609527920614  0.025209297180026756  \\
            0.2617993877991494  0.030641294920038895  \\
            0.2792526803190927  0.03725723886650568  \\
            0.29670597283903605  0.04699828485379302  \\
            0.3141592653589793  0.057372636106898585  \\
            0.3316125578789226  0.0703436170277454  \\
            0.3490658503988659  0.08604955516972278  \\
            0.3665191429188092  0.10569798553689437  \\
            0.3839724354387525  0.12405262936920293  \\
            0.40142572795869574  0.14354519513993408  \\
            0.41887902047863906  0.16465358911032568  \\
            0.4363323129985824  0.18725407754431073  \\
        }
        ;
    \addplot[color={rgb,1:red,0.7529;green,0.3255;blue,0.4039}, name path={88e754e4-fc36-4506-89b8-6ccd1282a179}, draw opacity={1.0}, line width={2}, dashed, forget plot]
        table[row sep={\\}]
        {
            \\
            -3.141592653589793  0.31364340692271403  \\
            -3.12413936106985  0.3118980776707197  \\
            -3.1066860685499065  0.31015274841872537  \\
            -3.0892327760299634  0.30840741916673103  \\
            -3.07177948351002  0.3066620899147367  \\
            -3.0543261909900767  0.30491676066274237  \\
            -3.036872898470133  0.30317143141074804  \\
            -3.01941960595019  0.3014261021587537  \\
            -3.0019663134302466  0.2996807729067594  \\
            -2.9845130209103035  0.29793544365476504  \\
            -2.9670597283903604  0.29619011440277077  \\
            -2.949606435870417  0.2944447851507764  \\
            -2.9321531433504737  0.2926994558987821  \\
            -2.91469985083053  0.2909541266467877  \\
            -2.897246558310587  0.28920879739479344  \\
            -2.8797932657906435  0.28746346814279905  \\
            -2.8623399732707004  0.2857181388908048  \\
            -2.844886680750757  0.2839728096388104  \\
            -2.827433388230814  0.2822274803868161  \\
            -2.8099800957108707  0.2804821511348218  \\
            -2.792526803190927  0.27873682188282745  \\
            -2.775073510670984  0.2769914926308331  \\
            -2.7576202181510405  0.2752461633788388  \\
            -2.7401669256310974  0.27350083412684445  \\
            -2.722713633111154  0.27175550487485006  \\
            -2.705260340591211  0.2700101756228558  \\
            -2.6878070480712672  0.2682648463708614  \\
            -2.670353755551324  0.2665195171188671  \\
            -2.652900463031381  0.2647741878668728  \\
            -2.6354471705114375  0.26302885861487846  \\
            -2.6179938779914944  0.2612835293628841  \\
            -2.600540585471551  0.2595382001108898  \\
            -2.5830872929516078  0.25779287085889546  \\
            -2.5656340004316642  0.25604754160690113  \\
            -2.548180707911721  0.2543022123549068  \\
            -2.5307274153917776  0.25255688310291247  \\
            -2.5132741228718345  0.25081155385091813  \\
            -2.4958208303518914  0.24906622459892383  \\
            -2.478367537831948  0.24732089534692947  \\
            -2.4609142453120048  0.24557556609493517  \\
            -2.443460952792061  0.2438302368429408  \\
            -2.426007660272118  0.24208490759094647  \\
            -2.4085543677521746  0.24033957833895214  \\
            -2.3911010752322315  0.2385942490869578  \\
            -2.373647782712288  0.23684891983496348  \\
            -2.356194490192345  0.23510359058296915  \\
            -2.3387411976724017  0.23335826133097484  \\
            -2.321287905152458  0.2316129320789805  \\
            -2.303834612632515  0.2298676028269862  \\
            -2.2863813201125716  0.22812227357499185  \\
            -2.2689280275926285  0.22637694432299754  \\
            -2.251474735072685  0.22463161507100318  \\
            -2.234021442552742  0.22288628581900888  \\
            -2.2165681500327983  0.22114095656701452  \\
            -2.199114857512855  0.21939562731502021  \\
            -2.181661564992912  0.2176502980630259  \\
            -2.1642082724729685  0.21590496881103155  \\
            -2.1467549799530254  0.21415963955903725  \\
            -2.129301687433082  0.2124143103070429  \\
            -2.111848394913139  0.21066898105504858  \\
            -2.0943951023931953  0.20892365180305422  \\
            -2.076941809873252  0.20717832255105992  \\
            -2.0594885173533086  0.20543299329906556  \\
            -2.0420352248333655  0.20368766404707125  \\
            -2.0245819323134224  0.20194233479507692  \\
            -2.007128639793479  0.20019700554308256  \\
            -1.9896753472735356  0.19845167629108823  \\
            -1.9722220547535922  0.1967063470390939  \\
            -1.9547687622336491  0.1949610177870996  \\
            -1.9373154697137058  0.19321568853510526  \\
            -1.9198621771937625  0.19147035928311093  \\
            -1.9024088846738192  0.1897250300311166  \\
            -1.8849555921538759  0.18797970077912227  \\
            -1.8675022996339325  0.18623437152712793  \\
            -1.8500490071139892  0.18448904227513363  \\
            -1.832595714594046  0.1827437130231393  \\
            -1.8151424220741026  0.18099838377114494  \\
            -1.7976891295541595  0.17925305451915066  \\
            -1.7802358370342162  0.17750772526715633  \\
            -1.7627825445142729  0.175762396015162  \\
            -1.7453292519943295  0.17401706676316767  \\
            -1.7278759594743862  0.17227173751117333  \\
            -1.710422666954443  0.17052640825917897  \\
            -1.6929693744344996  0.16878107900718464  \\
            -1.6755160819145563  0.1670357497551903  \\
            -1.658062789394613  0.16529042050319598  \\
            -1.6406094968746698  0.16354509125120167  \\
            -1.6231562043547265  0.16179976199920734  \\
            -1.6057029118347832  0.160054432747213  \\
            -1.5882496193148399  0.15830910349521868  \\
            -1.5707963267948966  0.15656377424322435  \\
            -1.5533430342749532  0.15481844499123001  \\
            -1.53588974175501  0.15307311573923568  \\
            -1.5184364492350666  0.15132778648724135  \\
            -1.5009831567151233  0.14958245723524702  \\
            -1.4835298641951802  0.14783712798325271  \\
            -1.4660765716752369  0.14609179873125838  \\
            -1.4486232791552935  0.14434646947926405  \\
            -1.4311699866353502  0.14260114022726972  \\
            -1.413716694115407  0.1408558109752754  \\
            -1.3962634015954636  0.13911048172328105  \\
            -1.3788101090755203  0.13736515247128672  \\
            -1.361356816555577  0.1356198232192924  \\
            -1.3439035240356336  0.13387449396729803  \\
            -1.3264502315156905  0.1321291647153037  \\
            -1.3089969389957472  0.1303838354633094  \\
            -1.2915436464758039  0.12863850621131506  \\
            -1.2740903539558606  0.12689317695932073  \\
            -1.2566370614359172  0.1251478477073264  \\
            -1.239183768915974  0.12340251845533208  \\
            -1.2217304763960306  0.12165718920333775  \\
            -1.2042771838760873  0.11991185995134343  \\
            -1.186823891356144  0.1181665306993491  \\
            -1.1693705988362009  0.1164212014473548  \\
            -1.1519173063162575  0.11467587219536046  \\
            -1.1344640137963142  0.11293054294336613  \\
            -1.117010721276371  0.1111852136913718  \\
            -1.0995574287564276  0.10943988443937747  \\
            -1.0821041362364843  0.10769455518738313  \\
            -1.064650843716541  0.1059492259353888  \\
            -1.0471975511965976  0.10420389668339447  \\
            -1.0297442586766543  0.10245856743140014  \\
            -1.0122909661567112  0.10071323817940582  \\
            -0.9948376736367678  0.09896790892741149  \\
            -0.9773843811168246  0.09722257967541716  \\
            -0.9599310885968813  0.09547725042342282  \\
            -0.9424777960769379  0.09373192117142849  \\
            -0.9250245035569946  0.09198659191943416  \\
            -0.9075712110370513  0.09024126266743983  \\
            -0.8901179185171081  0.08849593341544551  \\
            -0.8726646259971648  0.08675060416345118  \\
            -0.8552113334772214  0.08500527491145685  \\
            -0.8377580409572781  0.08325994565946251  \\
            -0.8203047484373349  0.08151461640746818  \\
            -0.8028514559173916  0.07976928715547385  \\
            -0.7853981633974483  0.07802395790347952  \\
            -0.767944870877505  0.07627862865148519  \\
            -0.7504915783575616  0.07453329939949085  \\
            -0.7330382858376184  0.07278797014749654  \\
            -0.7155849933176751  0.0710426408955022  \\
            -0.6981317007977318  0.06929731164350787  \\
            -0.6806784082777885  0.06755198239151354  \\
            -0.6632251157578453  0.06580665313951921  \\
            -0.6457718232379019  0.06406132388752488  \\
            -0.6283185307179586  0.062315994635530544  \\
            -0.6108652381980153  0.06057066538353621  \\
            -0.593411945678072  0.05882533613154188  \\
            -0.5759586531581288  0.05708000687954756  \\
            -0.5585053606381855  0.05533467762755323  \\
            -0.5410520681182421  0.05358934837555889  \\
            -0.5235987755982988  0.05184401912356456  \\
            -0.5061454830783556  0.05009868987157024  \\
            -0.4886921905584123  0.0483533606195759  \\
            -0.47123889803846897  0.04660803136758157  \\
            -0.45378560551852565  0.044862702115587245  \\
            -0.4363323129985824  0.043117372863592913  \\
            -0.41887902047863906  0.041372043611598575  \\
            -0.4014257279586958  0.03962671435960428  \\
            -0.3839724354387525  0.037881385107610015  \\
            -0.3665191429188092  0.03613605585561567  \\
            -0.3490658503988659  0.03439072660362114  \\
            -0.33161255787892263  0.03264539735162631  \\
            -0.3141592653589793  0.030900068099631053  \\
            -0.296705972839036  0.02915473884763525  \\
            -0.2792526803190927  0.027409409595638798  \\
            -0.2617993877991494  0.025664080343641565  \\
            -0.24434609527920614  0.021381756049354342  \\
            -0.22689280275926282  0.018077666799427967  \\
            -0.20943951023931953  0.015489824897380025  \\
            -0.19198621771937624  0.012819123344961037  \\
            -0.17453292519943295  0.011336669082074114  \\
            -0.15707963267948966  0.009944022674818361  \\
            -0.13962634015954636  0.009140264030071683  \\
            -0.12217304763960307  0.008282846425113606  \\
            -0.10471975511965977  0.007661202482599106  \\
            -0.08726646259971647  0.0071839396486049955  \\
            -0.06981317007977318  0.0068326372921262086  \\
            -0.05235987755982988  0.006599898909126306  \\
            -0.03490658503988659  0.006422137396965193  \\
            -0.017453292519943295  0.006329308259119943  \\
            2.524354896707238e-29  0.006296372963981781  \\
            0.017453292519943295  0.006067901858370478  \\
            0.03490658503988659  0.0053175212373090695  \\
            0.05235987755982988  0.0056339788324660045  \\
            0.06981317007977318  0.006056328441783342  \\
            0.08726646259971647  0.006576935465426823  \\
            0.10471975511965977  0.00743215589418155  \\
            0.12217304763960307  0.008757882259679462  \\
            0.13962634015954636  0.01057707645643728  \\
            0.15707963267948966  0.012318066694881766  \\
            0.17453292519943295  0.014020982635329127  \\
            0.19198621771937624  0.01598477991285393  \\
            0.20943951023931953  0.018477708515915715  \\
            0.22689280275926282  0.021694068619070514  \\
            0.24434609527920614  0.025954161152826474  \\
            0.2617993877991494  0.03210274905833797  \\
            0.2792526803190927  0.039728294893018025  \\
            0.296705972839036  0.04954125894403976  \\
            0.3141592653589793  0.057372636106884686  \\
            0.33161255787892263  0.059117965358884715  \\
            0.3490658503988659  0.06086329461088306  \\
            0.3665191429188092  0.06260862386287998  \\
            0.3839724354387525  0.06435395311487578  \\
            0.4014257279586958  0.06609928236687074  \\
            0.41887902047863906  0.06784461161886511  \\
            0.4363323129985824  0.06958994087085918  \\
            0.45378560551852565  0.07133527012285326  \\
            0.47123889803846897  0.07308059937484754  \\
            0.4886921905584123  0.07482592862684187  \\
            0.5061454830783556  0.0765712578788362  \\
            0.5235987755982988  0.07831658713083053  \\
            0.5410520681182421  0.08006191638282487  \\
            0.5585053606381855  0.0818072456348192  \\
            0.5759586531581288  0.08355257488681353  \\
            0.593411945678072  0.08529790413880785  \\
            0.6108652381980153  0.08704323339080218  \\
            0.6283185307179586  0.08878856264279651  \\
            0.6457718232379019  0.09053389189479084  \\
            0.6632251157578453  0.09227922114678518  \\
            0.6806784082777885  0.09402455039877951  \\
            0.6981317007977318  0.09576987965077384  \\
            0.7155849933176751  0.09751520890276817  \\
            0.7330382858376184  0.09926053815476252  \\
            0.7504915783575616  0.10100586740675684  \\
            0.767944870877505  0.10275119665875115  \\
            0.7853981633974483  0.10449652591074549  \\
            0.8028514559173916  0.10624185516273983  \\
            0.8203047484373349  0.10798718441473416  \\
            0.8377580409572781  0.10973251366672848  \\
            0.8552113334772214  0.11147784291872281  \\
            0.8726646259971648  0.11322317217071715  \\
            0.8901179185171081  0.11496850142271148  \\
            0.9075712110370513  0.11671383067470581  \\
            0.9250245035569946  0.11845915992670013  \\
            0.9424777960769379  0.12020448917869446  \\
            0.9599310885968813  0.12194981843068879  \\
            0.9773843811168246  0.12369514768268312  \\
            0.9948376736367678  0.12544047693467744  \\
            1.0122909661567112  0.12718580618667177  \\
            1.0297442586766543  0.12893113543866605  \\
            1.0471975511965976  0.1306764646906604  \\
            1.064650843716541  0.13242179394265474  \\
            1.0821041362364843  0.13416712319464907  \\
            1.0995574287564276  0.1359124524466434  \\
            1.117010721276371  0.13765778169863774  \\
            1.1344640137963142  0.13940311095063207  \\
            1.1519173063162575  0.1411484402026264  \\
            1.1693705988362009  0.14289376945462073  \\
            1.186823891356144  0.14463909870661507  \\
            1.2042771838760873  0.1463844279586094  \\
            1.2217304763960306  0.1481297572106037  \\
            1.239183768915974  0.14987508646259803  \\
            1.2566370614359172  0.15162041571459237  \\
            1.2740903539558606  0.1533657449665867  \\
            1.2915436464758039  0.15511107421858103  \\
            1.3089969389957472  0.1568564034705754  \\
            1.3264502315156905  0.15860173272256972  \\
            1.3439035240356336  0.16034706197456403  \\
            1.361356816555577  0.16209239122655836  \\
            1.3788101090755203  0.1638377204785527  \\
            1.3962634015954636  0.16558304973054702  \\
            1.413716694115407  0.16732837898254135  \\
            1.4311699866353502  0.16907370823453569  \\
            1.4486232791552935  0.17081903748653002  \\
            1.4660765716752369  0.17256436673852435  \\
            1.4835298641951802  0.17430969599051868  \\
            1.5009831567151233  0.17605502524251299  \\
            1.5184364492350666  0.17780035449450732  \\
            1.53588974175501  0.17954568374650165  \\
            1.5533430342749532  0.18129101299849598  \\
            1.5707963267948966  0.1830363422504903  \\
            1.5882496193148399  0.18478167150248465  \\
            1.6057029118347832  0.18652700075447898  \\
            1.6231562043547265  0.1882723300064733  \\
            1.6406094968746698  0.19001765925846764  \\
            1.658062789394613  0.19176298851046195  \\
            1.6755160819145563  0.19350831776245628  \\
            1.6929693744344996  0.1952536470144506  \\
            1.710422666954443  0.19699897626644494  \\
            1.7278759594743862  0.19874430551843927  \\
            1.7453292519943295  0.2004896347704336  \\
            1.7627825445142729  0.20223496402242794  \\
            1.7802358370342162  0.20398029327442227  \\
            1.7976891295541595  0.2057256225264166  \\
            1.8151424220741026  0.2074709517784109  \\
            1.832595714594046  0.20921628103040524  \\
            1.8500490071139892  0.21096161028239957  \\
            1.8675022996339325  0.2127069395343939  \\
            1.8849555921538759  0.21445226878638823  \\
            1.9024088846738192  0.21619759803838257  \\
            1.9198621771937625  0.2179429272903769  \\
            1.9373154697137058  0.21968825654237123  \\
            1.9547687622336491  0.22143358579436556  \\
            1.9722220547535922  0.22317891504635987  \\
            1.9896753472735356  0.2249242442983542  \\
            2.007128639793479  0.22666957355034853  \\
            2.0245819323134224  0.2284149028023429  \\
            2.0420352248333655  0.2301602320543372  \\
            2.0594885173533086  0.23190556130633153  \\
            2.076941809873252  0.23365089055832586  \\
            2.0943951023931953  0.2353962198103202  \\
            2.111848394913139  0.23714154906231455  \\
            2.129301687433082  0.23888687831430885  \\
            2.1467549799530254  0.2406322075663032  \\
            2.1642082724729685  0.24237753681829752  \\
            2.181661564992912  0.24412286607029188  \\
            2.199114857512855  0.24586819532228618  \\
            2.2165681500327983  0.24761352457428049  \\
            2.234021442552742  0.24935885382627485  \\
            2.251474735072685  0.25110418307826915  \\
            2.2689280275926285  0.2528495123302635  \\
            2.2863813201125716  0.2545948415822578  \\
            2.303834612632515  0.25634017083425215  \\
            2.321287905152458  0.2580855000862465  \\
            2.3387411976724017  0.2598308293382408  \\
            2.356194490192345  0.26157615859023514  \\
            2.373647782712288  0.2633214878422295  \\
            2.3911010752322315  0.2650668170942238  \\
            2.4085543677521746  0.26681214634621814  \\
            2.426007660272118  0.26855747559821247  \\
            2.443460952792061  0.2703028048502068  \\
            2.4609142453120048  0.27204813410220113  \\
            2.478367537831948  0.27379346335419547  \\
            2.4958208303518914  0.2755387926061898  \\
            2.5132741228718345  0.27728412185818413  \\
            2.5307274153917776  0.2790294511101784  \\
            2.548180707911721  0.2807747803621728  \\
            2.5656340004316642  0.28252010961416707  \\
            2.5830872929516078  0.28426543886616146  \\
            2.600540585471551  0.28601076811815573  \\
            2.6179938779914944  0.2877560973701501  \\
            2.6354471705114375  0.2895014266221444  \\
            2.652900463031381  0.2912467558741388  \\
            2.670353755551324  0.29299208512613306  \\
            2.6878070480712672  0.2947374143781274  \\
            2.705260340591211  0.2964827436301217  \\
            2.722713633111154  0.29822807288211606  \\
            2.7401669256310974  0.2999734021341104  \\
            2.7576202181510405  0.3017187313861047  \\
            2.775073510670984  0.30346406063809905  \\
            2.792526803190927  0.3052093898900934  \\
            2.8099800957108707  0.3069547191420877  \\
            2.827433388230814  0.30870004839408205  \\
            2.844886680750757  0.3104453776460763  \\
            2.8623399732707004  0.3121907068980707  \\
            2.8797932657906435  0.313936036150065  \\
            2.897246558310587  0.3156813654020594  \\
            2.91469985083053  0.31742669465405365  \\
            2.9321531433504737  0.31917202390604804  \\
            2.949606435870417  0.3209173531580423  \\
            2.9670597283903604  0.3226626824100367  \\
            2.9845130209103035  0.324408011662031  \\
            3.0019663134302466  0.3261533409140253  \\
            3.01941960595019  0.32789867016601965  \\
            3.036872898470133  0.329643999418014  \\
            3.0543261909900767  0.3313893286700083  \\
            3.07177948351002  0.33313465792200264  \\
            3.0892327760299634  0.334879987173997  \\
            3.1066860685499065  0.3366253164259913  \\
            3.12413936106985  0.33837064567798564  \\
            3.141592653589793  0.34011597492997997  \\
        }
        ;
    \node[right, , color={rgb,1:red,0.0;green,0.3608;blue,0.6706}, draw opacity={1.0}, rotate={0.0}, font={{\fontsize{8 pt}{10.4 pt}\selectfont}}]  at (axis cs:-0.7,0.3) {Nominal};
    \node[right, , color={rgb,1:red,0.7529;green,0.3255;blue,0.4039}, draw opacity={1.0}, rotate={0.0}, font={{\fontsize{8 pt}{10.4 pt}\selectfont}}]  at (axis cs:-0.7,0.265) {Stall Limited};
\end{axis}
\end{tikzpicture}

         \caption{We cutoff Drag data at the same angles of attack as the lift data.}
        \label{fig:dragstallcutoff}
     \end{subfigure}
        \caption{We cut off airfoil data outside the range of minimum and maximum lift coefficient and replace/extend the data using a prescribed lift curve slope in order to avoid numerical difficulties associated with multiple angles of attack resulting in equal lift coefficients.}
     \label{fig:stall-cutoff}
\end{figure}


%---------------------------------#
%       Cascade Corrections       #
%---------------------------------#

\subsection{Solidity and Stagger Corrections}

Isolated airfoil data needs to be corrected to account for cascade, or multi-plane interference, effects since the airfoils along a rotor blade section are not actually isolated.
%
This is especially true for higher solidities, where blades are relatively close together.
%
We apply corrections based on a simple model published by Wallis\scite{Wallis_1968,Wallis_1977,Wallis_1983}, which assumes smooth transition between isolated airfoil and cascade data as solidity increases, as well as circular camber line airfoil geometries.
%
\begin{marginfigure}
	% Recommended preamble:
% \usetikzlibrary{arrows.meta}
% \usetikzlibrary{backgrounds}
% \usepgfplotslibrary{patchplots}
% \usepgfplotslibrary{fillbetween}
% \pgfplotsset{%
%     layers/standard/.define layer set={%
%         background,axis background,axis grid,axis ticks,axis lines,axis tick labels,pre main,main,axis descriptions,axis foreground%
%     }{
%         grid style={/pgfplots/on layer=axis grid},%
%         tick style={/pgfplots/on layer=axis ticks},%
%         axis line style={/pgfplots/on layer=axis lines},%
%         label style={/pgfplots/on layer=axis descriptions},%
%         legend style={/pgfplots/on layer=axis descriptions},%
%         title style={/pgfplots/on layer=axis descriptions},%
%         colorbar style={/pgfplots/on layer=axis descriptions},%
%         ticklabel style={/pgfplots/on layer=axis tick labels},%
%         axis background@ style={/pgfplots/on layer=axis background},%
%         3d box foreground style={/pgfplots/on layer=axis foreground},%
%     },
% }

\begin{tikzpicture}[/tikz/background rectangle/.style={fill={rgb,1:red,1.0;green,1.0;blue,1.0}, fill opacity={1.0}, draw opacity={1.0}}, show background rectangle]
\begin{axis}[point meta max={nan}, point meta min={nan}, legend cell align={left}, legend columns={1}, title={}, title style={at={{(0.5,1)}}, anchor={south}, font={{\fontsize{14 pt}{18.2 pt}\selectfont}}, color={rgb,1:red,0.0;green,0.0;blue,0.0}, draw opacity={1.0}, rotate={0.0}, align={center}}, legend style={color={rgb,1:red,0.0;green,0.0;blue,0.0}, draw opacity={0.0}, line width={1}, solid, fill={rgb,1:red,0.0;green,0.0;blue,0.0}, fill opacity={0.0}, text opacity={1.0}, font={{\fontsize{8 pt}{10.4 pt}\selectfont}}, text={rgb,1:red,0.0;green,0.0;blue,0.0}, cells={anchor={center}}, at={(1.02, 1)}, anchor={north west}}, axis background/.style={fill={rgb,1:red,0.0;green,0.0;blue,0.0}, opacity={0.0}}, anchor={north west}, xshift={0.0mm}, yshift={-0.0mm}, width={45.8mm}, height={50.8mm}, scaled x ticks={false}, xlabel={}, x tick style={color={rgb,1:red,0.0;green,0.0;blue,0.0}, opacity={1.0}}, x tick label style={color={rgb,1:red,0.0;green,0.0;blue,0.0}, opacity={1.0}, rotate={0}}, xlabel style={at={(ticklabel cs:0.5)}, anchor=near ticklabel, at={{(ticklabel cs:0.5)}}, anchor={near ticklabel}, font={{\fontsize{11 pt}{14.3 pt}\selectfont}}, color={rgb,1:red,0.0;green,0.0;blue,0.0}, draw opacity={1.0}, rotate={0.0}}, xmajorticks={false}, xmajorgrids={false}, xmin={-18.080000000000002}, xmax={20.080000000000002}, axis x line*={left}, separate axis lines, x axis line style={{draw opacity = 0}}, scaled y ticks={false}, ylabel={}, y tick style={color={rgb,1:red,0.0;green,0.0;blue,0.0}, opacity={1.0}}, y tick label style={color={rgb,1:red,0.0;green,0.0;blue,0.0}, opacity={1.0}, rotate={0}}, ylabel style={at={(ticklabel cs:0.5)}, anchor=near ticklabel, at={{(ticklabel cs:0.5)}}, anchor={near ticklabel}, font={{\fontsize{11 pt}{14.3 pt}\selectfont}}, color={rgb,1:red,0.0;green,0.0;blue,0.0}, draw opacity={1.0}, rotate={0.0}}, ymajorticks={false}, ymajorgrids={false}, ymin={-1.4228193239977105}, ymax={1.6767320105016155}, axis y line*={left}, y axis line style={{draw opacity = 0}}, colorbar={false}]
    \addplot[color={rgb,1:red,0.0;green,0.0;blue,0.0}, name path={c37b5ee6-e462-4939-b378-50da5a1dcce9}, draw opacity={1.0}, line width={0.25}, solid, forget plot]
        table[row sep={\\}]
        {
            \\
            -17.0  0.0  \\
            19.0  0.0  \\
        }
        ;
    \addplot[color={rgb,1:red,0.0;green,0.0;blue,0.0}, name path={22e71cf3-ce75-43a9-8fa8-06348b08d8f4}, draw opacity={1.0}, line width={0.25}, solid, forget plot]
        table[row sep={\\}]
        {
            \\
            0.0  -1.3350961730213144  \\
            0.0  1.5890088595252194  \\
        }
        ;
    \addplot[color={rgb,1:red,0.0;green,0.3608;blue,0.6706}, name path={06fd42f4-9da8-4a8a-9dcb-705cdc8b1f07}, draw opacity={1.0}, line width={1.0}, solid, forget plot]
        table[row sep={\\}]
        {
            \\
            -17.0  -1.3263761926191644  \\
            -16.9  -1.3284019384451278  \\
            -16.8  -1.3301543349855023  \\
            -16.7  -1.331636378557983  \\
            -16.6  -1.3328510654802654  \\
            -16.5  -1.3338013920700442  \\
            -16.4  -1.3344903546450155  \\
            -16.3  -1.3349209495228738  \\
            -16.2  -1.3350961730213144  \\
            -16.1  -1.335019021458033  \\
            -16.0  -1.334692491150725  \\
            -15.9  -1.3341195784170845  \\
            -15.8  -1.3333032795748079  \\
            -15.7  -1.3322465909415897  \\
            -15.6  -1.3309525088351257  \\
            -15.5  -1.3294240295731108  \\
            -15.4  -1.3276641494732404  \\
            -15.3  -1.3256758648532094  \\
            -15.2  -1.3234621720307134  \\
            -15.1  -1.3210260673234475  \\
            -15.0  -1.3183705470491074  \\
            -14.9  -1.3155885053659153  \\
            -14.8  -1.3127454998545929  \\
            -14.7  -1.3098035219665838  \\
            -14.6  -1.3067245631533313  \\
            -14.5  -1.3034706148662791  \\
            -14.4  -1.3000036685568699  \\
            -14.3  -1.296285715676548  \\
            -14.2  -1.292278747676756  \\
            -14.1  -1.287944756008938  \\
            -14.0  -1.2832457321245372  \\
            -13.9  -1.2781436674749966  \\
            -13.8  -1.2726005535117604  \\
            -13.7  -1.2665783816862708  \\
            -13.6  -1.2600391434499725  \\
            -13.5  -1.252944830254308  \\
            -13.4  -1.2452574335507216  \\
            -13.3  -1.236938944790656  \\
            -13.2  -1.2279513554255543  \\
            -13.1  -1.218256656906861  \\
            -13.0  -1.2078168406860186  \\
            -12.9  -1.1969545268352275  \\
            -12.8  -1.186018188207181  \\
            -12.7  -1.1750085954240634  \\
            -12.6  -1.1639265191080592  \\
            -12.5  -1.1527727298813533  \\
            -12.4  -1.1415479983661292  \\
            -12.3  -1.1302530951845715  \\
            -12.2  -1.1188887909588647  \\
            -12.1  -1.1074558563111927  \\
            -12.0  -1.0959550618637404  \\
            -11.9  -1.0843871782386918  \\
            -11.8  -1.072752976058231  \\
            -11.7  -1.0610532259445424  \\
            -11.6  -1.0492886985198109  \\
            -11.5  -1.0374601644062207  \\
            -11.4  -1.0255683942259552  \\
            -11.3  -1.0136141586011995  \\
            -11.2  -1.001598228154138  \\
            -11.1  -0.9895213735069548  \\
            -11.0  -0.9773843652818341  \\
            -10.9  -0.9652104621446128  \\
            -10.8  -0.9530235929458453  \\
            -10.7  -0.9408255335847938  \\
            -10.6  -0.9286180599607206  \\
            -10.5  -0.9164029479728876  \\
            -10.4  -0.9041819735205565  \\
            -10.3  -0.8919569125029898  \\
            -10.2  -0.8797295408194489  \\
            -10.1  -0.8675016343691964  \\
            -10.0  -0.855274969051494  \\
            -9.9  -0.8430513207656039  \\
            -9.8  -0.8308324654107879  \\
            -9.7  -0.818620178886308  \\
            -9.6  -0.8064162370914264  \\
            -9.5  -0.7942224159254052  \\
            -9.4  -0.7820404912875063  \\
            -9.3  -0.7698722390769915  \\
            -9.2  -0.757719435193123  \\
            -9.1  -0.7455838555351627  \\
            -9.0  -0.7334672760023729  \\
            -8.9  -0.7213476671605121  \\
            -8.8  -0.7092064757887554  \\
            -8.7  -0.6970506921064885  \\
            -8.6  -0.6848873063330984  \\
            -8.5  -0.6727233086879707  \\
            -8.4  -0.660565689390492  \\
            -8.3  -0.6484214386600484  \\
            -8.2  -0.6362975467160258  \\
            -8.1  -0.624201003777811  \\
            -8.0  -0.6121388000647897  \\
            -7.9  -0.6001179257963486  \\
            -7.8  -0.5881453711918734  \\
            -7.7  -0.5762281264707507  \\
            -7.6  -0.5643731818523667  \\
            -7.5  -0.5525875275561075  \\
            -7.4  -0.5408781538013594  \\
            -7.3  -0.5292520508075083  \\
            -7.2  -0.517716208793941  \\
            -7.1  -0.5062776179800431  \\
            -7.0  -0.49494326858520143  \\
            -6.9  -0.48371911033487036  \\
            -6.8  -0.4726022124991179  \\
            -6.7  -0.4615862446142507  \\
            -6.6  -0.45066487621657486  \\
            -6.5  -0.43983177684239705  \\
            -6.4  -0.4290806160280234  \\
            -6.3  -0.41840506330976024  \\
            -6.2  -0.4077987882239142  \\
            -6.1  -0.3972554603067914  \\
            -6.0  -0.3867687490946983  \\
            -5.9  -0.3763323241239414  \\
            -5.8  -0.36593985493082676  \\
            -5.7  -0.355585011051661  \\
            -5.6  -0.3452614620227504  \\
            -5.5  -0.3349628773804013  \\
            -5.4  -0.3246829266609202  \\
            -5.3  -0.31441527940061326  \\
            -5.2  -0.304153605135787  \\
            -5.1  -0.29389157340274763  \\
            -5.0  -0.28362285373780177  \\
            -4.9  -0.27334848544104173  \\
            -4.8  -0.2630738322468798  \\
            -4.7  -0.2527990503431024  \\
            -4.6  -0.2425242959174956  \\
            -4.5  -0.23224972515784578  \\
            -4.4  -0.2219754942519391  \\
            -4.3  -0.2117017593875617  \\
            -4.2  -0.20142867675249995  \\
            -4.1  -0.19115640253453994  \\
            -4.0  -0.1808850929214681  \\
            -3.9  -0.1706149041010705  \\
            -3.8  -0.16034599226113339  \\
            -3.7  -0.15007851358944305  \\
            -3.6  -0.13981262427378563  \\
            -3.5  -0.12954848050194748  \\
            -3.4  -0.11928623846171468  \\
            -3.3  -0.10902605434087356  \\
            -3.2  -0.09876808432721038  \\
            -3.1  -0.08851248460851127  \\
            -3.0  -0.07825941137256248  \\
            -2.9  -0.06800815075249923  \\
            -2.8  -0.05775790171309017  \\
            -2.7  -0.04750868968957163  \\
            -2.6  -0.03726054011717993  \\
            -2.5  -0.02701347843115147  \\
            -2.4  -0.016767530066722613  \\
            -2.3  -0.006522720459129719  \\
            -2.2  0.003720924956390808  \\
            -2.1  0.013963380744602669  \\
            -2.0  0.024204621470269486  \\
            -1.9  0.034444621698154876  \\
            -1.8  0.044683355993022464  \\
            -1.7  0.054920798919635916  \\
            -1.6  0.06515692504275884  \\
            -1.5  0.07539170892715487  \\
            -1.4  0.0856251251375877  \\
            -1.3  0.09585714823882088  \\
            -1.2  0.10608775279561815  \\
            -1.1  0.11631691337274302  \\
            -1.0  0.12654460453495925  \\
            -0.9  0.13675114698078158  \\
            -0.8  0.1469191062590442  \\
            -0.7  0.15705182420998975  \\
            -0.6  0.1671526426738609  \\
            -0.5  0.17722490349090023  \\
            -0.4  0.18727194850135045  \\
            -0.3  0.1972971195454542  \\
            -0.2  0.20730375846345414  \\
            -0.1  0.21729520709559283  \\
            0.0  0.22727480728211297  \\
            0.1  0.2372459008632572  \\
            0.2  0.24721182967926816  \\
            0.3  0.25717593557038854  \\
            0.4  0.26714156037686093  \\
            0.5  0.27711204593892796  \\
            0.6  0.28709073409683233  \\
            0.7  0.2970809666908167  \\
            0.8  0.3070860855611236  \\
            0.9  0.31710943254799584  \\
            1.0  0.32715434949167593  \\
            1.1  0.3372311618514844  \\
            1.2  0.34734784498727106  \\
            1.3  0.357504215590073  \\
            1.4  0.36770009035092716  \\
            1.5  0.37793528596087045  \\
            1.6  0.38820961911093976  \\
            1.7  0.39852290649217215  \\
            1.8  0.40887496479560453  \\
            1.9  0.4192656107122737  \\
            2.0  0.4296946609332169  \\
            2.1  0.4401619321494708  \\
            2.2  0.45066724105207245  \\
            2.3  0.46121040433205873  \\
            2.4  0.4717912386804668  \\
            2.5  0.48240956078833336  \\
            2.6  0.4930651873466955  \\
            2.7  0.50375793504659  \\
            2.8  0.5144876205790542  \\
            2.9  0.5252540606351245  \\
            3.0  0.5360570719058383  \\
            3.1  0.5472176917030886  \\
            3.2  0.5590142298647537  \\
            3.3  0.5713824118708475  \\
            3.4  0.5842579632013845  \\
            3.5  0.5975766093363787  \\
            3.6  0.6112740757558444  \\
            3.7  0.6252860879397959  \\
            3.8  0.6395483713682468  \\
            3.9  0.653996651521212  \\
            4.0  0.6685666538787055  \\
            4.1  0.6831941039207411  \\
            4.2  0.6978147271273336  \\
            4.3  0.7123642489784966  \\
            4.4  0.7267783949542447  \\
            4.5  0.7409928905345917  \\
            4.6  0.7549434611995522  \\
            4.7  0.7685658324291402  \\
            4.8  0.7817957297033699  \\
            4.9  0.7945688785022555  \\
            5.0  0.806821004305811  \\
            5.1  0.818768964189323  \\
            5.2  0.8306621191887592  \\
            5.3  0.8424849507251556  \\
            5.4  0.8542219402195486  \\
            5.5  0.8658575690929744  \\
            5.6  0.8773763187664694  \\
            5.7  0.8887626706610697  \\
            5.8  0.9000011061978116  \\
            5.9  0.9110761067977312  \\
            6.0  0.9219721538818648  \\
            6.1  0.9326737288712489  \\
            6.2  0.9431653131869197  \\
            6.3  0.9534313882499131  \\
            6.4  0.9634564354812657  \\
            6.5  0.9732249363020133  \\
            6.6  0.9827213721331928  \\
            6.7  0.9919302243958399  \\
            6.8  1.0008359745109914  \\
            6.9  1.0094231038996828  \\
            7.0  1.017676093982951  \\
            7.1  1.0257363947197207  \\
            7.2  1.0337563825947784  \\
            7.3  1.041736928817951  \\
            7.4  1.0496789045990664  \\
            7.5  1.057583181147952  \\
            7.6  1.0654506296744348  \\
            7.7  1.0732821213883428  \\
            7.8  1.081078527499503  \\
            7.9  1.0888407192177432  \\
            8.0  1.0965695677528906  \\
            8.1  1.104265944314773  \\
            8.2  1.1119307201132174  \\
            8.3  1.1195647663580517  \\
            8.4  1.127168954259103  \\
            8.5  1.1347441550261987  \\
            8.6  1.1422912398691667  \\
            8.7  1.1498110799978338  \\
            8.8  1.1573045466220284  \\
            8.9  1.1647725109515767  \\
            9.0  1.1722158441963073  \\
            9.1  1.179650461534768  \\
            9.2  1.1870895824612524  \\
            9.3  1.1945300346592058  \\
            9.4  1.2019686458120726  \\
            9.5  1.2094022436032983  \\
            9.6  1.216827655716328  \\
            9.7  1.224241709834606  \\
            9.8  1.2316412336415778  \\
            9.9  1.2390230548206882  \\
            10.0  1.2463840010553824  \\
            10.1  1.2537209000291054  \\
            10.2  1.2610305794253018  \\
            10.3  1.2683098669274173  \\
            10.4  1.2755555902188962  \\
            10.5  1.2827645769831837  \\
            10.6  1.2899336549037248  \\
            10.7  1.2970596516639645  \\
            10.8  1.304139394947348  \\
            10.9  1.3111697124373198  \\
            11.0  1.3181474318173256  \\
            11.1  1.325026174277794  \\
            11.2  1.3317668927068351  \\
            11.3  1.338377412334413  \\
            11.4  1.3448655583904925  \\
            11.5  1.351239156105038  \\
            11.6  1.3575060307080147  \\
            11.7  1.3636740074293865  \\
            11.8  1.3697509114991193  \\
            11.9  1.3757445681471765  \\
            12.0  1.3816628026035231  \\
            12.1  1.3875134400981242  \\
            12.2  1.3933043058609442  \\
            12.3  1.399043225121948  \\
            12.4  1.4047380231110997  \\
            12.5  1.4103965250583645  \\
            12.6  1.4160265561937067  \\
            12.7  1.4216359417470912  \\
            12.8  1.4272325069484826  \\
            12.9  1.4328240770278458  \\
            13.0  1.438418477215145  \\
            13.1  1.44397194969458  \\
            13.2  1.449437093844378  \\
            13.3  1.454816270685545  \\
            13.4  1.4601118412390868  \\
            13.5  1.4653261665260098  \\
            13.6  1.4704616075673198  \\
            13.7  1.475520525384023  \\
            13.8  1.4805052809971249  \\
            13.9  1.485418235427632  \\
            14.0  1.49026174969655  \\
            14.1  1.4950381848248855  \\
            14.2  1.499749901833644  \\
            14.3  1.5043992617438318  \\
            14.4  1.5089886255764546  \\
            14.5  1.5135203543525189  \\
            14.6  1.51799680909303  \\
            14.7  1.5224203508189946  \\
            14.8  1.5267933405514187  \\
            14.9  1.5311181393113076  \\
            15.0  1.5353971081196682  \\
            15.1  1.539551204323609  \\
            15.2  1.5435032907074437  \\
            15.3  1.5472585864479875  \\
            15.4  1.5508223107220547  \\
            15.5  1.554199682706459  \\
            15.6  1.557395921578015  \\
            15.7  1.5604162465135376  \\
            15.8  1.5632658766898406  \\
            15.9  1.5659500312837387  \\
            16.0  1.5684739294720458  \\
            16.1  1.5708427904315765  \\
            16.2  1.573061833339145  \\
            16.3  1.5751362773715663  \\
            16.4  1.5770713417056537  \\
            16.5  1.5788722455182223  \\
            16.6  1.580544207986086  \\
            16.7  1.5820924482860599  \\
            16.8  1.5835221855949575  \\
            16.9  1.5848386390895934  \\
            17.0  1.5860470279467822  \\
            17.1  1.5870871836027531  \\
            17.2  1.587898652660948  \\
            17.3  1.5884862270490012  \\
            17.4  1.588854698694547  \\
            17.5  1.5890088595252194  \\
            17.6  1.5889535014686522  \\
            17.7  1.5886934164524797  \\
            17.8  1.5882333964043356  \\
            17.9  1.5875782332518544  \\
            18.0  1.58673271892267  \\
            18.1  1.5857016453444164  \\
            18.2  1.5844898044447273  \\
            18.3  1.5831019881512376  \\
            18.4  1.5815429883915808  \\
            18.5  1.5798175970933912  \\
            18.6  1.5779306061843026  \\
            18.7  1.575886807591949  \\
            18.8  1.5736909932439649  \\
            18.9  1.571347955067984  \\
            19.0  1.5688624849916406  \\
        }
        ;
    \addplot[color={rgb,1:red,0.7529;green,0.3255;blue,0.4039}, name path={d8b7a671-3991-4e01-a52f-ed49d8c39773}, draw opacity={1.0}, line width={1.0}, solid, forget plot]
        table[row sep={\\}]
        {
            \\
            -17.0  -1.0644508048104238  \\
            -16.9  -1.0660765176261295  \\
            -16.8  -1.067482860651682  \\
            -16.7  -1.0686722385086351  \\
            -16.6  -1.069647055818544  \\
            -16.5  -1.0704097172029627  \\
            -16.4  -1.0709626272834463  \\
            -16.3  -1.0713081906815485  \\
            -16.2  -1.0714488120188244  \\
            -16.1  -1.0713868959168282  \\
            -16.0  -1.071124846997115  \\
            -15.9  -1.0706650698812386  \\
            -15.8  -1.070009969190754  \\
            -15.7  -1.0691619495472153  \\
            -15.6  -1.0681234155721775  \\
            -15.5  -1.0668967718871947  \\
            -15.4  -1.0654844231138216  \\
            -15.3  -1.0638887738736125  \\
            -15.2  -1.0621122287881222  \\
            -15.1  -1.060157192478905  \\
            -15.0  -1.0580260695675159  \\
            -14.9  -1.0557934099908672  \\
            -14.8  -1.0535118253835383  \\
            -14.7  -1.051150812913583  \\
            -14.6  -1.0486798697490551  \\
            -14.5  -1.0460684930580089  \\
            -14.4  -1.0432861800084978  \\
            -14.3  -1.0403024277685764  \\
            -14.2  -1.037086733506298  \\
            -14.1  -1.033608594389717  \\
            -14.0  -1.029837507586887  \\
            -13.9  -1.025742970265862  \\
            -13.8  -1.0212944795946963  \\
            -13.7  -1.0164615327414428  \\
            -13.6  -1.0112136268741567  \\
            -13.5  -1.005520259160891  \\
            -13.4  -0.9993509267697003  \\
            -13.3  -0.992675126868638  \\
            -13.2  -0.985462356625758  \\
            -13.1  -0.9776821132091147  \\
            -13.0  -0.9693038937867616  \\
            -12.9  -0.9605866092147686  \\
            -12.8  -0.9518099178665049  \\
            -12.7  -0.9429744381859756  \\
            -12.6  -0.9340807886171852  \\
            -12.5  -0.9251295876041391  \\
            -12.4  -0.9161214535908413  \\
            -12.3  -0.9070570050212966  \\
            -12.2  -0.8979368603395099  \\
            -12.1  -0.8887616379894857  \\
            -12.0  -0.879531956415229  \\
            -11.9  -0.8702484340607443  \\
            -11.8  -0.8609116893700361  \\
            -11.7  -0.8515223407871093  \\
            -11.6  -0.8420810067559689  \\
            -11.5  -0.8325883057206195  \\
            -11.4  -0.8230448561250653  \\
            -11.3  -0.8134512764133113  \\
            -11.2  -0.8038081850293625  \\
            -11.1  -0.7941162004172233  \\
            -11.0  -0.7843759410208985  \\
            -10.9  -0.7746060725143548  \\
            -10.8  -0.7648257984119288  \\
            -10.7  -0.7550365439181815  \\
            -10.6  -0.7452397342376735  \\
            -10.5  -0.7354367945749654  \\
            -10.4  -0.7256291501346173  \\
            -10.3  -0.7158182261211902  \\
            -10.2  -0.706005447739244  \\
            -10.1  -0.6961922401933399  \\
            -10.0  -0.6863800286880383  \\
            -9.9  -0.6765702384278997  \\
            -9.8  -0.6667642946174844  \\
            -9.7  -0.6569636224613531  \\
            -9.6  -0.6471696471640663  \\
            -9.5  -0.6373837939301848  \\
            -9.4  -0.6276074879642691  \\
            -9.3  -0.6178421544708794  \\
            -9.2  -0.6080892186545763  \\
            -9.1  -0.5983501057199205  \\
            -9.0  -0.5886262408714726  \\
            -8.9  -0.5788999449250478  \\
            -8.8  -0.5691563284466008  \\
            -8.7  -0.5594010012659013  \\
            -8.6  -0.5496395732127198  \\
            -8.5  -0.5398776541168258  \\
            -8.4  -0.5301208538079893  \\
            -8.3  -0.5203747821159804  \\
            -8.2  -0.5106450488705686  \\
            -8.1  -0.5009372639015242  \\
            -8.0  -0.4912570370386169  \\
            -7.9  -0.4816099781116168  \\
            -7.8  -0.4720016969502936  \\
            -7.7  -0.46243780338441737  \\
            -7.6  -0.4529239072437581  \\
            -7.5  -0.4434656183580855  \\
            -7.4  -0.43406854655716964  \\
            -7.3  -0.42473830167078014  \\
            -7.2  -0.4154804935286875  \\
            -7.1  -0.40630073196066097  \\
            -7.0  -0.39720462679647106  \\
            -6.9  -0.38819695284290595  \\
            -6.8  -0.37927535811426805  \\
            -6.7  -0.37043476225159827  \\
            -6.6  -0.36167008489593716  \\
            -6.5  -0.35297624568832575  \\
            -6.4  -0.3443481642698045  \\
            -6.3  -0.3357807602814143  \\
            -6.2  -0.32726895336419587  \\
            -6.1  -0.31880766315918996  \\
            -6.0  -0.31039180930743737  \\
            -5.9  -0.30201631144997887  \\
            -5.8  -0.293676089227855  \\
            -5.7  -0.2853660622821067  \\
            -5.6  -0.2770811502537746  \\
            -5.5  -0.26881627278389963  \\
            -5.4  -0.2605663495135224  \\
            -5.3  -0.2523263000836836  \\
            -5.2  -0.24409104413542415  \\
            -5.1  -0.2358555013097846  \\
            -5.0  -0.22761459124780592  \\
            -4.9  -0.219369148014383  \\
            -4.8  -0.21112347614351137  \\
            -4.7  -0.20287770097988275  \\
            -4.6  -0.19463194786818852  \\
            -4.5  -0.18638634215312036  \\
            -4.4  -0.17814100917936984  \\
            -4.3  -0.16989607429162837  \\
            -4.2  -0.16165166283458768  \\
            -4.1  -0.15340790015293918  \\
            -4.0  -0.14516491159137457  \\
            -3.9  -0.13692282249458532  \\
            -3.8  -0.128681758207263  \\
            -3.7  -0.1204418440740992  \\
            -3.6  -0.11220320543978542  \\
            -3.5  -0.10396596764901331  \\
            -3.4  -0.0957302560464743  \\
            -3.3  -0.08749619597686004  \\
            -3.2  -0.07926391278486208  \\
            -3.1  -0.07103353181517197  \\
            -3.0  -0.0628051784124812  \\
            -2.9  -0.05457827968038008  \\
            -2.8  -0.04635219276761608  \\
            -2.7  -0.03812693808661665  \\
            -2.6  -0.029902536049809216  \\
            -2.5  -0.021679007069621267  \\
            -2.4  -0.013456371558480264  \\
            -2.3  -0.005234649928813662  \\
            -2.2  0.0029861374069510417  \\
            -2.1  0.01120597003638644  \\
            -2.0  0.01942482754706506  \\
            -1.9  0.027642689526559426  \\
            -1.8  0.03585953556244206  \\
            -1.7  0.04407534524228555  \\
            -1.6  0.05229009815366238  \\
            -1.5  0.0605037738841451  \\
            -1.4  0.0687163520213063  \\
            -1.3  0.07692781215271845  \\
            -1.2  0.08513813386595416  \\
            -1.1  0.09334729674858588  \\
            -1.0  0.10155528038818624  \\
            -0.9  0.10974629164219084  \\
            -0.8  0.11790633891780859  \\
            -0.7  0.12603810412727198  \\
            -0.6  0.13414426918281347  \\
            -0.5  0.14222751599666544  \\
            -0.4  0.1502905264810605  \\
            -0.3  0.15833598254823103  \\
            -0.2  0.1663665661104095  \\
            -0.1  0.17438495907982834  \\
            0.0  0.18239384336872003  \\
            0.1  0.19039590088931707  \\
            0.2  0.19839381355385186  \\
            0.3  0.20639026327455692  \\
            0.4  0.21438793196366468  \\
            0.5  0.2223895015334076  \\
            0.6  0.23039765389601816  \\
            0.7  0.23841507096372883  \\
            0.8  0.246444434648772  \\
            0.9  0.25448842686338025  \\
            1.0  0.26254972951978595  \\
            1.1  0.27063662906307523  \\
            1.2  0.27875552592342634  \\
            1.3  0.2869062729907099  \\
            1.4  0.29508872315479673  \\
            1.5  0.30330272930555735  \\
            1.6  0.3115481443328624  \\
            1.7  0.31982482112658256  \\
            1.8  0.3281326125765886  \\
            1.9  0.336471371572751  \\
            2.0  0.34484095100494055  \\
            2.1  0.35324120376302787  \\
            2.2  0.36167198273688356  \\
            2.3  0.37013314081637827  \\
            2.4  0.37862453089138287  \\
            2.5  0.3871460058517678  \\
            2.6  0.3956974185874038  \\
            2.7  0.40427862198816145  \\
            2.8  0.4128894689439116  \\
            2.9  0.4215298123445246  \\
            3.0  0.4301995050798715  \\
            3.1  0.4391561878004104  \\
            3.2  0.44862321126632887  \\
            3.3  0.45854899353209094  \\
            3.4  0.46888195265216104  \\
            3.5  0.47957050668100315  \\
            3.6  0.4905630736730815  \\
            3.7  0.5018080716828605  \\
            3.8  0.5132539187648036  \\
            3.9  0.5248490329733757  \\
            4.0  0.5365418323630409  \\
            4.1  0.5482807349882628  \\
            4.2  0.5600141589035064  \\
            4.3  0.5716905221632351  \\
            4.4  0.5832582428219135  \\
            4.5  0.5946657389340056  \\
            4.6  0.6058614285539758  \\
            4.7  0.616793729736288  \\
            4.8  0.6274110605354065  \\
            4.9  0.6376618390057956  \\
            5.0  0.647494483201919  \\
            5.1  0.6570830264708788  \\
            5.2  0.666627587419222  \\
            5.3  0.6761157119905808  \\
            5.4  0.6855349461285877  \\
            5.5  0.6948728357768753  \\
            5.6  0.7041169268790759  \\
            5.7  0.7132547653788219  \\
            5.8  0.7222738972197459  \\
            5.9  0.73116186834548  \\
            6.0  0.739906224699657  \\
            6.1  0.748494512225909  \\
            6.2  0.7569142768678689  \\
            6.3  0.7651530645691688  \\
            6.4  0.773198421273441  \\
            6.5  0.781037892924318  \\
            6.6  0.7886590254654325  \\
            6.7  0.7960493648404167  \\
            6.8  0.8031964569929033  \\
            6.9  0.8100878478665243  \\
            7.0  0.8167110834049125  \\
            7.1  0.8231796808164253  \\
            7.2  0.8296159261257726  \\
            7.3  0.8360205185011138  \\
            7.4  0.8423941571106094  \\
            7.5  0.8487375411224191  \\
            7.6  0.8550513697047025  \\
            7.7  0.8613363420256201  \\
            7.8  0.8675931572533314  \\
            7.9  0.8738225145559968  \\
            8.0  0.8800251131017758  \\
            8.1  0.8862016520588288  \\
            8.2  0.8923528305953151  \\
            8.3  0.8984793478793954  \\
            8.4  0.9045819030792291  \\
            8.5  0.9106611953629763  \\
            8.6  0.916717923898797  \\
            8.7  0.9227527878548509  \\
            8.8  0.9287664863992987  \\
            8.9  0.9347597187002992  \\
            9.0  0.9407331839260133  \\
            9.1  0.9466996544141143  \\
            9.2  0.9526697391467434  \\
            9.3  0.9586408922587618  \\
            9.4  0.9646105678850299  \\
            9.5  0.9705762201604089  \\
            9.6  0.9765353032197601  \\
            9.7  0.9824852711979435  \\
            9.8  0.988423578229821  \\
            9.9  0.9943476784502527  \\
            10.0  1.0002550259941  \\
            10.1  1.0061430749962241  \\
            10.2  1.0120092795914852  \\
            10.3  1.017851093914745  \\
            10.4  1.0236659721008636  \\
            10.5  1.0294513682847024  \\
            10.6  1.035204736601122  \\
            10.7  1.0409235311849838  \\
            10.8  1.0466052061711486  \\
            10.9  1.0522472156944769  \\
            11.0  1.0578470138898304  \\
            11.1  1.0633673805768036  \\
            11.2  1.0687769794498236  \\
            11.3  1.0740820904559971  \\
            11.4  1.079288993542431  \\
            11.5  1.0844039686562328  \\
            11.6  1.089433295744509  \\
            11.7  1.094383254754367  \\
            11.8  1.0992601256329142  \\
            11.9  1.104070188327257  \\
            12.0  1.1088197227845027  \\
            12.1  1.1135150089517583  \\
            12.2  1.1181623267761314  \\
            12.3  1.1227679562047284  \\
            12.4  1.1273381771846567  \\
            12.5  1.131879269663023  \\
            12.6  1.1363975135869349  \\
            12.7  1.1408991889034992  \\
            12.8  1.1453905755598228  \\
            12.9  1.149877953503013  \\
            13.0  1.1543676026801768  \\
            13.1  1.1588244063254187  \\
            13.2  1.1632103242279181  \\
            13.3  1.1675272511673946  \\
            13.4  1.171777081923567  \\
            13.5  1.1759617112761553  \\
            13.6  1.180083034004879  \\
            13.7  1.1841429448894567  \\
            13.8  1.188143338709608  \\
            13.9  1.1920861102450526  \\
            14.0  1.1959731542755097  \\
            14.1  1.1998063655806988  \\
            14.2  1.203587638940339  \\
            14.3  1.20731886913415  \\
            14.4  1.211001950941851  \\
            14.5  1.2146387791431614  \\
            14.6  1.2182312485178004  \\
            14.7  1.2217812538454875  \\
            14.8  1.2252906899059424  \\
            14.9  1.2287614514788838  \\
            15.0  1.2321954333440317  \\
            15.1  1.2355292017516304  \\
            15.2  1.2387008521139955  \\
            15.3  1.2417145729539554  \\
            15.4  1.2445745527943373  \\
            15.5  1.247284980157968  \\
            15.6  1.2498500435676752  \\
            15.7  1.2522739315462865  \\
            15.8  1.2545608326166293  \\
            15.9  1.2567149353015308  \\
            16.0  1.2587404281238181  \\
            16.1  1.260641499606319  \\
            16.2  1.262422338271861  \\
            16.3  1.2640871326432712  \\
            16.4  1.2656400712433766  \\
            16.5  1.2670853425950053  \\
            16.6  1.2684271352209844  \\
            16.7  1.2696696376441412  \\
            16.8  1.2708170383873032  \\
            16.9  1.2718735259732976  \\
            17.0  1.2728432889249521  \\
            17.1  1.2736780404315666  \\
            17.2  1.274329265096847  \\
            17.3  1.2748008085652307  \\
            17.4  1.275096516481155  \\
            17.5  1.275220234489057  \\
            17.6  1.2751758082333735  \\
            17.7  1.274967083358542  \\
            17.8  1.274597905508999  \\
            17.9  1.2740721203291827  \\
            18.0  1.2733935734635293  \\
            18.1  1.2725661105564758  \\
            18.2  1.2715935772524598  \\
            18.3  1.2704798191959186  \\
            18.4  1.2692286820312888  \\
            18.5  1.2678440114030078  \\
            18.6  1.2663296529555128  \\
            18.7  1.2646894523332404  \\
            18.8  1.2629272551806283  \\
            18.9  1.2610469071421133  \\
            19.0  1.2590522538621327  \\
        }
        ;
    \node[left, , color={rgb,1:red,0.0;green,0.3608;blue,0.6706}, draw opacity={1.0}, rotate={0.0}, font={{\fontsize{6 pt}{7.800000000000001 pt}\selectfont}}]  at (axis cs:15.0,1.5890088595252194) {Nominal};
    \node[left, , color={rgb,1:red,0.7529;green,0.3255;blue,0.4039}, draw opacity={1.0}, rotate={0.0}, font={{\fontsize{6 pt}{7.800000000000001 pt}\selectfont}}]  at (axis cs:21.0,0.6376101172445285) {Corrected};
\end{axis}
\end{tikzpicture}

    \caption{Example curves demonstrating the changes to the lift coefficient vs angle of attack for the nominal polar when corrections for a solidity of 1.0 at a stagger angle of \(\pi/4\) are applied.}
	\label{fig:soliditystagger-correction}
\end{marginfigure}
%
The corrections depend both on solidity and stagger, though stagger only begin to effect the correction after 20 degrees.
%
These corrections are somewhat limited as they assume the airfoil camber is well matched to the operating conditions such that the deviation angle isn't overly large, but they should be sufficient for our purposes.
%
Wallis gives his corrections in the form of a line plot, to which quadratic fits are made.
%
Specifically, we use the quadratic fits provided in the DFDC source code.
%
The model is also applied only for stagger angles less than 90 degrees, and stagger effects are held constant after that.
%
Furthermore, the total correction factor is set to a maximum of 1, since, as stated by Wallis, ``there are no documented examples of factors exceeding unity,'' and the tendency of theoretical models predict values above 1 appears to be due to not capturing increased deviation angles completely.

For implementation, since the solidity and stagger corrections only apply for stagger angles between 20 and 90 degrees, and we also set a maximum adjustment factor of 1, we limit stagger angles below 20 degrees to 20 degrees, and above 90 degrees to 90 degrees.
%
We apply these limits using a sigmoid blending function between the limited ranges and the nominal range.
%
For the limit of the overall adjustment factor, we subtract the difference of the unlimited adjustment factor and the factor limit of 1 from the unlimited factor.
%
To keep the end product smooth, we actually apply another sigmoid blending function to the difference to be subtracted and zero, centered just before the point of limitation to mitigate overshoot by the blending function.
%
\Cref{fig:cascadesmoothed} shows the limited correction curves with respect to changes in solidity and stagger.
%
We also see in \cref{fig:cascadesmoothed} that our applications of smooth blending functions have minimal effect on the calculated correction values.

\begin{figure}[htb]
     \centering
     \begin{subfigure}[t]{0.45\textwidth}
         \centering
        % Recommended preamble:
% \usetikzlibrary{arrows.meta}
% \usetikzlibrary{backgrounds}
% \usepgfplotslibrary{patchplots}
% \usepgfplotslibrary{fillbetween}
% \pgfplotsset{%
%     layers/standard/.define layer set={%
%         background,axis background,axis grid,axis ticks,axis lines,axis tick labels,pre main,main,axis descriptions,axis foreground%
%     }{
%         grid style={/pgfplots/on layer=axis grid},%
%         tick style={/pgfplots/on layer=axis ticks},%
%         axis line style={/pgfplots/on layer=axis lines},%
%         label style={/pgfplots/on layer=axis descriptions},%
%         legend style={/pgfplots/on layer=axis descriptions},%
%         title style={/pgfplots/on layer=axis descriptions},%
%         colorbar style={/pgfplots/on layer=axis descriptions},%
%         ticklabel style={/pgfplots/on layer=axis tick labels},%
%         axis background@ style={/pgfplots/on layer=axis background},%
%         3d box foreground style={/pgfplots/on layer=axis foreground},%
%     },
% }

\begin{tikzpicture}[/tikz/background rectangle/.style={fill={rgb,1:red,1.0;green,1.0;blue,1.0}, fill opacity={1.0}, draw opacity={1.0}}, show background rectangle]
\begin{axis}[point meta max={nan}, point meta min={nan}, legend cell align={left}, legend columns={1}, title={}, title style={at={{(0.5,1)}}, anchor={south}, font={{\fontsize{14 pt}{18.2 pt}\selectfont}}, color={rgb,1:red,0.0;green,0.0;blue,0.0}, draw opacity={1.0}, rotate={0.0}, align={center}}, legend style={color={rgb,1:red,0.0;green,0.0;blue,0.0}, draw opacity={0.0}, line width={1}, solid, fill={rgb,1:red,0.0;green,0.0;blue,0.0}, fill opacity={0.0}, text opacity={1.0}, font={{\fontsize{8 pt}{10.4 pt}\selectfont}}, text={rgb,1:red,0.0;green,0.0;blue,0.0}, cells={anchor={center}}, at={(1.02, 1)}, anchor={north west}}, axis background/.style={fill={rgb,1:red,0.0;green,0.0;blue,0.0}, opacity={0.0}}, anchor={north west}, xshift={0.0mm}, yshift={-0.0mm}, width={45.8mm}, height={50.8mm}, scaled x ticks={false}, xlabel={Solidity}, x tick style={color={rgb,1:red,0.0;green,0.0;blue,0.0}, opacity={1.0}}, x tick label style={color={rgb,1:red,0.0;green,0.0;blue,0.0}, opacity={1.0}, rotate={0}}, xlabel style={at={(ticklabel cs:0.5)}, anchor=near ticklabel, at={{(ticklabel cs:0.5)}}, anchor={near ticklabel}, font={{\fontsize{11 pt}{14.3 pt}\selectfont}}, color={rgb,1:red,0.0;green,0.0;blue,0.0}, draw opacity={1.0}, rotate={0.0}}, xmajorgrids={false}, xmin={-0.09000000000000008}, xmax={3.09}, xticklabels={{$0$,$1$,$2$,$3$}}, xtick={{0.0,1.0,2.0,3.0}}, xtick align={inside}, xticklabel style={font={{\fontsize{8 pt}{10.4 pt}\selectfont}}, color={rgb,1:red,0.0;green,0.0;blue,0.0}, draw opacity={1.0}, rotate={0.0}}, x grid style={color={rgb,1:red,0.0;green,0.0;blue,0.0}, draw opacity={0.1}, line width={0.5}, solid}, axis x line*={left}, x axis line style={color={rgb,1:red,0.0;green,0.0;blue,0.0}, draw opacity={1.0}, line width={1}, solid}, scaled y ticks={false}, ylabel={$c_{\ell_\mathrm{ss}}$}, y tick style={color={rgb,1:red,0.0;green,0.0;blue,0.0}, opacity={1.0}}, y tick label style={color={rgb,1:red,0.0;green,0.0;blue,0.0}, opacity={1.0}, rotate={0}}, ylabel style={at={(ticklabel cs:0.5)}, anchor=near ticklabel, at={{(ticklabel cs:0.5)}}, anchor={near ticklabel}, font={{\fontsize{11 pt}{14.3 pt}\selectfont}}, color={rgb,1:red,0.0;green,0.0;blue,0.0}, draw opacity={1.0}, rotate={0.0}}, ymajorgrids={false}, ymin={0.2903697512727046}, ymax={1.0214225391507328}, yticklabels={{$0.4$,$0.6$,$0.8$,$1.0$}}, ytick={{0.4,0.6000000000000001,0.8,1.0}}, ytick align={inside}, yticklabel style={font={{\fontsize{8 pt}{10.4 pt}\selectfont}}, color={rgb,1:red,0.0;green,0.0;blue,0.0}, draw opacity={1.0}, rotate={0.0}}, y grid style={color={rgb,1:red,0.0;green,0.0;blue,0.0}, draw opacity={0.1}, line width={0.5}, solid}, axis y line*={left}, y axis line style={color={rgb,1:red,0.0;green,0.0;blue,0.0}, draw opacity={1.0}, line width={1}, solid}, colorbar={false}]
    \addplot[color={rgb,1:red,0.0;green,0.3608;blue,0.6706}, name path={9b84746f-50be-4224-b746-5f30b8b6b4df}, draw opacity={1.0}, line width={1.0}, solid, forget plot]
        table[row sep={\\}]
        {
            \\
            0.030303030303030304  1.0  \\
            0.06060606060606061  1.0  \\
            0.09090909090909091  1.0  \\
            0.12121212121212122  1.0  \\
            0.15151515151515152  1.0  \\
            0.18181818181818182  1.0  \\
            0.21212121212121213  1.0  \\
            0.24242424242424243  1.0  \\
            0.2727272727272727  1.0  \\
            0.30303030303030304  1.0  \\
            0.3333333333333333  1.0  \\
            0.36363636363636365  1.0  \\
            0.3939393939393939  1.0  \\
            0.42424242424242425  1.000000000000031  \\
            0.45454545454545453  1.000000000008658  \\
            0.48484848484848486  1.0000000011321493  \\
            0.5151515151515151  1.0000000778635134  \\
            0.5454545454545454  1.0000030314595953  \\
            0.5757575757575758  1.0000679042742788  \\
            0.6060606060606061  1.0007323659089018  \\
            0.6363636363636364  0.9984568021146492  \\
            0.6666666666666666  0.9724965708391975  \\
            0.696969696969697  0.945379359428743  \\
            0.7272727272727273  0.9185979355561048  \\
            0.7575757575757576  0.8893094001743276  \\
            0.7878787878787878  0.8603956688887157  \\
            0.8181818181818182  0.832466667413929  \\
            0.8484848484848485  0.806403449739479  \\
            0.8787878787878788  0.7820127710636436  \\
            0.9090909090909091  0.7592481391855712  \\
            0.9393939393939394  0.7379534532295529  \\
            0.9696969696969697  0.7179896851619572  \\
            1.0  0.6992358424338074  \\
            1.0303030303030303  0.6821546599372428  \\
            1.0606060606060606  0.6660495450119536  \\
            1.0909090909090908  0.6508391586936328  \\
            1.121212121212121  0.6359803791819263  \\
            1.1515151515151516  0.620987255392311  \\
            1.1818181818181819  0.6067630097457533  \\
            1.2121212121212122  0.5932499763815233  \\
            1.2424242424242424  0.5803961153765241  \\
            1.2727272727272727  0.569477161793638  \\
            1.303030303030303  0.5594967517085511  \\
            1.3333333333333333  0.5499699966273316  \\
            1.3636363636363635  0.5408666528830552  \\
            1.393939393939394  0.5321591066928779  \\
            1.4242424242424243  0.5238220943831335  \\
            1.4545454545454546  0.515369974042599  \\
            1.4848484848484849  0.5071873306884214  \\
            1.5151515151515151  0.49933199306841114  \\
            1.5454545454545454  0.4917847079040875  \\
            1.5757575757575757  0.48452770293839176  \\
            1.606060606060606  0.4775445472166845  \\
            1.6363636363636365  0.4708200268920773  \\
            1.6666666666666667  0.46434003457927414  \\
            1.696969696969697  0.4581814587284587  \\
            1.7272727272727273  0.4522389732583737  \\
            1.7575757575757576  0.4465014010803607  \\
            1.7878787878787878  0.4409583228744835  \\
            1.8181818181818181  0.4356000139421357  \\
            1.8484848484848484  0.43041738726986484  \\
            1.878787878787879  0.425401942103151  \\
            1.9090909090909092  0.42054571741792035  \\
            1.9393939393939394  0.4158412497541031  \\
            1.9696969696969697  0.41128153494148006  \\
            2.0  0.4068599933049973  \\
            2.0303030303030303  0.4025704379860213  \\
            2.0606060606060606  0.39840704605878  \\
            2.090909090909091  0.39436433215841527  \\
            2.121212121212121  0.3904371243694897  \\
            2.1515151515151514  0.38662054215208286  \\
            2.1818181818181817  0.382909976107382  \\
            2.212121212121212  0.37930106940637126  \\
            2.242424242424242  0.37578970072430706  \\
            2.272727272727273  0.3723719685404311  \\
            2.303030303030303  0.3690441766771835  \\
            2.3333333333333335  0.365802820966228  \\
            2.3636363636363638  0.36264457694016883  \\
            2.393939393939394  0.3595662884590732  \\
            2.4242424242424243  0.35656495719000497  \\
            2.4545454545454546  0.35363773286585204  \\
            2.484848484848485  0.35078190425692224  \\
            2.515151515151515  0.3479948907951956  \\
            2.5454545454545454  0.34527423479684355  \\
            2.5757575757575757  0.34261759423374677  \\
            2.606060606060606  0.3400227360093265  \\
            2.6363636363636362  0.3374875296981113  \\
            2.6666666666666665  0.335009941712151  \\
            2.696969696969697  0.332588029860707  \\
            2.727272727272727  0.3302199382726279  \\
            2.757575757575758  0.3279038926535179  \\
            2.787878787878788  0.32563819585221454  \\
            2.8181818181818183  0.3234212237133049  \\
            2.8484848484848486  0.3212514211943721  \\
            2.878787878787879  0.31912729872846923  \\
            2.909090909090909  0.3170474288139396  \\
            2.9393939393939394  0.31501044281517354  \\
            2.9696969696969697  0.31301502795923913  \\
            3.0  0.31105992451453557  \\
        }
        ;
    \addplot[color={rgb,1:red,0.7529;green,0.3255;blue,0.4039}, name path={2bed704e-652e-45db-a856-393b6dfcd944}, draw opacity={1.0}, line width={2}, dashed, forget plot]
        table[row sep={\\}]
        {
            \\
            0.030303030303030304  1.0  \\
            0.06060606060606061  1.0  \\
            0.09090909090909091  1.0  \\
            0.12121212121212122  1.0  \\
            0.15151515151515152  1.0  \\
            0.18181818181818182  1.0  \\
            0.21212121212121213  1.0  \\
            0.24242424242424243  1.0  \\
            0.2727272727272727  1.0  \\
            0.30303030303030304  1.0  \\
            0.3333333333333333  1.0  \\
            0.36363636363636365  1.0  \\
            0.3939393939393939  1.0  \\
            0.42424242424242425  1.0  \\
            0.45454545454545453  1.0  \\
            0.48484848484848486  1.0  \\
            0.5151515151515151  1.0  \\
            0.5454545454545454  1.0  \\
            0.5757575757575758  1.0  \\
            0.6060606060606061  1.0  \\
            0.6363636363636364  0.9957220048705054  \\
            0.6666666666666666  0.9690960827710882  \\
            0.696969696969697  0.9447854582455333  \\
            0.7272727272727273  0.9185338294984133  \\
            0.7575757575757576  0.8893047123647337  \\
            0.7878787878787878  0.8603953406120209  \\
            0.8181818181818182  0.8324666432863096  \\
            0.8484848484848485  0.8064034476816908  \\
            0.8787878787878788  0.7820127708614998  \\
            0.9090909090909091  0.759248139162655  \\
            0.9393939393939394  0.7379534532265871  \\
            0.9696969696969697  0.7179896851615236  \\
            1.0  0.6992358424337366  \\
            1.0303030303030303  0.6821546599372292  \\
            1.0606060606060606  0.6660495450119507  \\
            1.0909090909090908  0.6508391586936322  \\
            1.121212121212121  0.6359803791819262  \\
            1.1515151515151516  0.620987255392311  \\
            1.1818181818181819  0.6067630097457533  \\
            1.2121212121212122  0.5932499763815233  \\
            1.2424242424242424  0.5803961153765241  \\
            1.2727272727272727  0.569477161793638  \\
            1.303030303030303  0.5594967517085511  \\
            1.3333333333333333  0.5499699966273316  \\
            1.3636363636363635  0.5408666528830552  \\
            1.393939393939394  0.5321591066928779  \\
            1.4242424242424243  0.5238220943831335  \\
            1.4545454545454546  0.515369974042599  \\
            1.4848484848484849  0.5071873306884214  \\
            1.5151515151515151  0.4993319930684111  \\
            1.5454545454545454  0.49178470790408746  \\
            1.5757575757575757  0.4845277029383917  \\
            1.606060606060606  0.47754454721668443  \\
            1.6363636363636365  0.47082002689207725  \\
            1.6666666666666667  0.4643400345792741  \\
            1.696969696969697  0.45818145872845867  \\
            1.7272727272727273  0.45223897325837364  \\
            1.7575757575757576  0.4465014010803606  \\
            1.7878787878787878  0.44095832287448344  \\
            1.8181818181818181  0.43560001394213566  \\
            1.8484848484848484  0.4304173872698648  \\
            1.878787878787879  0.42540194210315097  \\
            1.9090909090909092  0.4205457174179203  \\
            1.9393939393939394  0.41584124975410297  \\
            1.9696969696969697  0.41128153494148  \\
            2.0  0.4068599933049972  \\
            2.0303030303030303  0.40257043798602127  \\
            2.0606060606060606  0.39840704605877997  \\
            2.090909090909091  0.3943643321584152  \\
            2.121212121212121  0.39043712436948963  \\
            2.1515151515151514  0.3866205421520828  \\
            2.1818181818181817  0.38290997610738187  \\
            2.212121212121212  0.3793010694063712  \\
            2.242424242424242  0.375789700724307  \\
            2.272727272727273  0.37237196854043103  \\
            2.303030303030303  0.3690441766771834  \\
            2.3333333333333335  0.36580282096622796  \\
            2.3636363636363638  0.3626445769401688  \\
            2.393939393939394  0.3595662884590731  \\
            2.4242424242424243  0.3565649571900049  \\
            2.4545454545454546  0.353637732865852  \\
            2.484848484848485  0.35078190425692213  \\
            2.515151515151515  0.3479948907951955  \\
            2.5454545454545454  0.3452742347968435  \\
            2.5757575757575757  0.3426175942337467  \\
            2.606060606060606  0.3400227360093264  \\
            2.6363636363636362  0.33748752969811124  \\
            2.6666666666666665  0.3350099417121509  \\
            2.696969696969697  0.33258802986070696  \\
            2.727272727272727  0.33021993827262786  \\
            2.757575757575758  0.3279038926535178  \\
            2.787878787878788  0.3256381958522145  \\
            2.8181818181818183  0.32342122371330484  \\
            2.8484848484848486  0.32125142119437206  \\
            2.878787878787879  0.3191272987284691  \\
            2.909090909090909  0.31704742881393955  \\
            2.9393939393939394  0.3150104428151735  \\
            2.9696969696969697  0.3130150279592391  \\
            3.0  0.3110599245145355  \\
        }
        ;
    \node[right, , color={rgb,1:red,0.0;green,0.3608;blue,0.6706}, draw opacity={1.0}, rotate={0.0}, font={{\fontsize{8 pt}{10.4 pt}\selectfont}}]  at (axis cs:1.5,0.9) {Nominal};
    \node[right, , color={rgb,1:red,0.7529;green,0.3255;blue,0.4039}, draw opacity={1.0}, rotate={0.0}, font={{\fontsize{8 pt}{10.4 pt}\selectfont}}]  at (axis cs:1.5,0.8) {Smoothed};
\end{axis}
\end{tikzpicture}

        \caption{Corrected value vs solidity for a stagger angle of \(\pi/4\).}
        \label{fig:solidtysmoothed}
     \end{subfigure}
     \hfill
     \begin{subfigure}[t]{0.45\textwidth}
         \centering
         % Recommended preamble:
% \usetikzlibrary{arrows.meta}
% \usetikzlibrary{backgrounds}
% \usepgfplotslibrary{patchplots}
% \usepgfplotslibrary{fillbetween}
% \pgfplotsset{%
%     layers/standard/.define layer set={%
%         background,axis background,axis grid,axis ticks,axis lines,axis tick labels,pre main,main,axis descriptions,axis foreground%
%     }{
%         grid style={/pgfplots/on layer=axis grid},%
%         tick style={/pgfplots/on layer=axis ticks},%
%         axis line style={/pgfplots/on layer=axis lines},%
%         label style={/pgfplots/on layer=axis descriptions},%
%         legend style={/pgfplots/on layer=axis descriptions},%
%         title style={/pgfplots/on layer=axis descriptions},%
%         colorbar style={/pgfplots/on layer=axis descriptions},%
%         ticklabel style={/pgfplots/on layer=axis tick labels},%
%         axis background@ style={/pgfplots/on layer=axis background},%
%         3d box foreground style={/pgfplots/on layer=axis foreground},%
%     },
% }

\begin{tikzpicture}[/tikz/background rectangle/.style={fill={rgb,1:red,1.0;green,1.0;blue,1.0}, fill opacity={1.0}, draw opacity={1.0}}, show background rectangle]
\begin{axis}[point meta max={nan}, point meta min={nan}, legend cell align={left}, legend columns={1}, title={}, title style={at={{(0.5,1)}}, anchor={south}, font={{\fontsize{14 pt}{18.2 pt}\selectfont}}, color={rgb,1:red,0.0;green,0.0;blue,0.0}, draw opacity={1.0}, rotate={0.0}, align={center}}, legend style={color={rgb,1:red,0.0;green,0.0;blue,0.0}, draw opacity={0.0}, line width={1}, solid, fill={rgb,1:red,0.0;green,0.0;blue,0.0}, fill opacity={0.0}, text opacity={1.0}, font={{\fontsize{8 pt}{10.4 pt}\selectfont}}, text={rgb,1:red,0.0;green,0.0;blue,0.0}, cells={anchor={center}}, at={(1.02, 1)}, anchor={north west}}, axis background/.style={fill={rgb,1:red,0.0;green,0.0;blue,0.0}, opacity={0.0}}, anchor={north west}, xshift={0.0mm}, yshift={-0.0mm}, width={45.8mm}, height={50.8mm}, scaled x ticks={false}, xlabel={Stagger (degrees)}, x tick style={color={rgb,1:red,0.0;green,0.0;blue,0.0}, opacity={1.0}}, x tick label style={color={rgb,1:red,0.0;green,0.0;blue,0.0}, opacity={1.0}, rotate={0}}, xlabel style={at={(ticklabel cs:0.5)}, anchor=near ticklabel, at={{(ticklabel cs:0.5)}}, anchor={near ticklabel}, font={{\fontsize{11 pt}{14.3 pt}\selectfont}}, color={rgb,1:red,0.0;green,0.0;blue,0.0}, draw opacity={1.0}, rotate={0.0}}, xmajorgrids={false}, xmin={-1.970000000000006}, xmax={102.97}, xticklabels={{$0$,$25$,$50$,$75$,$100$}}, xtick={{0.0,25.0,50.0,75.0,100.0}}, xtick align={inside}, xticklabel style={font={{\fontsize{8 pt}{10.4 pt}\selectfont}}, color={rgb,1:red,0.0;green,0.0;blue,0.0}, draw opacity={1.0}, rotate={0.0}}, x grid style={color={rgb,1:red,0.0;green,0.0;blue,0.0}, draw opacity={0.1}, line width={0.5}, solid}, axis x line*={left}, x axis line style={color={rgb,1:red,0.0;green,0.0;blue,0.0}, draw opacity={1.0}, line width={1}, solid}, scaled y ticks={false}, ylabel={$c_{\ell_\mathrm{ss}}$}, y tick style={color={rgb,1:red,0.0;green,0.0;blue,0.0}, opacity={1.0}}, y tick label style={color={rgb,1:red,0.0;green,0.0;blue,0.0}, opacity={1.0}, rotate={0}}, ylabel style={at={(ticklabel cs:0.5)}, anchor=near ticklabel, at={{(ticklabel cs:0.5)}}, anchor={near ticklabel}, font={{\fontsize{11 pt}{14.3 pt}\selectfont}}, color={rgb,1:red,0.0;green,0.0;blue,0.0}, draw opacity={1.0}, rotate={0.0}}, ymajorgrids={false}, ymin={0.38871965277900666}, ymax={1.01895092559128}, yticklabels={{$0.4$,$0.5$,$0.6$,$0.7$,$0.8$,$0.9$,$1.0$}}, ytick={{0.4,0.5,0.6000000000000001,0.7000000000000001,0.8,0.9,1.0}}, ytick align={inside}, yticklabel style={font={{\fontsize{8 pt}{10.4 pt}\selectfont}}, color={rgb,1:red,0.0;green,0.0;blue,0.0}, draw opacity={1.0}, rotate={0.0}}, y grid style={color={rgb,1:red,0.0;green,0.0;blue,0.0}, draw opacity={0.1}, line width={0.5}, solid}, axis y line*={left}, y axis line style={color={rgb,1:red,0.0;green,0.0;blue,0.0}, draw opacity={1.0}, line width={1}, solid}, colorbar={false}]
    \addplot[color={rgb,1:red,0.0;green,0.3608;blue,0.6706}, name path={fb0ed5e4-35ab-434d-930f-5830a4134cf8}, draw opacity={1.0}, line width={1.0}, solid, forget plot]
        table[row sep={\\}]
        {
            \\
            1.0  0.4068599933049972  \\
            2.0  0.4068599933049972  \\
            3.0  0.4068599933049972  \\
            4.0  0.4068599933049972  \\
            5.0  0.4068599933049972  \\
            6.0  0.4068599933049972  \\
            7.0  0.4068599933049972  \\
            8.0  0.4068599933049972  \\
            9.0  0.4068599933049972  \\
            10.0  0.4068599933049972  \\
            11.0  0.4068599933049972  \\
            12.0  0.4068599933049972  \\
            13.0  0.4068599933049972  \\
            14.0  0.4068599933049972  \\
            15.0  0.4068599933049972  \\
            16.0  0.4068599933049972  \\
            17.0  0.4068599933049972  \\
            18.0  0.4068599933049972  \\
            19.0  0.4068599933049972  \\
            20.0  0.4068599933049972  \\
            21.0  0.40655909024010684  \\
            22.0  0.406556386915203  \\
            23.0  0.4068518833302859  \\
            24.0  0.4074455794853555  \\
            25.0  0.40833747538041165  \\
            26.0  0.40952757101545445  \\
            27.0  0.41101586639048393  \\
            28.0  0.41280236150550004  \\
            29.0  0.41488705636050277  \\
            30.0  0.41726995095549213  \\
            31.0  0.41995104529046823  \\
            32.0  0.4229303393654308  \\
            33.0  0.42620783318038014  \\
            34.0  0.4297835267353161  \\
            35.0  0.43365742003023866  \\
            36.0  0.43782951306514795  \\
            37.0  0.4422998058400438  \\
            38.0  0.44706829835492634  \\
            39.0  0.45213499060979545  \\
            40.0  0.4574998826046513  \\
            41.0  0.4631629743394937  \\
            42.0  0.46912426581432287  \\
            43.0  0.47538375702913854  \\
            44.0  0.4819414479839409  \\
            45.0  0.48879733867873  \\
            46.0  0.49595142911350565  \\
            47.0  0.5034037192882679  \\
            48.0  0.5111542092030168  \\
            49.0  0.5192028988577524  \\
            50.0  0.5275497882524747  \\
            51.0  0.5361948773871836  \\
            52.0  0.545138166261879  \\
            53.0  0.5543796548765612  \\
            54.0  0.5639193432312299  \\
            55.0  0.5737572313258854  \\
            56.0  0.5838933191605276  \\
            57.0  0.5943276067351562  \\
            58.0  0.6050600940497717  \\
            59.0  0.6160907811043737  \\
            60.0  0.6274196678989623  \\
            61.0  0.6390467544335374  \\
            62.0  0.6509720407080996  \\
            63.0  0.663195526722648  \\
            64.0  0.6757172124771833  \\
            65.0  0.6885370979717051  \\
            66.0  0.7016551832062137  \\
            67.0  0.7150714681807089  \\
            68.0  0.7287859528951905  \\
            69.0  0.742798637349659  \\
            70.0  0.7571095215441139  \\
            71.0  0.7717186054785556  \\
            72.0  0.7866258891529841  \\
            73.0  0.8018313725673992  \\
            74.0  0.8173350557218009  \\
            75.0  0.8331369386161891  \\
            76.0  0.849237021250564  \\
            77.0  0.8656353036249254  \\
            78.0  0.8823317857392736  \\
            79.0  0.8993264675936086  \\
            80.0  0.9166193491879302  \\
            81.0  0.9342104305222383  \\
            82.0  0.9520997115965331  \\
            83.0  0.9702871924108143  \\
            84.0  0.9887728729650826  \\
            85.0  1.0  \\
            86.0  1.0  \\
            87.0  1.0  \\
            88.0  1.0  \\
            89.0  1.0  \\
            90.0  1.0  \\
            91.0  1.0  \\
            92.0  1.0  \\
            93.0  1.0  \\
            94.0  1.0  \\
            95.0  1.0  \\
            96.0  1.0  \\
            97.0  1.0  \\
            98.0  1.0  \\
            99.0  1.0  \\
            100.0  1.0  \\
        }
        ;
    \addplot[color={rgb,1:red,0.7529;green,0.3255;blue,0.4039}, name path={072b17de-446a-455c-a5bc-605b488bc77d}, draw opacity={1.0}, line width={2}, dashed, forget plot]
        table[row sep={\\}]
        {
            \\
            1.0  0.40685999330499745  \\
            2.0  0.4068599933049985  \\
            3.0  0.40685999330500383  \\
            4.0  0.40685999330503103  \\
            5.0  0.4068599933051692  \\
            6.0  0.4068599933058656  \\
            7.0  0.40685999330934436  \\
            8.0  0.4068599933265465  \\
            9.0  0.40685999341060713  \\
            10.0  0.4068599938156871  \\
            11.0  0.40685999573538106  \\
            12.0  0.4068600046493546  \\
            13.0  0.4068600450003877  \\
            14.0  0.4068602217657351  \\
            15.0  0.40686096273322053  \\
            16.0  0.4068638784565553  \\
            17.0  0.4068742426389738  \\
            18.0  0.40690425577280304  \\
            19.0  0.4069490423192419  \\
            20.0  0.4068599933049972  \\
            21.0  0.406603815654666  \\
            22.0  0.4065653673482496  \\
            23.0  0.4068519262595962  \\
            24.0  0.4074450359346553  \\
            25.0  0.4083372357635006  \\
            26.0  0.4095274954743027  \\
            27.0  0.4110158458433105  \\
            28.0  0.4128023563761271  \\
            29.0  0.4148870551508107  \\
            30.0  0.4172699506815995  \\
            31.0  0.41995104523033433  \\
            32.0  0.4229303393525429  \\
            33.0  0.4262078331776712  \\
            34.0  0.4297835267347557  \\
            35.0  0.4336574200301243  \\
            36.0  0.43782951306512485  \\
            37.0  0.4422998058400392  \\
            38.0  0.44706829835492545  \\
            39.0  0.4521349906097953  \\
            40.0  0.45749988260465124  \\
            41.0  0.4631629743394937  \\
            42.0  0.4691242658143228  \\
            43.0  0.47538375702913854  \\
            44.0  0.48194144798394095  \\
            45.0  0.48879733867873  \\
            46.0  0.4959514291135056  \\
            47.0  0.5034037192882679  \\
            48.0  0.5111542092030168  \\
            49.0  0.5192028988577524  \\
            50.0  0.5275497882524747  \\
            51.0  0.5361948773871836  \\
            52.0  0.545138166261879  \\
            53.0  0.5543796548765612  \\
            54.0  0.5639193432312299  \\
            55.0  0.5737572313258854  \\
            56.0  0.5838933191605276  \\
            57.0  0.5943276067351562  \\
            58.0  0.6050600940497717  \\
            59.0  0.6160907811043737  \\
            60.0  0.6274196678989624  \\
            61.0  0.6390467544335376  \\
            62.0  0.6509720407081001  \\
            63.0  0.6631955267226501  \\
            64.0  0.6757172124771906  \\
            65.0  0.6885370979717303  \\
            66.0  0.7016551832063033  \\
            67.0  0.7150714681810361  \\
            68.0  0.7287859528964178  \\
            69.0  0.7427986373543845  \\
            70.0  0.7571095215627827  \\
            71.0  0.7717186055541758  \\
            72.0  0.7866258894668372  \\
            73.0  0.8018313739008841  \\
            74.0  0.8173350615151281  \\
            75.0  0.8331369643144433  \\
            76.0  0.8492371374112971  \\
            77.0  0.8656358372309199  \\
            78.0  0.88233426721634  \\
            79.0  0.8993380820598444  \\
            80.0  0.9166735420249442  \\
            81.0  0.9344579766627344  \\
            82.0  0.9531582455614829  \\
            83.0  0.9739204269430451  \\
            84.0  0.9940451286417074  \\
            85.0  1.0011141914550836  \\
            86.0  1.0006628266944737  \\
            87.0  1.0001650879107247  \\
            88.0  1.0000306153011447  \\
            89.0  1.0000045907307484  \\
            90.0  1.0000009760809583  \\
            91.0  1.0000007396734782  \\
            92.0  1.0000008734593988  \\
            93.0  1.0000009472142009  \\
            94.0  1.0000009692061975  \\
            95.0  1.000000974564723  \\
            96.0  1.000000975760825  \\
            97.0  1.0000009760152857  \\
            98.0  1.0000009760677633  \\
            99.0  1.0000009760783488  \\
            100.0  1.0000009760804485  \\
        }
        ;
    \node[right, , color={rgb,1:red,0.0;green,0.3608;blue,0.6706}, draw opacity={1.0}, rotate={0.0}, font={{\fontsize{8 pt}{10.4 pt}\selectfont}}]  at (axis cs:5,1.0) {Nominal};
    \node[right, , color={rgb,1:red,0.7529;green,0.3255;blue,0.4039}, draw opacity={1.0}, rotate={0.0}, font={{\fontsize{8 pt}{10.4 pt}\selectfont}}]  at (axis cs:5,0.9) {Smoothed};
\end{axis}
\end{tikzpicture}

         \caption{Corrected value vs stagger angle for a solidity of \(\sigma=2\)}
         \label{fig:staggersmoothed}
     \end{subfigure}
     \caption{Nominal (with cutoffs) and smoothed solidity and stagger corrections for a nominal lift coefficient of 1.}
        \label{fig:cascadesmoothed}
\end{figure}



%---------------------------------#
%   Compressibility Corrections   #
%---------------------------------#
\subsection{Compressibility Lift Corrections}

% \subsubsection{Subsonic Corrections}

For subsonic compressibility corrections, we apply the well-used Pradtl-Glauert correction, which is based off of compressible potential flow and thin airfoil theories\scite{Glauert_1928}.
%
The Pradtl-Glauert correction states that for the nominal lift coefficient (which in our case is already corrected for solidty and stagger effects), \(c_{\ell_\text{ss}}\), one can apply a correction factor of \(\beta = \left[1-M^2\right]^{-1/2}\) to correct for compressibility affects of lift on the airfoil for Mach numbers, \(M\), up to about 0.7.

\begin{marginfigure}
	% Recommended preamble:
% \usetikzlibrary{arrows.meta}
% \usetikzlibrary{backgrounds}
% \usepgfplotslibrary{patchplots}
% \usepgfplotslibrary{fillbetween}
% \pgfplotsset{%
%     layers/standard/.define layer set={%
%         background,axis background,axis grid,axis ticks,axis lines,axis tick labels,pre main,main,axis descriptions,axis foreground%
%     }{
%         grid style={/pgfplots/on layer=axis grid},%
%         tick style={/pgfplots/on layer=axis ticks},%
%         axis line style={/pgfplots/on layer=axis lines},%
%         label style={/pgfplots/on layer=axis descriptions},%
%         legend style={/pgfplots/on layer=axis descriptions},%
%         title style={/pgfplots/on layer=axis descriptions},%
%         colorbar style={/pgfplots/on layer=axis descriptions},%
%         ticklabel style={/pgfplots/on layer=axis tick labels},%
%         axis background@ style={/pgfplots/on layer=axis background},%
%         3d box foreground style={/pgfplots/on layer=axis foreground},%
%     },
% }

\begin{tikzpicture}[/tikz/background rectangle/.style={fill={rgb,1:red,1.0;green,1.0;blue,1.0}, fill opacity={1.0}, draw opacity={1.0}}, show background rectangle]
\begin{axis}[point meta max={nan}, point meta min={nan}, legend cell align={left}, legend columns={1}, title={}, title style={at={{(0.5,1)}}, anchor={south}, font={{\fontsize{14 pt}{18.2 pt}\selectfont}}, color={rgb,1:red,0.0;green,0.0;blue,0.0}, draw opacity={1.0}, rotate={0.0}, align={center}}, legend style={color={rgb,1:red,0.0;green,0.0;blue,0.0}, draw opacity={0.0}, line width={1}, solid, fill={rgb,1:red,0.0;green,0.0;blue,0.0}, fill opacity={0.0}, text opacity={1.0}, font={{\fontsize{8 pt}{10.4 pt}\selectfont}}, text={rgb,1:red,0.0;green,0.0;blue,0.0}, cells={anchor={center}}, at={(1.02, 1)}, anchor={north west}}, axis background/.style={fill={rgb,1:red,0.0;green,0.0;blue,0.0}, opacity={0.0}}, anchor={north west}, xshift={0.0mm}, yshift={-0.0mm}, width={45.8mm}, height={50.8mm}, scaled x ticks={false}, xlabel={}, x tick style={color={rgb,1:red,0.0;green,0.0;blue,0.0}, opacity={1.0}}, x tick label style={color={rgb,1:red,0.0;green,0.0;blue,0.0}, opacity={1.0}, rotate={0}}, xlabel style={at={(ticklabel cs:0.5)}, anchor=near ticklabel, at={{(ticklabel cs:0.5)}}, anchor={near ticklabel}, font={{\fontsize{11 pt}{14.3 pt}\selectfont}}, color={rgb,1:red,0.0;green,0.0;blue,0.0}, draw opacity={1.0}, rotate={0.0}}, xmajorticks={false}, xmajorgrids={false}, xmin={-18.080000000000002}, xmax={20.080000000000002}, axis x line*={left}, separate axis lines, x axis line style={{draw opacity = 0}}, scaled y ticks={false}, ylabel={}, y tick style={color={rgb,1:red,0.0;green,0.0;blue,0.0}, opacity={1.0}}, y tick label style={color={rgb,1:red,0.0;green,0.0;blue,0.0}, opacity={1.0}, rotate={0}}, ylabel style={at={(ticklabel cs:0.5)}, anchor=near ticklabel, at={{(ticklabel cs:0.5)}}, anchor={near ticklabel}, font={{\fontsize{11 pt}{14.3 pt}\selectfont}}, color={rgb,1:red,0.0;green,0.0;blue,0.0}, draw opacity={1.0}, rotate={0.0}}, ymajorticks={false}, ymajorgrids={false}, ymin={-1.5416362697757904}, ymax={2.201795262699806}, axis y line*={left}, y axis line style={{draw opacity = 0}}, colorbar={false}]
    \addplot[color={rgb,1:red,0.0;green,0.0;blue,0.0}, name path={2cab6e4e-e498-4976-991f-c278d890d435}, draw opacity={1.0}, line width={0.25}, solid, forget plot]
        table[row sep={\\}]
        {
            \\
            -17.0  0.0  \\
            19.0  0.0  \\
        }
        ;
    \addplot[color={rgb,1:red,0.0;green,0.0;blue,0.0}, name path={fa9a4f84-1564-4c28-bc6d-addb4faba393}, draw opacity={1.0}, line width={0.25}, solid, forget plot]
        table[row sep={\\}]
        {
            \\
            0.0  -1.3350961730213144  \\
            0.0  1.5890088595252194  \\
        }
        ;
    \addplot[color={rgb,1:red,0.0;green,0.3608;blue,0.6706}, name path={bf2a32ed-db82-45fb-a157-136b106574bf}, draw opacity={1.0}, line width={1.0}, solid, forget plot]
        table[row sep={\\}]
        {
            \\
            -17.0  -1.3263761926191644  \\
            -16.9  -1.3284019384451278  \\
            -16.8  -1.3301543349855023  \\
            -16.7  -1.331636378557983  \\
            -16.6  -1.3328510654802654  \\
            -16.5  -1.3338013920700442  \\
            -16.4  -1.3344903546450155  \\
            -16.3  -1.3349209495228738  \\
            -16.2  -1.3350961730213144  \\
            -16.1  -1.335019021458033  \\
            -16.0  -1.334692491150725  \\
            -15.9  -1.3341195784170845  \\
            -15.8  -1.3333032795748079  \\
            -15.7  -1.3322465909415897  \\
            -15.6  -1.3309525088351257  \\
            -15.5  -1.3294240295731108  \\
            -15.4  -1.3276641494732404  \\
            -15.3  -1.3256758648532094  \\
            -15.2  -1.3234621720307134  \\
            -15.1  -1.3210260673234475  \\
            -15.0  -1.3183705470491074  \\
            -14.9  -1.3155885053659153  \\
            -14.8  -1.3127454998545929  \\
            -14.7  -1.3098035219665838  \\
            -14.6  -1.3067245631533313  \\
            -14.5  -1.3034706148662791  \\
            -14.4  -1.3000036685568699  \\
            -14.3  -1.296285715676548  \\
            -14.2  -1.292278747676756  \\
            -14.1  -1.287944756008938  \\
            -14.0  -1.2832457321245372  \\
            -13.9  -1.2781436674749966  \\
            -13.8  -1.2726005535117604  \\
            -13.7  -1.2665783816862708  \\
            -13.6  -1.2600391434499725  \\
            -13.5  -1.252944830254308  \\
            -13.4  -1.2452574335507216  \\
            -13.3  -1.236938944790656  \\
            -13.2  -1.2279513554255543  \\
            -13.1  -1.218256656906861  \\
            -13.0  -1.2078168406860186  \\
            -12.9  -1.1969545268352275  \\
            -12.8  -1.186018188207181  \\
            -12.7  -1.1750085954240634  \\
            -12.6  -1.1639265191080592  \\
            -12.5  -1.1527727298813533  \\
            -12.4  -1.1415479983661292  \\
            -12.3  -1.1302530951845715  \\
            -12.2  -1.1188887909588647  \\
            -12.1  -1.1074558563111927  \\
            -12.0  -1.0959550618637404  \\
            -11.9  -1.0843871782386918  \\
            -11.8  -1.072752976058231  \\
            -11.7  -1.0610532259445424  \\
            -11.6  -1.0492886985198109  \\
            -11.5  -1.0374601644062207  \\
            -11.4  -1.0255683942259552  \\
            -11.3  -1.0136141586011995  \\
            -11.2  -1.001598228154138  \\
            -11.1  -0.9895213735069548  \\
            -11.0  -0.9773843652818341  \\
            -10.9  -0.9652104621446128  \\
            -10.8  -0.9530235929458453  \\
            -10.7  -0.9408255335847938  \\
            -10.6  -0.9286180599607206  \\
            -10.5  -0.9164029479728876  \\
            -10.4  -0.9041819735205565  \\
            -10.3  -0.8919569125029898  \\
            -10.2  -0.8797295408194489  \\
            -10.1  -0.8675016343691964  \\
            -10.0  -0.855274969051494  \\
            -9.9  -0.8430513207656039  \\
            -9.8  -0.8308324654107879  \\
            -9.7  -0.818620178886308  \\
            -9.6  -0.8064162370914264  \\
            -9.5  -0.7942224159254052  \\
            -9.4  -0.7820404912875063  \\
            -9.3  -0.7698722390769915  \\
            -9.2  -0.757719435193123  \\
            -9.1  -0.7455838555351627  \\
            -9.0  -0.7334672760023729  \\
            -8.9  -0.7213476671605121  \\
            -8.8  -0.7092064757887554  \\
            -8.7  -0.6970506921064885  \\
            -8.6  -0.6848873063330984  \\
            -8.5  -0.6727233086879707  \\
            -8.4  -0.660565689390492  \\
            -8.3  -0.6484214386600484  \\
            -8.2  -0.6362975467160258  \\
            -8.1  -0.624201003777811  \\
            -8.0  -0.6121388000647897  \\
            -7.9  -0.6001179257963486  \\
            -7.8  -0.5881453711918734  \\
            -7.7  -0.5762281264707507  \\
            -7.6  -0.5643731818523667  \\
            -7.5  -0.5525875275561075  \\
            -7.4  -0.5408781538013594  \\
            -7.3  -0.5292520508075083  \\
            -7.2  -0.517716208793941  \\
            -7.1  -0.5062776179800431  \\
            -7.0  -0.49494326858520143  \\
            -6.9  -0.48371911033487036  \\
            -6.8  -0.4726022124991179  \\
            -6.7  -0.4615862446142507  \\
            -6.6  -0.45066487621657486  \\
            -6.5  -0.43983177684239705  \\
            -6.4  -0.4290806160280234  \\
            -6.3  -0.41840506330976024  \\
            -6.2  -0.4077987882239142  \\
            -6.1  -0.3972554603067914  \\
            -6.0  -0.3867687490946983  \\
            -5.9  -0.3763323241239414  \\
            -5.8  -0.36593985493082676  \\
            -5.7  -0.355585011051661  \\
            -5.6  -0.3452614620227504  \\
            -5.5  -0.3349628773804013  \\
            -5.4  -0.3246829266609202  \\
            -5.3  -0.31441527940061326  \\
            -5.2  -0.304153605135787  \\
            -5.1  -0.29389157340274763  \\
            -5.0  -0.28362285373780177  \\
            -4.9  -0.27334848544104173  \\
            -4.8  -0.2630738322468798  \\
            -4.7  -0.2527990503431024  \\
            -4.6  -0.2425242959174956  \\
            -4.5  -0.23224972515784578  \\
            -4.4  -0.2219754942519391  \\
            -4.3  -0.2117017593875617  \\
            -4.2  -0.20142867675249995  \\
            -4.1  -0.19115640253453994  \\
            -4.0  -0.1808850929214681  \\
            -3.9  -0.1706149041010705  \\
            -3.8  -0.16034599226113339  \\
            -3.7  -0.15007851358944305  \\
            -3.6  -0.13981262427378563  \\
            -3.5  -0.12954848050194748  \\
            -3.4  -0.11928623846171468  \\
            -3.3  -0.10902605434087356  \\
            -3.2  -0.09876808432721038  \\
            -3.1  -0.08851248460851127  \\
            -3.0  -0.07825941137256248  \\
            -2.9  -0.06800815075249923  \\
            -2.8  -0.05775790171309017  \\
            -2.7  -0.04750868968957163  \\
            -2.6  -0.03726054011717993  \\
            -2.5  -0.02701347843115147  \\
            -2.4  -0.016767530066722613  \\
            -2.3  -0.006522720459129719  \\
            -2.2  0.003720924956390808  \\
            -2.1  0.013963380744602669  \\
            -2.0  0.024204621470269486  \\
            -1.9  0.034444621698154876  \\
            -1.8  0.044683355993022464  \\
            -1.7  0.054920798919635916  \\
            -1.6  0.06515692504275884  \\
            -1.5  0.07539170892715487  \\
            -1.4  0.0856251251375877  \\
            -1.3  0.09585714823882088  \\
            -1.2  0.10608775279561815  \\
            -1.1  0.11631691337274302  \\
            -1.0  0.12654460453495925  \\
            -0.9  0.13675114698078158  \\
            -0.8  0.1469191062590442  \\
            -0.7  0.15705182420998975  \\
            -0.6  0.1671526426738609  \\
            -0.5  0.17722490349090023  \\
            -0.4  0.18727194850135045  \\
            -0.3  0.1972971195454542  \\
            -0.2  0.20730375846345414  \\
            -0.1  0.21729520709559283  \\
            0.0  0.22727480728211297  \\
            0.1  0.2372459008632572  \\
            0.2  0.24721182967926816  \\
            0.3  0.25717593557038854  \\
            0.4  0.26714156037686093  \\
            0.5  0.27711204593892796  \\
            0.6  0.28709073409683233  \\
            0.7  0.2970809666908167  \\
            0.8  0.3070860855611236  \\
            0.9  0.31710943254799584  \\
            1.0  0.32715434949167593  \\
            1.1  0.3372311618514844  \\
            1.2  0.34734784498727106  \\
            1.3  0.357504215590073  \\
            1.4  0.36770009035092716  \\
            1.5  0.37793528596087045  \\
            1.6  0.38820961911093976  \\
            1.7  0.39852290649217215  \\
            1.8  0.40887496479560453  \\
            1.9  0.4192656107122737  \\
            2.0  0.4296946609332169  \\
            2.1  0.4401619321494708  \\
            2.2  0.45066724105207245  \\
            2.3  0.46121040433205873  \\
            2.4  0.4717912386804668  \\
            2.5  0.48240956078833336  \\
            2.6  0.4930651873466955  \\
            2.7  0.50375793504659  \\
            2.8  0.5144876205790542  \\
            2.9  0.5252540606351245  \\
            3.0  0.5360570719058383  \\
            3.1  0.5472176917030886  \\
            3.2  0.5590142298647537  \\
            3.3  0.5713824118708475  \\
            3.4  0.5842579632013845  \\
            3.5  0.5975766093363787  \\
            3.6  0.6112740757558444  \\
            3.7  0.6252860879397959  \\
            3.8  0.6395483713682468  \\
            3.9  0.653996651521212  \\
            4.0  0.6685666538787055  \\
            4.1  0.6831941039207411  \\
            4.2  0.6978147271273336  \\
            4.3  0.7123642489784966  \\
            4.4  0.7267783949542447  \\
            4.5  0.7409928905345917  \\
            4.6  0.7549434611995522  \\
            4.7  0.7685658324291402  \\
            4.8  0.7817957297033699  \\
            4.9  0.7945688785022555  \\
            5.0  0.806821004305811  \\
            5.1  0.818768964189323  \\
            5.2  0.8306621191887592  \\
            5.3  0.8424849507251556  \\
            5.4  0.8542219402195486  \\
            5.5  0.8658575690929744  \\
            5.6  0.8773763187664694  \\
            5.7  0.8887626706610697  \\
            5.8  0.9000011061978116  \\
            5.9  0.9110761067977312  \\
            6.0  0.9219721538818648  \\
            6.1  0.9326737288712489  \\
            6.2  0.9431653131869197  \\
            6.3  0.9534313882499131  \\
            6.4  0.9634564354812657  \\
            6.5  0.9732249363020133  \\
            6.6  0.9827213721331928  \\
            6.7  0.9919302243958399  \\
            6.8  1.0008359745109914  \\
            6.9  1.0094231038996828  \\
            7.0  1.017676093982951  \\
            7.1  1.0257363947197207  \\
            7.2  1.0337563825947784  \\
            7.3  1.041736928817951  \\
            7.4  1.0496789045990664  \\
            7.5  1.057583181147952  \\
            7.6  1.0654506296744348  \\
            7.7  1.0732821213883428  \\
            7.8  1.081078527499503  \\
            7.9  1.0888407192177432  \\
            8.0  1.0965695677528906  \\
            8.1  1.104265944314773  \\
            8.2  1.1119307201132174  \\
            8.3  1.1195647663580517  \\
            8.4  1.127168954259103  \\
            8.5  1.1347441550261987  \\
            8.6  1.1422912398691667  \\
            8.7  1.1498110799978338  \\
            8.8  1.1573045466220284  \\
            8.9  1.1647725109515767  \\
            9.0  1.1722158441963073  \\
            9.1  1.179650461534768  \\
            9.2  1.1870895824612524  \\
            9.3  1.1945300346592058  \\
            9.4  1.2019686458120726  \\
            9.5  1.2094022436032983  \\
            9.6  1.216827655716328  \\
            9.7  1.224241709834606  \\
            9.8  1.2316412336415778  \\
            9.9  1.2390230548206882  \\
            10.0  1.2463840010553824  \\
            10.1  1.2537209000291054  \\
            10.2  1.2610305794253018  \\
            10.3  1.2683098669274173  \\
            10.4  1.2755555902188962  \\
            10.5  1.2827645769831837  \\
            10.6  1.2899336549037248  \\
            10.7  1.2970596516639645  \\
            10.8  1.304139394947348  \\
            10.9  1.3111697124373198  \\
            11.0  1.3181474318173256  \\
            11.1  1.325026174277794  \\
            11.2  1.3317668927068351  \\
            11.3  1.338377412334413  \\
            11.4  1.3448655583904925  \\
            11.5  1.351239156105038  \\
            11.6  1.3575060307080147  \\
            11.7  1.3636740074293865  \\
            11.8  1.3697509114991193  \\
            11.9  1.3757445681471765  \\
            12.0  1.3816628026035231  \\
            12.1  1.3875134400981242  \\
            12.2  1.3933043058609442  \\
            12.3  1.399043225121948  \\
            12.4  1.4047380231110997  \\
            12.5  1.4103965250583645  \\
            12.6  1.4160265561937067  \\
            12.7  1.4216359417470912  \\
            12.8  1.4272325069484826  \\
            12.9  1.4328240770278458  \\
            13.0  1.438418477215145  \\
            13.1  1.44397194969458  \\
            13.2  1.449437093844378  \\
            13.3  1.454816270685545  \\
            13.4  1.4601118412390868  \\
            13.5  1.4653261665260098  \\
            13.6  1.4704616075673198  \\
            13.7  1.475520525384023  \\
            13.8  1.4805052809971249  \\
            13.9  1.485418235427632  \\
            14.0  1.49026174969655  \\
            14.1  1.4950381848248855  \\
            14.2  1.499749901833644  \\
            14.3  1.5043992617438318  \\
            14.4  1.5089886255764546  \\
            14.5  1.5135203543525189  \\
            14.6  1.51799680909303  \\
            14.7  1.5224203508189946  \\
            14.8  1.5267933405514187  \\
            14.9  1.5311181393113076  \\
            15.0  1.5353971081196682  \\
            15.1  1.539551204323609  \\
            15.2  1.5435032907074437  \\
            15.3  1.5472585864479875  \\
            15.4  1.5508223107220547  \\
            15.5  1.554199682706459  \\
            15.6  1.557395921578015  \\
            15.7  1.5604162465135376  \\
            15.8  1.5632658766898406  \\
            15.9  1.5659500312837387  \\
            16.0  1.5684739294720458  \\
            16.1  1.5708427904315765  \\
            16.2  1.573061833339145  \\
            16.3  1.5751362773715663  \\
            16.4  1.5770713417056537  \\
            16.5  1.5788722455182223  \\
            16.6  1.580544207986086  \\
            16.7  1.5820924482860599  \\
            16.8  1.5835221855949575  \\
            16.9  1.5848386390895934  \\
            17.0  1.5860470279467822  \\
            17.1  1.5870871836027531  \\
            17.2  1.587898652660948  \\
            17.3  1.5884862270490012  \\
            17.4  1.588854698694547  \\
            17.5  1.5890088595252194  \\
            17.6  1.5889535014686522  \\
            17.7  1.5886934164524797  \\
            17.8  1.5882333964043356  \\
            17.9  1.5875782332518544  \\
            18.0  1.58673271892267  \\
            18.1  1.5857016453444164  \\
            18.2  1.5844898044447273  \\
            18.3  1.5831019881512376  \\
            18.4  1.5815429883915808  \\
            18.5  1.5798175970933912  \\
            18.6  1.5779306061843026  \\
            18.7  1.575886807591949  \\
            18.8  1.5736909932439649  \\
            18.9  1.571347955067984  \\
            19.0  1.5688624849916406  \\
        }
        ;
    \addplot[color={rgb,1:red,0.7529;green,0.3255;blue,0.4039}, name path={459b75dc-3534-46fa-8962-fb009290776d}, draw opacity={1.0}, line width={1.0}, solid, forget plot]
        table[row sep={\\}]
        {
            \\
            -17.0  -1.531567303710771  \\
            -16.9  -1.5339064335066306  \\
            -16.8  -1.535929926735255  \\
            -16.7  -1.5376412432463  \\
            -16.6  -1.5390438428894218  \\
            -16.5  -1.5401411855142755  \\
            -16.4  -1.540936730970518  \\
            -16.3  -1.5414339391078042  \\
            -16.2  -1.5416362697757904  \\
            -16.1  -1.5415471828241325  \\
            -16.0  -1.5411701381024867  \\
            -15.9  -1.5405085954605078  \\
            -15.8  -1.5395660147478525  \\
            -15.7  -1.5383458558141763  \\
            -15.6  -1.5368515785091355  \\
            -15.5  -1.5350866426823853  \\
            -15.4  -1.533054508183582  \\
            -15.3  -1.530758634862381  \\
            -15.2  -1.5282024825684386  \\
            -15.1  -1.5253895111514104  \\
            -15.0  -1.522323180460953  \\
            -14.9  -1.5191107554315775  \\
            -14.8  -1.5158279354370383  \\
            -14.7  -1.5124308319858544  \\
            -14.6  -1.5088755565865442  \\
            -14.5  -1.5051182207476268  \\
            -14.4  -1.50111493597762  \\
            -14.3  -1.4968218137850435  \\
            -14.2  -1.4921949656784155  \\
            -14.1  -1.4871905031662547  \\
            -14.0  -1.48176453775708  \\
            -13.9  -1.47587318095941  \\
            -13.8  -1.4694725442817633  \\
            -13.7  -1.4625187392326582  \\
            -13.6  -1.4549678773206143  \\
            -13.5  -1.4467760700541497  \\
            -13.4  -1.4378994289417835  \\
            -13.3  -1.428294065492034  \\
            -13.2  -1.4179160912134194  \\
            -13.1  -1.4067216176144597  \\
            -13.0  -1.3946667562036725  \\
            -12.9  -1.3821240365521197  \\
            -12.8  -1.3694958404504165  \\
            -12.7  -1.3567830577364144  \\
            -12.6  -1.3439865782479643  \\
            -12.5  -1.3311072918229183  \\
            -12.4  -1.3181460882991265  \\
            -12.3  -1.3051038575144402  \\
            -12.2  -1.2919814893067112  \\
            -12.1  -1.2787798735137894  \\
            -12.0  -1.2654998999735272  \\
            -11.9  -1.252142458523775  \\
            -11.8  -1.238708439002384  \\
            -11.7  -1.225198731247205  \\
            -11.6  -1.21161422509609  \\
            -11.5  -1.19795581038689  \\
            -11.4  -1.1842243769574552  \\
            -11.3  -1.1704208146456372  \\
            -11.2  -1.1565460132892877  \\
            -11.1  -1.1426008627262574  \\
            -11.0  -1.128586252794397  \\
            -10.9  -1.1145290402876706  \\
            -10.8  -1.1004568558626964  \\
            -10.7  -1.0863717501513082  \\
            -10.6  -1.0722757737853403  \\
            -10.5  -1.0581709773966266  \\
            -10.4  -1.044059411617001  \\
            -10.3  -1.0299431270782975  \\
            -10.2  -1.0158241744123495  \\
            -10.1  -1.0017046042509918  \\
            -10.0  -0.987586467226058  \\
            -9.9  -0.973471813969382  \\
            -9.8  -0.9593626951127978  \\
            -9.7  -0.9452611612881392  \\
            -9.6  -0.9311692631272404  \\
            -9.5  -0.9170890512619354  \\
            -9.4  -0.903022576324058  \\
            -9.3  -0.8889718889454421  \\
            -9.2  -0.8749390397579216  \\
            -9.1  -0.8609260793933305  \\
            -9.0  -0.8469350584835033  \\
            -8.9  -0.8329405396288606  \\
            -8.8  -0.8189210994153275  \\
            -8.7  -0.8048848094529922  \\
            -8.6  -0.7908397413519441  \\
            -8.5  -0.7767939667222713  \\
            -8.4  -0.7627555571740626  \\
            -8.3  -0.7487325843174069  \\
            -8.2  -0.734733119762392  \\
            -8.1  -0.7207652351191076  \\
            -8.0  -0.7068370019976418  \\
            -7.9  -0.6929564920080835  \\
            -7.8  -0.679131776760521  \\
            -7.7  -0.6653709278650434  \\
            -7.6  -0.6516820169317392  \\
            -7.5  -0.6380731155706969  \\
            -7.4  -0.6245522953920054  \\
            -7.3  -0.6111276280057528  \\
            -7.2  -0.5978071850220287  \\
            -7.1  -0.584599038050921  \\
            -7.0  -0.5715112587025186  \\
            -6.9  -0.5585507171280074  \\
            -6.8  -0.5457140292119569  \\
            -6.7  -0.5329938851645323  \\
            -6.6  -0.5203829751958978  \\
            -6.5  -0.5078739895162188  \\
            -6.4  -0.4954596183356596  \\
            -6.3  -0.48313255186438503  \\
            -6.2  -0.47088548031256017  \\
            -6.1  -0.4587110938903495  \\
            -6.0  -0.44660208280791786  \\
            -5.9  -0.4345511372754302  \\
            -5.8  -0.4225509475030509  \\
            -5.7  -0.4105942037009451  \\
            -5.6  -0.39867359607927744  \\
            -5.5  -0.38678181484821267  \\
            -5.4  -0.3749115502179156  \\
            -5.3  -0.36305549239855095  \\
            -5.2  -0.3512063316002836  \\
            -5.1  -0.339356758033278  \\
            -5.0  -0.32749946190769946  \\
            -4.9  -0.3156356433039239  \\
            -4.8  -0.30377149572896506  \\
            -4.7  -0.2919071995329439  \\
            -4.6  -0.2800429350659811  \\
            -4.5  -0.2681788826781978  \\
            -4.4  -0.2563152227197146  \\
            -4.3  -0.2444521355406523  \\
            -4.2  -0.23258980149113195  \\
            -4.1  -0.22072840092127421  \\
            -4.0  -0.2088681141812002  \\
            -3.9  -0.1970091216210305  \\
            -3.8  -0.18515160359088603  \\
            -3.7  -0.17329574044088772  \\
            -3.6  -0.1614417125211563  \\
            -3.5  -0.14958970018181275  \\
            -3.4  -0.13773988377297774  \\
            -3.3  -0.12589244364477226  \\
            -3.2  -0.11404756014731715  \\
            -3.1  -0.1022054136307332  \\
            -3.0  -0.09036618444514123  \\
            -2.9  -0.07852904828808817  \\
            -2.8  -0.06669308020376112  \\
            -2.7  -0.054858309562241164  \\
            -2.6  -0.04302476573360937  \\
            -2.5  -0.031192478087946906  \\
            -2.4  -0.01936147599533489  \\
            -2.3  -0.007531788825854446  \\
            -2.2  0.00429655405041326  \\
            -2.1  0.016123523263387177  \\
            -2.0  0.02794908944298617  \\
            -1.9  0.039773223219129086  \\
            -1.8  0.0515958952217348  \\
            -1.7  0.06341707608072221  \\
            -1.6  0.07523673642601018  \\
            -1.5  0.08705484688751756  \\
            -1.4  0.09887137809516332  \\
            -1.3  0.11068630067886621  \\
            -1.2  0.12249958526854524  \\
            -1.1  0.13431120249411915  \\
            -1.0  0.14612112298550695  \\
            -0.9  0.15790662304268868  \\
            -0.8  0.1696475710955168  \\
            -0.7  0.18134782596871876  \\
            -0.6  0.19301124648702186  \\
            -0.5  0.2046416914751534  \\
            -0.4  0.21624301975784085  \\
            -0.3  0.22781909015981155  \\
            -0.2  0.23937376150579284  \\
            -0.1  0.250910892620512  \\
            0.0  0.2624343423286965  \\
            0.1  0.27394796945507366  \\
            0.2  0.2854556328243708  \\
            0.3  0.29696119126131537  \\
            0.4  0.3084685035906347  \\
            0.5  0.31998142863705603  \\
            0.6  0.33150382522530686  \\
            0.7  0.34303955218011456  \\
            0.8  0.3545924683262064  \\
            0.9  0.36616643248830977  \\
            1.0  0.37776530349115206  \\
            1.1  0.3894010041481696  \\
            1.2  0.40108274361167473  \\
            1.3  0.4128103102147094  \\
            1.4  0.42458349229031506  \\
            1.5  0.43640207817153354  \\
            1.6  0.44826585619140635  \\
            1.7  0.4601746146829754  \\
            1.8  0.47212814197928216  \\
            1.9  0.48412622641336817  \\
            2.0  0.49616865631827556  \\
            2.1  0.5082552200270456  \\
            2.2  0.52038570587272  \\
            2.3  0.5325599021883406  \\
            2.4  0.5447775973069491  \\
            2.5  0.5570385795615869  \\
            2.6  0.5693426372852959  \\
            2.7  0.5816895588111176  \\
            2.8  0.594079132472094  \\
            2.9  0.6065111466012664  \\
            3.0  0.6189853895316767  \\
            3.1  0.6318725632202078  \\
            3.2  0.6454940321864938  \\
            3.3  0.659775578607703  \\
            3.4  0.6746429846610037  \\
            3.5  0.6900220325235642  \\
            3.6  0.7058385043725529  \\
            3.7  0.7220181823851385  \\
            3.8  0.7384868487384882  \\
            3.9  0.7551702856097714  \\
            4.0  0.771994275176156  \\
            4.1  0.7888845996148102  \\
            4.2  0.8057670411029028  \\
            4.3  0.8225673818176014  \\
            4.4  0.8392114039360747  \\
            4.5  0.855624889635491  \\
            4.6  0.8717336210930187  \\
            4.7  0.887463380485826  \\
            4.8  0.9027399499910811  \\
            4.9  0.9174891117859527  \\
            5.0  0.9316366480476086  \\
            5.1  0.9454329637576335  \\
            5.2  0.9591659962385104  \\
            5.3  0.9728178261787546  \\
            5.4  0.9863705342668817  \\
            5.5  0.9998062011914076  \\
            5.6  1.013106907640848  \\
            5.7  1.0262547343037187  \\
            5.8  1.0392317618685352  \\
            5.9  1.0520200710238128  \\
            6.0  1.0646017424580676  \\
            6.1  1.0769588568598154  \\
            6.2  1.0890734949175715  \\
            6.3  1.100927737319852  \\
            6.4  1.1125036647551723  \\
            6.5  1.1237833579120478  \\
            6.6  1.1347488974789945  \\
            6.7  1.1453823641445282  \\
            6.8  1.1556658385971648  \\
            6.9  1.1655814015254191  \\
            7.0  1.1751111336178075  \\
            7.1  1.1844183672180542  \\
            7.2  1.1936790515351783  \\
            7.3  1.2028941925556362  \\
            7.4  1.2120647962658853  \\
            7.5  1.2211918686523817  \\
            7.6  1.2302764157015826  \\
            7.7  1.2393194433999448  \\
            7.8  1.2483219577339246  \\
            7.9  1.2572849646899797  \\
            8.0  1.266209470254566  \\
            8.1  1.2750964804141414  \\
            8.2  1.283947001155161  \\
            8.3  1.2927620384640834  \\
            8.4  1.3015425983273643  \\
            8.5  1.3102896867314606  \\
            8.6  1.3190043096628297  \\
            8.7  1.3276874731079276  \\
            8.8  1.336340183053212  \\
            8.9  1.3449634454851385  \\
            9.0  1.353558266390165  \\
            9.1  1.3621430230335294  \\
            9.2  1.3707329799724093  \\
            9.3  1.3793244741311712  \\
            9.4  1.3879138424341804  \\
            9.5  1.3964974218058035  \\
            9.6  1.4050715491704067  \\
            9.7  1.413632561452355  \\
            9.8  1.4221767955760156  \\
            9.9  1.4307005884657538  \\
            10.0  1.4392002770459358  \\
            10.1  1.447672198240928  \\
            10.2  1.4561126889750957  \\
            10.3  1.4645180861728058  \\
            10.4  1.4728847267584235  \\
            10.5  1.4812089476563153  \\
            10.6  1.4894870857908469  \\
            10.7  1.4977154780863844  \\
            10.8  1.5058904614672943  \\
            10.9  1.5140083728579419  \\
            11.0  1.5220655491826938  \\
            11.1  1.530008436805169  \\
            11.2  1.5377919480042457  \\
            11.3  1.5454251185771763  \\
            11.4  1.5529169843212147  \\
            11.5  1.5602765810336132  \\
            11.6  1.5675129445116256  \\
            11.7  1.5746351105525043  \\
            11.8  1.5816521149535039  \\
            11.9  1.5885729935118758  \\
            12.0  1.5954067820248738  \\
            12.1  1.6021625162897515  \\
            12.2  1.608849232103762  \\
            12.3  1.615475965264158  \\
            12.4  1.6220517515681925  \\
            12.5  1.6285856268131194  \\
            12.6  1.6350866267961908  \\
            12.7  1.6415637873146607  \\
            12.8  1.648026144165782  \\
            12.9  1.654482733146808  \\
            13.0  1.6609425900549912  \\
            13.1  1.6673551877168693  \\
            13.2  1.673665792608961  \\
            13.3  1.679877131003494  \\
            13.4  1.685991929172694  \\
            13.5  1.6920129133887885  \\
            13.6  1.6979428099240041  \\
            13.7  1.7037843450505676  \\
            13.8  1.7095402450407053  \\
            13.9  1.7152132361666446  \\
            14.0  1.720806044700612  \\
            14.1  1.7263213969148345  \\
            14.2  1.7317620190815386  \\
            14.3  1.7371306374729514  \\
            14.4  1.7424299783612993  \\
            14.5  1.7476627680188093  \\
            14.6  1.7528317327177079  \\
            14.7  1.7579395987302224  \\
            14.8  1.7629890923285794  \\
            14.9  1.7679829397850049  \\
            15.0  1.772923867371727  \\
            15.1  1.7777206044948968  \\
            15.2  1.782284080770032  \\
            15.3  1.7866203227834112  \\
            15.4  1.7907353571213116  \\
            15.5  1.7946352103700103  \\
            15.6  1.7983259091157848  \\
            15.7  1.801813479944913  \\
            15.8  1.8051039494436718  \\
            15.9  1.808203344198339  \\
            16.0  1.8111176907951918  \\
            16.1  1.8138530158205073  \\
            16.2  1.8164153458605634  \\
            16.3  1.8188107075016378  \\
            16.4  1.821045127330007  \\
            16.5  1.8231246319319492  \\
            16.6  1.8250552478937416  \\
            16.7  1.8268430018016617  \\
            16.8  1.8284939202419868  \\
            16.9  1.8300140298009941  \\
            17.0  1.8314093570649614  \\
            17.1  1.8326104253609092  \\
            17.2  1.833547429119285  \\
            17.3  1.834225901581508  \\
            17.4  1.8346513759889973  \\
            17.5  1.8348293855831717  \\
            17.6  1.8347654636054498  \\
            17.7  1.834465143297251  \\
            17.8  1.8339339578999938  \\
            17.9  1.8331774406550976  \\
            18.0  1.832201124803981  \\
            18.1  1.8310105435880628  \\
            18.2  1.829611230248762  \\
            18.3  1.8280087180274978  \\
            18.4  1.826208540165689  \\
            18.5  1.8242162299047546  \\
            18.6  1.822037320486113  \\
            18.7  1.8196773451511836  \\
            18.8  1.8171418371413857  \\
            18.9  1.8144363296981374  \\
            19.0  1.811566356062858  \\
        }
        ;
    \node[right, , color={rgb,1:red,0.0;green,0.3608;blue,0.6706}, draw opacity={1.0}, rotate={0.0}, font={{\fontsize{6 pt}{7.800000000000001 pt}\selectfont}}]  at (axis cs:5.8,0.7945044297626097) {Nominal};
    \node[left, , color={rgb,1:red,0.7529;green,0.3255;blue,0.4039}, draw opacity={1.0}, rotate={0.0}, font={{\fontsize{6 pt}{7.800000000000001 pt}\selectfont}}]  at (axis cs:12.6,1.7890088595252194) {Corrected};
\end{axis}
\end{tikzpicture}

    \caption{Example curves demonstrating the changes to the lift coefficient vs angle of attack for the nominal polar when the Prandtl-Glauert correction applied.}
	\label{fig:prandtlglauert-correction}
\end{marginfigure}

\begin{equation}
    \label{eqn:prandtlglauertlift}
    c_{\ell_\text{pg}} = \frac{c_{\ell_\text{ss}}}{\left[1-M^2\right]^{1/2}}
\end{equation}

\where the Mach number is defined as

\begin{equation}
    M = \frac{W}{V_s}
\end{equation}

\where \(W\) is the local inflow velocity magnitude and \(V_s\) is the local speed of sound (which we assume to be the freestream speed of sound).
%
\Cref{fig:prandtlglauert-correction} shows an example application of the Pradtl-Glauert correction applied to an arbitrary set of airfoil data for a Mach number of 0.5.

\begin{marginfigure}
	% Recommended preamble:
% \usetikzlibrary{arrows.meta}
% \usetikzlibrary{backgrounds}
% \usepgfplotslibrary{patchplots}
% \usepgfplotslibrary{fillbetween}
% \pgfplotsset{%
%     layers/standard/.define layer set={%
%         background,axis background,axis grid,axis ticks,axis lines,axis tick labels,pre main,main,axis descriptions,axis foreground%
%     }{
%         grid style={/pgfplots/on layer=axis grid},%
%         tick style={/pgfplots/on layer=axis ticks},%
%         axis line style={/pgfplots/on layer=axis lines},%
%         label style={/pgfplots/on layer=axis descriptions},%
%         legend style={/pgfplots/on layer=axis descriptions},%
%         title style={/pgfplots/on layer=axis descriptions},%
%         colorbar style={/pgfplots/on layer=axis descriptions},%
%         ticklabel style={/pgfplots/on layer=axis tick labels},%
%         axis background@ style={/pgfplots/on layer=axis background},%
%         3d box foreground style={/pgfplots/on layer=axis foreground},%
%     },
% }

\begin{tikzpicture}[/tikz/background rectangle/.style={fill={rgb,1:red,1.0;green,1.0;blue,1.0}, fill opacity={1.0}, draw opacity={1.0}}, show background rectangle]
\begin{axis}[point meta max={nan}, point meta min={nan}, legend cell align={left}, legend columns={1}, title={}, title style={at={{(0.5,1)}}, anchor={south}, font={{\fontsize{14 pt}{18.2 pt}\selectfont}}, color={rgb,1:red,0.0;green,0.0;blue,0.0}, draw opacity={1.0}, rotate={0.0}, align={center}}, legend style={color={rgb,1:red,0.0;green,0.0;blue,0.0}, draw opacity={0.0}, line width={1}, solid, fill={rgb,1:red,0.0;green,0.0;blue,0.0}, fill opacity={0.0}, text opacity={1.0}, font={{\fontsize{8 pt}{10.4 pt}\selectfont}}, text={rgb,1:red,0.0;green,0.0;blue,0.0}, cells={anchor={center}}, at={(1.02, 1)}, anchor={north west}}, axis background/.style={fill={rgb,1:red,0.0;green,0.0;blue,0.0}, opacity={0.0}}, anchor={north west}, xshift={0.0mm}, yshift={-0.0mm}, width={45.8mm}, height={50.8mm}, scaled x ticks={false}, xlabel={Mach number}, x tick style={color={rgb,1:red,0.0;green,0.0;blue,0.0}, opacity={1.0}}, x tick label style={color={rgb,1:red,0.0;green,0.0;blue,0.0}, opacity={1.0}, rotate={0}}, xlabel style={at={(ticklabel cs:0.5)}, anchor=near ticklabel, at={{(ticklabel cs:0.5)}}, anchor={near ticklabel}, font={{\fontsize{11 pt}{14.3 pt}\selectfont}}, color={rgb,1:red,0.0;green,0.0;blue,0.0}, draw opacity={1.0}, rotate={0.0}}, xmajorgrids={false}, xmin={-0.04500000000000004}, xmax={1.545}, xticklabels={{$0.0$,$0.5$,$1.0$,$1.5$}}, xtick={{0.0,0.5,1.0,1.5}}, xtick align={inside}, xticklabel style={font={{\fontsize{8 pt}{10.4 pt}\selectfont}}, color={rgb,1:red,0.0;green,0.0;blue,0.0}, draw opacity={1.0}, rotate={0.0}}, x grid style={color={rgb,1:red,0.0;green,0.0;blue,0.0}, draw opacity={0.1}, line width={0.5}, solid}, axis x line*={left}, x axis line style={color={rgb,1:red,0.0;green,0.0;blue,0.0}, draw opacity={1.0}, line width={1}, solid}, scaled y ticks={false}, ylabel={$c_{\ell_\text{pg}}$}, y tick style={color={rgb,1:red,0.0;green,0.0;blue,0.0}, opacity={1.0}}, y tick label style={color={rgb,1:red,0.0;green,0.0;blue,0.0}, opacity={1.0}, rotate={0}}, ylabel style={at={(ticklabel cs:0.5)}, anchor=near ticklabel, at={{(ticklabel cs:0.5)}}, anchor={near ticklabel}, font={{\fontsize{11 pt}{14.3 pt}\selectfont}}, color={rgb,1:red,0.0;green,0.0;blue,0.0}, draw opacity={1.0}, rotate={0.0}}, ymajorgrids={false}, ymin={0.8173356384974997}, ymax={7.271476411585855}, yticklabels={{$1$,$2$,$3$,$4$,$5$,$6$,$7$}}, ytick={{1.0,2.0,3.0,4.0,5.0,6.0,7.0}}, ytick align={inside}, yticklabel style={font={{\fontsize{8 pt}{10.4 pt}\selectfont}}, color={rgb,1:red,0.0;green,0.0;blue,0.0}, draw opacity={1.0}, rotate={0.0}}, y grid style={color={rgb,1:red,0.0;green,0.0;blue,0.0}, draw opacity={0.1}, line width={0.5}, solid}, axis y line*={left}, y axis line style={color={rgb,1:red,0.0;green,0.0;blue,0.0}, draw opacity={1.0}, line width={1}, solid}, colorbar={false}]
    \addplot[color={rgb,1:red,0.0;green,0.3608;blue,0.6706}, name path={d890538f-559b-460f-80a1-85c30887408f}, draw opacity={1.0}, line width={1.0}, solid, forget plot]
        table[row sep={\\}]
        {
            \\
            0.0  1.0  \\
            0.015151515151515152  1.0001148039725956  \\
            0.030303030303030304  1.0004594532748736  \\
            0.045454545454545456  1.0010346614252381  \\
            0.06060606060606061  1.0018416221862119  \\
            0.07575757575757576  1.0028820164664152  \\
            0.09090909090909091  1.0041580220928046  \\
            0.10606060606060606  1.005672326565096  \\
            0.12121212121212122  1.0074281429398948  \\
            0.13636363636363635  1.0094292290304718  \\
            0.15151515151515152  1.0116799101501937  \\
            0.16666666666666666  1.01418510567422  \\
            0.18181818181818182  1.0169503597462535  \\
            0.19696969696969696  1.0199818765160997  \\
            0.21212121212121213  1.0232865603609698  \\
            0.22727272727272727  1.0268720616205405  \\
            0.24242424242424243  1.0307468284648  \\
            0.25757575757575757  1.0349201656170806  \\
            0.2727272727272727  1.0394023007753748  \\
            0.2878787878787879  1.0442044597166376  \\
            0.30303030303030304  1.049338951235674  \\
            0.3181818181818182  1.0548192632678453  \\
            0.3333333333333333  1.0606601717798212  \\
            0.3484848484848485  1.0668778642933074  \\
            0.36363636363636365  1.0734900802433864  \\
            0.3787878787878788  1.080516270778858  \\
            0.3939393939393939  1.0879777811030487  \\
            0.4090909090909091  1.0958980590507794  \\
            0.42424242424242425  1.1043028943269202  \\
            0.4393939393939394  1.113220693728123  \\
            0.45454545454545453  1.1226827987756234  \\
            0.4696969696969697  1.1327238535601487  \\
            0.48484848484848486  1.1433822323141412  \\
            0.5  1.1547005383792517  \\
            0.5151515151515151  1.1667261889578033  \\
            0.5303030303030303  1.1795121034985545  \\
            0.5454545454545454  1.1931175180026088  \\
            0.5606060606060606  1.2076089532614942  \\
            0.5757575757575758  1.223061372490817  \\
            0.5909090909090909  1.2395595736018243  \\
            0.6060606060606061  1.257199874303108  \\
            0.6212121212121212  1.276092165540276  \\
            0.6363636363636364  1.296362432175337  \\
            0.6515151515151515  1.3181558717669293  \\
            0.6666666666666666  1.3416407864998738  \\
            0.6818181818181818  1.36701348520264  \\
            0.696969696969697  1.3945045203020525  \\
            0.7121212121212122  1.424386711357996  \\
            0.7272727272727273  1.4569855927715483  \\
            0.7424242424242424  1.492693201005272  \\
            0.7575757575757576  1.5319865399606778  \\
            0.7727272727272727  1.575452722886752  \\
            0.7878787878787878  1.6238238433069057  \\
            0.803030303030303  1.6780263593970406  \\
            0.8181818181818182  1.739252713092609  \\
            0.8333333333333334  1.8090680674665818  \\
            0.8484848484848485  1.8895745032357651  \\
            0.8636363636363636  1.9836731962683514  \\
            0.8787878787878788  2.095502095503143  \\
            0.8939393939393939  2.2312072324832832  \\
            0.9090909090909091  2.4003967925959153  \\
            0.9242424242424242  2.619130107437398  \\
            0.9393939393939394  2.9168154723945094  \\
            0.9545454545454546  3.354968547317305  \\
            0.9696969696969697  4.093146241443883  \\
            0.9848484848484849  5.7664467739628495  \\
            1.0  7.088812050083354  \\
            1.0151515151515151  7.088812050083354  \\
            1.0303030303030303  7.088812050083354  \\
            1.0454545454545454  7.088812050083354  \\
            1.0606060606060606  7.088812050083354  \\
            1.0757575757575757  7.088812050083354  \\
            1.0909090909090908  7.088812050083354  \\
            1.106060606060606  7.088812050083354  \\
            1.121212121212121  7.088812050083354  \\
            1.1363636363636365  7.088812050083354  \\
            1.1515151515151516  7.088812050083354  \\
            1.1666666666666667  7.088812050083354  \\
            1.1818181818181819  7.088812050083354  \\
            1.196969696969697  7.088812050083354  \\
            1.2121212121212122  7.088812050083354  \\
            1.2272727272727273  7.088812050083354  \\
            1.2424242424242424  7.088812050083354  \\
            1.2575757575757576  7.088812050083354  \\
            1.2727272727272727  7.088812050083354  \\
            1.2878787878787878  7.088812050083354  \\
            1.303030303030303  7.088812050083354  \\
            1.3181818181818181  7.088812050083354  \\
            1.3333333333333333  7.088812050083354  \\
            1.3484848484848484  7.088812050083354  \\
            1.3636363636363635  7.088812050083354  \\
            1.378787878787879  7.088812050083354  \\
            1.393939393939394  7.088812050083354  \\
            1.4090909090909092  7.088812050083354  \\
            1.4242424242424243  7.088812050083354  \\
            1.4393939393939394  7.088812050083354  \\
            1.4545454545454546  7.088812050083354  \\
            1.4696969696969697  7.088812050083354  \\
            1.4848484848484849  7.088812050083354  \\
            1.5  7.088812050083354  \\
        }
        ;
    \addplot[color={rgb,1:red,0.7529;green,0.3255;blue,0.4039}, name path={72119b9c-9c0e-4c13-8257-7438b4d2c36f}, draw opacity={1.0}, line width={2}, dashed, forget plot]
        table[row sep={\\}]
        {
            \\
            0.0  1.0  \\
            0.015151515151515152  1.0001148039725956  \\
            0.030303030303030304  1.0004594532748736  \\
            0.045454545454545456  1.0010346614252381  \\
            0.06060606060606061  1.0018416221862119  \\
            0.07575757575757576  1.0028820164664152  \\
            0.09090909090909091  1.0041580220928046  \\
            0.10606060606060606  1.005672326565096  \\
            0.12121212121212122  1.0074281429398948  \\
            0.13636363636363635  1.0094292290304718  \\
            0.15151515151515152  1.0116799101501937  \\
            0.16666666666666666  1.01418510567422  \\
            0.18181818181818182  1.0169503597462535  \\
            0.19696969696969696  1.0199818765160997  \\
            0.21212121212121213  1.0232865603609698  \\
            0.22727272727272727  1.0268720616205405  \\
            0.24242424242424243  1.0307468284648  \\
            0.25757575757575757  1.0349201656170806  \\
            0.2727272727272727  1.0394023007753748  \\
            0.2878787878787879  1.0442044597166376  \\
            0.30303030303030304  1.049338951235674  \\
            0.3181818181818182  1.0548192632678453  \\
            0.3333333333333333  1.0606601717798212  \\
            0.3484848484848485  1.0668778642933074  \\
            0.36363636363636365  1.0734900802433864  \\
            0.3787878787878788  1.080516270778858  \\
            0.3939393939393939  1.0879777811030487  \\
            0.4090909090909091  1.0958980590507794  \\
            0.42424242424242425  1.1043028943269202  \\
            0.4393939393939394  1.113220693728123  \\
            0.45454545454545453  1.1226827987756234  \\
            0.4696969696969697  1.1327238535601487  \\
            0.48484848484848486  1.1433822323141412  \\
            0.5  1.1547005383792517  \\
            0.5151515151515151  1.1667261889578033  \\
            0.5303030303030303  1.1795121034985545  \\
            0.5454545454545454  1.1931175180026088  \\
            0.5606060606060606  1.2076089532614942  \\
            0.5757575757575758  1.223061372490817  \\
            0.5909090909090909  1.2395595736018243  \\
            0.6060606060606061  1.257199874303108  \\
            0.6212121212121212  1.276092165540276  \\
            0.6363636363636364  1.296362432175337  \\
            0.6515151515151515  1.3181558717669293  \\
            0.6666666666666666  1.3416407864998738  \\
            0.6818181818181818  1.36701348520264  \\
            0.696969696969697  1.3945045203020525  \\
            0.7121212121212122  1.424386711357996  \\
            0.7272727272727273  1.4569855927715483  \\
            0.7424242424242424  1.492693201005272  \\
            0.7575757575757576  1.5319865399606778  \\
            0.7727272727272727  1.575452722886752  \\
            0.7878787878787878  1.6238238433069057  \\
            0.803030303030303  1.6780263593970406  \\
            0.8181818181818182  1.739252713092609  \\
            0.8333333333333334  1.8090680674665818  \\
            0.8484848484848485  1.8895745032357651  \\
            0.8636363636363636  1.9836731962683514  \\
            0.8787878787878788  2.095502095503143  \\
            0.8939393939393939  2.2312072324832832  \\
            0.9090909090909091  2.4003967925959153  \\
            0.9242424242424242  2.619130107437398  \\
            0.9393939393939394  2.9168154723945094  \\
            0.9545454545454546  3.354968547317305  \\
            0.9696969696969697  4.8804841869986735  \\
            0.9848484848484849  6.946590424580144  \\
            1.0  7.088812050083354  \\
            1.0151515151515151  7.088812050083354  \\
            1.0303030303030303  7.088812050083354  \\
            1.0454545454545454  7.088812050083354  \\
            1.0606060606060606  7.088812050083354  \\
            1.0757575757575757  7.088812050083354  \\
            1.0909090909090908  7.088812050083354  \\
            1.106060606060606  7.088812050083354  \\
            1.121212121212121  7.088812050083354  \\
            1.1363636363636365  7.088812050083354  \\
            1.1515151515151516  7.088812050083354  \\
            1.1666666666666667  7.088812050083354  \\
            1.1818181818181819  7.088812050083354  \\
            1.196969696969697  7.088812050083354  \\
            1.2121212121212122  7.088812050083354  \\
            1.2272727272727273  7.088812050083354  \\
            1.2424242424242424  7.088812050083354  \\
            1.2575757575757576  7.088812050083354  \\
            1.2727272727272727  7.088812050083354  \\
            1.2878787878787878  7.088812050083354  \\
            1.303030303030303  7.088812050083354  \\
            1.3181818181818181  7.088812050083354  \\
            1.3333333333333333  7.088812050083354  \\
            1.3484848484848484  7.088812050083354  \\
            1.3636363636363635  7.088812050083354  \\
            1.378787878787879  7.088812050083354  \\
            1.393939393939394  7.088812050083354  \\
            1.4090909090909092  7.088812050083354  \\
            1.4242424242424243  7.088812050083354  \\
            1.4393939393939394  7.088812050083354  \\
            1.4545454545454546  7.088812050083354  \\
            1.4696969696969697  7.088812050083354  \\
            1.4848484848484849  7.088812050083354  \\
            1.5  7.088812050083354  \\
        }
        ;
    \node[, above, color={rgb,1:red,0.0;green,0.3608;blue,0.6706}, draw opacity={1.0}, rotate={0.0}, font={{\fontsize{6 pt}{7.800000000000001 pt}\selectfont}}]  at (axis cs:0.5,6) {Nominal};
    \node[, below, color={rgb,1:red,0.7529;green,0.3255;blue,0.4039}, draw opacity={1.0}, rotate={0.0}, font={{\fontsize{6 pt}{7.800000000000001 pt}\selectfont}}]  at (axis cs:0.5,6) {Smoothed};
\end{axis}
\end{tikzpicture}

    \caption{Nominal (with cutoff Mach number) and smoothed versions of the Prandtl-Glauert correction compared for a nominal lift coefficient of 1.}
	\label{fig:prandtlglauert-smoothed-correction}
\end{marginfigure}

For implementation in a gradient-based optimization setting, we note that \cref{eqn:prandtlglauertlift} is only valid for Mach numbers less than 1.
%
At \(M=1\) we get infinity, and for \(M>1\) the output is not a real number.
%
In order to remedy these issues, we first set a limit on the Mach numbers that can be input, say \(M=0.999\), we then apply a quintic polynomial blend between \cref{eqn:prandtlglauertlift} and the output for the limit of \(M=0.999\) centered at \(M=0.975\) with an interpolation range of 0.02 on either side of the center point.
%
This provides a smooth transition to the cutoff value as well as avoids the possibility of \cref{eqn:prandtlglauertlift} being evaluated at or above \(M=1\).
%
Although, as may be seen in \cref{fig:prandtlglauert-smoothed-correction}, this adjustments causes a slight deviation from the nominal correction for high subsonic Mach numbers, the deviations are small and in ranges that we do not expect to operate frequently.


%---------------------------------#
%      Reynolds Corrections       #
%---------------------------------#
\subsection{Reynolds Number Drag Adjustments}

If we have airfoil data at one Reynolds number, but we need to know how the airfoil behaves at a slightly different Reynolds number, we can apply an adjustment to the drag coefficient based on similarity between flat plate skin friction drag.
%
The limitation here is that we assume that the flow regimes between the Reynolds numbers are similar, in that they have similar laminar vs turbulent behavior, so that we can cancel out unknown constants due to airfoil shape and flow regime to arrive at

\begin{marginfigure}
	% Recommended preamble:
% \usetikzlibrary{arrows.meta}
% \usetikzlibrary{backgrounds}
% \usepgfplotslibrary{patchplots}
% \usepgfplotslibrary{fillbetween}
% \pgfplotsset{%
%     layers/standard/.define layer set={%
%         background,axis background,axis grid,axis ticks,axis lines,axis tick labels,pre main,main,axis descriptions,axis foreground%
%     }{
%         grid style={/pgfplots/on layer=axis grid},%
%         tick style={/pgfplots/on layer=axis ticks},%
%         axis line style={/pgfplots/on layer=axis lines},%
%         label style={/pgfplots/on layer=axis descriptions},%
%         legend style={/pgfplots/on layer=axis descriptions},%
%         title style={/pgfplots/on layer=axis descriptions},%
%         colorbar style={/pgfplots/on layer=axis descriptions},%
%         ticklabel style={/pgfplots/on layer=axis tick labels},%
%         axis background@ style={/pgfplots/on layer=axis background},%
%         3d box foreground style={/pgfplots/on layer=axis foreground},%
%     },
% }

\begin{tikzpicture}[/tikz/background rectangle/.style={fill={rgb,1:red,1.0;green,1.0;blue,1.0}, fill opacity={1.0}, draw opacity={1.0}}, show background rectangle]
\begin{axis}[point meta max={nan}, point meta min={nan}, legend cell align={left}, legend columns={1}, title={}, title style={at={{(0.5,1)}}, anchor={south}, font={{\fontsize{14 pt}{18.2 pt}\selectfont}}, color={rgb,1:red,0.0;green,0.0;blue,0.0}, draw opacity={1.0}, rotate={0.0}, align={center}}, legend style={color={rgb,1:red,0.0;green,0.0;blue,0.0}, draw opacity={0.0}, line width={1}, solid, fill={rgb,1:red,0.0;green,0.0;blue,0.0}, fill opacity={0.0}, text opacity={1.0}, font={{\fontsize{8 pt}{10.4 pt}\selectfont}}, text={rgb,1:red,0.0;green,0.0;blue,0.0}, cells={anchor={center}}, at={(1.02, 1)}, anchor={north west}}, axis background/.style={fill={rgb,1:red,0.0;green,0.0;blue,0.0}, opacity={0.0}}, anchor={north west}, xshift={0.0mm}, yshift={-0.0mm}, width={45.8mm}, height={50.8mm}, scaled x ticks={false}, xlabel={}, x tick style={color={rgb,1:red,0.0;green,0.0;blue,0.0}, opacity={1.0}}, x tick label style={color={rgb,1:red,0.0;green,0.0;blue,0.0}, opacity={1.0}, rotate={0}}, xlabel style={at={(ticklabel cs:0.5)}, anchor=near ticklabel, at={{(ticklabel cs:0.5)}}, anchor={near ticklabel}, font={{\fontsize{11 pt}{14.3 pt}\selectfont}}, color={rgb,1:red,0.0;green,0.0;blue,0.0}, draw opacity={1.0}, rotate={0.0}}, xmajorticks={false}, xmajorgrids={false}, xmin={-18.080000000000002}, xmax={20.080000000000002}, axis x line*={left}, separate axis lines, x axis line style={{draw opacity = 0}}, scaled y ticks={false}, ylabel={}, y tick style={color={rgb,1:red,0.0;green,0.0;blue,0.0}, opacity={1.0}}, y tick label style={color={rgb,1:red,0.0;green,0.0;blue,0.0}, opacity={1.0}, rotate={0}}, ylabel style={at={(ticklabel cs:0.5)}, anchor=near ticklabel, at={{(ticklabel cs:0.5)}}, anchor={near ticklabel}, font={{\fontsize{11 pt}{14.3 pt}\selectfont}}, color={rgb,1:red,0.0;green,0.0;blue,0.0}, draw opacity={1.0}, rotate={0.0}}, ymajorticks={false}, ymajorgrids={false}, ymin={-0.0018372668804019568}, ymax={0.06307949622713371}, axis y line*={left}, y axis line style={{draw opacity = 0}}, colorbar={false}]
    \addplot[color={rgb,1:red,0.0;green,0.0;blue,0.0}, name path={fad1d026-ce77-4fb1-92d6-e7157c036095}, draw opacity={1.0}, line width={0.25}, solid, forget plot]
        table[row sep={\\}]
        {
            \\
            -17.0  0.0  \\
            19.0  0.0  \\
        }
        ;
    \addplot[color={rgb,1:red,0.0;green,0.0;blue,0.0}, name path={6f5691ad-e4db-484e-bc8e-9ad6ca9be59f}, draw opacity={1.0}, line width={0.25}, solid, forget plot]
        table[row sep={\\}]
        {
            \\
            0.0  0.0  \\
            0.0  0.06124222934673175  \\
        }
        ;
    \addplot[color={rgb,1:red,0.0;green,0.3608;blue,0.6706}, name path={bb79e361-34d3-4aa9-974d-211f829d6c53}, draw opacity={1.0}, line width={1.0}, solid, forget plot]
        table[row sep={\\}]
        {
            \\
            -17.0  0.03274774225137401  \\
            -16.9  0.032078105905770565  \\
            -16.8  0.0314152645136106  \\
            -16.7  0.0307604786412915  \\
            -16.6  0.030115008855210748  \\
            -16.5  0.029480115721765728  \\
            -16.4  0.0288570598073539  \\
            -16.3  0.02824710167837272  \\
            -16.2  0.027651501901219573  \\
            -16.1  0.02707152104229195  \\
            -16.0  0.026508419667987226  \\
            -15.9  0.025963458344702876  \\
            -15.8  0.02543789763883633  \\
            -15.7  0.024932998116785003  \\
            -15.6  0.02445002034494635  \\
            -15.5  0.023990224889717793  \\
            -15.4  0.023554872317496773  \\
            -15.3  0.023145223194680722  \\
            -15.2  0.02276253808766707  \\
            -15.1  0.02240807756285326  \\
            -15.0  0.02208310218663672  \\
            -14.9  0.021775799369370918  \\
            -14.8  0.021473570332794213  \\
            -14.7  0.021176496360381335  \\
            -14.6  0.020884658735607062  \\
            -14.5  0.02059813874194613  \\
            -14.4  0.020317017662873298  \\
            -14.3  0.020041376781863313  \\
            -14.2  0.01977129738239093  \\
            -14.1  0.019506860747930913  \\
            -14.0  0.019248148161958002  \\
            -13.9  0.018995240907946947  \\
            -13.8  0.018748220269372513  \\
            -13.7  0.01850716752970944  \\
            -13.6  0.01827216397243249  \\
            -13.5  0.018043290881016413  \\
            -13.4  0.017820629538935956  \\
            -13.3  0.017604261229665883  \\
            -13.2  0.017394267236680933  \\
            -13.1  0.017190728843455867  \\
            -13.0  0.016993727333465436  \\
            -12.9  0.016799744583548246  \\
            -12.8  0.016605353818553277  \\
            -12.7  0.016410773343970862  \\
            -12.6  0.016216221465291326  \\
            -12.5  0.016021916488005  \\
            -12.4  0.015828076717602207  \\
            -12.3  0.01563492045957328  \\
            -12.2  0.015442666019408535  \\
            -12.1  0.015251531702598313  \\
            -12.0  0.015061735814632937  \\
            -11.9  0.01487349666100273  \\
            -11.8  0.014687032547198026  \\
            -11.7  0.014502561778709145  \\
            -11.6  0.014320302661026421  \\
            -11.5  0.014140473499640182  \\
            -11.4  0.013963292600040752  \\
            -11.3  0.013788978267718457  \\
            -11.2  0.013617748808163627  \\
            -11.1  0.01344982252686659  \\
            -11.0  0.013285417729317676  \\
            -10.9  0.013123256442351834  \\
            -10.8  0.012961988738554107  \\
            -10.7  0.012801724992039944  \\
            -10.6  0.01264257557692481  \\
            -10.5  0.012484650867324159  \\
            -10.4  0.012328061237353444  \\
            -10.3  0.012172917061128125  \\
            -10.2  0.012019328712763655  \\
            -10.1  0.011867406566375494  \\
            -10.0  0.011717260996079096  \\
            -9.9  0.01156900237598992  \\
            -9.8  0.01142274108022342  \\
            -9.7  0.011278587482895047  \\
            -9.6  0.01113665195812027  \\
            -9.5  0.010997044880014536  \\
            -9.4  0.010859876622693302  \\
            -9.3  0.01072525756027203  \\
            -9.2  0.010593298066866166  \\
            -9.1  0.010464108516591176  \\
            -9.0  0.010337799283562514  \\
            -8.9  0.010213971815691194  \\
            -8.8  0.010092188424284365  \\
            -8.7  0.009972500778551687  \\
            -8.6  0.00985496054770282  \\
            -8.5  0.009739619400947424  \\
            -8.4  0.009626529007495154  \\
            -8.3  0.009515741036555675  \\
            -8.2  0.009407307157338637  \\
            -8.1  0.009301279039053707  \\
            -8.0  0.00919770835091054  \\
            -7.9  0.009096646762118799  \\
            -7.8  0.008998145941888138  \\
            -7.7  0.008902257559428221  \\
            -7.6  0.008809033283948702  \\
            -7.5  0.00871852478465924  \\
            -7.4  0.0086307837307695  \\
            -7.3  0.008545861791489135  \\
            -7.2  0.008463810636027806  \\
            -7.1  0.00838468193359517  \\
            -7.0  0.00830852735340089  \\
            -6.9  0.008234355627174653  \\
            -6.8  0.008161172967873453  \\
            -6.7  0.008089027266547895  \\
            -6.6  0.008017966414248583  \\
            -6.5  0.007948038302026126  \\
            -6.4  0.007879290820931133  \\
            -6.3  0.007811771862014209  \\
            -6.2  0.00774552931632596  \\
            -6.1  0.007680611074916994  \\
            -6.0  0.007617065028837918  \\
            -5.9  0.007554939069139339  \\
            -5.8  0.007494281086871864  \\
            -5.7  0.007435138973086099  \\
            -5.6  0.007377560618832652  \\
            -5.5  0.007321593915162128  \\
            -5.4  0.0072672867531251364  \\
            -5.3  0.007214687023772282  \\
            -5.2  0.007163842618154173  \\
            -5.1  0.007114801427321416  \\
            -5.0  0.007067611342324617  \\
            -4.9  0.0070212995895903435  \\
            -4.8  0.006974935693871966  \\
            -4.7  0.0069286309937103  \\
            -4.6  0.006882496827646157  \\
            -4.5  0.006836644534220352  \\
            -4.4  0.006791185451973701  \\
            -4.3  0.006746230919447015  \\
            -4.2  0.006701892275181112  \\
            -4.1  0.006658280857716803  \\
            -4.0  0.006615508005594903  \\
            -3.9  0.006573685057356227  \\
            -3.8  0.006532923351541588  \\
            -3.7  0.006493334226691801  \\
            -3.6  0.006455029021347679  \\
            -3.5  0.006418119074050036  \\
            -3.4  0.006382715723339688  \\
            -3.3  0.006348930307757448  \\
            -3.2  0.006316874165844132  \\
            -3.1  0.0062866586361405505  \\
            -3.0  0.006258395057187519  \\
            -2.9  0.0062312952293119015  \\
            -2.8  0.006204477971590604  \\
            -2.7  0.006177915150689514  \\
            -2.6  0.006151578633274512  \\
            -2.5  0.0061254402860114856  \\
            -2.4  0.006099471975566318  \\
            -2.3  0.006073645568604896  \\
            -2.2  0.0060479329317931  \\
            -2.1  0.006022305931796817  \\
            -2.0  0.005996736435281932  \\
            -1.9  0.005971196308914328  \\
            -1.8  0.0059456574193598915  \\
            -1.7  0.005920091633284505  \\
            -1.6  0.005894470817354054  \\
            -1.5  0.005868766838234422  \\
            -1.4  0.005842951562591497  \\
            -1.3  0.005816996857091157  \\
            -1.2  0.0057908745883992925  \\
            -1.1  0.005764556623181786  \\
            -1.0  0.005738014828104521  \\
            -0.9  0.005711186890429283  \\
            -0.8  0.005684049557190802  \\
            -0.7  0.005656633284714391  \\
            -0.6  0.005628968529325357  \\
            -0.5  0.005601085747349011  \\
            -0.4  0.005573015395110662  \\
            -0.3  0.005544787928935617  \\
            -0.2  0.005516433805149188  \\
            -0.1  0.005487983480076683  \\
            0.0  0.005459467410043413  \\
            0.1  0.005430916051374686  \\
            0.2  0.005402359860395811  \\
            0.3  0.005373829293432098  \\
            0.4  0.005345354806808856  \\
            0.5  0.005316966856851396  \\
            0.6  0.005288695899885024  \\
            0.7  0.0052605723922350515  \\
            0.8  0.00523262679022679  \\
            0.9  0.005204889550185544  \\
            1.0  0.005177391128436626  \\
            1.1  0.005150751995126851  \\
            1.2  0.0051257346245434775  \\
            1.3  0.005102582479222478  \\
            1.4  0.005081539021699825  \\
            1.5  0.005062847714511489  \\
            1.6  0.005046752020193444  \\
            1.7  0.005033495401281663  \\
            1.8  0.005023321320312118  \\
            1.9  0.00501647323982078  \\
            2.0  0.005013194622343622  \\
            2.1  0.005013728930416618  \\
            2.2  0.005018319626575739  \\
            2.3  0.005027210173356958  \\
            2.4  0.0050406440332962455  \\
            2.5  0.005058864668929577  \\
            2.6  0.0050821155427929225  \\
            2.7  0.005110640117422256  \\
            2.8  0.0051446818553535474  \\
            2.9  0.0051844842191227725  \\
            3.0  0.005230290671265901  \\
            3.1  0.0052823010756966925  \\
            3.2  0.005340369027180724  \\
            3.3  0.005404218584531699  \\
            3.4  0.0054735738065633185  \\
            3.5  0.005548158752089284  \\
            3.6  0.005627697479923298  \\
            3.7  0.005711914048879063  \\
            3.8  0.0058005325177702795  \\
            3.9  0.00589327694541065  \\
            4.0  0.005989871390613878  \\
            4.1  0.006090039912193662  \\
            4.2  0.006193506568963708  \\
            4.3  0.006299995419737717  \\
            4.4  0.006409230523329389  \\
            4.5  0.006520935938552425  \\
            4.6  0.006634835724220529  \\
            4.7  0.0067506539391474045  \\
            4.8  0.006868114642146751  \\
            4.9  0.0069869418920322736  \\
            5.0  0.007106859747617668  \\
            5.1  0.007227914763315581  \\
            5.2  0.007350227880041849  \\
            5.3  0.007473634736364967  \\
            5.4  0.007597970970853431  \\
            5.5  0.007723072222075736  \\
            5.6  0.007848774128600378  \\
            5.7  0.007974912328995856  \\
            5.8  0.00810132246183066  \\
            5.9  0.00822784016567329  \\
            6.0  0.00835430107909224  \\
            6.1  0.008480540840656003  \\
            6.2  0.00860639508893308  \\
            6.3  0.008731699462491964  \\
            6.4  0.008856289599901151  \\
            6.5  0.008980001139729138  \\
            6.6  0.009102669720544417  \\
            6.7  0.00922413098091549  \\
            6.8  0.009344220559410846  \\
            6.9  0.009462774094598985  \\
            7.0  0.0095796272250484  \\
            7.1  0.009695121805801218  \\
            7.2  0.009809767555493974  \\
            7.3  0.009923651908086778  \\
            7.4  0.010036862297539738  \\
            7.5  0.010149486157812967  \\
            7.6  0.010261610922866576  \\
            7.7  0.010373324026660675  \\
            7.8  0.010484712903155372  \\
            7.9  0.01059586498631078  \\
            8.0  0.010706867710087008  \\
            8.1  0.010817808508444169  \\
            8.2  0.01092877481534237  \\
            8.3  0.01103985406474173  \\
            8.4  0.011151133690602345  \\
            8.5  0.011262701126884338  \\
            8.6  0.01137464380754781  \\
            8.7  0.011487049166552878  \\
            8.8  0.011600004637859656  \\
            8.9  0.011713597655428244  \\
            9.0  0.01182791565321876  \\
            9.1  0.01194349340864211  \\
            9.2  0.0120607446751984  \\
            9.3  0.012179575350981542  \\
            9.4  0.01229989133408544  \\
            9.5  0.012421598522603999  \\
            9.6  0.012544602814631132  \\
            9.7  0.012668810108260741  \\
            9.8  0.012794126301586743  \\
            9.9  0.012920457292703035  \\
            10.0  0.013047708979703528  \\
            10.1  0.013175787260682132  \\
            10.2  0.013304598033732752  \\
            10.3  0.0134340471969493  \\
            10.4  0.013564040648425675  \\
            10.5  0.013694484286255791  \\
            10.6  0.013825284008533554  \\
            10.7  0.013956345713352872  \\
            10.8  0.014087575298807654  \\
            10.9  0.014218878662991801  \\
            11.0  0.014350161703999229  \\
            11.1  0.014485361106380894  \\
            11.2  0.014628176014337205  \\
            11.3  0.014778156015436212  \\
            11.4  0.014934850697245977  \\
            11.5  0.015097809647334563  \\
            11.6  0.015266582453270025  \\
            11.7  0.015440718702620426  \\
            11.8  0.01561976798295383  \\
            11.9  0.015803279881838286  \\
            12.0  0.015990803986841867  \\
            12.1  0.016181889885532617  \\
            12.2  0.01637608716547861  \\
            12.3  0.016572945414247902  \\
            12.4  0.016772014219408545  \\
            12.5  0.016972843168528613  \\
            12.6  0.01717498184917615  \\
            12.7  0.01737797984891922  \\
            12.8  0.017581386755325897  \\
            12.9  0.01778475215596422  \\
            13.0  0.017987625638402265  \\
            13.1  0.01819853488415434  \\
            13.2  0.018425739152365075  \\
            13.3  0.018668385397047998  \\
            13.4  0.01892562057221663  \\
            13.5  0.019196591631884495  \\
            13.6  0.019480445530065123  \\
            13.7  0.01977632922077204  \\
            13.8  0.020083389658018775  \\
            13.9  0.020400773795818845  \\
            14.0  0.020727628588185772  \\
            14.1  0.021063100989133096  \\
            14.2  0.021406337952674333  \\
            14.3  0.02175648643282302  \\
            14.4  0.022112693383592664  \\
            14.5  0.022474105758996797  \\
            14.6  0.02283987051304895  \\
            14.7  0.02320913459976265  \\
            14.8  0.023581044973151423  \\
            14.9  0.02395474858722878  \\
            15.0  0.024329392396008263  \\
            15.1  0.02471917093040247  \\
            15.2  0.025138357414664334  \\
            15.3  0.025586216842817883  \\
            15.4  0.026062014208887128  \\
            15.5  0.026565014506896086  \\
            15.6  0.027094482730868775  \\
            15.7  0.027649683874829224  \\
            15.8  0.02822988293280145  \\
            15.9  0.02883434489880946  \\
            16.0  0.029462334766877277  \\
            16.1  0.030113117531028934  \\
            16.2  0.030785958185288422  \\
            16.3  0.031480121723679796  \\
            16.4  0.03219487314022703  \\
            16.5  0.03292947742895419  \\
            16.6  0.03368319958388527  \\
            16.7  0.03445530459904428  \\
            16.8  0.035245057468455264  \\
            16.9  0.0360517231861422  \\
            17.0  0.03687456674612916  \\
            17.1  0.03772901534643799  \\
            17.2  0.03863006485462826  \\
            17.3  0.03957633326903071  \\
            17.4  0.04056643858797592  \\
            17.5  0.0415989988097946  \\
            17.6  0.04267263193281738  \\
            17.7  0.043785955955374864  \\
            17.8  0.04493758887579779  \\
            17.9  0.046126148692416734  \\
            18.0  0.04735025340356243  \\
            18.1  0.04860852100756548  \\
            18.2  0.049899569502756494  \\
            18.3  0.051222016887466235  \\
            18.4  0.052574481160025235  \\
            18.5  0.05395558031876427  \\
            18.6  0.05536393236201392  \\
            18.7  0.0567981552881048  \\
            18.8  0.05825686709536767  \\
            18.9  0.059738685782133066  \\
            19.0  0.06124222934673175  \\
        }
        ;
    \addplot[color={rgb,1:red,0.7529;green,0.3255;blue,0.4039}, name path={e6530695-eeb3-4c87-a96f-50b636e03f8c}, draw opacity={1.0}, line width={1.0}, solid, forget plot]
        table[row sep={\\}]
        {
            \\
            -17.0  0.020711490748494436  \\
            -16.9  0.020287975537266724  \\
            -16.8  0.019868757831934248  \\
            -16.7  0.01945463488468854  \\
            -16.6  0.019046403947721176  \\
            -16.5  0.018644862273223678  \\
            -16.4  0.018250807113387616  \\
            -16.3  0.017865035720404557  \\
            -16.2  0.017488345346466025  \\
            -16.1  0.017121533243763608  \\
            -16.0  0.01676539666448882  \\
            -15.9  0.01642073286083324  \\
            -15.8  0.01608833908498842  \\
            -15.7  0.015769012589145898  \\
            -15.6  0.015463550625497242  \\
            -15.5  0.015172750446234  \\
            -15.4  0.014897409303547725  \\
            -15.3  0.014638324449629972  \\
            -15.2  0.014396293136672287  \\
            -15.1  0.014172112616866232  \\
            -15.0  0.013966580142403357  \\
            -14.9  0.013772224775614067  \\
            -14.8  0.013581078349489923  \\
            -14.7  0.013393192272214178  \\
            -14.6  0.013208617951970121  \\
            -14.5  0.013027406796941012  \\
            -14.4  0.012849610215310121  \\
            -14.3  0.01267527961526072  \\
            -14.2  0.01250446640497608  \\
            -14.1  0.012337221992639475  \\
            -14.0  0.012173597786434169  \\
            -13.9  0.012013645194543432  \\
            -13.8  0.011857415625150539  \\
            -13.7  0.011704960486438756  \\
            -13.6  0.011556331186591355  \\
            -13.5  0.011411579133791608  \\
            -13.4  0.01127075573622278  \\
            -13.3  0.011133912402068149  \\
            -13.2  0.011001100539510977  \\
            -13.1  0.010872371556734537  \\
            -13.0  0.010747776861922102  \\
            -12.9  0.010625091398617871  \\
            -12.8  0.010502147883920544  \\
            -12.7  0.010379084386345157  \\
            -12.6  0.010256038974406741  \\
            -12.5  0.010133149716620327  \\
            -12.4  0.010010554681500942  \\
            -12.3  0.009888391937563622  \\
            -12.2  0.009766799553323392  \\
            -12.1  0.009645915597295291  \\
            -12.0  0.009525878137994345  \\
            -11.9  0.009406825243935584  \\
            -11.8  0.009288894983634041  \\
            -11.7  0.009172225425604745  \\
            -11.6  0.00905695463836273  \\
            -11.5  0.008943220690423026  \\
            -11.4  0.008831161650300663  \\
            -11.3  0.00872091558651067  \\
            -11.2  0.008612620567568082  \\
            -11.1  0.008506414661987928  \\
            -11.0  0.008402435938285241  \\
            -10.9  0.008299876135261994  \\
            -10.8  0.008197881483856754  \\
            -10.7  0.008096521790789428  \\
            -10.6  0.007995866862779937  \\
            -10.5  0.007895986506548193  \\
            -10.4  0.007796950528814109  \\
            -10.3  0.007698828736297599  \\
            -10.2  0.007601690935718574  \\
            -10.1  0.007505606933796952  \\
            -10.0  0.007410646537252644  \\
            -9.9  0.007316879552805565  \\
            -9.8  0.007224375787175628  \\
            -9.7  0.007133205047082744  \\
            -9.6  0.007043437139246833  \\
            -9.5  0.006955141870387805  \\
            -9.4  0.0068683890472255715  \\
            -9.3  0.006783248476480051  \\
            -9.2  0.006699789964871151  \\
            -9.1  0.006618083319118791  \\
            -9.0  0.006538198345942883  \\
            -8.9  0.006459882978869944  \\
            -8.8  0.006382860399264873  \\
            -8.7  0.006307163285605154  \\
            -8.6  0.006232824316368273  \\
            -8.5  0.006159876170031715  \\
            -8.4  0.006088351525072961  \\
            -8.3  0.006018283059969502  \\
            -8.2  0.005949703453198815  \\
            -8.1  0.00588264538323839  \\
            -8.0  0.00581714152856571  \\
            -7.9  0.00575322456765826  \\
            -7.8  0.005690927178993524  \\
            -7.7  0.005630282041048988  \\
            -7.6  0.005571321832302136  \\
            -7.5  0.00551407923123045  \\
            -7.4  0.005458586916311418  \\
            -7.3  0.005404877566022523  \\
            -7.2  0.00535298385884125  \\
            -7.1  0.005302938473245083  \\
            -7.0  0.0052547740877115084  \\
            -6.9  0.005207863769139238  \\
            -6.8  0.005161578991415258  \\
            -6.7  0.00511595004349946  \\
            -6.6  0.005071007214351732  \\
            -6.5  0.005026780792931967  \\
            -6.4  0.004983301068200058  \\
            -6.3  0.004940598329115895  \\
            -6.2  0.004898702864639369  \\
            -6.1  0.004857644963730371  \\
            -6.0  0.004817454915348792  \\
            -5.9  0.004778163008454525  \\
            -5.8  0.004739799532007459  \\
            -5.7  0.004702394774967487  \\
            -5.6  0.0046659790262945  \\
            -5.5  0.004630582574948388  \\
            -5.4  0.004596235709889043  \\
            -5.3  0.004562968720076356  \\
            -5.2  0.004530811894470219  \\
            -5.1  0.004499795522030522  \\
            -5.0  0.0044699498917171576  \\
            -4.9  0.004440659767502191  \\
            -4.8  0.004411336665168471  \\
            -4.7  0.004382051001392064  \\
            -4.6  0.004352873192849036  \\
            -4.5  0.004323873656215455  \\
            -4.4  0.0042951228081673865  \\
            -4.3  0.004266691065380896  \\
            -4.2  0.004238648844532053  \\
            -4.1  0.00421106656229692  \\
            -4.0  0.004184014635351566  \\
            -3.9  0.004157563480372058  \\
            -3.8  0.00413178351403446  \\
            -3.7  0.004106745153014841  \\
            -3.6  0.004082518813989264  \\
            -3.5  0.004059174913633798  \\
            -3.4  0.004036783868624511  \\
            -3.3  0.004015416095637467  \\
            -3.2  0.003995142011348733  \\
            -3.1  0.003976032032434375  \\
            -3.0  0.00395815657557046  \\
            -2.9  0.003941017139513365  \\
            -2.8  0.003924056416513558  \\
            -2.7  0.003907256613488243  \\
            -2.6  0.0038905999373546243  \\
            -2.5  0.003874068595029906  \\
            -2.4  0.003857644793431292  \\
            -2.3  0.003841310739475987  \\
            -2.2  0.0038250486400811944  \\
            -2.1  0.003808840702164118  \\
            -2.0  0.003792669132641963  \\
            -1.9  0.0037765161384319327  \\
            -1.8  0.0037603639264512326  \\
            -1.7  0.003744194703617065  \\
            -1.6  0.0037279906768466343  \\
            -1.5  0.0037117340530571454  \\
            -1.4  0.003695407039165803  \\
            -1.3  0.0036789918420898084  \\
            -1.2  0.0036624706687463683  \\
            -1.1  0.003645825726052687  \\
            -1.0  0.003629039220925966  \\
            -0.9  0.0036120717433302074  \\
            -0.8  0.0035949085867988885  \\
            -0.7  0.00357756901360344  \\
            -0.6  0.0035600722860152866  \\
            -0.5  0.003542437666305858  \\
            -0.4  0.00352468441674658  \\
            -0.3  0.0035068317996088793  \\
            -0.2  0.003488899077164185  \\
            -0.1  0.003470905511683923  \\
            0.0  0.0034528703654395213  \\
            0.1  0.003434812900702407  \\
            0.2  0.0034167523797440074  \\
            0.3  0.0033987080648357503  \\
            0.4  0.0033806992182490616  \\
            0.5  0.0033627451022553707  \\
            0.6  0.003344864979126103  \\
            0.7  0.0033270781111326866  \\
            0.8  0.00330940376054655  \\
            0.9  0.003291861189639118  \\
            1.0  0.00327446966068182  \\
            1.1  0.00325762159345147  \\
            1.2  0.003241799219029079  \\
            1.3  0.0032271565166423653  \\
            1.4  0.0032138474655190475  \\
            1.5  0.0032020260448868436  \\
            1.6  0.0031918462339734733  \\
            1.7  0.003183462012006655  \\
            1.8  0.0031770273582141075  \\
            1.9  0.003172696251823549  \\
            2.0  0.0031706226720626986  \\
            2.1  0.003170960598159275  \\
            2.2  0.0031738640093409965  \\
            2.3  0.003179486884835583  \\
            2.4  0.0031879832038707507  \\
            2.5  0.003199506945674221  \\
            2.6  0.003214212089473711  \\
            2.7  0.0032322526144969404  \\
            2.8  0.003253782499971627  \\
            2.9  0.0032789557251254897  \\
            3.0  0.003307926269186247  \\
            3.1  0.00334082053719181  \\
            3.2  0.003377545934341749  \\
            3.3  0.0034179279401062744  \\
            3.4  0.003461792033955596  \\
            3.5  0.0035089636953599236  \\
            3.6  0.0035592684037894667  \\
            3.7  0.003612531638714436  \\
            3.8  0.0036685788796050394  \\
            3.9  0.003727235605931489  \\
            4.0  0.003788327297163994  \\
            4.1  0.003851679432772763  \\
            4.2  0.003917117492228009  \\
            4.3  0.003984466954999939  \\
            4.4  0.004053553300558763  \\
            4.5  0.004124202008374692  \\
            4.6  0.004196238557917934  \\
            4.7  0.0042694884286587015  \\
            4.8  0.004343777100067203  \\
            4.9  0.00441893005161365  \\
            5.0  0.004494772762768248  \\
            5.1  0.004571334677126816  \\
            5.2  0.004648692284440625  \\
            5.3  0.004726741633413066  \\
            5.4  0.004805378772747532  \\
            5.5  0.004884499751147413  \\
            5.6  0.004964000617316102  \\
            5.7  0.005043777419956995  \\
            5.8  0.005123726207773478  \\
            5.9  0.005203743029468948  \\
            6.0  0.005283723933746795  \\
            6.1  0.005363564969310409  \\
            6.2  0.005443162184863187  \\
            6.3  0.005522411629108517  \\
            6.4  0.005601209350749793  \\
            6.5  0.005679451398490408  \\
            6.6  0.0057570338210337505  \\
            6.7  0.005833852667083219  \\
            6.8  0.005909803985342199  \\
            6.9  0.005984783824514087  \\
            7.0  0.006058688233302272  \\
            7.1  0.006131733419819302  \\
            7.2  0.006204241758436634  \\
            7.3  0.006276268547246026  \\
            7.4  0.006347869084339238  \\
            7.5  0.006419098667808029  \\
            7.6  0.00649001259574416  \\
            7.7  0.00656066616623939  \\
            7.8  0.006631114677385477  \\
            7.9  0.006701413427274182  \\
            8.0  0.006771617713997264  \\
            8.1  0.006841782835646483  \\
            8.2  0.0069119640903135965  \\
            8.3  0.006982216776090369  \\
            8.4  0.007052596191068554  \\
            8.5  0.007123157633339915  \\
            8.6  0.007193956400996207  \\
            8.7  0.007265047792129194  \\
            8.8  0.007336487104830636  \\
            8.9  0.0074083296371922885  \\
            9.0  0.007480630687305914  \\
            9.1  0.007553728478103447  \\
            9.2  0.007627884690274928  \\
            9.3  0.007703039808549276  \\
            9.4  0.007779134317655406  \\
            9.5  0.007856108702322234  \\
            9.6  0.00793390344727868  \\
            9.7  0.008012459037253658  \\
            9.8  0.00809171595697609  \\
            9.9  0.008171614691174885  \\
            10.0  0.008252095724578965  \\
            10.1  0.008333099541917247  \\
            10.2  0.008414566627918646  \\
            10.3  0.008496437467312082  \\
            10.4  0.008578652544826466  \\
            10.5  0.008661152345190722  \\
            10.6  0.00874387735313376  \\
            10.7  0.008826768053384502  \\
            10.8  0.008909764930671865  \\
            10.9  0.008992808469724762  \\
            11.0  0.009075839155272113  \\
            11.1  0.009161346765236044  \\
            11.2  0.009251670843829893  \\
            11.3  0.009346526525219378  \\
            11.4  0.00944562894357022  \\
            11.5  0.009548693233048145  \\
            11.6  0.009655434527818875  \\
            11.7  0.009765567962048131  \\
            11.8  0.009878808669901642  \\
            11.9  0.00999487178554512  \\
            12.0  0.010113472443144297  \\
            12.1  0.01023432577686489  \\
            12.2  0.010357146920872626  \\
            12.3  0.010481651009333225  \\
            12.4  0.010607553176412408  \\
            12.5  0.010734568556275905  \\
            12.6  0.010862412283089427  \\
            12.7  0.010990799491018704  \\
            12.8  0.011119445314229462  \\
            12.9  0.011248064886887416  \\
            13.0  0.011376373343158294  \\
            13.1  0.011509764062391242  \\
            13.2  0.011653460658722786  \\
            13.3  0.011806923618499696  \\
            13.4  0.01196961342806875  \\
            13.5  0.012140990573776718  \\
            13.6  0.01232051554197038  \\
            13.7  0.012507648818996511  \\
            13.8  0.012701850891201889  \\
            13.9  0.012902582244933281  \\
            14.0  0.013109303366537463  \\
            14.1  0.013321474742361217  \\
            14.2  0.013538556858751313  \\
            14.3  0.013760010202054534  \\
            14.4  0.013985295258617643  \\
            14.5  0.014213872514787418  \\
            14.6  0.014445202456910639  \\
            14.7  0.01467874557133408  \\
            14.8  0.014913962344404522  \\
            14.9  0.015150313262468724  \\
            15.0  0.015387258811873474  \\
            15.1  0.01563377640221907  \\
            15.2  0.015898893213144232  \\
            15.3  0.016182144386053383  \\
            15.4  0.016483065062350928  \\
            15.5  0.016801190383441283  \\
            15.6  0.017136055490728854  \\
            15.7  0.017487195525618067  \\
            15.8  0.017854145629513325  \\
            15.9  0.018236440943819043  \\
            16.0  0.018633616609939636  \\
            16.1  0.019045207769279517  \\
            16.2  0.019470749563243087  \\
            16.3  0.019909777133234784  \\
            16.4  0.020361825620658986  \\
            16.5  0.020826430166920144  \\
            16.6  0.02130312591342265  \\
            16.7  0.021791448001570908  \\
            16.8  0.022290931572769356  \\
            16.9  0.022801111768422373  \\
            17.0  0.023321523729934412  \\
            17.1  0.02386192447403816  \\
            17.2  0.024431798220129322  \\
            17.3  0.025030270913606885  \\
            17.4  0.02565646849986975  \\
            17.5  0.026309516924316892  \\
            17.6  0.026988542132347245  \\
            17.7  0.027692670069359708  \\
            17.8  0.028421026680753286  \\
            17.9  0.029172737911926868  \\
            18.0  0.02994692970827945  \\
            18.1  0.030742728015209934  \\
            18.2  0.031559258778117244  \\
            18.3  0.03239564794240039  \\
            18.4  0.03325102145345823  \\
            18.5  0.03412450525668979  \\
            18.6  0.035015225297493964  \\
            18.7  0.03592230752126966  \\
            18.8  0.03684487787341591  \\
            18.9  0.03778206229933156  \\
            19.0  0.03873298674441563  \\
        }
        ;
    \node[left, , color={rgb,1:red,0.0;green,0.3608;blue,0.6706}, draw opacity={1.0}, rotate={0.0}, font={{\fontsize{6 pt}{7.800000000000001 pt}\selectfont}}]  at (axis cs:19.0,0.05501319462234362) {Nominal};
    \node[left, , color={rgb,1:red,0.7529;green,0.3255;blue,0.4039}, draw opacity={1.0}, rotate={0.0}, font={{\fontsize{6 pt}{7.800000000000001 pt}\selectfont}}]  at (axis cs:19.0,0.0031706226720626986) {Adjusted};
\end{axis}
\end{tikzpicture}

    \caption{Example curves demonstrating the changes to the drag coefficient vs angle of attack for the nominal polar when the Reynolds number adjustment is applied for a slightly higher Reynolds number.}
	\label{fig:redrag-correction}
\end{marginfigure}

\begin{equation}
    c_{d_{Re}} = c_{d_o} \left(\frac{Re_o}{Re}\right)^p;
\end{equation}

\where \(Re\) is the local Reynolds number, \(Re_o\) is the Reynolds number at which the data was generated, and the exponent terms are defined, for example, as \(p=0.5\) for fully laminar flow and \(p=0.2\) for fully turbulent flow.
%
\Cref{fig:redrag-correction} shows an example of the Reynolds number drag adjustment for an arbitrary drag curve applied for use at a Reynolds number 2.5 times larger than the nominal case.
%
Note that we do not have to apply a similar correction to the lift coefficient, because within the constraint of similar flow regimes (that is, relatively small changes in Reynolds number), the lift does not actually change significantly.
%
Also note that in practice, it may be better to simply utilize an interpolation between data at various Reynolds numbers, especially if the laminar vs turbulent regime is not fully characterized a priori.



%---------------------------------#
%      Transonic Corrections      #
%---------------------------------#

%%%%% ----- Transonic Lift Adjustments ----- %%%%%

\subsection{Transonic Effects on Lift and Drag Coefficients}

Above a critical Mach number, often around 0.7, the Pradtl-Glauert correction begins to break down due to transonic effects over the airfoil.
%
If shock waves are present on the airfoil, we can expect a decrease in lift as early separation can occur.
%
For these high subsonic and transonic cases, we apply limiters to the maximum and minimum lift coefficients.
%
We choose to employ the method used in XROTOR\sidenote{\url{https://web.mit.edu/drela/Public/web/xrotor/}} and DFDC\sidenote{\url{https://web.mit.edu/drela/Public/web/dfdc/}}.
%
The lift curve limiter correction used in these codes takes the form:

\begin{equation}
    c_{\ell_\text{corr}} = c_{\ell_\text{pg}} - (1-f_\text{stall}) c_{\ell_\text{lim}},
\end{equation}

\where

\begin{marginfigure}
	% Recommended preamble:
% \usetikzlibrary{arrows.meta}
% \usetikzlibrary{backgrounds}
% \usepgfplotslibrary{patchplots}
% \usepgfplotslibrary{fillbetween}
% \pgfplotsset{%
%     layers/standard/.define layer set={%
%         background,axis background,axis grid,axis ticks,axis lines,axis tick labels,pre main,main,axis descriptions,axis foreground%
%     }{
%         grid style={/pgfplots/on layer=axis grid},%
%         tick style={/pgfplots/on layer=axis ticks},%
%         axis line style={/pgfplots/on layer=axis lines},%
%         label style={/pgfplots/on layer=axis descriptions},%
%         legend style={/pgfplots/on layer=axis descriptions},%
%         title style={/pgfplots/on layer=axis descriptions},%
%         colorbar style={/pgfplots/on layer=axis descriptions},%
%         ticklabel style={/pgfplots/on layer=axis tick labels},%
%         axis background@ style={/pgfplots/on layer=axis background},%
%         3d box foreground style={/pgfplots/on layer=axis foreground},%
%     },
% }

\begin{tikzpicture}[/tikz/background rectangle/.style={fill={rgb,1:red,1.0;green,1.0;blue,1.0}, fill opacity={1.0}, draw opacity={1.0}}, show background rectangle]
\begin{axis}[point meta max={nan}, point meta min={nan}, legend cell align={left}, legend columns={1}, title={}, title style={at={{(0.5,1)}}, anchor={south}, font={{\fontsize{14 pt}{18.2 pt}\selectfont}}, color={rgb,1:red,0.0;green,0.0;blue,0.0}, draw opacity={1.0}, rotate={0.0}, align={center}}, legend style={color={rgb,1:red,0.0;green,0.0;blue,0.0}, draw opacity={0.0}, line width={1}, solid, fill={rgb,1:red,0.0;green,0.0;blue,0.0}, fill opacity={0.0}, text opacity={1.0}, font={{\fontsize{8 pt}{10.4 pt}\selectfont}}, text={rgb,1:red,0.0;green,0.0;blue,0.0}, cells={anchor={center}}, at={(1.02, 1)}, anchor={north west}}, axis background/.style={fill={rgb,1:red,0.0;green,0.0;blue,0.0}, opacity={0.0}}, anchor={north west}, xshift={0.0mm}, yshift={-0.0mm}, width={45.8mm}, height={50.8mm}, scaled x ticks={false}, xlabel={}, x tick style={color={rgb,1:red,0.0;green,0.0;blue,0.0}, opacity={1.0}}, x tick label style={color={rgb,1:red,0.0;green,0.0;blue,0.0}, opacity={1.0}, rotate={0}}, xlabel style={at={(ticklabel cs:0.5)}, anchor=near ticklabel, at={{(ticklabel cs:0.5)}}, anchor={near ticklabel}, font={{\fontsize{11 pt}{14.3 pt}\selectfont}}, color={rgb,1:red,0.0;green,0.0;blue,0.0}, draw opacity={1.0}, rotate={0.0}}, xmajorticks={false}, xmajorgrids={false}, xmin={-18.080000000000002}, xmax={20.080000000000002}, axis x line*={left}, separate axis lines, x axis line style={{draw opacity = 0}}, scaled y ticks={false}, ylabel={}, y tick style={color={rgb,1:red,0.0;green,0.0;blue,0.0}, opacity={1.0}}, y tick label style={color={rgb,1:red,0.0;green,0.0;blue,0.0}, opacity={1.0}, rotate={0}}, ylabel style={at={(ticklabel cs:0.5)}, anchor=near ticklabel, at={{(ticklabel cs:0.5)}}, anchor={near ticklabel}, font={{\fontsize{11 pt}{14.3 pt}\selectfont}}, color={rgb,1:red,0.0;green,0.0;blue,0.0}, draw opacity={1.0}, rotate={0.0}}, ymajorticks={false}, ymajorgrids={false}, ymin={-1.4228193239977105}, ymax={1.6767320105016155}, axis y line*={left}, y axis line style={{draw opacity = 0}}, colorbar={false}]
    \addplot[color={rgb,1:red,0.0;green,0.0;blue,0.0}, name path={ba9c24f6-3f79-44a0-8c46-7cab81e48987}, draw opacity={1.0}, line width={0.25}, solid, forget plot]
        table[row sep={\\}]
        {
            \\
            -17.0  0.0  \\
            19.0  0.0  \\
        }
        ;
    \addplot[color={rgb,1:red,0.0;green,0.0;blue,0.0}, name path={c922f216-1d0f-4909-96fd-ded32bac61fc}, draw opacity={1.0}, line width={0.25}, solid, forget plot]
        table[row sep={\\}]
        {
            \\
            0.0  -1.3350961730213144  \\
            0.0  1.5890088595252194  \\
        }
        ;
    \addplot[color={rgb,1:red,0.0;green,0.3608;blue,0.6706}, name path={25786f64-6396-49b7-9a80-c5464f7a7db5}, draw opacity={1.0}, line width={1.0}, solid, forget plot]
        table[row sep={\\}]
        {
            \\
            -17.0  -1.3263761926191644  \\
            -16.9  -1.3284019384451278  \\
            -16.8  -1.3301543349855023  \\
            -16.7  -1.331636378557983  \\
            -16.6  -1.3328510654802654  \\
            -16.5  -1.3338013920700442  \\
            -16.4  -1.3344903546450155  \\
            -16.3  -1.3349209495228738  \\
            -16.2  -1.3350961730213144  \\
            -16.1  -1.335019021458033  \\
            -16.0  -1.334692491150725  \\
            -15.9  -1.3341195784170845  \\
            -15.8  -1.3333032795748079  \\
            -15.7  -1.3322465909415897  \\
            -15.6  -1.3309525088351257  \\
            -15.5  -1.3294240295731108  \\
            -15.4  -1.3276641494732404  \\
            -15.3  -1.3256758648532094  \\
            -15.2  -1.3234621720307134  \\
            -15.1  -1.3210260673234475  \\
            -15.0  -1.3183705470491074  \\
            -14.9  -1.3155885053659153  \\
            -14.8  -1.3127454998545929  \\
            -14.7  -1.3098035219665838  \\
            -14.6  -1.3067245631533313  \\
            -14.5  -1.3034706148662791  \\
            -14.4  -1.3000036685568699  \\
            -14.3  -1.296285715676548  \\
            -14.2  -1.292278747676756  \\
            -14.1  -1.287944756008938  \\
            -14.0  -1.2832457321245372  \\
            -13.9  -1.2781436674749966  \\
            -13.8  -1.2726005535117604  \\
            -13.7  -1.2665783816862708  \\
            -13.6  -1.2600391434499725  \\
            -13.5  -1.252944830254308  \\
            -13.4  -1.2452574335507216  \\
            -13.3  -1.236938944790656  \\
            -13.2  -1.2279513554255543  \\
            -13.1  -1.218256656906861  \\
            -13.0  -1.2078168406860186  \\
            -12.9  -1.1969545268352275  \\
            -12.8  -1.186018188207181  \\
            -12.7  -1.1750085954240634  \\
            -12.6  -1.1639265191080592  \\
            -12.5  -1.1527727298813533  \\
            -12.4  -1.1415479983661292  \\
            -12.3  -1.1302530951845715  \\
            -12.2  -1.1188887909588647  \\
            -12.1  -1.1074558563111927  \\
            -12.0  -1.0959550618637404  \\
            -11.9  -1.0843871782386918  \\
            -11.8  -1.072752976058231  \\
            -11.7  -1.0610532259445424  \\
            -11.6  -1.0492886985198109  \\
            -11.5  -1.0374601644062207  \\
            -11.4  -1.0255683942259552  \\
            -11.3  -1.0136141586011995  \\
            -11.2  -1.001598228154138  \\
            -11.1  -0.9895213735069548  \\
            -11.0  -0.9773843652818341  \\
            -10.9  -0.9652104621446128  \\
            -10.8  -0.9530235929458453  \\
            -10.7  -0.9408255335847938  \\
            -10.6  -0.9286180599607206  \\
            -10.5  -0.9164029479728876  \\
            -10.4  -0.9041819735205565  \\
            -10.3  -0.8919569125029898  \\
            -10.2  -0.8797295408194489  \\
            -10.1  -0.8675016343691964  \\
            -10.0  -0.855274969051494  \\
            -9.9  -0.8430513207656039  \\
            -9.8  -0.8308324654107879  \\
            -9.7  -0.818620178886308  \\
            -9.6  -0.8064162370914264  \\
            -9.5  -0.7942224159254052  \\
            -9.4  -0.7820404912875063  \\
            -9.3  -0.7698722390769915  \\
            -9.2  -0.757719435193123  \\
            -9.1  -0.7455838555351627  \\
            -9.0  -0.7334672760023729  \\
            -8.9  -0.7213476671605121  \\
            -8.8  -0.7092064757887554  \\
            -8.7  -0.6970506921064885  \\
            -8.6  -0.6848873063330984  \\
            -8.5  -0.6727233086879707  \\
            -8.4  -0.660565689390492  \\
            -8.3  -0.6484214386600484  \\
            -8.2  -0.6362975467160258  \\
            -8.1  -0.624201003777811  \\
            -8.0  -0.6121388000647897  \\
            -7.9  -0.6001179257963486  \\
            -7.8  -0.5881453711918734  \\
            -7.7  -0.5762281264707507  \\
            -7.6  -0.5643731818523667  \\
            -7.5  -0.5525875275561075  \\
            -7.4  -0.5408781538013594  \\
            -7.3  -0.5292520508075083  \\
            -7.2  -0.517716208793941  \\
            -7.1  -0.5062776179800431  \\
            -7.0  -0.49494326858520143  \\
            -6.9  -0.48371911033487036  \\
            -6.8  -0.4726022124991179  \\
            -6.7  -0.4615862446142507  \\
            -6.6  -0.45066487621657486  \\
            -6.5  -0.43983177684239705  \\
            -6.4  -0.4290806160280234  \\
            -6.3  -0.41840506330976024  \\
            -6.2  -0.4077987882239142  \\
            -6.1  -0.3972554603067914  \\
            -6.0  -0.3867687490946983  \\
            -5.9  -0.3763323241239414  \\
            -5.8  -0.36593985493082676  \\
            -5.7  -0.355585011051661  \\
            -5.6  -0.3452614620227504  \\
            -5.5  -0.3349628773804013  \\
            -5.4  -0.3246829266609202  \\
            -5.3  -0.31441527940061326  \\
            -5.2  -0.304153605135787  \\
            -5.1  -0.29389157340274763  \\
            -5.0  -0.28362285373780177  \\
            -4.9  -0.27334848544104173  \\
            -4.8  -0.2630738322468798  \\
            -4.7  -0.2527990503431024  \\
            -4.6  -0.2425242959174956  \\
            -4.5  -0.23224972515784578  \\
            -4.4  -0.2219754942519391  \\
            -4.3  -0.2117017593875617  \\
            -4.2  -0.20142867675249995  \\
            -4.1  -0.19115640253453994  \\
            -4.0  -0.1808850929214681  \\
            -3.9  -0.1706149041010705  \\
            -3.8  -0.16034599226113339  \\
            -3.7  -0.15007851358944305  \\
            -3.6  -0.13981262427378563  \\
            -3.5  -0.12954848050194748  \\
            -3.4  -0.11928623846171468  \\
            -3.3  -0.10902605434087356  \\
            -3.2  -0.09876808432721038  \\
            -3.1  -0.08851248460851127  \\
            -3.0  -0.07825941137256248  \\
            -2.9  -0.06800815075249923  \\
            -2.8  -0.05775790171309017  \\
            -2.7  -0.04750868968957163  \\
            -2.6  -0.03726054011717993  \\
            -2.5  -0.02701347843115147  \\
            -2.4  -0.016767530066722613  \\
            -2.3  -0.006522720459129719  \\
            -2.2  0.003720924956390808  \\
            -2.1  0.013963380744602669  \\
            -2.0  0.024204621470269486  \\
            -1.9  0.034444621698154876  \\
            -1.8  0.044683355993022464  \\
            -1.7  0.054920798919635916  \\
            -1.6  0.06515692504275884  \\
            -1.5  0.07539170892715487  \\
            -1.4  0.0856251251375877  \\
            -1.3  0.09585714823882088  \\
            -1.2  0.10608775279561815  \\
            -1.1  0.11631691337274302  \\
            -1.0  0.12654460453495925  \\
            -0.9  0.13675114698078158  \\
            -0.8  0.1469191062590442  \\
            -0.7  0.15705182420998975  \\
            -0.6  0.1671526426738609  \\
            -0.5  0.17722490349090023  \\
            -0.4  0.18727194850135045  \\
            -0.3  0.1972971195454542  \\
            -0.2  0.20730375846345414  \\
            -0.1  0.21729520709559283  \\
            0.0  0.22727480728211297  \\
            0.1  0.2372459008632572  \\
            0.2  0.24721182967926816  \\
            0.3  0.25717593557038854  \\
            0.4  0.26714156037686093  \\
            0.5  0.27711204593892796  \\
            0.6  0.28709073409683233  \\
            0.7  0.2970809666908167  \\
            0.8  0.3070860855611236  \\
            0.9  0.31710943254799584  \\
            1.0  0.32715434949167593  \\
            1.1  0.3372311618514844  \\
            1.2  0.34734784498727106  \\
            1.3  0.357504215590073  \\
            1.4  0.36770009035092716  \\
            1.5  0.37793528596087045  \\
            1.6  0.38820961911093976  \\
            1.7  0.39852290649217215  \\
            1.8  0.40887496479560453  \\
            1.9  0.4192656107122737  \\
            2.0  0.4296946609332169  \\
            2.1  0.4401619321494708  \\
            2.2  0.45066724105207245  \\
            2.3  0.46121040433205873  \\
            2.4  0.4717912386804668  \\
            2.5  0.48240956078833336  \\
            2.6  0.4930651873466955  \\
            2.7  0.50375793504659  \\
            2.8  0.5144876205790542  \\
            2.9  0.5252540606351245  \\
            3.0  0.5360570719058383  \\
            3.1  0.5472176917030886  \\
            3.2  0.5590142298647537  \\
            3.3  0.5713824118708475  \\
            3.4  0.5842579632013845  \\
            3.5  0.5975766093363787  \\
            3.6  0.6112740757558444  \\
            3.7  0.6252860879397959  \\
            3.8  0.6395483713682468  \\
            3.9  0.653996651521212  \\
            4.0  0.6685666538787055  \\
            4.1  0.6831941039207411  \\
            4.2  0.6978147271273336  \\
            4.3  0.7123642489784966  \\
            4.4  0.7267783949542447  \\
            4.5  0.7409928905345917  \\
            4.6  0.7549434611995522  \\
            4.7  0.7685658324291402  \\
            4.8  0.7817957297033699  \\
            4.9  0.7945688785022555  \\
            5.0  0.806821004305811  \\
            5.1  0.818768964189323  \\
            5.2  0.8306621191887592  \\
            5.3  0.8424849507251556  \\
            5.4  0.8542219402195486  \\
            5.5  0.8658575690929744  \\
            5.6  0.8773763187664694  \\
            5.7  0.8887626706610697  \\
            5.8  0.9000011061978116  \\
            5.9  0.9110761067977312  \\
            6.0  0.9219721538818648  \\
            6.1  0.9326737288712489  \\
            6.2  0.9431653131869197  \\
            6.3  0.9534313882499131  \\
            6.4  0.9634564354812657  \\
            6.5  0.9732249363020133  \\
            6.6  0.9827213721331928  \\
            6.7  0.9919302243958399  \\
            6.8  1.0008359745109914  \\
            6.9  1.0094231038996828  \\
            7.0  1.017676093982951  \\
            7.1  1.0257363947197207  \\
            7.2  1.0337563825947784  \\
            7.3  1.041736928817951  \\
            7.4  1.0496789045990664  \\
            7.5  1.057583181147952  \\
            7.6  1.0654506296744348  \\
            7.7  1.0732821213883428  \\
            7.8  1.081078527499503  \\
            7.9  1.0888407192177432  \\
            8.0  1.0965695677528906  \\
            8.1  1.104265944314773  \\
            8.2  1.1119307201132174  \\
            8.3  1.1195647663580517  \\
            8.4  1.127168954259103  \\
            8.5  1.1347441550261987  \\
            8.6  1.1422912398691667  \\
            8.7  1.1498110799978338  \\
            8.8  1.1573045466220284  \\
            8.9  1.1647725109515767  \\
            9.0  1.1722158441963073  \\
            9.1  1.179650461534768  \\
            9.2  1.1870895824612524  \\
            9.3  1.1945300346592058  \\
            9.4  1.2019686458120726  \\
            9.5  1.2094022436032983  \\
            9.6  1.216827655716328  \\
            9.7  1.224241709834606  \\
            9.8  1.2316412336415778  \\
            9.9  1.2390230548206882  \\
            10.0  1.2463840010553824  \\
            10.1  1.2537209000291054  \\
            10.2  1.2610305794253018  \\
            10.3  1.2683098669274173  \\
            10.4  1.2755555902188962  \\
            10.5  1.2827645769831837  \\
            10.6  1.2899336549037248  \\
            10.7  1.2970596516639645  \\
            10.8  1.304139394947348  \\
            10.9  1.3111697124373198  \\
            11.0  1.3181474318173256  \\
            11.1  1.325026174277794  \\
            11.2  1.3317668927068351  \\
            11.3  1.338377412334413  \\
            11.4  1.3448655583904925  \\
            11.5  1.351239156105038  \\
            11.6  1.3575060307080147  \\
            11.7  1.3636740074293865  \\
            11.8  1.3697509114991193  \\
            11.9  1.3757445681471765  \\
            12.0  1.3816628026035231  \\
            12.1  1.3875134400981242  \\
            12.2  1.3933043058609442  \\
            12.3  1.399043225121948  \\
            12.4  1.4047380231110997  \\
            12.5  1.4103965250583645  \\
            12.6  1.4160265561937067  \\
            12.7  1.4216359417470912  \\
            12.8  1.4272325069484826  \\
            12.9  1.4328240770278458  \\
            13.0  1.438418477215145  \\
            13.1  1.44397194969458  \\
            13.2  1.449437093844378  \\
            13.3  1.454816270685545  \\
            13.4  1.4601118412390868  \\
            13.5  1.4653261665260098  \\
            13.6  1.4704616075673198  \\
            13.7  1.475520525384023  \\
            13.8  1.4805052809971249  \\
            13.9  1.485418235427632  \\
            14.0  1.49026174969655  \\
            14.1  1.4950381848248855  \\
            14.2  1.499749901833644  \\
            14.3  1.5043992617438318  \\
            14.4  1.5089886255764546  \\
            14.5  1.5135203543525189  \\
            14.6  1.51799680909303  \\
            14.7  1.5224203508189946  \\
            14.8  1.5267933405514187  \\
            14.9  1.5311181393113076  \\
            15.0  1.5353971081196682  \\
            15.1  1.539551204323609  \\
            15.2  1.5435032907074437  \\
            15.3  1.5472585864479875  \\
            15.4  1.5508223107220547  \\
            15.5  1.554199682706459  \\
            15.6  1.557395921578015  \\
            15.7  1.5604162465135376  \\
            15.8  1.5632658766898406  \\
            15.9  1.5659500312837387  \\
            16.0  1.5684739294720458  \\
            16.1  1.5708427904315765  \\
            16.2  1.573061833339145  \\
            16.3  1.5751362773715663  \\
            16.4  1.5770713417056537  \\
            16.5  1.5788722455182223  \\
            16.6  1.580544207986086  \\
            16.7  1.5820924482860599  \\
            16.8  1.5835221855949575  \\
            16.9  1.5848386390895934  \\
            17.0  1.5860470279467822  \\
            17.1  1.5870871836027531  \\
            17.2  1.587898652660948  \\
            17.3  1.5884862270490012  \\
            17.4  1.588854698694547  \\
            17.5  1.5890088595252194  \\
            17.6  1.5889535014686522  \\
            17.7  1.5886934164524797  \\
            17.8  1.5882333964043356  \\
            17.9  1.5875782332518544  \\
            18.0  1.58673271892267  \\
            18.1  1.5857016453444164  \\
            18.2  1.5844898044447273  \\
            18.3  1.5831019881512376  \\
            18.4  1.5815429883915808  \\
            18.5  1.5798175970933912  \\
            18.6  1.5779306061843026  \\
            18.7  1.575886807591949  \\
            18.8  1.5736909932439649  \\
            18.9  1.571347955067984  \\
            19.0  1.5688624849916406  \\
        }
        ;
    \addplot[color={rgb,1:red,0.7529;green,0.3255;blue,0.4039}, name path={2feea139-c798-4240-98c3-4c619f877b20}, draw opacity={1.0}, line width={1.0}, solid, forget plot]
        table[row sep={\\}]
        {
            \\
            -17.0  -0.05811342042506862  \\
            -16.9  -0.05815417246592469  \\
            -16.8  -0.05818942544794847  \\
            -16.7  -0.05821923968194298  \\
            -16.6  -0.058243675472333445  \\
            -16.5  -0.058262793117743295  \\
            -16.4  -0.05827665291147999  \\
            -16.3  -0.05828531514193336  \\
            -16.2  -0.05828884009290225  \\
            -16.1  -0.058287288043856744  \\
            -16.0  -0.05828071927013867  \\
            -15.9  -0.05826919404310815  \\
            -15.8  -0.05825277263023976  \\
            -15.7  -0.05823151529517223  \\
            -15.6  -0.058205482297710276  \\
            -15.5  -0.05817473389378636  \\
            -15.4  -0.05813933033537633  \\
            -15.3  -0.058099331870372195  \\
            -15.2  -0.05805479874241426  \\
            -15.1  -0.05800579119067262  \\
            -15.0  -0.057952369449588925  \\
            -14.9  -0.05789640225462911  \\
            -14.8  -0.05783920842255341  \\
            -14.7  -0.0577800232964778  \\
            -14.6  -0.05771808220531671  \\
            -14.5  -0.05765262046138564  \\
            -14.4  -0.057582873357237  \\
            -14.3  -0.05750807616158071  \\
            -14.2  -0.05742746411406885  \\
            -14.1  -0.05734027241868733  \\
            -14.0  -0.057245736235393974  \\
            -13.9  -0.05714309066954004  \\
            -13.8  -0.057031570758459216  \\
            -13.7  -0.05691041145440745  \\
            -13.6  -0.05677884760276153  \\
            -13.5  -0.056636113914008135  \\
            -13.4  -0.05648144492754792  \\
            -13.3  -0.05631407496463314  \\
            -13.2  -0.05613323806676784  \\
            -13.1  -0.05593816791453432  \\
            -13.0  -0.055728097719871084  \\
            -12.9  -0.055509517121132834  \\
            -12.8  -0.05528943672558362  \\
            -12.7  -0.05506787063839225  \\
            -12.6  -0.0548448327770501  \\
            -12.5  -0.05462033684527978  \\
            -12.4  -0.054394396303214965  \\
            -12.3  -0.05416702433330611  \\
            -12.2  -0.05393823380132501  \\
            -12.1  -0.053708037211744886  \\
            -12.0  -0.05347644665667284  \\
            -11.9  -0.053243473757377924  \\
            -11.8  -0.05300912959731763  \\
            -11.7  -0.052773424645406575  \\
            -11.6  -0.05253636866807243  \\
            -11.5  -0.05229797062842856  \\
            -11.4  -0.05205823857063763  \\
            -11.3  -0.05181717948725839  \\
            -11.2  -0.051574799167012464  \\
            -11.1  -0.05133110202004043  \\
            -11.0  -0.05108609087725269  \\
            -10.9  -0.050840221053246815  \\
            -10.8  -0.05059396034955199  \\
            -10.7  -0.05034732751943849  \\
            -10.6  -0.05010033913235756  \\
            -10.5  -0.04985300930027592  \\
            -10.4  -0.0496053493711921  \\
            -10.3  -0.04935736758616027  \\
            -10.2  -0.049109068695795344  \\
            -10.1  -0.04886045353183377  \\
            -10.0  -0.04861151852890744  \\
            -9.9  -0.04836225519124493  \\
            -9.8  -0.048112649498529114  \\
            -9.7  -0.04786268124464399  \\
            -9.6  -0.04761232330250387  \\
            -9.5  -0.04736154080761146  \\
            -9.4  -0.04711029025240543  \\
            -9.3  -0.04685851848287037  \\
            -9.2  -0.04660616158826969  \\
            -9.1  -0.046353143674256136  \\
            -9.0  -0.046099375509006  \\
            -8.9  -0.04584425054190544  \\
            -8.8  -0.04558720741024391  \\
            -8.7  -0.045328198550129484  \\
            -8.6  -0.04506715448413756  \\
            -8.5  -0.044803981714644614  \\
            -8.4  -0.04453856050785221  \\
            -8.3  -0.044270742583498834  \\
            -8.2  -0.04400034873099856  \\
            -8.1  -0.04372716637949614  \\
            -8.0  -0.04345094715703701  \\
            -7.9  -0.043171404482824416  \\
            -7.8  -0.04288821124631492  \\
            -7.7  -0.04260099763768199  \\
            -7.6  -0.04230934920584806  \\
            -7.5  -0.04201280523268813  \\
            -7.4  -0.04171085752490988  \\
            -7.3  -0.0414029497382159  \\
            -7.2  -0.04108847736119292  \\
            -7.1  -0.04076678849846316  \\
            -7.0  -0.0404371856032919  \\
            -6.9  -0.04009889636854813  \\
            -6.8  -0.03975082106505401  \\
            -6.7  -0.0393916590432421  \\
            -6.6  -0.039019995093199145  \\
            -6.5  -0.0386342874065716  \\
            -6.4  -0.03823285419393013  \\
            -6.3  -0.03781385881627608  \\
            -6.2  -0.03737529327845157  \\
            -6.1  -0.036914959921637125  \\
            -6.0  -0.03643045114237298  \\
            -5.9  -0.03591912695726279  \\
            -5.8  -0.0353780902265759  \\
            -5.7  -0.03480415934746628  \\
            -5.6  -0.03419383822989697  \\
            -5.5  -0.03354328337740037  \\
            -5.4  -0.0328482679127563  \\
            -5.3  -0.03210414241828263  \\
            -5.2  -0.03130579250503013  \\
            -5.1  -0.03044759308869438  \\
            -5.0  -0.029523359437085783  \\
            -4.9  -0.028527037048378973  \\
            -4.8  -0.027452608768346404  \\
            -4.7  -0.02629315950498365  \\
            -4.6  -0.025041325012072152  \\
            -4.5  -0.023689295433155177  \\
            -4.4  -0.022228826425510134  \\
            -4.3  -0.020651259123708104  \\
            -4.2  -0.018947550192224216  \\
            -4.1  -0.017108313144791026  \\
            -4.0  -0.015123871960699087  \\
            -3.9  -0.012984327792309625  \\
            -3.8  -0.010679639224029924  \\
            -3.7  -0.008199716106454508  \\
            -3.6  -0.005534526453248367  \\
            -3.5  -0.0026742152650976736  \\
            -3.4  0.0003907665417589995  \\
            -3.3  0.003669525639942356  \\
            -3.2  0.007170585462663331  \\
            -3.1  0.01090174490073667  \\
            -3.0  0.014869940873664014  \\
            -2.9  0.019081487377391626  \\
            -2.8  0.023541716151121675  \\
            -2.7  0.028254460847778856  \\
            -2.6  0.03322228912488197  \\
            -2.5  0.038446449040080516  \\
            -2.4  0.043926842249746315  \\
            -2.3  0.04966202518204501  \\
            -2.2  0.055649238127734375  \\
            -2.1  0.06188446096791752  \\
            -2.0  0.06836249312926337  \\
            -1.9  0.07507705440422052  \\
            -1.8  0.08202090255762905  \\
            -1.7  0.08918596319811511  \\
            -1.6  0.09656346723207866  \\
            -1.5  0.10414409132398057  \\
            -1.4  0.11191809712228748  \\
            -1.3  0.11987546552482559  \\
            -1.2  0.12800602289232987  \\
            -1.1  0.13629955681622147  \\
            -1.0  0.14474591975293655  \\
            -0.9  0.1533184798626236  \\
            -0.8  0.16199177266277057  \\
            -0.7  0.1707580173169689  \\
            -0.6  0.17961011697295498  \\
            -0.5  0.18854165278847743  \\
            -0.4  0.1975468692990009  \\
            -0.3  0.20662065298492066  \\
            -0.2  0.21575850568646712  \\
            -0.1  0.2249565142941017  \\
            0.0  0.23421131792373198  \\
            0.1  0.24352007357839  \\
            0.2  0.2528804211068139  \\
            0.3  0.2622904480977546  \\
            0.4  0.2717486551979912  \\
            0.5  0.28125392221177786  \\
            0.6  0.2908054752286122  \\
            0.7  0.3004028549330933  \\
            0.8  0.3100458861731803  \\
            0.9  0.3197346487992043  \\
            1.0  0.3294694497333775  \\
            1.1  0.33925758687958113  \\
            1.2  0.3491042943917821  \\
            1.3  0.3590070141238999  \\
            1.4  0.36896327841324184  \\
            1.5  0.3789706953066607  \\
            1.6  0.38902693319359877  \\
            1.7  0.3991297048277102  \\
            1.8  0.40927675069719816  \\
            1.9  0.419465821683971  \\
            2.0  0.4296946609332169  \\
            2.1  0.43996098483803764  \\
            2.2  0.45026246302850187  \\
            2.3  0.4605966972410349  \\
            2.4  0.4709611989328067  \\
            2.5  0.48135336549714597  \\
            2.6  0.4917704549306612  \\
            2.7  0.5022095588015308  \\
            2.8  0.5126675733724755  \\
            2.9  0.5231411687426396  \\
            3.0  0.5336267558917779  \\
            3.1  0.5444309736808883  \\
            3.2  0.5558148557075325  \\
            3.3  0.5677050923223858  \\
            3.4  0.5800268512164539  \\
            3.5  0.5927037316148505  \\
            3.6  0.6056577672926441  \\
            3.7  0.6188095000283121  \\
            3.8  0.632078149584206  \\
            3.9  0.6453819094436177  \\
            4.0  0.6586383981063773  \\
            4.1  0.6717652923069073  \\
            4.2  0.684681159635278  \\
            4.3  0.6973064926645041  \\
            4.4  0.7095649246739275  \\
            4.5  0.7213845797274966  \\
            4.6  0.7326994803788964  \\
            4.7  0.7434509095795443  \\
            4.8  0.7535886054674891  \\
            4.9  0.7630716642658097  \\
            5.0  0.7718690410083715  \\
            5.1  0.7801509657051896  \\
            5.2  0.7880901432987933  \\
            5.3  0.7956692303039499  \\
            5.4  0.802874510166535  \\
            5.5  0.8096961234427514  \\
            5.6  0.816128183402937  \\
            5.7  0.8221687684271607  \\
            5.8  0.8278197912021554  \\
            5.9  0.8330867533774968  \\
            6.0  0.8379784019568302  \\
            6.1  0.8425063094358844  \\
            6.2  0.8466844030267983  \\
            6.3  0.8505284690896358  \\
            6.4  0.8540556573543197  \\
            6.5  0.8572840061621533  \\
            6.6  0.8602320054362408  \\
            6.7  0.8629182090740728  \\
            6.8  0.8653609035287735  \\
            6.9  0.8675778349459449  \\
            7.0  0.8695859936194239  \\
            7.1  0.8714364747825988  \\
            7.2  0.8731739636454693  \\
            7.3  0.8748050058590868  \\
            7.4  0.876335917705236  \\
            7.5  0.877772770730729  \\
            7.6  0.8791213806973512  \\
            7.7  0.8803873004111094  \\
            7.8  0.8815758159835183  \\
            7.9  0.8826919460811289  \\
            8.0  0.8837404437342493  \\
            8.1  0.8847258002989623  \\
            8.2  0.8856522511955329  \\
            8.3  0.8865237830789672  \\
            8.4  0.8873441421319982  \\
            8.5  0.8881168432057044  \\
            8.6  0.8888451795671948  \\
            8.7  0.8895322330465001  \\
            8.8  0.8901808844054279  \\
            8.9  0.8907938237793127  \\
            9.0  0.8913735610681244  \\
            9.1  0.8919235209284788  \\
            9.2  0.8924465705668143  \\
            9.3  0.892944214997184  \\
            9.4  0.8934179033778942  \\
            9.5  0.8938690261245548  \\
            9.6  0.8942989128659269  \\
            9.7  0.894708831152471  \\
            9.8  0.8950999858299423  \\
            9.9  0.8954735189940075  \\
            10.0  0.8958305104463925  \\
            10.1  0.8961719785782307  \\
            10.2  0.8964988816118629  \\
            10.3  0.8968121191381382  \\
            10.4  0.8971125338921342  \\
            10.5  0.8974009137160025  \\
            10.6  0.8976779936632797  \\
            10.7  0.8979444582043811  \\
            10.8  0.898200943498077  \\
            10.9  0.898448039698486  \\
            11.0  0.8986862932714954  \\
            11.1  0.8989147931677818  \\
            11.2  0.8991329656473982  \\
            11.3  0.8993417547454143  \\
            11.4  0.8995420124642917  \\
            11.5  0.899734510535066  \\
            11.6  0.8999199505500945  \\
            11.7  0.9000989727060779  \\
            11.8  0.900272163359588  \\
            11.9  0.9004400615665895  \\
            12.0  0.9006031647515387  \\
            12.1  0.900761933629803  \\
            12.2  0.9009167964887207  \\
            12.3  0.9010681529170492  \\
            12.4  0.9012163770594058  \\
            12.5  0.9013618204611712  \\
            12.6  0.9015048145598916  \\
            12.7  0.9016456728712201  \\
            12.8  0.9017846929106405  \\
            12.9  0.9019221578864272  \\
            13.0  0.9020583381943722  \\
            13.1  0.902192254571576  \\
            13.2  0.9023228738725528  \\
            13.3  0.902450367615254  \\
            13.4  0.9025748959797616  \\
            13.5  0.9026966089450261  \\
            13.6  0.902815647302526  \\
            13.7  0.9029321435610348  \\
            13.8  0.903046222754963  \\
            13.9  0.9031580031672334  \\
            14.0  0.9032675969763301  \\
            14.1  0.9033751108360234  \\
            14.2  0.9034806463952513  \\
            14.3  0.9035843007647667  \\
            14.4  0.903686166936385  \\
            14.5  0.903786334159989  \\
            14.6  0.9038848882828491  \\
            14.7  0.9039819120553035  \\
            14.8  0.9040774854063742  \\
            14.9  0.904171685692494  \\
            15.0  0.9042645879221662  \\
            15.1  0.9043545055491098  \\
            15.2  0.9044398105838207  \\
            15.3  0.9045206594992049  \\
            15.4  0.9045972030910282  \\
            15.5  0.9046695871812983  \\
            15.6  0.9047379532276059  \\
            15.7  0.9048024388520449  \\
            15.8  0.9048631783012393  \\
            15.9  0.9049203028472451  \\
            16.0  0.9049739411376181  \\
            16.1  0.9050242195017081  \\
            16.2  0.9050712622191833  \\
            16.3  0.905115191755922  \\
            16.4  0.9051561289716487  \\
            16.5  0.905194193303075  \\
            16.6  0.9052295029257583  \\
            16.7  0.9052621748974412  \\
            16.8  0.9052923252852464  \\
            16.9  0.905320069278765  \\
            17.0  0.9053455212907996  \\
            17.1  0.9053674188308889  \\
            17.2  0.9053844950155591  \\
            17.3  0.9053968558615261  \\
            17.4  0.9054046058073644  \\
            17.5  0.9054078478535996  \\
            17.6  0.9054066836828949  \\
            17.7  0.9054012137629546  \\
            17.8  0.9053915374343914  \\
            17.9  0.9053777529854908  \\
            18.0  0.9053599577155279  \\
            18.1  0.9053382479880809  \\
            18.2  0.9053127192755865  \\
            18.3  0.9052834661962319  \\
            18.4  0.9052505825441433  \\
            18.5  0.9052141613137231  \\
            18.6  0.9051742947189024  \\
            18.7  0.9051310742079938  \\
            18.8  0.9050845904747851  \\
            18.9  0.905034933466458  \\
            19.0  0.904982192388889  \\
        }
        ;
    \node[left, , color={rgb,1:red,0.0;green,0.3608;blue,0.6706}, draw opacity={1.0}, rotate={0.0}, font={{\fontsize{6 pt}{7.800000000000001 pt}\selectfont}}]  at (axis cs:14.5,1.5890088595252194) {Nominal};
    \node[right, , color={rgb,1:red,0.7529;green,0.3255;blue,0.4039}, draw opacity={1.0}, rotate={0.0}, font={{\fontsize{6 pt}{7.800000000000001 pt}\selectfont}}]  at (axis cs:4.9,0.6945044297626097) {Limited};
\end{axis}
\end{tikzpicture}

    \caption{Example curves demonstrating the changes to the lift coefficient vs angle of attack for the nominal polar when the critical mach limiter is applied.}
	\label{fig:clminmax-correction}
\end{marginfigure}

\begin{equation}
    f_\text{stall} = \frac{\left.\frac{\d c_\ell}{\d \alpha}\right|_\text{stall}}{\frac{\d c_\ell}{\d \alpha}},
\end{equation}
%
and
%
\begin{equation}
    c_{\ell_\text{lim}} = \Delta c_{\ell_\text{stall}} \ln \left[\frac{1 + exp\left(\frac{c_{\ell_\text{pg}}-c_{\ell_\text{max}}'}{\Delta c_{\ell_\text{stall}}}\right)}{1+exp\left(\frac{c_{\ell_\text{min}}'-c_{\ell_\text{pg}}}{\Delta c_{\ell_\text{stall}}}\right)}\right];
\end{equation}

\where \(\left.\frac{\d c_\ell}{\d \alpha}\right|_\text{stall}\) is the lift curve slope to apply in the post stall region as part of this limiting correction, and \(\frac{\d c_\ell}{\d \alpha}\) is the nominal lift curve slope.
%
The \(\Delta c_{\ell_\text{stall}}\) term is the change in \(c_\ell\) between incipient and total stall.
%
The \(c_{\ell_\text{max}}'\) and \(c_{\ell_\text{min}}'\) values are the minimum and maximum of the nominal \(c_{\ell_{\text{max}_o}}\) and \(c_{\ell_{\text{min}_o}}\) and the following expressions, respectively:

\begin{subequations}
    \begin{align}
        c_{\ell_\text{max}}' &= min \left[\left.c_\ell\right|_{c_{d_\text{min}}}+4\left(M_\text{crit} - M +\Delta M_\text{stall}\right),~~c_{\ell_{\text{max}_o}}\right] \\
        c_{\ell_\text{min}}' &= max \left[\left.c_\ell\right|_{c_{d_\text{min}}}-4\left(M_\text{crit} - M +\Delta M_\text{stall}\right),~~c_{\ell_{\text{min}_o}} \right],
    \end{align}
\end{subequations}

\where \(M_\text{crit}\) is the critical Mach number, \(\left.c_\ell\right|_{c_{d_min}}\) is the lift coefficient at the minimum drag angle of attack, and

\begin{equation}
    \Delta M_\text{stall} = \left(\frac{0.1}{10}\right)^{1/3}
\end{equation}

\noindent is comprised of chosen factors that yield reasonable results.\sidenote{These numbers are hard coded into XROTOR and DFDC.}
%
\Cref{fig:clminmax-correction} shows an example transonic limit adjustment for an arbitrary lift curve given a critical Mach number of 0.7 and an operational Mach number of 0.8.


%%%%% ----- Transonic Drag Adjustments ----- %%%%%


Along with the limiters placed on the lift curve due to transonic effects for Mach numbers above the critical Mach number for the airfoil, there are accompanying increases in the drag coefficients.
%
Again, we turn to the corrections provided in the XROTOR and DFDC codes, which add compressibility drag based on the limited lift coefficients described previously.
%
The added compressibility drag takes the form

\begin{equation}
    c_{d_c} = c_{d_{Re}} + 10 \left(M-M_\text{crit}(c_\ell)\right)^{3},
\end{equation}

\begin{marginfigure}
	% Recommended preamble:
% \usetikzlibrary{arrows.meta}
% \usetikzlibrary{backgrounds}
% \usepgfplotslibrary{patchplots}
% \usepgfplotslibrary{fillbetween}
% \pgfplotsset{%
%     layers/standard/.define layer set={%
%         background,axis background,axis grid,axis ticks,axis lines,axis tick labels,pre main,main,axis descriptions,axis foreground%
%     }{
%         grid style={/pgfplots/on layer=axis grid},%
%         tick style={/pgfplots/on layer=axis ticks},%
%         axis line style={/pgfplots/on layer=axis lines},%
%         label style={/pgfplots/on layer=axis descriptions},%
%         legend style={/pgfplots/on layer=axis descriptions},%
%         title style={/pgfplots/on layer=axis descriptions},%
%         colorbar style={/pgfplots/on layer=axis descriptions},%
%         ticklabel style={/pgfplots/on layer=axis tick labels},%
%         axis background@ style={/pgfplots/on layer=axis background},%
%         3d box foreground style={/pgfplots/on layer=axis foreground},%
%     },
% }

\begin{tikzpicture}[/tikz/background rectangle/.style={fill={rgb,1:red,1.0;green,1.0;blue,1.0}, fill opacity={1.0}, draw opacity={1.0}}, show background rectangle]
\begin{axis}[point meta max={nan}, point meta min={nan}, legend cell align={left}, legend columns={1}, title={}, title style={at={{(0.5,1)}}, anchor={south}, font={{\fontsize{14 pt}{18.2 pt}\selectfont}}, color={rgb,1:red,0.0;green,0.0;blue,0.0}, draw opacity={1.0}, rotate={0.0}, align={center}}, legend style={color={rgb,1:red,0.0;green,0.0;blue,0.0}, draw opacity={0.0}, line width={1}, solid, fill={rgb,1:red,0.0;green,0.0;blue,0.0}, fill opacity={0.0}, text opacity={1.0}, font={{\fontsize{8 pt}{10.4 pt}\selectfont}}, text={rgb,1:red,0.0;green,0.0;blue,0.0}, cells={anchor={center}}, at={(1.02, 1)}, anchor={north west}}, axis background/.style={fill={rgb,1:red,0.0;green,0.0;blue,0.0}, opacity={0.0}}, anchor={north west}, xshift={0.0mm}, yshift={-0.0mm}, width={45.8mm}, height={50.8mm}, scaled x ticks={false}, xlabel={}, x tick style={color={rgb,1:red,0.0;green,0.0;blue,0.0}, opacity={1.0}}, x tick label style={color={rgb,1:red,0.0;green,0.0;blue,0.0}, opacity={1.0}, rotate={0}}, xlabel style={at={(ticklabel cs:0.5)}, anchor=near ticklabel, at={{(ticklabel cs:0.5)}}, anchor={near ticklabel}, font={{\fontsize{11 pt}{14.3 pt}\selectfont}}, color={rgb,1:red,0.0;green,0.0;blue,0.0}, draw opacity={1.0}, rotate={0.0}}, xmajorticks={false}, xmajorgrids={false}, xmin={-18.080000000000002}, xmax={20.080000000000002}, axis x line*={left}, separate axis lines, x axis line style={{draw opacity = 0}}, scaled y ticks={false}, ylabel={}, y tick style={color={rgb,1:red,0.0;green,0.0;blue,0.0}, opacity={1.0}}, y tick label style={color={rgb,1:red,0.0;green,0.0;blue,0.0}, opacity={1.0}, rotate={0}}, ylabel style={at={(ticklabel cs:0.5)}, anchor=near ticklabel, at={{(ticklabel cs:0.5)}}, anchor={near ticklabel}, font={{\fontsize{11 pt}{14.3 pt}\selectfont}}, color={rgb,1:red,0.0;green,0.0;blue,0.0}, draw opacity={1.0}, rotate={0.0}}, ymajorticks={false}, ymajorgrids={false}, ymin={-0.008234340913512272}, ymax={0.282712371363921}, axis y line*={left}, y axis line style={{draw opacity = 0}}, colorbar={false}]
    \addplot[color={rgb,1:red,0.0;green,0.0;blue,0.0}, name path={41abde91-89d4-405b-a1dd-eed3b16517fc}, draw opacity={1.0}, line width={0.25}, solid, forget plot]
        table[row sep={\\}]
        {
            \\
            -17.0  0.0  \\
            19.0  0.0  \\
        }
        ;
    \addplot[color={rgb,1:red,0.0;green,0.0;blue,0.0}, name path={742b8e84-373d-4302-95b2-9337d7c63d2f}, draw opacity={1.0}, line width={0.25}, solid, forget plot]
        table[row sep={\\}]
        {
            \\
            0.0  0.0  \\
            0.0  0.06124222934673175  \\
        }
        ;
    \addplot[color={rgb,1:red,0.0;green,0.3608;blue,0.6706}, name path={965088b2-bedd-482f-895f-83851f1ad588}, draw opacity={1.0}, line width={1.0}, solid, forget plot]
        table[row sep={\\}]
        {
            \\
            -17.0  0.03274774225137401  \\
            -16.9  0.032078105905770565  \\
            -16.8  0.0314152645136106  \\
            -16.7  0.0307604786412915  \\
            -16.6  0.030115008855210748  \\
            -16.5  0.029480115721765728  \\
            -16.4  0.0288570598073539  \\
            -16.3  0.02824710167837272  \\
            -16.2  0.027651501901219573  \\
            -16.1  0.02707152104229195  \\
            -16.0  0.026508419667987226  \\
            -15.9  0.025963458344702876  \\
            -15.8  0.02543789763883633  \\
            -15.7  0.024932998116785003  \\
            -15.6  0.02445002034494635  \\
            -15.5  0.023990224889717793  \\
            -15.4  0.023554872317496773  \\
            -15.3  0.023145223194680722  \\
            -15.2  0.02276253808766707  \\
            -15.1  0.02240807756285326  \\
            -15.0  0.02208310218663672  \\
            -14.9  0.021775799369370918  \\
            -14.8  0.021473570332794213  \\
            -14.7  0.021176496360381335  \\
            -14.6  0.020884658735607062  \\
            -14.5  0.02059813874194613  \\
            -14.4  0.020317017662873298  \\
            -14.3  0.020041376781863313  \\
            -14.2  0.01977129738239093  \\
            -14.1  0.019506860747930913  \\
            -14.0  0.019248148161958002  \\
            -13.9  0.018995240907946947  \\
            -13.8  0.018748220269372513  \\
            -13.7  0.01850716752970944  \\
            -13.6  0.01827216397243249  \\
            -13.5  0.018043290881016413  \\
            -13.4  0.017820629538935956  \\
            -13.3  0.017604261229665883  \\
            -13.2  0.017394267236680933  \\
            -13.1  0.017190728843455867  \\
            -13.0  0.016993727333465436  \\
            -12.9  0.016799744583548246  \\
            -12.8  0.016605353818553277  \\
            -12.7  0.016410773343970862  \\
            -12.6  0.016216221465291326  \\
            -12.5  0.016021916488005  \\
            -12.4  0.015828076717602207  \\
            -12.3  0.01563492045957328  \\
            -12.2  0.015442666019408535  \\
            -12.1  0.015251531702598313  \\
            -12.0  0.015061735814632937  \\
            -11.9  0.01487349666100273  \\
            -11.8  0.014687032547198026  \\
            -11.7  0.014502561778709145  \\
            -11.6  0.014320302661026421  \\
            -11.5  0.014140473499640182  \\
            -11.4  0.013963292600040752  \\
            -11.3  0.013788978267718457  \\
            -11.2  0.013617748808163627  \\
            -11.1  0.01344982252686659  \\
            -11.0  0.013285417729317676  \\
            -10.9  0.013123256442351834  \\
            -10.8  0.012961988738554107  \\
            -10.7  0.012801724992039944  \\
            -10.6  0.01264257557692481  \\
            -10.5  0.012484650867324159  \\
            -10.4  0.012328061237353444  \\
            -10.3  0.012172917061128125  \\
            -10.2  0.012019328712763655  \\
            -10.1  0.011867406566375494  \\
            -10.0  0.011717260996079096  \\
            -9.9  0.01156900237598992  \\
            -9.8  0.01142274108022342  \\
            -9.7  0.011278587482895047  \\
            -9.6  0.01113665195812027  \\
            -9.5  0.010997044880014536  \\
            -9.4  0.010859876622693302  \\
            -9.3  0.01072525756027203  \\
            -9.2  0.010593298066866166  \\
            -9.1  0.010464108516591176  \\
            -9.0  0.010337799283562514  \\
            -8.9  0.010213971815691194  \\
            -8.8  0.010092188424284365  \\
            -8.7  0.009972500778551687  \\
            -8.6  0.00985496054770282  \\
            -8.5  0.009739619400947424  \\
            -8.4  0.009626529007495154  \\
            -8.3  0.009515741036555675  \\
            -8.2  0.009407307157338637  \\
            -8.1  0.009301279039053707  \\
            -8.0  0.00919770835091054  \\
            -7.9  0.009096646762118799  \\
            -7.8  0.008998145941888138  \\
            -7.7  0.008902257559428221  \\
            -7.6  0.008809033283948702  \\
            -7.5  0.00871852478465924  \\
            -7.4  0.0086307837307695  \\
            -7.3  0.008545861791489135  \\
            -7.2  0.008463810636027806  \\
            -7.1  0.00838468193359517  \\
            -7.0  0.00830852735340089  \\
            -6.9  0.008234355627174653  \\
            -6.8  0.008161172967873453  \\
            -6.7  0.008089027266547895  \\
            -6.6  0.008017966414248583  \\
            -6.5  0.007948038302026126  \\
            -6.4  0.007879290820931133  \\
            -6.3  0.007811771862014209  \\
            -6.2  0.00774552931632596  \\
            -6.1  0.007680611074916994  \\
            -6.0  0.007617065028837918  \\
            -5.9  0.007554939069139339  \\
            -5.8  0.007494281086871864  \\
            -5.7  0.007435138973086099  \\
            -5.6  0.007377560618832652  \\
            -5.5  0.007321593915162128  \\
            -5.4  0.0072672867531251364  \\
            -5.3  0.007214687023772282  \\
            -5.2  0.007163842618154173  \\
            -5.1  0.007114801427321416  \\
            -5.0  0.007067611342324617  \\
            -4.9  0.0070212995895903435  \\
            -4.8  0.006974935693871966  \\
            -4.7  0.0069286309937103  \\
            -4.6  0.006882496827646157  \\
            -4.5  0.006836644534220352  \\
            -4.4  0.006791185451973701  \\
            -4.3  0.006746230919447015  \\
            -4.2  0.006701892275181112  \\
            -4.1  0.006658280857716803  \\
            -4.0  0.006615508005594903  \\
            -3.9  0.006573685057356227  \\
            -3.8  0.006532923351541588  \\
            -3.7  0.006493334226691801  \\
            -3.6  0.006455029021347679  \\
            -3.5  0.006418119074050036  \\
            -3.4  0.006382715723339688  \\
            -3.3  0.006348930307757448  \\
            -3.2  0.006316874165844132  \\
            -3.1  0.0062866586361405505  \\
            -3.0  0.006258395057187519  \\
            -2.9  0.0062312952293119015  \\
            -2.8  0.006204477971590604  \\
            -2.7  0.006177915150689514  \\
            -2.6  0.006151578633274512  \\
            -2.5  0.0061254402860114856  \\
            -2.4  0.006099471975566318  \\
            -2.3  0.006073645568604896  \\
            -2.2  0.0060479329317931  \\
            -2.1  0.006022305931796817  \\
            -2.0  0.005996736435281932  \\
            -1.9  0.005971196308914328  \\
            -1.8  0.0059456574193598915  \\
            -1.7  0.005920091633284505  \\
            -1.6  0.005894470817354054  \\
            -1.5  0.005868766838234422  \\
            -1.4  0.005842951562591497  \\
            -1.3  0.005816996857091157  \\
            -1.2  0.0057908745883992925  \\
            -1.1  0.005764556623181786  \\
            -1.0  0.005738014828104521  \\
            -0.9  0.005711186890429283  \\
            -0.8  0.005684049557190802  \\
            -0.7  0.005656633284714391  \\
            -0.6  0.005628968529325357  \\
            -0.5  0.005601085747349011  \\
            -0.4  0.005573015395110662  \\
            -0.3  0.005544787928935617  \\
            -0.2  0.005516433805149188  \\
            -0.1  0.005487983480076683  \\
            0.0  0.005459467410043413  \\
            0.1  0.005430916051374686  \\
            0.2  0.005402359860395811  \\
            0.3  0.005373829293432098  \\
            0.4  0.005345354806808856  \\
            0.5  0.005316966856851396  \\
            0.6  0.005288695899885024  \\
            0.7  0.0052605723922350515  \\
            0.8  0.00523262679022679  \\
            0.9  0.005204889550185544  \\
            1.0  0.005177391128436626  \\
            1.1  0.005150751995126851  \\
            1.2  0.0051257346245434775  \\
            1.3  0.005102582479222478  \\
            1.4  0.005081539021699825  \\
            1.5  0.005062847714511489  \\
            1.6  0.005046752020193444  \\
            1.7  0.005033495401281663  \\
            1.8  0.005023321320312118  \\
            1.9  0.00501647323982078  \\
            2.0  0.005013194622343622  \\
            2.1  0.005013728930416618  \\
            2.2  0.005018319626575739  \\
            2.3  0.005027210173356958  \\
            2.4  0.0050406440332962455  \\
            2.5  0.005058864668929577  \\
            2.6  0.0050821155427929225  \\
            2.7  0.005110640117422256  \\
            2.8  0.0051446818553535474  \\
            2.9  0.0051844842191227725  \\
            3.0  0.005230290671265901  \\
            3.1  0.0052823010756966925  \\
            3.2  0.005340369027180724  \\
            3.3  0.005404218584531699  \\
            3.4  0.0054735738065633185  \\
            3.5  0.005548158752089284  \\
            3.6  0.005627697479923298  \\
            3.7  0.005711914048879063  \\
            3.8  0.0058005325177702795  \\
            3.9  0.00589327694541065  \\
            4.0  0.005989871390613878  \\
            4.1  0.006090039912193662  \\
            4.2  0.006193506568963708  \\
            4.3  0.006299995419737717  \\
            4.4  0.006409230523329389  \\
            4.5  0.006520935938552425  \\
            4.6  0.006634835724220529  \\
            4.7  0.0067506539391474045  \\
            4.8  0.006868114642146751  \\
            4.9  0.0069869418920322736  \\
            5.0  0.007106859747617668  \\
            5.1  0.007227914763315581  \\
            5.2  0.007350227880041849  \\
            5.3  0.007473634736364967  \\
            5.4  0.007597970970853431  \\
            5.5  0.007723072222075736  \\
            5.6  0.007848774128600378  \\
            5.7  0.007974912328995856  \\
            5.8  0.00810132246183066  \\
            5.9  0.00822784016567329  \\
            6.0  0.00835430107909224  \\
            6.1  0.008480540840656003  \\
            6.2  0.00860639508893308  \\
            6.3  0.008731699462491964  \\
            6.4  0.008856289599901151  \\
            6.5  0.008980001139729138  \\
            6.6  0.009102669720544417  \\
            6.7  0.00922413098091549  \\
            6.8  0.009344220559410846  \\
            6.9  0.009462774094598985  \\
            7.0  0.0095796272250484  \\
            7.1  0.009695121805801218  \\
            7.2  0.009809767555493974  \\
            7.3  0.009923651908086778  \\
            7.4  0.010036862297539738  \\
            7.5  0.010149486157812967  \\
            7.6  0.010261610922866576  \\
            7.7  0.010373324026660675  \\
            7.8  0.010484712903155372  \\
            7.9  0.01059586498631078  \\
            8.0  0.010706867710087008  \\
            8.1  0.010817808508444169  \\
            8.2  0.01092877481534237  \\
            8.3  0.01103985406474173  \\
            8.4  0.011151133690602345  \\
            8.5  0.011262701126884338  \\
            8.6  0.01137464380754781  \\
            8.7  0.011487049166552878  \\
            8.8  0.011600004637859656  \\
            8.9  0.011713597655428244  \\
            9.0  0.01182791565321876  \\
            9.1  0.01194349340864211  \\
            9.2  0.0120607446751984  \\
            9.3  0.012179575350981542  \\
            9.4  0.01229989133408544  \\
            9.5  0.012421598522603999  \\
            9.6  0.012544602814631132  \\
            9.7  0.012668810108260741  \\
            9.8  0.012794126301586743  \\
            9.9  0.012920457292703035  \\
            10.0  0.013047708979703528  \\
            10.1  0.013175787260682132  \\
            10.2  0.013304598033732752  \\
            10.3  0.0134340471969493  \\
            10.4  0.013564040648425675  \\
            10.5  0.013694484286255791  \\
            10.6  0.013825284008533554  \\
            10.7  0.013956345713352872  \\
            10.8  0.014087575298807654  \\
            10.9  0.014218878662991801  \\
            11.0  0.014350161703999229  \\
            11.1  0.014485361106380894  \\
            11.2  0.014628176014337205  \\
            11.3  0.014778156015436212  \\
            11.4  0.014934850697245977  \\
            11.5  0.015097809647334563  \\
            11.6  0.015266582453270025  \\
            11.7  0.015440718702620426  \\
            11.8  0.01561976798295383  \\
            11.9  0.015803279881838286  \\
            12.0  0.015990803986841867  \\
            12.1  0.016181889885532617  \\
            12.2  0.01637608716547861  \\
            12.3  0.016572945414247902  \\
            12.4  0.016772014219408545  \\
            12.5  0.016972843168528613  \\
            12.6  0.01717498184917615  \\
            12.7  0.01737797984891922  \\
            12.8  0.017581386755325897  \\
            12.9  0.01778475215596422  \\
            13.0  0.017987625638402265  \\
            13.1  0.01819853488415434  \\
            13.2  0.018425739152365075  \\
            13.3  0.018668385397047998  \\
            13.4  0.01892562057221663  \\
            13.5  0.019196591631884495  \\
            13.6  0.019480445530065123  \\
            13.7  0.01977632922077204  \\
            13.8  0.020083389658018775  \\
            13.9  0.020400773795818845  \\
            14.0  0.020727628588185772  \\
            14.1  0.021063100989133096  \\
            14.2  0.021406337952674333  \\
            14.3  0.02175648643282302  \\
            14.4  0.022112693383592664  \\
            14.5  0.022474105758996797  \\
            14.6  0.02283987051304895  \\
            14.7  0.02320913459976265  \\
            14.8  0.023581044973151423  \\
            14.9  0.02395474858722878  \\
            15.0  0.024329392396008263  \\
            15.1  0.02471917093040247  \\
            15.2  0.025138357414664334  \\
            15.3  0.025586216842817883  \\
            15.4  0.026062014208887128  \\
            15.5  0.026565014506896086  \\
            15.6  0.027094482730868775  \\
            15.7  0.027649683874829224  \\
            15.8  0.02822988293280145  \\
            15.9  0.02883434489880946  \\
            16.0  0.029462334766877277  \\
            16.1  0.030113117531028934  \\
            16.2  0.030785958185288422  \\
            16.3  0.031480121723679796  \\
            16.4  0.03219487314022703  \\
            16.5  0.03292947742895419  \\
            16.6  0.03368319958388527  \\
            16.7  0.03445530459904428  \\
            16.8  0.035245057468455264  \\
            16.9  0.0360517231861422  \\
            17.0  0.03687456674612916  \\
            17.1  0.03772901534643799  \\
            17.2  0.03863006485462826  \\
            17.3  0.03957633326903071  \\
            17.4  0.04056643858797592  \\
            17.5  0.0415989988097946  \\
            17.6  0.04267263193281738  \\
            17.7  0.043785955955374864  \\
            17.8  0.04493758887579779  \\
            17.9  0.046126148692416734  \\
            18.0  0.04735025340356243  \\
            18.1  0.04860852100756548  \\
            18.2  0.049899569502756494  \\
            18.3  0.051222016887466235  \\
            18.4  0.052574481160025235  \\
            18.5  0.05395558031876427  \\
            18.6  0.05536393236201392  \\
            18.7  0.0567981552881048  \\
            18.8  0.05825686709536767  \\
            18.9  0.059738685782133066  \\
            19.0  0.06124222934673175  \\
        }
        ;
    \addplot[color={rgb,1:red,0.0;green,0.0;blue,0.0}, name path={b4a82ecb-300a-4a4a-b00b-3166df5ba377}, draw opacity={1.0}, line width={0.25}, solid, forget plot]
        table[row sep={\\}]
        {
            \\
            0.0  0.0  \\
            0.0  0.27447803045040875  \\
        }
        ;
    \addplot[color={rgb,1:red,0.7529;green,0.3255;blue,0.4039}, name path={4a4622ae-3e91-4d66-baf7-2434b3b52f94}, draw opacity={1.0}, line width={1.0}, solid, forget plot]
        table[row sep={\\}]
        {
            \\
            -17.0  0.25328625966903234  \\
            -16.9  0.25264066045853595  \\
            -16.8  0.2519986140520971  \\
            -16.7  0.2513614159835972  \\
            -16.6  0.2507303618847072  \\
            -16.5  0.2501067474818096  \\
            -16.4  0.24949186859290198  \\
            -16.3  0.24888702112448624  \\
            -16.2  0.2482935010684516  \\
            -16.1  0.24771260449895632  \\
            -16.0  0.24714562756930816  \\
            -15.9  0.24659386650885257  \\
            -15.8  0.2460586176198646  \\
            -15.7  0.24554117727445246  \\
            -15.6  0.24504284191147022  \\
            -15.5  0.2445649080334416  \\
            -15.4  0.244108672203497  \\
            -15.3  0.2436754310423208  \\
            -15.2  0.24326648122511269  \\
            -15.1  0.24288311947855648  \\
            -15.0  0.2425266425778048  \\
            -14.9  0.24218634046131032  \\
            -14.8  0.24185039227989058  \\
            -14.7  0.24151842879795837  \\
            -14.6  0.24119008097176092  \\
            -14.5  0.24086497996702297  \\
            -14.4  0.24054275718228055  \\
            -14.3  0.24022304427781369  \\
            -14.2  0.23990547321004746  \\
            -14.1  0.23958967627126737  \\
            -14.0  0.2392752861344329  \\
            -13.9  0.23896193590281464  \\
            -13.8  0.2386492591640914  \\
            -13.7  0.23833689004842065  \\
            -13.6  0.23802446328984245  \\
            -13.5  0.23771161429015122  \\
            -13.4  0.23739797918407657  \\
            -13.3  0.2370831949042053  \\
            -13.2  0.23676689924349803  \\
            -13.1  0.23644873091246651  \\
            -13.0  0.23612832958695085  \\
            -12.9  0.2358059969852943  \\
            -12.8  0.23548242633677702  \\
            -12.7  0.23515784525008013  \\
            -12.6  0.23483248121730232  \\
            -12.5  0.23450656159866948  \\
            -12.4  0.2341803136051018  \\
            -12.3  0.23385396427832617  \\
            -12.2  0.2335277404681776  \\
            -12.1  0.23320186880667254  \\
            -12.0  0.23287657567838485  \\
            -11.9  0.2325520871865756  \\
            -11.8  0.23222862911445197  \\
            -11.7  0.23190642688083515  \\
            -11.6  0.23158570548940705  \\
            -11.5  0.2312666894705824  \\
            -11.4  0.23094960281490817  \\
            -11.3  0.2306346688967291  \\
            -11.2  0.23032211038666137  \\
            -11.1  0.23001214915120238  \\
            -11.0  0.22970500613754496  \\
            -10.9  0.2293996694492716  \\
            -10.8  0.2290950620374104  \\
            -10.7  0.22879130547213206  \\
            -10.6  0.22848852003488682  \\
            -10.5  0.22818682456211725  \\
            -10.4  0.22788633627034713  \\
            -10.3  0.22758717056056782  \\
            -10.2  0.22728944079965577  \\
            -10.1  0.22699325807632528  \\
            -10.0  0.2266987309288943  \\
            -9.9  0.22640596504189223  \\
            -9.8  0.22611506290826677  \\
            -9.7  0.2258261234536815  \\
            -9.6  0.22553924161908998  \\
            -9.5  0.22525450789747523  \\
            -9.4  0.2249720078203301  \\
            -9.3  0.2246918213891218  \\
            -9.2  0.22441402244666914  \\
            -9.1  0.22413867798302717  \\
            -9.0  0.2238658473701573  \\
            -8.9  0.22359478266388738  \\
            -8.8  0.2233247234920088  \\
            -8.7  0.22305569562797578  \\
            -8.6  0.22278771228152938  \\
            -8.5  0.22252077291636052  \\
            -8.4  0.2222548620090819  \\
            -8.3  0.2219899477585277  \\
            -8.2  0.22172598075771102  \\
            -8.1  0.2214628926446656  \\
            -8.0  0.22120059475281478  \\
            -7.9  0.22093897678654761  \\
            -7.8  0.22067790555326433  \\
            -7.7  0.22041722378929804  \\
            -7.6  0.2201567491237419  \\
            -7.5  0.21989627323122635  \\
            -7.4  0.21963556123193478  \\
            -7.3  0.21937435140446676  \\
            -7.2  0.21911235528424866  \\
            -7.1  0.21884925822676882  \\
            -7.0  0.21858472052058794  \\
            -6.9  0.21831731795777445  \\
            -6.8  0.2180454382605251  \\
            -6.7  0.217768398099605  \\
            -6.6  0.21748545113515466  \\
            -6.5  0.2171957815446735  \\
            -6.4  0.21689849685989782  \\
            -6.3  0.21659262004701044  \\
            -6.2  0.2162770807614629  \\
            -6.1  0.21595070570633443  \\
            -6.0  0.21561220802203984  \\
            -5.9  0.21526017563579056  \\
            -5.8  0.21489305850221516  \\
            -5.7  0.214509154672735  \\
            -5.6  0.21410659514173153  \\
            -5.5  0.2136833274334777  \\
            -5.4  0.21323709791676126  \\
            -5.3  0.21276543286595934  \\
            -5.2  0.21226561833015498  \\
            -5.1  0.21173467892820644  \\
            -5.0  0.21116935576025234  \\
            -4.9  0.2105654778478092  \\
            -4.8  0.20991897682131166  \\
            -4.7  0.20922636994193292  \\
            -4.6  0.20848397817220818  \\
            -4.5  0.20768793727347304  \\
            -4.4  0.20683421421847667  \\
            -4.3  0.20591862962723817  \\
            -4.2  0.2049368868861591  \\
            -4.1  0.20388460851605844  \\
            -4.0  0.2027573802059795  \\
            -3.9  0.20155080271972092  \\
            -3.8  0.20026055160755862  \\
            -3.7  0.19888244431745278  \\
            -3.6  0.19741251390501607  \\
            -3.5  0.1958470881036968  \\
            -3.4  0.19418287205785717  \\
            -3.3  0.1924170325714017  \\
            -3.2  0.1905472813194774  \\
            -3.1  0.18857195415081415  \\
            -3.0  0.18649008341394205  \\
            -2.9  0.184300372486656  \\
            -2.8  0.18200218313395664  \\
            -2.7  0.17959672881957275  \\
            -2.6  0.1770860782299011  \\
            -2.5  0.17447315419820092  \\
            -2.4  0.17176171327549988  \\
            -2.3  0.1689563062614434  \\
            -2.2  0.1660622208935604  \\
            -2.1  0.1630854087087608  \\
            -2.0  0.1600323987738342  \\
            -1.9  0.15691020148353133  \\
            -1.8  0.15372620591554642  \\
            -1.7  0.15048807430274969  \\
            -1.6  0.1472036370462098  \\
            -1.5  0.143880791377246  \\
            -1.4  0.14052740632448482  \\
            -1.3  0.1371512361006385  \\
            -1.2  0.1337598434418723  \\
            -1.1  0.13036053385433333  \\
            -1.0  0.12696030118386517  \\
            -0.9  0.12357221342295151  \\
            -0.8  0.1202081424793613  \\
            -0.7  0.11687264072706605  \\
            -0.6  0.11356976284201603  \\
            -0.5  0.11030309192241493  \\
            -0.4  0.10707576904582214  \\
            -0.3  0.10389052498513386  \\
            -0.2  0.10074971301433447  \\
            -0.1  0.09765534193363061  \\
            0.0  0.09460910862557148  \\
            0.1  0.09161242961532312  \\
            0.2  0.08866647124786406  \\
            0.3  0.08577217821263708  \\
            0.4  0.08293030024327046  \\
            0.5  0.08014141689821552  \\
            0.6  0.0774059603897085  \\
            0.7  0.0747242364756415  \\
            0.8  0.07209644346393912  \\
            0.9  0.06952268940398841  \\
            1.0  0.06700300755642308  \\
            1.1  0.06453629006387668  \\
            1.2  0.062122238363479396  \\
            1.3  0.05976196311380098  \\
            1.4  0.05746212286079291  \\
            1.5  0.055450783266054876  \\
            1.6  0.053831270400863995  \\
            1.7  0.052586512199060464  \\
            1.8  0.051703984964772534  \\
            1.9  0.05117577688296234  \\
            2.0  0.05099867915431009  \\
            2.1  0.0511743104817724  \\
            2.2  0.05170927966284601  \\
            2.3  0.05261538993906033  \\
            2.4  0.05390988749190715  \\
            2.5  0.05561575501590092  \\
            2.6  0.05776204960296734  \\
            2.7  0.060196366286318026  \\
            2.8  0.0627132549757148  \\
            2.9  0.06531327458608592  \\
            3.0  0.06799716119747032  \\
            3.1  0.07084698448052218  \\
            3.2  0.07394163658895068  \\
            3.3  0.07727575149468892  \\
            3.4  0.08084202736898335  \\
            3.5  0.08463083436481145  \\
            3.6  0.08862984494526926  \\
            3.7  0.09282370253945543  \\
            3.8  0.09719374900626761  \\
            3.9  0.10171783591450619  \\
            4.0  0.10637024821782226  \\
            4.1  0.11112177041845141  \\
            4.2  0.11593992344817151  \\
            4.3  0.12078939390983295  \\
            4.4  0.12563266504058715  \\
            4.5  0.13043084065532107  \\
            4.6  0.1351446306257088  \\
            4.7  0.13973544198286084  \\
            4.8  0.14416649777075954  \\
            4.9  0.1484038912450755  \\
            5.0  0.15241748017495202  \\
            5.1  0.15626877809498346  \\
            5.2  0.1600263748715622  \\
            5.3  0.1636748843828311  \\
            5.4  0.167200384565834  \\
            5.5  0.17059071026843972  \\
            5.6  0.173835684917269  \\
            5.7  0.17692727819005358  \\
            5.8  0.17985968233005817  \\
            5.9  0.18262930586228024  \\
            6.0  0.18523468949681582  \\
            6.1  0.18767635420879195  \\
            6.2  0.18995659531365622  \\
            6.3  0.19207923851249098  \\
            6.4  0.19404937433955785  \\
            6.5  0.19587308641520873  \\
            6.6  0.1975571867629665  \\
            6.7  0.1991089686299912  \\
            6.8  0.2005359841815327  \\
            6.9  0.201845851475644  \\
            7.0  0.20304609251446268  \\
            7.1  0.20416350297460534  \\
            7.2  0.20522202978909151  \\
            7.3  0.20622474332365695  \\
            7.4  0.20717466561572723  \\
            7.5  0.20807475385419325  \\
            7.6  0.20892788676732255  \\
            7.7  0.20973685370693182  \\
            7.8  0.21050434619299663  \\
            7.9  0.21123295167031347  \\
            8.0  0.21192514922569902  \\
            8.1  0.21258330701857356  \\
            8.2  0.21320968118782174  \\
            8.3  0.213806416011994  \\
            8.4  0.21437554511679563  \\
            8.5  0.21491899354230112  \\
            8.6  0.21543858050148096  \\
            8.7  0.21593602268073228  \\
            8.8  0.21641293795161548  \\
            8.9  0.21687084938054807  \\
            9.0  0.21731118943951114  \\
            9.1  0.21773636263757  \\
            9.2  0.21814834882834905  \\
            9.3  0.2185478598798212  \\
            9.4  0.2189355812292152  \\
            9.5  0.2193121697477588  \\
            9.6  0.2196782521146288  \\
            9.7  0.2200344236502275  \\
            9.8  0.22038124755942803  \\
            9.9  0.22071925453681585  \\
            10.0  0.22104894268798586  \\
            10.1  0.22137077772349023  \\
            10.2  0.2216851933849163  \\
            10.3  0.22199259206566668  \\
            10.4  0.22229334559224168  \\
            10.5  0.2225877961350503  \\
            10.6  0.22287625722098015  \\
            10.7  0.2231590148230378  \\
            10.8  0.22343632850532852  \\
            10.9  0.22370843260441178  \\
            11.0  0.2239755374306498  \\
            11.1  0.22424105345488846  \\
            11.2  0.2245083455093057  \\
            11.3  0.22477749512407255  \\
            11.4  0.22504853237690356  \\
            11.5  0.22532144243382649  \\
            11.6  0.22559617119171768  \\
            11.7  0.22587263015280035  \\
            11.8  0.22615070064172507  \\
            11.9  0.22643023745928548  \\
            12.0  0.22671107205280355  \\
            12.1  0.22699301527135818  \\
            12.2  0.22727585976400064  \\
            12.3  0.22755938207058599  \\
            12.4  0.2278433444476628  \\
            12.5  0.2281274964657375  \\
            12.6  0.22841157640905532  \\
            12.7  0.22869531250461944  \\
            12.8  0.22897842400342255  \\
            12.9  0.2292606221336678  \\
            13.0  0.2295416109430162  \\
            13.1  0.22982935564362833  \\
            13.2  0.23013152156681327  \\
            13.3  0.23044735279590525  \\
            13.4  0.2307760870257612  \\
            13.5  0.23111695620311865  \\
            13.6  0.2314691870978572  \\
            13.7  0.2318320018130846  \\
            13.8  0.2322046182410226  \\
            13.9  0.23258625047082013  \\
            14.0  0.23297610915369654  \\
            14.1  0.23337340183017835  \\
            14.2  0.23377733322362723  \\
            14.3  0.23418710550376753  \\
            14.4  0.23460191852349344  \\
            14.5  0.23502097003185074  \\
            14.6  0.23544345586575982  \\
            14.7  0.23586857012275503  \\
            14.8  0.23629550531675217  \\
            14.9  0.23672345251863638  \\
            15.0  0.23715160148325629  \\
            15.1  0.2375931747936965  \\
            15.2  0.2380615068542019  \\
            15.3  0.2385559515990992  \\
            15.4  0.23907585980665771  \\
            15.5  0.23962057949529347  \\
            15.6  0.24018945626646462  \\
            15.7  0.24078183360197988  \\
            15.8  0.24139705312225543  \\
            15.9  0.24203445481106103  \\
            16.0  0.24269337721146012  \\
            16.1  0.24337315759695524  \\
            16.2  0.244073132121243  \\
            16.3  0.2447926359495043  \\
            16.4  0.24553100337371161  \\
            16.5  0.24628756791409384  \\
            16.6  0.2470616624085862  \\
            16.7  0.24785261909183592  \\
            16.8  0.24865976966511671  \\
            16.9  0.24948244535830993  \\
            17.0  0.2503199769849567  \\
            17.1  0.251187062945865  \\
            17.2  0.2520979676926964  \\
            17.3  0.2530513701549533  \\
            17.4  0.25404594842718065  \\
            17.5  0.2550803798446571  \\
            17.6  0.25615334104782156  \\
            17.7  0.2572635080369281  \\
            17.8  0.25840955621821193  \\
            17.9  0.25959016044267136  \\
            18.0  0.2608039950384076  \\
            18.1  0.2620497338373508  \\
            18.2  0.26332605019708133  \\
            18.3  0.2646316170183748  \\
            18.4  0.26596510675901536  \\
            18.5  0.26732519144437233  \\
            18.6  0.26871054267516936  \\
            18.7  0.2701198316328477  \\
            18.8  0.2715517290828857  \\
            18.9  0.2730049053764081  \\
            19.0  0.27447803045040875  \\
        }
        ;
    \node[left, , color={rgb,1:red,0.0;green,0.3608;blue,0.6706}, draw opacity={1.0}, rotate={0.0}, font={{\fontsize{6 pt}{7.800000000000001 pt}\selectfont}}]  at (axis cs:19.0,0.05501319462234362) {Nominal};
    \node[left, , color={rgb,1:red,0.7529;green,0.3255;blue,0.4039}, draw opacity={1.0}, rotate={0.0}, font={{\fontsize{6 pt}{7.800000000000001 pt}\selectfont}}]  at (axis cs:21.0,0.13723901522520437) {Augmented};
\end{axis}
\end{tikzpicture}

    \caption{Example curves demonstrating the changes to the drag coefficient vs angle of attack for the nominal polar when the transonic compressibility corrections are added for a Mach number of 0.1 above \(M_\text{crit}\).}
	\label{fig:transdrag-correction}
\end{marginfigure}
%

\where the critical Mach adjusted for the limited lift coefficient takes the form

\begin{equation}
    \label{eqn:mcritcl}
    M_\text{crit}(c_\ell) = M_\text{crit} - \frac{\bigg|c_{\ell_\text{lim}} - \left.c_\ell\right|_{c_{d_\text{min}}}\bigg|}{4} - \Delta M_\text{crit},
\end{equation}

\where

\begin{equation}
    \Delta M_\text{crit} = \left(\frac{0.002}{10}\right)^{1/3}
\end{equation}

\noindent comes from the difference in Mach corresponding to a rise in \(c_d\) of 0.002 at \(M_\text{crit}\), which is chosen to match empirical experience.\sidenote{Again, these values are hard coded into XROTOR and DFDC.}
%
Similarly, as before, the other constants are chosen to yield reasonable results.
%
\Cref{fig:transdrag-correction} shows an example comparison between a nominal drag curve and one for which the transonic compressibility augmentations have been applied for a Mach number 0.1 above \(M_\text{crit}\).


For smooth implementation there are several min/max operations in the lift limiter function, these have been smoothed with sigmoid blending functions, and very little change is introduced from the nominal function as seen in \cref{fig:translim-smoothed}.
%
In addition, the nominal drag limiter function only adds drag after the critical Mach number is reached.
%
We smoothed this transition, which is perhaps less physical, but the differences are minimal as seen in \cref{fig:translim-smoothed}.
%
Furthermore, we used a smoothed absolute value with relatively tight smoothing range.
%
In this case, there is a slight over-prediction of the corrected drag for values at and just above the critical mach number, which actually counters the under prediction introduced by smoothing across the critical mach.

\begin{figure}[htb]
     \centering
     \begin{subfigure}[t]{0.45\textwidth}
         \centering
        % Recommended preamble:
% \usetikzlibrary{arrows.meta}
% \usetikzlibrary{backgrounds}
% \usepgfplotslibrary{patchplots}
% \usepgfplotslibrary{fillbetween}
% \pgfplotsset{%
%     layers/standard/.define layer set={%
%         background,axis background,axis grid,axis ticks,axis lines,axis tick labels,pre main,main,axis descriptions,axis foreground%
%     }{
%         grid style={/pgfplots/on layer=axis grid},%
%         tick style={/pgfplots/on layer=axis ticks},%
%         axis line style={/pgfplots/on layer=axis lines},%
%         label style={/pgfplots/on layer=axis descriptions},%
%         legend style={/pgfplots/on layer=axis descriptions},%
%         title style={/pgfplots/on layer=axis descriptions},%
%         colorbar style={/pgfplots/on layer=axis descriptions},%
%         ticklabel style={/pgfplots/on layer=axis tick labels},%
%         axis background@ style={/pgfplots/on layer=axis background},%
%         3d box foreground style={/pgfplots/on layer=axis foreground},%
%     },
% }

\begin{tikzpicture}[/tikz/background rectangle/.style={fill={rgb,1:red,1.0;green,1.0;blue,1.0}, fill opacity={1.0}, draw opacity={1.0}}, show background rectangle]
\begin{axis}[point meta max={nan}, point meta min={nan}, legend cell align={left}, legend columns={1}, title={}, title style={at={{(0.5,1)}}, anchor={south}, font={{\fontsize{14 pt}{18.2 pt}\selectfont}}, color={rgb,1:red,0.0;green,0.0;blue,0.0}, draw opacity={1.0}, rotate={0.0}, align={center}}, legend style={color={rgb,1:red,0.0;green,0.0;blue,0.0}, draw opacity={0.0}, line width={1}, solid, fill={rgb,1:red,0.0;green,0.0;blue,0.0}, fill opacity={0.0}, text opacity={1.0}, font={{\fontsize{8 pt}{10.4 pt}\selectfont}}, text={rgb,1:red,0.0;green,0.0;blue,0.0}, cells={anchor={center}}, at={(1.02, 1)}, anchor={north west}}, axis background/.style={fill={rgb,1:red,0.0;green,0.0;blue,0.0}, opacity={0.0}}, anchor={north west}, xshift={0.0mm}, yshift={-0.0mm}, width={45.8mm}, height={50.8mm}, scaled x ticks={false}, xlabel={Mach Number}, x tick style={color={rgb,1:red,0.0;green,0.0;blue,0.0}, opacity={1.0}}, x tick label style={color={rgb,1:red,0.0;green,0.0;blue,0.0}, opacity={1.0}, rotate={0}}, xlabel style={at={(ticklabel cs:0.5)}, anchor=near ticklabel, at={{(ticklabel cs:0.5)}}, anchor={near ticklabel}, font={{\fontsize{11 pt}{14.3 pt}\selectfont}}, color={rgb,1:red,0.0;green,0.0;blue,0.0}, draw opacity={1.0}, rotate={0.0}}, xmajorgrids={false}, xmin={-0.030000000000000027}, xmax={1.03}, xticklabels={{$0.00$,$0.25$,$0.50$,$0.75$,$1.00$}}, xtick={{0.0,0.25,0.5,0.75,1.0}}, xtick align={inside}, xticklabel style={font={{\fontsize{8 pt}{10.4 pt}\selectfont}}, color={rgb,1:red,0.0;green,0.0;blue,0.0}, draw opacity={1.0}, rotate={0.0}}, x grid style={color={rgb,1:red,0.0;green,0.0;blue,0.0}, draw opacity={0.1}, line width={0.5}, solid}, axis x line*={left}, x axis line style={color={rgb,1:red,0.0;green,0.0;blue,0.0}, draw opacity={1.0}, line width={1}, solid}, scaled y ticks={false}, ylabel={$c_{\ell_\mathrm{lim}}$}, y tick style={color={rgb,1:red,0.0;green,0.0;blue,0.0}, opacity={1.0}}, y tick label style={color={rgb,1:red,0.0;green,0.0;blue,0.0}, opacity={1.0}, rotate={0}}, ylabel style={at={(ticklabel cs:0.5)}, anchor=near ticklabel, at={{(ticklabel cs:0.5)}}, anchor={near ticklabel}, font={{\fontsize{11 pt}{14.3 pt}\selectfont}}, color={rgb,1:red,0.0;green,0.0;blue,0.0}, draw opacity={1.0}, rotate={0.0}}, ymajorgrids={false}, ymin={0.6327278367197582}, ymax={1.0106890661150496}, yticklabels={{$0.7$,$0.8$,$0.9$,$1.0$}}, ytick={{0.7000000000000001,0.8,0.9,1.0}}, ytick align={inside}, yticklabel style={font={{\fontsize{8 pt}{10.4 pt}\selectfont}}, color={rgb,1:red,0.0;green,0.0;blue,0.0}, draw opacity={1.0}, rotate={0.0}}, y grid style={color={rgb,1:red,0.0;green,0.0;blue,0.0}, draw opacity={0.1}, line width={0.5}, solid}, axis y line*={left}, y axis line style={color={rgb,1:red,0.0;green,0.0;blue,0.0}, draw opacity={1.0}, line width={1}, solid}, colorbar={false}]
    \addplot[color={rgb,1:red,0.0;green,0.3608;blue,0.6706}, name path={3872b832-ced2-4ab2-9152-be36ac4aefa0}, draw opacity={1.0}, line width={1.0}, solid, forget plot]
        table[row sep={\\}]
        {
            \\
            0.0  0.9999917438488518  \\
            0.002004008016032064  0.9999917438488518  \\
            0.004008016032064128  0.9999917438488518  \\
            0.006012024048096192  0.9999917438488518  \\
            0.008016032064128256  0.9999917438488518  \\
            0.01002004008016032  0.9999917438488518  \\
            0.012024048096192385  0.9999917438488518  \\
            0.014028056112224449  0.9999917438488518  \\
            0.01603206412825651  0.9999917438488518  \\
            0.018036072144288578  0.9999917438488518  \\
            0.02004008016032064  0.9999917438488518  \\
            0.022044088176352707  0.9999917438488518  \\
            0.02404809619238477  0.9999917438488518  \\
            0.026052104208416832  0.9999917438488518  \\
            0.028056112224448898  0.9999917438488518  \\
            0.03006012024048096  0.9999917438488518  \\
            0.03206412825651302  0.9999917438488518  \\
            0.03406813627254509  0.9999917438488518  \\
            0.036072144288577156  0.9999917438488518  \\
            0.03807615230460922  0.9999917438488518  \\
            0.04008016032064128  0.9999917438488518  \\
            0.04208416833667335  0.9999917438488518  \\
            0.04408817635270541  0.9999917438488518  \\
            0.04609218436873747  0.9999917438488518  \\
            0.04809619238476954  0.9999917438488518  \\
            0.050100200400801605  0.9999917438488518  \\
            0.052104208416833664  0.9999917438488518  \\
            0.05410821643286573  0.9999917438488518  \\
            0.056112224448897796  0.9999917438488518  \\
            0.05811623246492986  0.9999917438488518  \\
            0.06012024048096192  0.9999917438488518  \\
            0.06212424849699399  0.9999917438488518  \\
            0.06412825651302605  0.9999917438488518  \\
            0.06613226452905811  0.9999917438488518  \\
            0.06813627254509018  0.9999917438488518  \\
            0.07014028056112225  0.9999917438488518  \\
            0.07214428857715431  0.9999917438488518  \\
            0.07414829659318638  0.9999917438488518  \\
            0.07615230460921844  0.9999917438488518  \\
            0.0781563126252505  0.9999917438488518  \\
            0.08016032064128256  0.9999917438488518  \\
            0.08216432865731463  0.9999917438488518  \\
            0.0841683366733467  0.9999917438488518  \\
            0.08617234468937876  0.9999917438488518  \\
            0.08817635270541083  0.9999917438488518  \\
            0.09018036072144289  0.9999917438488518  \\
            0.09218436873747494  0.9999917438488518  \\
            0.09418837675350701  0.9999917438488518  \\
            0.09619238476953908  0.9999917438488518  \\
            0.09819639278557114  0.9999917438488518  \\
            0.10020040080160321  0.9999917438488518  \\
            0.10220440881763528  0.9999917438488518  \\
            0.10420841683366733  0.9999917438488518  \\
            0.1062124248496994  0.9999917438488518  \\
            0.10821643286573146  0.9999917438488518  \\
            0.11022044088176353  0.9999917438488518  \\
            0.11222444889779559  0.9999917438488518  \\
            0.11422845691382766  0.9999917438488518  \\
            0.11623246492985972  0.9999917438488518  \\
            0.11823647294589178  0.9999917438488518  \\
            0.12024048096192384  0.9999917438488518  \\
            0.12224448897795591  0.9999917438488518  \\
            0.12424849699398798  0.9999917438488518  \\
            0.12625250501002003  0.9999917438488518  \\
            0.1282565130260521  0.9999917438488518  \\
            0.13026052104208416  0.9999917438488518  \\
            0.13226452905811623  0.9999917438488518  \\
            0.1342685370741483  0.9999917438488518  \\
            0.13627254509018036  0.9999917438488518  \\
            0.13827655310621242  0.9999917438488518  \\
            0.1402805611222445  0.9999917438488518  \\
            0.14228456913827656  0.9999917438488518  \\
            0.14428857715430862  0.9999917438488518  \\
            0.1462925851703407  0.9999917438488518  \\
            0.14829659318637275  0.9999917438488518  \\
            0.15030060120240482  0.9999917438488518  \\
            0.1523046092184369  0.9999917438488518  \\
            0.15430861723446893  0.9999917438488518  \\
            0.156312625250501  0.9999917438488518  \\
            0.15831663326653306  0.9999917438488518  \\
            0.16032064128256512  0.9999917438488518  \\
            0.1623246492985972  0.9999917438488518  \\
            0.16432865731462926  0.9999917438488518  \\
            0.16633266533066132  0.9999917438488518  \\
            0.1683366733466934  0.9999917438488518  \\
            0.17034068136272545  0.9999917438488518  \\
            0.17234468937875752  0.9999917438488518  \\
            0.1743486973947896  0.9999917438488518  \\
            0.17635270541082165  0.9999917438488518  \\
            0.17835671342685372  0.9999917438488518  \\
            0.18036072144288579  0.9999917438488518  \\
            0.18236472945891782  0.9999917438488518  \\
            0.1843687374749499  0.9999917438488518  \\
            0.18637274549098196  0.9999917438488518  \\
            0.18837675350701402  0.9999917438488518  \\
            0.1903807615230461  0.9999917438488518  \\
            0.19238476953907815  0.9999917438488518  \\
            0.19438877755511022  0.9999917438488518  \\
            0.1963927855711423  0.9999917438488518  \\
            0.19839679358717435  0.9999917438488518  \\
            0.20040080160320642  0.9999917438488518  \\
            0.20240480961923848  0.9999917438490858  \\
            0.20440881763527055  0.9999917438493704  \\
            0.20641282565130262  0.9999917438496786  \\
            0.20841683366733466  0.9999917438500125  \\
            0.21042084168336672  0.9999917438503745  \\
            0.2124248496993988  0.9999917438507665  \\
            0.21442885771543085  0.9999917438511913  \\
            0.21643286573146292  0.9999917438516515  \\
            0.218436873747495  0.9999917438521502  \\
            0.22044088176352705  0.9999917438526904  \\
            0.22244488977955912  0.9999917438532757  \\
            0.22444889779559118  0.99999174385391  \\
            0.22645290581162325  0.9999917438545971  \\
            0.22845691382765532  0.9999917438553416  \\
            0.23046092184368738  0.9999917438561482  \\
            0.23246492985971945  0.9999917438570222  \\
            0.23446893787575152  0.999991743857969  \\
            0.23647294589178355  0.9999917438589949  \\
            0.23847695390781562  0.9999917438601065  \\
            0.24048096192384769  0.9999917438613107  \\
            0.24248496993987975  0.9999917438626156  \\
            0.24448897795591182  0.9999917438640292  \\
            0.24649298597194388  0.999991743865561  \\
            0.24849699398797595  0.9999917438672206  \\
            0.250501002004008  0.9999917438690186  \\
            0.25250501002004005  0.9999917438709667  \\
            0.2545090180360721  0.9999917438730773  \\
            0.2565130260521042  0.9999917438753642  \\
            0.25851703406813625  0.999991743877842  \\
            0.2605210420841683  0.9999917438805265  \\
            0.2625250501002004  0.999991743883435  \\
            0.26452905811623245  0.9999917438865863  \\
            0.2665330661322645  0.9999917438900007  \\
            0.2685370741482966  0.9999917438936999  \\
            0.27054108216432865  0.999991743897708  \\
            0.2725450901803607  0.9999917439020505  \\
            0.2745490981963928  0.9999917439067555  \\
            0.27655310621242485  0.9999917439118532  \\
            0.2785571142284569  0.9999917439173763  \\
            0.280561122244489  0.9999917439233604  \\
            0.28256513026052105  0.9999917439298439  \\
            0.2845691382765531  0.9999917439368685  \\
            0.2865731462925852  0.9999917439444794  \\
            0.28857715430861725  0.9999917439527256  \\
            0.2905811623246493  0.9999917439616599  \\
            0.2925851703406814  0.9999917439713399  \\
            0.29458917835671344  0.9999917439818279  \\
            0.2965931863727455  0.9999917439931911  \\
            0.2985971943887776  0.9999917440055028  \\
            0.30060120240480964  0.999991744018842  \\
            0.3026052104208417  0.9999917440332945  \\
            0.3046092184368738  0.9999917440489532  \\
            0.3066132264529058  0.9999917440659188  \\
            0.30861723446893785  0.9999917440843004  \\
            0.3106212424849699  0.9999917441042161  \\
            0.312625250501002  0.999991744125794  \\
            0.31462925851703405  0.9999917441491727  \\
            0.3166332665330661  0.9999917441745027  \\
            0.3186372745490982  0.9999917442019467  \\
            0.32064128256513025  0.9999917442316812  \\
            0.3226452905811623  0.9999917442638975  \\
            0.3246492985971944  0.9999917442988024  \\
            0.32665330661322645  0.9999917443366206  \\
            0.3286573146292585  0.9999917443775952  \\
            0.3306613226452906  0.9999917444219895  \\
            0.33266533066132264  0.999991744470089  \\
            0.3346693386773547  0.9999917445222029  \\
            0.3366733466933868  0.9999917445786662  \\
            0.33867735470941884  0.9999917446398421  \\
            0.3406813627254509  0.9999917447061237  \\
            0.342685370741483  0.9999917447779373  \\
            0.34468937875751504  0.9999917448557444  \\
            0.3466933867735471  0.9999917449400454  \\
            0.3486973947895792  0.9999917450313822  \\
            0.35070140280561124  0.9999917451303421  \\
            0.3527054108216433  0.9999917452375612  \\
            0.35470941883767537  0.9999917453537289  \\
            0.35671342685370744  0.999991745479592  \\
            0.3587174348697395  0.9999917456159598  \\
            0.36072144288577157  0.999991745763709  \\
            0.3627254509018036  0.9999917459237894  \\
            0.36472945891783565  0.9999917460972302  \\
            0.3667334669338677  0.9999917462851465  \\
            0.3687374749498998  0.9999917464887464  \\
            0.37074148296593185  0.9999917467093391  \\
            0.3727454909819639  0.9999917469483425  \\
            0.374749498997996  0.9999917472072933  \\
            0.37675350701402804  0.9999917474878564  \\
            0.3787575150300601  0.9999917477918354  \\
            0.3807615230460922  0.9999917481211847  \\
            0.38276553106212424  0.9999917484780217  \\
            0.3847695390781563  0.9999917488646408  \\
            0.3867735470941884  0.9999917492835273  \\
            0.38877755511022044  0.9999917497373744  \\
            0.3907815631262525  0.9999917502290999  \\
            0.3927855711422846  0.9999911560012378  \\
            0.39478957915831664  0.9999904179099592  \\
            0.3967935871743487  0.9999896182232041  \\
            0.39879759519038077  0.9999887518012414  \\
            0.40080160320641284  0.9999878130755613  \\
            0.4028056112224449  0.9999867960131215  \\
            0.40480961923847697  0.9999856940776144  \\
            0.40681362725450904  0.9999845001875093  \\
            0.4088176352705411  0.9999832066706004  \\
            0.41082164328657317  0.99998180521477  \\
            0.41282565130260523  0.9999802868146531  \\
            0.4148296593186373  0.999978641713864  \\
            0.4168336673346693  0.9999768593424148  \\
            0.4188376753507014  0.9999749282489282  \\
            0.42084168336673344  0.9999728360272142  \\
            0.4228456913827655  0.9999705692367415  \\
            0.4248496993987976  0.9999681133164997  \\
            0.42685370741482964  0.9999654524917052  \\
            0.4288577154308617  0.9999625696727599  \\
            0.4308617234468938  0.999959446345821  \\
            0.43286573146292584  0.9999560624542922  \\
            0.4348697394789579  0.9999523962704853  \\
            0.43687374749499  0.9999484242566435  \\
            0.43887775551102204  0.9999441209144536  \\
            0.4408817635270541  0.9999394586220961  \\
            0.44288577154308617  0.9999344074578155  \\
            0.44488977955911824  0.9999289350089045  \\
            0.4468937875751503  0.9999230061649068  \\
            0.44889779559118237  0.9999165828937544  \\
            0.45090180360721444  0.9999096239994415  \\
            0.4529058116232465  0.9999020848597394  \\
            0.45490981963927857  0.9998939171423281  \\
            0.45691382765531063  0.9998850684975976  \\
            0.4589178356713427  0.9998754822262347  \\
            0.46092184368737477  0.9998650969195634  \\
            0.46292585170340683  0.9998538460704521  \\
            0.4649298597194389  0.9998416576524312  \\
            0.46693386773547096  0.9998284536644845  \\
            0.46893787575150303  0.999814149638792  \\
            0.4709418837675351  0.9997986541084898  \\
            0.4729458917835671  0.9997818680323026  \\
            0.4749498997995992  0.9997636841726725  \\
            0.47695390781563124  0.9997439864237609  \\
            0.4789579158316633  0.9997226490854488  \\
            0.48096192384769537  0.999699536079183  \\
            0.48296593186372744  0.9996745001012343  \\
            0.4849699398797595  0.9996473817086342  \\
            0.48697394789579157  0.9996180083327431  \\
            0.48897795591182364  0.9995861932150798  \\
            0.4909819639278557  0.9995517342597093  \\
            0.49298597194388777  0.9995144127961367  \\
            0.49498997995991983  0.9994739922463111  \\
            0.4969939879759519  0.9994302166889816  \\
            0.49899799599198397  0.9993828093142997  \\
            0.501002004008016  0.9993314707612054  \\
            0.503006012024048  0.9992758773298005  \\
            0.5050100200400801  0.9992156790605844  \\
            0.5070140280561122  0.9991504976721364  \\
            0.5090180360721442  0.9990799243485624  \\
            0.5110220440881763  0.9990035173678153  \\
            0.5130260521042084  0.9989207995618441  \\
            0.5150300601202404  0.9988312555994594  \\
            0.5170340681362725  0.9987343290828286  \\
            0.5190380761523046  0.9986294194486641  \\
            0.5210420841683366  0.998515878665468  \\
            0.5230460921843687  0.9983930077186733  \\
            0.5250501002004008  0.9982600528762142  \\
            0.5270541082164328  0.9981162017280041  \\
            0.5290581162324649  0.9979605789940426  \\
            0.531062124248497  0.9977922420974695  \\
            0.533066132264529  0.9976101765008748  \\
            0.5350701402805611  0.9974132908066395  \\
            0.5370741482965932  0.9972004116250747  \\
            0.5390781563126252  0.996970278217719  \\
            0.5410821643286573  0.9967215369274324  \\
            0.5430861723446894  0.9964527354119478  \\
            0.5450901803607214  0.9961623167034146  \\
            0.5470941883767535  0.9958486131232401  \\
            0.5490981963927856  0.9955098400893051  \\
            0.5511022044088176  0.9951440898614492  \\
            0.5531062124248497  0.9947493252810405  \\
            0.5551102204408818  0.9943233735715047  \\
            0.5571142284569138  0.9938639202788789  \\
            0.5591182364729459  0.9933685034447535  \\
            0.561122244488978  0.9928345081182763  \\
            0.56312625250501  0.9922591613290888  \\
            0.5651302605210421  0.9916395276589064  \\
            0.5671342685370742  0.9909725055656766  \\
            0.5691382765531062  0.9902548246304038  \\
            0.5711422845691383  0.9894830439123449  \\
            0.5731462925851704  0.988653551612696  \\
            0.5751503006012024  0.9877625662593388  \\
            0.5771543086172345  0.9868061396348167  \\
            0.5791583166332666  0.9857801616754216  \\
            0.5811623246492986  0.9846803675699871  \\
            0.5831663326653307  0.9835023472814779  \\
            0.5851703406813628  0.9822415577015163  \\
            0.5871743486973948  0.9808933376263822  \\
            0.5891783567134269  0.9794529257116518  \\
            0.591182364729459  0.9779154815206208  \\
            0.593186372745491  0.9762761097283854  \\
            0.5951903807615231  0.9745298874787712  \\
            0.5971943887775552  0.9726718948156041  \\
            0.5991983967935872  0.9706972480241332  \\
            0.6012024048096193  0.9686011356245585  \\
            0.6032064128256514  0.9663788566601571  \\
            0.6052104208416834  0.9640258608208415  \\
            0.6072144288577155  0.9615377898432599  \\
            0.6092184368737475  0.9589105195354745  \\
            0.6112224448897795  0.9561402016929712  \\
            0.6132264529058116  0.9532233051084653  \\
            0.6152304609218436  0.9501566548356482  \\
            0.6172344689378757  0.9469374688509943  \\
            0.6192384769539078  0.9435633912713213  \\
            0.6212424849699398  0.9400325213298812  \\
            0.6232464929859719  0.9363434373905974  \\
            0.625250501002004  0.9324952153870674  \\
            0.627254509018036  0.928487441206658  \\
            0.6292585170340681  0.9243202166952331  \\
            0.6312625250501002  0.9199941591281333  \\
            0.6332665330661322  0.9155103941703019  \\
            0.6352705410821643  0.91087054252475  \\
            0.6372745490981964  0.9060767006357697  \\
            0.6392785571142284  0.9011314159639756  \\
            0.6412825651302605  0.8960376574780462  \\
            0.6432865731462926  0.8907987821081464  \\
            0.6452905811623246  0.8854184979754293  \\
            0.6472945891783567  0.8799008252495809  \\
            0.6492985971943888  0.8742500554927702  \\
            0.6513026052104208  0.8684707103258498  \\
            0.6533066132264529  0.8625675002048234  \\
            0.655310621242485  0.8565452840269618  \\
            0.657314629258517  0.8504090302015653  \\
            0.6593186372745491  0.8441637797254826  \\
            0.6613226452905812  0.8378146117031282  \\
            0.6633266533066132  0.8313666116495156  \\
            0.6653306613226453  0.8248248428167023  \\
            0.6673346693386774  0.8181943206922137  \\
            0.6693386773547094  0.8114799907348967  \\
            0.6713426853707415  0.8046867093408281  \\
            0.6733466933867736  0.7978192279702635  \\
            0.6753507014028056  0.7908821803163979  \\
            0.6773547094188377  0.7838800723576285  \\
            0.6793587174348698  0.776817275106408  \\
            0.6813627254509018  0.7696980198486656  \\
            0.6833667334669339  0.7625263956570305  \\
            0.685370741482966  0.7553063489574757  \\
            0.687374749498998  0.7480416849312717  \\
            0.6893787575150301  0.7407360705410941  \\
            0.6913827655310621  0.7333930389806589  \\
            0.6933867735470942  0.7260159953603273  \\
            0.6953907815631263  0.718608223455895  \\
            0.6973947895791583  0.7111728933634304  \\
            0.6993987975951904  0.7037130699189957  \\
            0.7014028056112225  0.6962317217577949  \\
            0.7034068136272545  0.6887317309023688  \\
            0.7054108216432866  0.6812159027835774  \\
            0.7074148296593187  0.6736869766110113  \\
            0.7094188376753507  0.6661476360210181  \\
            0.7114228456913828  0.658600519940548  \\
            0.7134268537074149  0.6510482336134751  \\
            0.7154308617234469  0.6434933597428429  \\
            0.717434869739479  0.6434458290403369  \\
            0.7194388777555111  0.6434458290403369  \\
            0.7214428857715431  0.6434458290403369  \\
            0.7234468937875751  0.6434458290403369  \\
            0.7254509018036072  0.6434458290403369  \\
            0.7274549098196392  0.6434458290403369  \\
            0.7294589178356713  0.6434458290403369  \\
            0.7314629258517034  0.6434458290403369  \\
            0.7334669338677354  0.6434458290403369  \\
            0.7354709418837675  0.6434458290403369  \\
            0.7374749498997996  0.6434458290403369  \\
            0.7394789579158316  0.6434458290403369  \\
            0.7414829659318637  0.6434458290403369  \\
            0.7434869739478958  0.6434458290403369  \\
            0.7454909819639278  0.6434458290403369  \\
            0.7474949899799599  0.6434458290403369  \\
            0.749498997995992  0.6434458290403369  \\
            0.751503006012024  0.6434458290403369  \\
            0.7535070140280561  0.6434458290403369  \\
            0.7555110220440882  0.6434458290403369  \\
            0.7575150300601202  0.6434458290403369  \\
            0.7595190380761523  0.6434458290403369  \\
            0.7615230460921844  0.6434458290403369  \\
            0.7635270541082164  0.6434458290403369  \\
            0.7655310621242485  0.6434458290403369  \\
            0.7675350701402806  0.6434458290403369  \\
            0.7695390781563126  0.6434458290403369  \\
            0.7715430861723447  0.6434458290403369  \\
            0.7735470941883767  0.6434458290403369  \\
            0.7755511022044088  0.6434458290403369  \\
            0.7775551102204409  0.6434458290403369  \\
            0.779559118236473  0.6434458290403369  \\
            0.781563126252505  0.6434458290403369  \\
            0.7835671342685371  0.6434458290403369  \\
            0.7855711422845691  0.6434458290403369  \\
            0.7875751503006012  0.6434458290403369  \\
            0.7895791583166333  0.6434458290403369  \\
            0.7915831663326653  0.6434458290403369  \\
            0.7935871743486974  0.6434458290403369  \\
            0.7955911823647295  0.6434458290403369  \\
            0.7975951903807615  0.6434458290403369  \\
            0.7995991983967936  0.6434458290403369  \\
            0.8016032064128257  0.6434458290403369  \\
            0.8036072144288577  0.6434458290403369  \\
            0.8056112224448898  0.6434458290403369  \\
            0.8076152304609219  0.6434458290403369  \\
            0.8096192384769539  0.6434458290403369  \\
            0.811623246492986  0.6434458290403369  \\
            0.8136272545090181  0.6434458290403369  \\
            0.8156312625250501  0.6434458290403369  \\
            0.8176352705410822  0.6434458290403369  \\
            0.8196392785571143  0.6434458290403369  \\
            0.8216432865731463  0.6434458290403369  \\
            0.8236472945891784  0.6434458290403369  \\
            0.8256513026052105  0.6434458290403369  \\
            0.8276553106212425  0.6434458290403369  \\
            0.8296593186372746  0.6434458290403369  \\
            0.8316633266533067  0.6434458290403369  \\
            0.8336673346693386  0.6434458290403369  \\
            0.8356713426853707  0.6434458290403369  \\
            0.8376753507014028  0.6434458290403369  \\
            0.8396793587174348  0.6434458290403369  \\
            0.8416833667334669  0.6434458290403369  \\
            0.843687374749499  0.6434458290403369  \\
            0.845691382765531  0.6434458290403369  \\
            0.8476953907815631  0.6434458290403369  \\
            0.8496993987975952  0.6434458290403369  \\
            0.8517034068136272  0.6434458290403369  \\
            0.8537074148296593  0.6434458290403369  \\
            0.8557114228456913  0.6434458290403369  \\
            0.8577154308617234  0.6434458290403369  \\
            0.8597194388777555  0.6434458290403369  \\
            0.8617234468937875  0.6434458290403369  \\
            0.8637274549098196  0.6434458290403369  \\
            0.8657314629258517  0.6434458290403369  \\
            0.8677354709418837  0.6434458290403369  \\
            0.8697394789579158  0.6434458290403369  \\
            0.8717434869739479  0.6434458290403369  \\
            0.87374749498998  0.6434458290403369  \\
            0.875751503006012  0.6434458290403369  \\
            0.8777555110220441  0.6434458290403369  \\
            0.8797595190380761  0.6434458290403369  \\
            0.8817635270541082  0.6434458290403369  \\
            0.8837675350701403  0.6434458290403369  \\
            0.8857715430861723  0.6434458290403369  \\
            0.8877755511022044  0.6434458290403369  \\
            0.8897795591182365  0.6434458290403369  \\
            0.8917835671342685  0.6434458290403369  \\
            0.8937875751503006  0.6434458290403369  \\
            0.8957915831663327  0.6434458290403369  \\
            0.8977955911823647  0.6434458290403369  \\
            0.8997995991983968  0.6434458290403369  \\
            0.9018036072144289  0.6434458290403369  \\
            0.9038076152304609  0.6434458290403369  \\
            0.905811623246493  0.6434458290403369  \\
            0.9078156312625251  0.6434458290403369  \\
            0.9098196392785571  0.6434458290403369  \\
            0.9118236472945892  0.6434458290403369  \\
            0.9138276553106213  0.6434458290403369  \\
            0.9158316633266533  0.6434458290403369  \\
            0.9178356713426854  0.6434458290403369  \\
            0.9198396793587175  0.6434458290403369  \\
            0.9218436873747495  0.6434458290403369  \\
            0.9238476953907816  0.6434458290403369  \\
            0.9258517034068137  0.6434458290403369  \\
            0.9278557114228457  0.6434458290403369  \\
            0.9298597194388778  0.6434458290403369  \\
            0.9318637274549099  0.6434458290403369  \\
            0.9338677354709419  0.6434458290403369  \\
            0.935871743486974  0.6434458290403369  \\
            0.9378757515030061  0.6434458290403369  \\
            0.9398797595190381  0.6434458290403369  \\
            0.9418837675350702  0.6434458290403369  \\
            0.9438877755511023  0.6434458290403369  \\
            0.9458917835671342  0.6434458290403369  \\
            0.9478957915831663  0.6434458290403369  \\
            0.9498997995991983  0.6434458290403369  \\
            0.9519038076152304  0.6434458290403369  \\
            0.9539078156312625  0.6434458290403369  \\
            0.9559118236472945  0.6434458290403369  \\
            0.9579158316633266  0.6434458290403369  \\
            0.9599198396793587  0.6434458290403369  \\
            0.9619238476953907  0.6434458290403369  \\
            0.9639278557114228  0.6434458290403369  \\
            0.9659318637274549  0.6434458290403369  \\
            0.9679358717434869  0.6434458290403369  \\
            0.969939879759519  0.6434458290403369  \\
            0.9719438877755511  0.6434458290403369  \\
            0.9739478957915831  0.6434458290403369  \\
            0.9759519038076152  0.6434458290403369  \\
            0.9779559118236473  0.6434458290403369  \\
            0.9799599198396793  0.6434458290403369  \\
            0.9819639278557114  0.6434458290403369  \\
            0.9839679358717435  0.6434458290403369  \\
            0.9859719438877755  0.6434458290403369  \\
            0.9879759519038076  0.6434458290403369  \\
            0.9899799599198397  0.6434458290403369  \\
            0.9919839679358717  0.6434458290403369  \\
            0.9939879759519038  0.6434458290403369  \\
            0.9959919839679359  0.6434458290403369  \\
            0.9979959919839679  0.6434458290403369  \\
            1.0  0.6434458290403369  \\
        }
        ;
    \addplot[color={rgb,1:red,0.7529;green,0.3255;blue,0.4039}, name path={ba17905d-0f18-4ebe-a413-abc0ac483e9c}, draw opacity={1.0}, line width={2}, dashed, forget plot]
        table[row sep={\\}]
        {
            \\
            0.0  0.9999917438488518  \\
            0.002004008016032064  0.9999917438488518  \\
            0.004008016032064128  0.9999917438488518  \\
            0.006012024048096192  0.9999917438488518  \\
            0.008016032064128256  0.9999917438488518  \\
            0.01002004008016032  0.9999917438488518  \\
            0.012024048096192385  0.9999917438488518  \\
            0.014028056112224449  0.9999917438488518  \\
            0.01603206412825651  0.9999917438488518  \\
            0.018036072144288578  0.9999917438488518  \\
            0.02004008016032064  0.9999917438488518  \\
            0.022044088176352707  0.9999917438488518  \\
            0.02404809619238477  0.9999917438488518  \\
            0.026052104208416832  0.9999917438488518  \\
            0.028056112224448898  0.9999917438488518  \\
            0.03006012024048096  0.9999917438488518  \\
            0.03206412825651302  0.9999917438488518  \\
            0.03406813627254509  0.9999917438488518  \\
            0.036072144288577156  0.9999917438488518  \\
            0.03807615230460922  0.9999917438488518  \\
            0.04008016032064128  0.9999917438488518  \\
            0.04208416833667335  0.9999917438488518  \\
            0.04408817635270541  0.9999917438488518  \\
            0.04609218436873747  0.9999917438488518  \\
            0.04809619238476954  0.9999917438488518  \\
            0.050100200400801605  0.9999917438488518  \\
            0.052104208416833664  0.9999917438488518  \\
            0.05410821643286573  0.9999917438488518  \\
            0.056112224448897796  0.9999917438488518  \\
            0.05811623246492986  0.9999917438488518  \\
            0.06012024048096192  0.9999917438488518  \\
            0.06212424849699399  0.9999917438488518  \\
            0.06412825651302605  0.9999917438488518  \\
            0.06613226452905811  0.9999917438488518  \\
            0.06813627254509018  0.9999917438488518  \\
            0.07014028056112225  0.9999917438488518  \\
            0.07214428857715431  0.9999917438488518  \\
            0.07414829659318638  0.9999917438488518  \\
            0.07615230460921844  0.9999917438488518  \\
            0.0781563126252505  0.9999917438488518  \\
            0.08016032064128256  0.9999917438488518  \\
            0.08216432865731463  0.9999917438488518  \\
            0.0841683366733467  0.9999917438488518  \\
            0.08617234468937876  0.9999917438488518  \\
            0.08817635270541083  0.9999917438488518  \\
            0.09018036072144289  0.9999917438488518  \\
            0.09218436873747494  0.9999917438488518  \\
            0.09418837675350701  0.9999917438488518  \\
            0.09619238476953908  0.9999917438488518  \\
            0.09819639278557114  0.9999917438488518  \\
            0.10020040080160321  0.9999917438488518  \\
            0.10220440881763528  0.9999917438488518  \\
            0.10420841683366733  0.9999917438488518  \\
            0.1062124248496994  0.9999917438488518  \\
            0.10821643286573146  0.9999917438488518  \\
            0.11022044088176353  0.9999917438488518  \\
            0.11222444889779559  0.9999917438488518  \\
            0.11422845691382766  0.9999917438488518  \\
            0.11623246492985972  0.9999917438488518  \\
            0.11823647294589178  0.9999917438488518  \\
            0.12024048096192384  0.9999917438488518  \\
            0.12224448897795591  0.9999917438488518  \\
            0.12424849699398798  0.9999917438488518  \\
            0.12625250501002003  0.9999917438488518  \\
            0.1282565130260521  0.9999917438488518  \\
            0.13026052104208416  0.9999917438488518  \\
            0.13226452905811623  0.9999917438488518  \\
            0.1342685370741483  0.9999917438488518  \\
            0.13627254509018036  0.9999917438488518  \\
            0.13827655310621242  0.9999917438488518  \\
            0.1402805611222445  0.9999917438488518  \\
            0.14228456913827656  0.9999917438488518  \\
            0.14428857715430862  0.9999917438488518  \\
            0.1462925851703407  0.9999917438488518  \\
            0.14829659318637275  0.9999917438488518  \\
            0.15030060120240482  0.9999917438488518  \\
            0.1523046092184369  0.9999917438488518  \\
            0.15430861723446893  0.9999917438488518  \\
            0.156312625250501  0.9999917438488518  \\
            0.15831663326653306  0.9999917438488517  \\
            0.16032064128256512  0.9999917438488517  \\
            0.1623246492985972  0.9999917438488517  \\
            0.16432865731462926  0.9999917438488517  \\
            0.16633266533066132  0.9999917438488516  \\
            0.1683366733466934  0.9999917438488515  \\
            0.17034068136272545  0.9999917438488514  \\
            0.17234468937875752  0.999991743848851  \\
            0.1743486973947896  0.9999917438488505  \\
            0.17635270541082165  0.9999917438488496  \\
            0.17835671342685372  0.9999917438488483  \\
            0.18036072144288579  0.9999917438488459  \\
            0.18236472945891782  0.9999917438488422  \\
            0.1843687374749499  0.9999917438488362  \\
            0.18637274549098196  0.999991743848827  \\
            0.18837675350701402  0.9999917438488135  \\
            0.1903807615230461  0.9999917438487947  \\
            0.19238476953907815  0.9999917438487713  \\
            0.19438877755511022  0.9999917438487478  \\
            0.1963927855711423  0.999991743848736  \\
            0.19839679358717435  0.9999917438487573  \\
            0.20040080160320642  0.9999917438488379  \\
            0.20240480961923848  0.9999917438489974  \\
            0.20440881763527055  0.9999917438492368  \\
            0.20641282565130262  0.9999917438495421  \\
            0.20841683366733466  0.9999917438498952  \\
            0.21042084168336672  0.9999917438502828  \\
            0.2124248496993988  0.9999917438506992  \\
            0.21442885771543085  0.9999917438511438  \\
            0.21643286573146292  0.9999917438516189  \\
            0.218436873747495  0.9999917438521282  \\
            0.22044088176352705  0.9999917438526759  \\
            0.22244488977955912  0.9999917438532663  \\
            0.22444889779559118  0.9999917438539038  \\
            0.22645290581162325  0.9999917438545931  \\
            0.22845691382765532  0.9999917438553391  \\
            0.23046092184368738  0.9999917438561465  \\
            0.23246492985971945  0.999991743857021  \\
            0.23446893787575152  0.9999917438579684  \\
            0.23647294589178355  0.9999917438589945  \\
            0.23847695390781562  0.9999917438601061  \\
            0.24048096192384769  0.9999917438613106  \\
            0.24248496993987975  0.9999917438626155  \\
            0.24448897795591182  0.9999917438640292  \\
            0.24649298597194388  0.9999917438655609  \\
            0.24849699398797595  0.9999917438672205  \\
            0.250501002004008  0.9999917438690186  \\
            0.25250501002004005  0.9999917438709667  \\
            0.2545090180360721  0.9999917438730773  \\
            0.2565130260521042  0.9999917438753642  \\
            0.25851703406813625  0.999991743877842  \\
            0.2605210420841683  0.9999917438805265  \\
            0.2625250501002004  0.999991743883435  \\
            0.26452905811623245  0.9999917438865863  \\
            0.2665330661322645  0.9999917438900007  \\
            0.2685370741482966  0.9999917438936999  \\
            0.27054108216432865  0.999991743897708  \\
            0.2725450901803607  0.9999917439020505  \\
            0.2745490981963928  0.9999917439067555  \\
            0.27655310621242485  0.9999917439118532  \\
            0.2785571142284569  0.9999917439173763  \\
            0.280561122244489  0.9999917439233604  \\
            0.28256513026052105  0.9999917439298439  \\
            0.2845691382765531  0.9999917439368685  \\
            0.2865731462925852  0.9999917439444794  \\
            0.28857715430861725  0.9999917439527256  \\
            0.2905811623246493  0.9999917439616599  \\
            0.2925851703406814  0.9999917439713399  \\
            0.29458917835671344  0.9999917439818279  \\
            0.2965931863727455  0.9999917439931912  \\
            0.2985971943887776  0.9999917440055028  \\
            0.30060120240480964  0.999991744018842  \\
            0.3026052104208417  0.9999917440332946  \\
            0.3046092184368738  0.9999917440489534  \\
            0.3066132264529058  0.9999917440659191  \\
            0.30861723446893785  0.9999917440843008  \\
            0.3106212424849699  0.999991744104217  \\
            0.312625250501002  0.9999917441257955  \\
            0.31462925851703405  0.9999917441491755  \\
            0.3166332665330661  0.9999917441745076  \\
            0.3186372745490982  0.9999917442019555  \\
            0.32064128256513025  0.9999917442316969  \\
            0.3226452905811623  0.9999917442639251  \\
            0.3246492985971944  0.9999917442988515  \\
            0.32665330661322645  0.9999917443367073  \\
            0.3286573146292585  0.9999917443777484  \\
            0.3306613226452906  0.99999174442226  \\
            0.33266533066132264  0.9999917444705662  \\
            0.3346693386773547  0.9999917445230436  \\
            0.3366733466933868  0.9999917445801454  \\
            0.33867735470941884  0.9999917446424411  \\
            0.3406813627254509  0.9999917447106837  \\
            0.342685370741483  0.9999917447859251  \\
            0.34468937875751504  0.9999917448697129  \\
            0.3466933867735471  0.9999917449644266  \\
            0.3486973947895792  0.9999917450738512  \\
            0.35070140280561124  0.9999917452041522  \\
            0.3527054108216433  0.9999917453655246  \\
            0.35470941883767537  0.9999917455749691  \\
            0.35671342685370744  0.9999917458609341  \\
            0.3587174348697395  0.9999917462710103  \\
            0.36072144288577157  0.9999917468845599  \\
            0.3627254509018036  0.9999917478331748  \\
            0.36472945891783565  0.9999917493332573  \\
            0.3667334669338677  0.9999917517367148  \\
            0.3687374749498998  0.9999917556072194  \\
            0.37074148296593185  0.9999917618290165  \\
            0.3727454909819639  0.9999917717482832  \\
            0.374749498997996  0.9999917873225113  \\
            0.37675350701402804  0.9999918111869223  \\
            0.3787575150300601  0.9999918463915959  \\
            0.3807615230460922  0.9999918952527759  \\
            0.38276553106212424  0.9999919562965133  \\
            0.3847695390781563  0.9999920180682516  \\
            0.3867735470941884  0.9999920501887678  \\
            0.38877755511022044  0.9999919977716275  \\
            0.3907815631262525  0.9999917921224031  \\
            0.3927855711422846  0.9999913823662137  \\
            0.39478957915831664  0.9999907639374094  \\
            0.3967935871743487  0.9999899734596749  \\
            0.39879759519038077  0.9999890583191463  \\
            0.40080160320641284  0.9999880527095456  \\
            0.4028056112224449  0.9999869721651804  \\
            0.40480961923847697  0.9999858184125128  \\
            0.40681362725450904  0.9999845855276372  \\
            0.4088176352705411  0.9999832640821095  \\
            0.41082164328657317  0.9999818432640034  \\
            0.41282565130260523  0.9999803117418649  \\
            0.4148296593186373  0.9999786578948251  \\
            0.4168336673346693  0.9999768697672158  \\
            0.4188376753507014  0.9999749349231791  \\
            0.42084168336673344  0.9999728402774947  \\
            0.4228456913827655  0.9999705719309537  \\
            0.4248496993987976  0.9999681150174808  \\
            0.42685370741482964  0.9999654535618127  \\
            0.4288577154308617  0.9999625703438585  \\
            0.4308617234468938  0.9999594467655022  \\
            0.43286573146292584  0.9999560627160793  \\
            0.4348697394789579  0.9999523964334058  \\
            0.43687374749499  0.9999484243578232  \\
            0.43887775551102204  0.9999441209771694  \\
            0.4408817635270541  0.9999394586609018  \\
            0.44288577154308617  0.9999344074817879  \\
            0.44488977955911824  0.9999289350236912  \\
            0.4468937875751503  0.9999230061740151  \\
            0.44889779559118237  0.9999165828993576  \\
            0.45090180360721444  0.9999096240028842  \\
            0.4529058116232465  0.9999020848618523  \\
            0.45490981963927857  0.9998939171436235  \\
            0.45691382765531063  0.999885068498391  \\
            0.4589178356713427  0.9998754822267202  \\
            0.46092184368737477  0.9998650969198601  \\
            0.46292585170340683  0.9998538460706334  \\
            0.4649298597194389  0.9998416576525418  \\
            0.46693386773547096  0.999828453664552  \\
            0.46893787575150303  0.9998141496388332  \\
            0.4709418837675351  0.9997986541085148  \\
            0.4729458917835671  0.9997818680323178  \\
            0.4749498997995992  0.9997636841726818  \\
            0.47695390781563124  0.9997439864237666  \\
            0.4789579158316633  0.9997226490854523  \\
            0.48096192384769537  0.999699536079185  \\
            0.48296593186372744  0.9996745001012355  \\
            0.4849699398797595  0.999647381708635  \\
            0.48697394789579157  0.9996180083327435  \\
            0.48897795591182364  0.99958619321508  \\
            0.4909819639278557  0.9995517342597094  \\
            0.49298597194388777  0.9995144127961368  \\
            0.49498997995991983  0.9994739922463111  \\
            0.4969939879759519  0.9994302166889816  \\
            0.49899799599198397  0.9993828093142997  \\
            0.501002004008016  0.9993314707612054  \\
            0.503006012024048  0.9992758773298005  \\
            0.5050100200400801  0.9992156790605844  \\
            0.5070140280561122  0.9991504976721364  \\
            0.5090180360721442  0.9990799243485624  \\
            0.5110220440881763  0.9990035173678153  \\
            0.5130260521042084  0.9989207995618441  \\
            0.5150300601202404  0.9988312555994594  \\
            0.5170340681362725  0.9987343290828286  \\
            0.5190380761523046  0.9986294194486641  \\
            0.5210420841683366  0.998515878665468  \\
            0.5230460921843687  0.9983930077186733  \\
            0.5250501002004008  0.9982600528762142  \\
            0.5270541082164328  0.9981162017280041  \\
            0.5290581162324649  0.9979605789940426  \\
            0.531062124248497  0.9977922420974695  \\
            0.533066132264529  0.9976101765008748  \\
            0.5350701402805611  0.9974132908066395  \\
            0.5370741482965932  0.9972004116250747  \\
            0.5390781563126252  0.996970278217719  \\
            0.5410821643286573  0.9967215369274324  \\
            0.5430861723446894  0.9964527354119478  \\
            0.5450901803607214  0.9961623167034146  \\
            0.5470941883767535  0.9958486131232401  \\
            0.5490981963927856  0.9955098400893051  \\
            0.5511022044088176  0.9951440898614492  \\
            0.5531062124248497  0.9947493252810405  \\
            0.5551102204408818  0.9943233735715047  \\
            0.5571142284569138  0.9938639202788789  \\
            0.5591182364729459  0.9933685034447535  \\
            0.561122244488978  0.9928345081182763  \\
            0.56312625250501  0.9922591613290888  \\
            0.5651302605210421  0.9916395276589064  \\
            0.5671342685370742  0.9909725055656766  \\
            0.5691382765531062  0.9902548246304038  \\
            0.5711422845691383  0.9894830439123449  \\
            0.5731462925851704  0.988653551612696  \\
            0.5751503006012024  0.9877625662593388  \\
            0.5771543086172345  0.9868061396348167  \\
            0.5791583166332666  0.9857801616754216  \\
            0.5811623246492986  0.9846803675699871  \\
            0.5831663326653307  0.9835023472814779  \\
            0.5851703406813628  0.9822415577015162  \\
            0.5871743486973948  0.9808933376263822  \\
            0.5891783567134269  0.9794529257116517  \\
            0.591182364729459  0.9779154815206206  \\
            0.593186372745491  0.9762761097283849  \\
            0.5951903807615231  0.9745298874787702  \\
            0.5971943887775552  0.9726718948156022  \\
            0.5991983967935872  0.9706972480241295  \\
            0.6012024048096193  0.9686011356245516  \\
            0.6032064128256514  0.9663788566601441  \\
            0.6052104208416834  0.9640258608208169  \\
            0.6072144288577155  0.9615377898432133  \\
            0.6092184368737475  0.9589105195353864  \\
            0.6112224448897795  0.9561402016928051  \\
            0.6132264529058116  0.9532233051081527  \\
            0.6152304609218436  0.950156654835061  \\
            0.6172344689378757  0.9469374688498928  \\
            0.6192384769539078  0.9435633912692598  \\
            0.6212424849699398  0.9400325213260304  \\
            0.6232464929859719  0.9363434373834191  \\
            0.625250501002004  0.9324952153737136  \\
            0.627254509018036  0.9284874411818684  \\
            0.6292585170340681  0.9243202166493123  \\
            0.6312625250501002  0.9199941590432499  \\
            0.6332665330661322  0.9155103940137355  \\
            0.6352705410821643  0.9108705422365904  \\
            0.6372745490981964  0.9060767001065669  \\
            0.6392785571142284  0.9011314149942207  \\
            0.6412825651302605  0.8960376557048755  \\
            0.6432865731462926  0.8907987788730638  \\
            0.6452905811623246  0.8854184920861172  \\
            0.6472945891783567  0.8799008145519959  \\
            0.6492985971943888  0.8742500361041871  \\
            0.6513026052104208  0.8684706752634461  \\
            0.6533066132264529  0.8625674369392715  \\
            0.655310621242485  0.856545170129053  \\
            0.657314629258517  0.8504088256148399  \\
            0.6593186372745491  0.8441634130854939  \\
            0.6613226452905812  0.8378139561820216  \\
            0.6633266533066132  0.831365442426389  \\
            0.6653306613226453  0.824822762420272  \\
            0.6673346693386774  0.8181906283620364  \\
            0.6693386773547094  0.8114734546053304  \\
            0.6713426853707415  0.8046751706433245  \\
            0.6733466933867736  0.7977989162373298  \\
            0.6753507014028056  0.7908465341279126  \\
            0.6773547094188377  0.7838177197155416  \\
            0.6793587174348698  0.7767085974805796  \\
            0.6813627254509018  0.7695093558792458  \\
            0.6833667334669339  0.7622003741307786  \\
            0.685370741482966  0.7547460304420233  \\
            0.687374749498998  0.7470852080429786  \\
            0.6893787575150301  0.7391178411316053  \\
            0.6913827655310621  0.7306889243909616  \\
            0.6933867735470942  0.721578301281462  \\
            0.6953907815631263  0.711521561600192  \\
            0.6973947895791583  0.7003156497508735  \\
            0.6993987975951904  0.6880685735921512  \\
            0.7014028056112225  0.6755234980583985  \\
            0.7034068136272545  0.6640776608734429  \\
            0.7054108216432866  0.6551245237235697  \\
            0.7074148296593187  0.6492002348268464  \\
            0.7094188376753507  0.6458657611103056  \\
            0.7114228456913828  0.6442655190580175  \\
            0.7134268537074149  0.6436308670667082  \\
            0.7154308617234469  0.643446272109572  \\
            0.717434869739479  0.64342485264604  \\
            0.7194388777555111  0.6434388870857466  \\
            0.7214428857715431  0.6434495293934724  \\
            0.7234468937875751  0.6434528529028823  \\
            0.7254509018036072  0.6434523588445773  \\
            0.7274549098196392  0.6434507492075339  \\
            0.7294589178356713  0.6434491907128141  \\
            0.7314629258517034  0.6434480064225732  \\
            0.7334669338677354  0.6434471948936554  \\
            0.7354709418837675  0.6434466682755972  \\
            0.7374749498997996  0.6434463373774428  \\
            0.7394789579158316  0.6434461337338258  \\
            0.7414829659318637  0.6434460101978247  \\
            0.7434869739478958  0.6434459360488323  \\
            0.7454909819639278  0.6434458919081008  \\
            0.7474949899799599  0.6434458658054527  \\
            0.749498997995992  0.6434458504549081  \\
            0.751503006012024  0.6434458414700046  \\
            0.7535070140280561  0.6434458362324715  \\
            0.7555110220440882  0.6434458331903208  \\
            0.7575150300601202  0.6434458314289464  \\
            0.7595190380761523  0.6434458304120283  \\
            0.7615230460921844  0.6434458298264201  \\
            0.7635270541082164  0.6434458294899698  \\
            0.7655310621242485  0.6434458292970762  \\
            0.7675350701402806  0.6434458291866998  \\
            0.7695390781563126  0.6434458291236526  \\
            0.7715430861723447  0.6434458290876984  \\
            0.7735470941883767  0.6434458290672256  \\
            0.7755511022044088  0.6434458290555842  \\
            0.7775551102204409  0.6434458290489734  \\
            0.779559118236473  0.6434458290452238  \\
            0.781563126252505  0.6434458290430993  \\
            0.7835671342685371  0.643445829041897  \\
            0.7855711422845691  0.6434458290412173  \\
            0.7875751503006012  0.6434458290408333  \\
            0.7895791583166333  0.6434458290406164  \\
            0.7915831663326653  0.6434458290404943  \\
            0.7935871743486974  0.6434458290404255  \\
            0.7955911823647295  0.6434458290403866  \\
            0.7975951903807615  0.6434458290403647  \\
            0.7995991983967936  0.6434458290403525  \\
            0.8016032064128257  0.6434458290403458  \\
            0.8036072144288577  0.6434458290403418  \\
            0.8056112224448898  0.6434458290403396  \\
            0.8076152304609219  0.6434458290403384  \\
            0.8096192384769539  0.6434458290403378  \\
            0.811623246492986  0.6434458290403373  \\
            0.8136272545090181  0.6434458290403371  \\
            0.8156312625250501  0.6434458290403369  \\
            0.8176352705410822  0.6434458290403369  \\
            0.8196392785571143  0.6434458290403369  \\
            0.8216432865731463  0.6434458290403368  \\
            0.8236472945891784  0.6434458290403369  \\
            0.8256513026052105  0.6434458290403369  \\
            0.8276553106212425  0.6434458290403369  \\
            0.8296593186372746  0.6434458290403369  \\
            0.8316633266533067  0.6434458290403369  \\
            0.8336673346693386  0.6434458290403369  \\
            0.8356713426853707  0.6434458290403369  \\
            0.8376753507014028  0.6434458290403369  \\
            0.8396793587174348  0.6434458290403369  \\
            0.8416833667334669  0.6434458290403369  \\
            0.843687374749499  0.6434458290403369  \\
            0.845691382765531  0.6434458290403369  \\
            0.8476953907815631  0.6434458290403369  \\
            0.8496993987975952  0.6434458290403369  \\
            0.8517034068136272  0.6434458290403369  \\
            0.8537074148296593  0.6434458290403369  \\
            0.8557114228456913  0.6434458290403369  \\
            0.8577154308617234  0.6434458290403369  \\
            0.8597194388777555  0.6434458290403369  \\
            0.8617234468937875  0.6434458290403369  \\
            0.8637274549098196  0.6434458290403369  \\
            0.8657314629258517  0.6434458290403369  \\
            0.8677354709418837  0.6434458290403369  \\
            0.8697394789579158  0.6434458290403369  \\
            0.8717434869739479  0.6434458290403369  \\
            0.87374749498998  0.6434458290403369  \\
            0.875751503006012  0.6434458290403369  \\
            0.8777555110220441  0.6434458290403369  \\
            0.8797595190380761  0.6434458290403369  \\
            0.8817635270541082  0.6434458290403369  \\
            0.8837675350701403  0.6434458290403369  \\
            0.8857715430861723  0.6434458290403369  \\
            0.8877755511022044  0.6434458290403369  \\
            0.8897795591182365  0.6434458290403369  \\
            0.8917835671342685  0.6434458290403369  \\
            0.8937875751503006  0.6434458290403369  \\
            0.8957915831663327  0.6434458290403369  \\
            0.8977955911823647  0.6434458290403369  \\
            0.8997995991983968  0.6434458290403369  \\
            0.9018036072144289  0.6434458290403369  \\
            0.9038076152304609  0.6434458290403369  \\
            0.905811623246493  0.6434458290403369  \\
            0.9078156312625251  0.6434458290403369  \\
            0.9098196392785571  0.6434458290403369  \\
            0.9118236472945892  0.6434458290403369  \\
            0.9138276553106213  0.6434458290403369  \\
            0.9158316633266533  0.6434458290403369  \\
            0.9178356713426854  0.6434458290403369  \\
            0.9198396793587175  0.6434458290403369  \\
            0.9218436873747495  0.6434458290403369  \\
            0.9238476953907816  0.6434458290403369  \\
            0.9258517034068137  0.6434458290403369  \\
            0.9278557114228457  0.6434458290403369  \\
            0.9298597194388778  0.6434458290403369  \\
            0.9318637274549099  0.6434458290403369  \\
            0.9338677354709419  0.6434458290403369  \\
            0.935871743486974  0.6434458290403369  \\
            0.9378757515030061  0.6434458290403369  \\
            0.9398797595190381  0.6434458290403369  \\
            0.9418837675350702  0.6434458290403369  \\
            0.9438877755511023  0.6434458290403369  \\
            0.9458917835671342  0.6434458290403369  \\
            0.9478957915831663  0.6434458290403369  \\
            0.9498997995991983  0.6434458290403369  \\
            0.9519038076152304  0.6434458290403369  \\
            0.9539078156312625  0.6434458290403369  \\
            0.9559118236472945  0.6434458290403369  \\
            0.9579158316633266  0.6434458290403369  \\
            0.9599198396793587  0.6434458290403369  \\
            0.9619238476953907  0.6434458290403369  \\
            0.9639278557114228  0.6434458290403369  \\
            0.9659318637274549  0.6434458290403369  \\
            0.9679358717434869  0.6434458290403369  \\
            0.969939879759519  0.6434458290403369  \\
            0.9719438877755511  0.6434458290403369  \\
            0.9739478957915831  0.6434458290403369  \\
            0.9759519038076152  0.6434458290403369  \\
            0.9779559118236473  0.6434458290403369  \\
            0.9799599198396793  0.6434458290403369  \\
            0.9819639278557114  0.6434458290403369  \\
            0.9839679358717435  0.6434458290403369  \\
            0.9859719438877755  0.6434458290403369  \\
            0.9879759519038076  0.6434458290403369  \\
            0.9899799599198397  0.6434458290403369  \\
            0.9919839679358717  0.6434458290403369  \\
            0.9939879759519038  0.6434458290403369  \\
            0.9959919839679359  0.6434458290403369  \\
            0.9979959919839679  0.6434458290403369  \\
            1.0  0.6434458290403369  \\
        }
        ;
    \node[right, , color={rgb,1:red,0.0;green,0.3608;blue,0.6706}, draw opacity={1.0}, rotate={0.0}, font={{\fontsize{8 pt}{10.4 pt}\selectfont}}]  at (axis cs:0.05,0.95) {Nominal};
    \node[right, , color={rgb,1:red,0.7529;green,0.3255;blue,0.4039}, draw opacity={1.0}, rotate={0.0}, font={{\fontsize{8 pt}{10.4 pt}\selectfont}}]  at (axis cs:0.05,0.9) {Smoothed};
\end{axis}
\end{tikzpicture}

        % \label{fig:cltranslim-smoothed}
     \end{subfigure}
     \hfill
     \begin{subfigure}[t]{0.45\textwidth}
         \centering
         % Recommended preamble:
% \usetikzlibrary{arrows.meta}
% \usetikzlibrary{backgrounds}
% \usepgfplotslibrary{patchplots}
% \usepgfplotslibrary{fillbetween}
% \pgfplotsset{%
%     layers/standard/.define layer set={%
%         background,axis background,axis grid,axis ticks,axis lines,axis tick labels,pre main,main,axis descriptions,axis foreground%
%     }{
%         grid style={/pgfplots/on layer=axis grid},%
%         tick style={/pgfplots/on layer=axis ticks},%
%         axis line style={/pgfplots/on layer=axis lines},%
%         label style={/pgfplots/on layer=axis descriptions},%
%         legend style={/pgfplots/on layer=axis descriptions},%
%         title style={/pgfplots/on layer=axis descriptions},%
%         colorbar style={/pgfplots/on layer=axis descriptions},%
%         ticklabel style={/pgfplots/on layer=axis tick labels},%
%         axis background@ style={/pgfplots/on layer=axis background},%
%         3d box foreground style={/pgfplots/on layer=axis foreground},%
%     },
% }

\begin{tikzpicture}[/tikz/background rectangle/.style={fill={rgb,1:red,1.0;green,1.0;blue,1.0}, fill opacity={1.0}, draw opacity={1.0}}, show background rectangle]
\begin{axis}[point meta max={nan}, point meta min={nan}, legend cell align={left}, legend columns={1}, title={}, title style={at={{(0.5,1)}}, anchor={south}, font={{\fontsize{14 pt}{18.2 pt}\selectfont}}, color={rgb,1:red,0.0;green,0.0;blue,0.0}, draw opacity={1.0}, rotate={0.0}, align={center}}, legend style={color={rgb,1:red,0.0;green,0.0;blue,0.0}, draw opacity={0.0}, line width={1}, solid, fill={rgb,1:red,0.0;green,0.0;blue,0.0}, fill opacity={0.0}, text opacity={1.0}, font={{\fontsize{8 pt}{10.4 pt}\selectfont}}, text={rgb,1:red,0.0;green,0.0;blue,0.0}, cells={anchor={center}}, at={(1.02, 1)}, anchor={north west}}, axis background/.style={fill={rgb,1:red,0.0;green,0.0;blue,0.0}, opacity={0.0}}, anchor={north west}, xshift={0.0mm}, yshift={-0.0mm}, width={45.8mm}, height={50.8mm}, scaled x ticks={false}, xlabel={Mach Number}, x tick style={color={rgb,1:red,0.0;green,0.0;blue,0.0}, opacity={1.0}}, x tick label style={color={rgb,1:red,0.0;green,0.0;blue,0.0}, opacity={1.0}, rotate={0}}, xlabel style={at={(ticklabel cs:0.5)}, anchor=near ticklabel, at={{(ticklabel cs:0.5)}}, anchor={near ticklabel}, font={{\fontsize{11 pt}{14.3 pt}\selectfont}}, color={rgb,1:red,0.0;green,0.0;blue,0.0}, draw opacity={1.0}, rotate={0.0}}, xmajorgrids={false}, xmin={-0.030000000000000027}, xmax={1.03}, xticklabels={{$0.00$,$0.25$,$0.50$,$0.75$,$1.00$}}, xtick={{0.0,0.25,0.5,0.75,1.0}}, xtick align={inside}, xticklabel style={font={{\fontsize{8 pt}{10.4 pt}\selectfont}}, color={rgb,1:red,0.0;green,0.0;blue,0.0}, draw opacity={1.0}, rotate={0.0}}, x grid style={color={rgb,1:red,0.0;green,0.0;blue,0.0}, draw opacity={0.1}, line width={0.5}, solid}, axis x line*={left}, x axis line style={color={rgb,1:red,0.0;green,0.0;blue,0.0}, draw opacity={1.0}, line width={1}, solid}, scaled y ticks={false}, ylabel={$c_{d_\mathrm{lim}}$}, y tick style={color={rgb,1:red,0.0;green,0.0;blue,0.0}, opacity={1.0}}, y tick label style={color={rgb,1:red,0.0;green,0.0;blue,0.0}, opacity={1.0}, rotate={0}}, ylabel style={at={(ticklabel cs:0.5)}, anchor=near ticklabel, at={{(ticklabel cs:0.5)}}, anchor={near ticklabel}, font={{\fontsize{11 pt}{14.3 pt}\selectfont}}, color={rgb,1:red,0.0;green,0.0;blue,0.0}, draw opacity={1.0}, rotate={0.0}}, ymajorgrids={false}, ymin={0.9455182151822091}, ymax={2.870538945855386}, yticklabels={{$1.0$,$1.5$,$2.0$,$2.5$}}, ytick={{1.0,1.5,2.0,2.5}}, ytick align={inside}, yticklabel style={font={{\fontsize{8 pt}{10.4 pt}\selectfont}}, color={rgb,1:red,0.0;green,0.0;blue,0.0}, draw opacity={1.0}, rotate={0.0}}, y grid style={color={rgb,1:red,0.0;green,0.0;blue,0.0}, draw opacity={0.1}, line width={0.5}, solid}, axis y line*={left}, y axis line style={color={rgb,1:red,0.0;green,0.0;blue,0.0}, draw opacity={1.0}, line width={1}, solid}, colorbar={false}]
    \addplot[color={rgb,1:red,0.0;green,0.3608;blue,0.6706}, name path={188d8285-9849-40ee-a6d6-9c958f9ec7d3}, draw opacity={1.0}, line width={1.0}, solid, forget plot]
        table[row sep={\\}]
        {
            \\
            0.0  1.0  \\
            0.002004008016032064  1.0  \\
            0.004008016032064128  1.0  \\
            0.006012024048096192  1.0  \\
            0.008016032064128256  1.0  \\
            0.01002004008016032  1.0  \\
            0.012024048096192385  1.0  \\
            0.014028056112224449  1.0  \\
            0.01603206412825651  1.0  \\
            0.018036072144288578  1.0  \\
            0.02004008016032064  1.0  \\
            0.022044088176352707  1.0  \\
            0.02404809619238477  1.0  \\
            0.026052104208416832  1.0  \\
            0.028056112224448898  1.0  \\
            0.03006012024048096  1.0  \\
            0.03206412825651302  1.0  \\
            0.03406813627254509  1.0  \\
            0.036072144288577156  1.0  \\
            0.03807615230460922  1.0  \\
            0.04008016032064128  1.0  \\
            0.04208416833667335  1.0  \\
            0.04408817635270541  1.0  \\
            0.04609218436873747  1.0  \\
            0.04809619238476954  1.0  \\
            0.050100200400801605  1.0  \\
            0.052104208416833664  1.0  \\
            0.05410821643286573  1.0  \\
            0.056112224448897796  1.0  \\
            0.05811623246492986  1.0  \\
            0.06012024048096192  1.0  \\
            0.06212424849699399  1.0  \\
            0.06412825651302605  1.0  \\
            0.06613226452905811  1.0  \\
            0.06813627254509018  1.0  \\
            0.07014028056112225  1.0  \\
            0.07214428857715431  1.0  \\
            0.07414829659318638  1.0  \\
            0.07615230460921844  1.0  \\
            0.0781563126252505  1.0  \\
            0.08016032064128256  1.0  \\
            0.08216432865731463  1.0  \\
            0.0841683366733467  1.0  \\
            0.08617234468937876  1.0  \\
            0.08817635270541083  1.0  \\
            0.09018036072144289  1.0  \\
            0.09218436873747494  1.0  \\
            0.09418837675350701  1.0  \\
            0.09619238476953908  1.0  \\
            0.09819639278557114  1.0  \\
            0.10020040080160321  1.0  \\
            0.10220440881763528  1.0  \\
            0.10420841683366733  1.0  \\
            0.1062124248496994  1.0  \\
            0.10821643286573146  1.0  \\
            0.11022044088176353  1.0  \\
            0.11222444889779559  1.0  \\
            0.11422845691382766  1.0  \\
            0.11623246492985972  1.0  \\
            0.11823647294589178  1.0  \\
            0.12024048096192384  1.0  \\
            0.12224448897795591  1.0  \\
            0.12424849699398798  1.0  \\
            0.12625250501002003  1.0  \\
            0.1282565130260521  1.0  \\
            0.13026052104208416  1.0  \\
            0.13226452905811623  1.0  \\
            0.1342685370741483  1.0  \\
            0.13627254509018036  1.0  \\
            0.13827655310621242  1.0  \\
            0.1402805611222445  1.0  \\
            0.14228456913827656  1.0  \\
            0.14428857715430862  1.0  \\
            0.1462925851703407  1.0  \\
            0.14829659318637275  1.0  \\
            0.15030060120240482  1.0  \\
            0.1523046092184369  1.0  \\
            0.15430861723446893  1.0  \\
            0.156312625250501  1.0  \\
            0.15831663326653306  1.0  \\
            0.16032064128256512  1.0  \\
            0.1623246492985972  1.0  \\
            0.16432865731462926  1.0  \\
            0.16633266533066132  1.0  \\
            0.1683366733466934  1.0  \\
            0.17034068136272545  1.0  \\
            0.17234468937875752  1.0  \\
            0.1743486973947896  1.0  \\
            0.17635270541082165  1.0  \\
            0.17835671342685372  1.0  \\
            0.18036072144288579  1.0  \\
            0.18236472945891782  1.0  \\
            0.1843687374749499  1.0  \\
            0.18637274549098196  1.0  \\
            0.18837675350701402  1.0  \\
            0.1903807615230461  1.0  \\
            0.19238476953907815  1.0  \\
            0.19438877755511022  1.0  \\
            0.1963927855711423  1.0  \\
            0.19839679358717435  1.0  \\
            0.20040080160320642  1.0  \\
            0.20240480961923848  1.0  \\
            0.20440881763527055  1.0  \\
            0.20641282565130262  1.0  \\
            0.20841683366733466  1.0  \\
            0.21042084168336672  1.0  \\
            0.2124248496993988  1.0  \\
            0.21442885771543085  1.0  \\
            0.21643286573146292  1.0  \\
            0.218436873747495  1.0  \\
            0.22044088176352705  1.0  \\
            0.22244488977955912  1.0  \\
            0.22444889779559118  1.0  \\
            0.22645290581162325  1.0  \\
            0.22845691382765532  1.0  \\
            0.23046092184368738  1.0  \\
            0.23246492985971945  1.0  \\
            0.23446893787575152  1.0  \\
            0.23647294589178355  1.0  \\
            0.23847695390781562  1.0  \\
            0.24048096192384769  1.0  \\
            0.24248496993987975  1.0  \\
            0.24448897795591182  1.0  \\
            0.24649298597194388  1.0  \\
            0.24849699398797595  1.0  \\
            0.250501002004008  1.0  \\
            0.25250501002004005  1.0  \\
            0.2545090180360721  1.0  \\
            0.2565130260521042  1.0  \\
            0.25851703406813625  1.0  \\
            0.2605210420841683  1.0  \\
            0.2625250501002004  1.0  \\
            0.26452905811623245  1.0  \\
            0.2665330661322645  1.0  \\
            0.2685370741482966  1.0  \\
            0.27054108216432865  1.0  \\
            0.2725450901803607  1.0  \\
            0.2745490981963928  1.0  \\
            0.27655310621242485  1.0  \\
            0.2785571142284569  1.0  \\
            0.280561122244489  1.0  \\
            0.28256513026052105  1.0  \\
            0.2845691382765531  1.0  \\
            0.2865731462925852  1.0  \\
            0.28857715430861725  1.0  \\
            0.2905811623246493  1.0  \\
            0.2925851703406814  1.0  \\
            0.29458917835671344  1.0  \\
            0.2965931863727455  1.0  \\
            0.2985971943887776  1.0  \\
            0.30060120240480964  1.0  \\
            0.3026052104208417  1.0  \\
            0.3046092184368738  1.0  \\
            0.3066132264529058  1.0  \\
            0.30861723446893785  1.0  \\
            0.3106212424849699  1.0  \\
            0.312625250501002  1.0  \\
            0.31462925851703405  1.0  \\
            0.3166332665330661  1.0  \\
            0.3186372745490982  1.0  \\
            0.32064128256513025  1.0  \\
            0.3226452905811623  1.0  \\
            0.3246492985971944  1.0  \\
            0.32665330661322645  1.0  \\
            0.3286573146292585  1.0  \\
            0.3306613226452906  1.0  \\
            0.33266533066132264  1.0  \\
            0.3346693386773547  1.0  \\
            0.3366733466933868  1.0  \\
            0.33867735470941884  1.0  \\
            0.3406813627254509  1.0  \\
            0.342685370741483  1.0  \\
            0.34468937875751504  1.0  \\
            0.3466933867735471  1.0  \\
            0.3486973947895792  1.0  \\
            0.35070140280561124  1.0  \\
            0.3527054108216433  1.0  \\
            0.35470941883767537  1.0  \\
            0.35671342685370744  1.0  \\
            0.3587174348697395  1.0  \\
            0.36072144288577157  1.0  \\
            0.3627254509018036  1.0  \\
            0.36472945891783565  1.0  \\
            0.3667334669338677  1.0  \\
            0.3687374749498998  1.0  \\
            0.37074148296593185  1.0  \\
            0.3727454909819639  1.0  \\
            0.374749498997996  1.0  \\
            0.37675350701402804  1.0  \\
            0.3787575150300601  1.0  \\
            0.3807615230460922  1.0  \\
            0.38276553106212424  1.0  \\
            0.3847695390781563  1.0  \\
            0.3867735470941884  1.0  \\
            0.38877755511022044  1.0  \\
            0.3907815631262525  1.0  \\
            0.3927855711422846  1.0  \\
            0.39478957915831664  1.0  \\
            0.3967935871743487  1.0  \\
            0.39879759519038077  1.0  \\
            0.40080160320641284  1.0  \\
            0.4028056112224449  1.0  \\
            0.40480961923847697  1.0  \\
            0.40681362725450904  1.0  \\
            0.4088176352705411  1.0  \\
            0.41082164328657317  1.0  \\
            0.41282565130260523  1.0  \\
            0.4148296593186373  1.0  \\
            0.4168336673346693  1.0  \\
            0.4188376753507014  1.0  \\
            0.42084168336673344  1.0  \\
            0.4228456913827655  1.0  \\
            0.4248496993987976  1.0  \\
            0.42685370741482964  1.0  \\
            0.4288577154308617  1.0  \\
            0.4308617234468938  1.0  \\
            0.43286573146292584  1.0  \\
            0.4348697394789579  1.0  \\
            0.43687374749499  1.0  \\
            0.43887775551102204  1.0  \\
            0.4408817635270541  1.0  \\
            0.44288577154308617  1.0  \\
            0.44488977955911824  1.0  \\
            0.4468937875751503  1.0  \\
            0.44889779559118237  1.0  \\
            0.45090180360721444  1.0  \\
            0.4529058116232465  1.0  \\
            0.45490981963927857  1.0  \\
            0.45691382765531063  1.0  \\
            0.4589178356713427  1.0  \\
            0.46092184368737477  1.0  \\
            0.46292585170340683  1.0  \\
            0.4649298597194389  1.0  \\
            0.46693386773547096  1.0  \\
            0.46893787575150303  1.0  \\
            0.4709418837675351  1.0  \\
            0.4729458917835671  1.0  \\
            0.4749498997995992  1.0  \\
            0.47695390781563124  1.0  \\
            0.4789579158316633  1.0  \\
            0.48096192384769537  1.0  \\
            0.48296593186372744  1.0  \\
            0.4849699398797595  1.0  \\
            0.48697394789579157  1.0  \\
            0.48897795591182364  1.0  \\
            0.4909819639278557  1.0  \\
            0.49298597194388777  1.0  \\
            0.49498997995991983  1.0  \\
            0.4969939879759519  1.0  \\
            0.49899799599198397  1.0  \\
            0.501002004008016  1.0329105552947742  \\
            0.503006012024048  1.0342492636763854  \\
            0.5050100200400801  1.0356229862288189  \\
            0.5070140280561122  1.0370320646016882  \\
            0.5090180360721442  1.0384768261988626  \\
            0.5110220440881763  1.0399575828038528  \\
            0.5130260521042084  1.0414746290814172  \\
            0.5150300601202404  1.0430282409457057  \\
            0.5170340681362725  1.0446186737847556  \\
            0.5190380761523046  1.0462461605306992  \\
            0.5210420841683366  1.0479109095646235  \\
            0.5230460921843687  1.0496131024446682  \\
            0.5250501002004008  1.0513528914456844  \\
            0.5270541082164328  1.0531303968986234  \\
            0.5290581162324649  1.0549457043178079  \\
            0.531062124248497  1.0567988613044108  \\
            0.533066132264529  1.0586898742148567  \\
            0.5350701402805611  1.0606187045835078  \\
            0.5370741482965932  1.0625852652899852  \\
            0.5390781563126252  1.0645894164628333  \\
            0.5410821643286573  1.0666309611130438  \\
            0.5430861723446894  1.0687096404933043  \\
            0.5450901803607214  1.0708251291817736  \\
            0.5470941883767535  1.0729770298928272  \\
            0.5490981963927856  1.0751648680216488  \\
            0.5511022044088176  1.0773880859348341  \\
            0.5531062124248497  1.0796460370254581  \\
            0.5551102204408818  1.081937979558376  \\
            0.5571142284569138  1.0842630703400018  \\
            0.5591182364729459  1.0866203582564595  \\
            0.561122244488978  1.089008777734894  \\
            0.56312625250501  1.0914271421948527  \\
            0.5651302605210421  1.0938741375699537  \\
            0.5671342685370742  1.0963483159944614  \\
            0.5691382765531062  1.0988480897646993  \\
            0.5711422845691383  1.1013717257011977  \\
            0.5731462925851704  1.1039173400537434  \\
            0.5751503006012024  1.10648289410758  \\
            0.5771543086172345  1.1090661906643278  \\
            0.5791583166332666  1.111664871584994  \\
            0.5811623246492986  1.1142764165938972  \\
            0.5831663326653307  1.1168981435504268  \\
            0.5851703406813628  1.1195272103992342  \\
            0.5871743486973948  1.1221606190075462  \\
            0.5891783567134269  1.1247952210896275  \\
            0.591182364729459  1.12742772640189  \\
            0.593186372745491  1.1300547133667296  \\
            0.5951903807615231  1.1326726422481288  \\
            0.5971943887775552  1.1352778709569242  \\
            0.5991983967935872  1.137866673508441  \\
            0.6012024048096193  1.140435261090471  \\
            0.6032064128256514  1.14297980562653  \\
            0.6052104208416834  1.1454964656398485  \\
            0.6072144288577155  1.147981414140234  \\
            0.6092184368737475  1.150430868172037  \\
            0.6112224448897795  1.1528411195808017  \\
            0.6132264529058116  1.1552085664829623  \\
            0.6152304609218436  1.1575297448614983  \\
            0.6172344689378757  1.159801359664997  \\
            0.6192384769539078  1.1620203147617583  \\
            0.6212424849699398  1.16418374109735  \\
            0.6232464929859719  1.1662890224251605  \\
            0.625250501002004  1.1683338180254688  \\
            0.627254509018036  1.1703160818983482  \\
            0.6292585170340681  1.1722340780068645  \\
            0.6312625250501002  1.1740863912556068  \\
            0.6332665330661322  1.1758719340106372  \\
            0.6352705410821643  1.1775899480945762  \\
            0.6372745490981964  1.1792400023185519  \\
            0.6392785571142284  1.180821985734846  \\
            0.6412825651302605  1.1823360969044059  \\
            0.6432865731462926  1.1837828295667032  \\
            0.6452905811623246  1.185162955171211  \\
            0.6472945891783567  1.1864775027761243  \\
            0.6492985971943888  1.187727736837044  \\
            0.6513026052104208  1.1889151333914278  \\
            0.6533066132264529  1.1900413550860836  \\
            0.655310621242485  1.191108225381442  \\
            0.657314629258517  1.1921177020731113  \\
            0.6593186372745491  1.193071849953736  \\
            0.6613226452905812  1.193972811917236  \\
            0.6633266533066132  1.1948227769445532  \\
            0.6653306613226453  1.1956239419664372  \\
            0.6673346693386774  1.1963784621682283  \\
            0.6693386773547094  1.1970883802004733  \\
            0.6713426853707415  1.19775551785219  \\
            0.6733466933867736  1.198381302175425  \\
            0.6753507014028056  1.1989664788579368  \\
            0.6773547094188377  1.199510634266273  \\
            0.6793587174348698  1.2000113974645166  \\
            0.6813627254509018  1.2004631165193034  \\
            0.6833667334669339  1.2008546934812891  \\
            0.685370741482966  1.2011661296295297  \\
            0.687374749498998  1.2013632467781064  \\
            0.6893787575150301  1.2013902626452824  \\
            0.6913827655310621  1.2011611314457702  \\
            0.6933867735470942  1.200554527150885  \\
            0.6953907815631263  1.1994268591244996  \\
            0.6973947895791583  1.19775998262362  \\
            0.6993987975951904  1.1965793655834611  \\
            0.7014028056112225  1.196625129633777  \\
            0.7034068136272545  1.1982553865262364  \\
            0.7054108216432866  1.2013114715530597  \\
            0.7074148296593187  1.2052660761241556  \\
            0.7094188376753507  1.2096463252190033  \\
            0.7114228456913828  1.214209334582183  \\
            0.7134268537074149  1.218870016387981  \\
            0.7154308617234469  1.2236056988537793  \\
            0.717434869739479  1.2284112134760405  \\
            0.7194388777555111  1.2332854916602924  \\
            0.7214428857715431  1.2382285880832635  \\
            0.7234468937875751  1.2432409364321662  \\
            0.7254509018036072  1.2483230572778725  \\
            0.7274549098196392  1.2534754644841741  \\
            0.7294589178356713  1.2586986547859793  \\
            0.7314629258517034  1.2639931151172148  \\
            0.7334669338677354  1.2693593285214892  \\
            0.7354709418837675  1.274797776943568  \\
            0.7374749498997996  1.2803089422584044  \\
            0.7394789579158316  1.2858933065563836  \\
            0.7414829659318637  1.2915513521665405  \\
            0.7434869739478958  1.297283561607668  \\
            0.7454909819639278  1.3030904175324758  \\
            0.7474949899799599  1.3089724026827643  \\
            0.749498997995992  1.3149299998576052  \\
            0.751503006012024  1.3209636918920804  \\
            0.7535070140280561  1.3270739616435638  \\
            0.7555110220440882  1.3332612919830718  \\
            0.7575150300601202  1.3395261657898914  \\
            0.7595190380761523  1.3458690659482877  \\
            0.7615230460921844  1.3522904753455052  \\
            0.7635270541082164  1.3587908768705588  \\
            0.7655310621242485  1.3653707534135133  \\
            0.7675350701402806  1.3720305878650534  \\
            0.7695390781563126  1.378770863116226  \\
            0.7715430861723447  1.3855920620582933  \\
            0.7735470941883767  1.3924946675826404  \\
            0.7755511022044088  1.3994791625807257  \\
            0.7775551102204409  1.4065460299440498  \\
            0.779559118236473  1.4136957525641376  \\
            0.781563126252505  1.4209288133325284  \\
            0.7835671342685371  1.4282456951407696  \\
            0.7855711422845691  1.4356468808804133  \\
            0.7875751503006012  1.4431328534430141  \\
            0.7895791583166333  1.4507040957201291  \\
            0.7915831663326653  1.4583610906033149  \\
            0.7935871743486974  1.4661043209841293  \\
            0.7955911823647295  1.4739342697541309  \\
            0.7975951903807615  1.481851419804878  \\
            0.7995991983967936  1.4898562540279285  \\
            0.8016032064128257  1.497949255314841  \\
            0.8036072144288577  1.5061309065571744  \\
            0.8056112224448898  1.5144016906464866  \\
            0.8076152304609219  1.5227620904743366  \\
            0.8096192384769539  1.5312125889322825  \\
            0.811623246492986  1.539753668911883  \\
            0.8136272545090181  1.5483858133046968  \\
            0.8156312625250501  1.557109505002282  \\
            0.8176352705410822  1.5659252268961974  \\
            0.8196392785571143  1.5748334618780016  \\
            0.8216432865731463  1.5838346928392526  \\
            0.8236472945891784  1.5929294026715095  \\
            0.8256513026052105  1.6021180742663308  \\
            0.8276553106212425  1.6114011905152745  \\
            0.8296593186372746  1.6207792343098995  \\
            0.8316633266533067  1.630252688541764  \\
            0.8336673346693386  1.6398220361024265  \\
            0.8356713426853707  1.6494877598834465  \\
            0.8376753507014028  1.6592503427763814  \\
            0.8396793587174348  1.6691102676727902  \\
            0.8416833667334669  1.6790680174642314  \\
            0.843687374749499  1.6891240750422636  \\
            0.845691382765531  1.6992789232984449  \\
            0.8476953907815631  1.709533045124334  \\
            0.8496993987975952  1.7198869234114897  \\
            0.8517034068136272  1.73034104105147  \\
            0.8537074148296593  1.7408958809358341  \\
            0.8557114228456913  1.7515519259561398  \\
            0.8577154308617234  1.7623096590039462  \\
            0.8597194388777555  1.7731695629708115  \\
            0.8617234468937875  1.784132120748294  \\
            0.8637274549098196  1.7951978152279526  \\
            0.8657314629258517  1.806367129301346  \\
            0.8677354709418837  1.8176405458600322  \\
            0.8697394789579158  1.8290185477955698  \\
            0.8717434869739479  1.8405016179995175  \\
            0.87374749498998  1.852090239363434  \\
            0.875751503006012  1.8637848947788775  \\
            0.8777555110220441  1.8755860671374065  \\
            0.8797595190380761  1.8874942393305796  \\
            0.8817635270541082  1.8995098942499553  \\
            0.8837675350701403  1.9116335147870922  \\
            0.8857715430861723  1.9238655838335488  \\
            0.8877755511022044  1.9362065842808835  \\
            0.8897795591182365  1.9486569990206548  \\
            0.8917835671342685  1.9612173109444218  \\
            0.8937875751503006  1.9738880029437418  \\
            0.8957915831663327  1.9866695579101745  \\
            0.8977955911823647  1.9995624587352778  \\
            0.8997995991983968  2.0125671883106104  \\
            0.9018036072144289  2.025684229527731  \\
            0.9038076152304609  2.0389140652781976  \\
            0.905811623246493  2.0522571784535693  \\
            0.9078156312625251  2.065714051945404  \\
            0.9098196392785571  2.079285168645261  \\
            0.9118236472945892  2.0929710114446975  \\
            0.9138276553106213  2.1067720632352733  \\
            0.9158316633266533  2.120688806908547  \\
            0.9178356713426854  2.1347217253560764  \\
            0.9198396793587175  2.14887130146942  \\
            0.9218436873747495  2.1631380181401365  \\
            0.9238476953907816  2.1775223582597842  \\
            0.9258517034068137  2.1920248047199227  \\
            0.9278557114228457  2.2066458404121088  \\
            0.9298597194388778  2.221385948227902  \\
            0.9318637274549099  2.2362456110588615  \\
            0.9338677354709419  2.251225311796544  \\
            0.935871743486974  2.26632553333251  \\
            0.9378757515030061  2.2815467585583162  \\
            0.9398797595190381  2.2968894703655227  \\
            0.9418837675350702  2.312354151645687  \\
            0.9438877755511023  2.327941285290368  \\
            0.9458917835671342  2.343651354191123  \\
            0.9478957915831663  2.359484841239513  \\
            0.9498997995991983  2.3754422293270947  \\
            0.9519038076152304  2.3915240013454273  \\
            0.9539078156312625  2.4077306401860685  \\
            0.9559118236472945  2.4240626287405784  \\
            0.9579158316633266  2.4405204499005135  \\
            0.9599198396793587  2.457104586557434  \\
            0.9619238476953907  2.473815521602898  \\
            0.9639278557114228  2.490653737928463  \\
            0.9659318637274549  2.5076197184256888  \\
            0.9679358717434869  2.524713945986133  \\
            0.969939879759519  2.541936903501355  \\
            0.9719438877755511  2.5592890738629124  \\
            0.9739478957915831  2.5767709399623646  \\
            0.9759519038076152  2.5943829846912694  \\
            0.9779559118236473  2.612125690941185  \\
            0.9799599198396793  2.629999541603671  \\
            0.9819639278557114  2.6480050195702853  \\
            0.9839679358717435  2.6661426077325867  \\
            0.9859719438877755  2.684412788982134  \\
            0.9879759519038076  2.7028160462104847  \\
            0.9899799599198397  2.721352862309198  \\
            0.9919839679358717  2.740023720169832  \\
            0.9939879759519038  2.7588291026839453  \\
            0.9959919839679359  2.777769492743097  \\
            0.9979959919839679  2.7968453732388454  \\
            1.0  2.816057227062749  \\
        }
        ;
    \addplot[color={rgb,1:red,0.7529;green,0.3255;blue,0.4039}, name path={1f786efc-24b2-4cfb-9204-fa6aa7a5ca8d}, draw opacity={1.0}, line width={2}, dashed, forget plot]
        table[row sep={\\}]
        {
            \\
            0.0  0.9999999999939382  \\
            0.002004008016032064  0.999999999993413  \\
            0.004008016032064128  0.9999999999928432  \\
            0.006012024048096192  0.9999999999922248  \\
            0.008016032064128256  0.9999999999915538  \\
            0.01002004008016032  0.9999999999908258  \\
            0.012024048096192385  0.9999999999900361  \\
            0.014028056112224449  0.9999999999891795  \\
            0.01603206412825651  0.9999999999882506  \\
            0.018036072144288578  0.9999999999872432  \\
            0.02004008016032064  0.9999999999861511  \\
            0.022044088176352707  0.999999999984967  \\
            0.02404809619238477  0.9999999999836835  \\
            0.026052104208416832  0.9999999999822924  \\
            0.028056112224448898  0.9999999999807848  \\
            0.03006012024048096  0.9999999999791513  \\
            0.03206412825651302  0.9999999999773816  \\
            0.03406813627254509  0.9999999999754646  \\
            0.036072144288577156  0.9999999999733883  \\
            0.03807615230460922  0.9999999999711396  \\
            0.04008016032064128  0.9999999999687049  \\
            0.04208416833667335  0.9999999999660689  \\
            0.04408817635270541  0.9999999999632155  \\
            0.04609218436873747  0.9999999999601272  \\
            0.04809619238476954  0.9999999999567852  \\
            0.050100200400801605  0.9999999999531692  \\
            0.052104208416833664  0.9999999999492574  \\
            0.05410821643286573  0.9999999999450261  \\
            0.056112224448897796  0.9999999999404501  \\
            0.05811623246492986  0.999999999935502  \\
            0.06012024048096192  0.9999999999301525  \\
            0.06212424849699399  0.99999999992437  \\
            0.06412825651302605  0.9999999999181206  \\
            0.06613226452905811  0.9999999999113677  \\
            0.06813627254509018  0.999999999904072  \\
            0.07014028056112225  0.9999999998961911  \\
            0.07214428857715431  0.99999999988768  \\
            0.07414829659318638  0.9999999998784895  \\
            0.07615230460921844  0.9999999998685677  \\
            0.0781563126252505  0.9999999998578581  \\
            0.08016032064128256  0.9999999998463007  \\
            0.08216432865731463  0.9999999998338306  \\
            0.0841683366733467  0.9999999998203783  \\
            0.08617234468937876  0.9999999998058698  \\
            0.08817635270541083  0.999999999790225  \\
            0.09018036072144289  0.9999999997733586  \\
            0.09218436873747494  0.9999999997551793  \\
            0.09418837675350701  0.9999999997355887  \\
            0.09619238476953908  0.9999999997144824  \\
            0.09819639278557114  0.9999999996917479  \\
            0.10020040080160321  0.9999999996672654  \\
            0.10220440881763528  0.9999999996409066  \\
            0.10420841683366733  0.9999999996125345  \\
            0.1062124248496994  0.9999999995820026  \\
            0.10821643286573146  0.9999999995491547  \\
            0.11022044088176353  0.999999999513824  \\
            0.11222444889779559  0.9999999994758325  \\
            0.11422845691382766  0.9999999994349906  \\
            0.11623246492985972  0.9999999993910962  \\
            0.11823647294589178  0.9999999993439337  \\
            0.12024048096192384  0.999999999293274  \\
            0.12224448897795591  0.9999999992388731  \\
            0.12424849699398798  0.9999999991804714  \\
            0.12625250501002003  0.9999999991177935  \\
            0.1282565130260521  0.9999999990505463  \\
            0.13026052104208416  0.9999999989784186  \\
            0.13226452905811623  0.9999999989010808  \\
            0.1342685370741483  0.9999999988181829  \\
            0.13627254509018036  0.9999999987293543  \\
            0.13827655310621242  0.9999999986342023  \\
            0.1402805611222445  0.9999999985323118  \\
            0.14228456913827656  0.9999999984232437  \\
            0.14428857715430862  0.999999998306534  \\
            0.1462925851703407  0.9999999981816929  \\
            0.14829659318637275  0.999999998048204  \\
            0.15030060120240482  0.999999997905523  \\
            0.1523046092184369  0.9999999977530765  \\
            0.15430861723446893  0.9999999975902617  \\
            0.156312625250501  0.999999997416445  \\
            0.15831663326653306  0.9999999972309613  \\
            0.16032064128256512  0.9999999970331134  \\
            0.1623246492985972  0.9999999968221703  \\
            0.16432865731462926  0.9999999965973682  \\
            0.16633266533066132  0.9999999963579083  \\
            0.1683366733466934  0.9999999961029574  \\
            0.17034068136272545  0.9999999958316473  \\
            0.17234468937875752  0.9999999955430747  \\
            0.1743486973947896  0.9999999952363013  \\
            0.17635270541082165  0.9999999949103541  \\
            0.17835671342685372  0.9999999945642255  \\
            0.18036072144288579  0.9999999941968749  \\
            0.18236472945891782  0.9999999938072295  \\
            0.1843687374749499  0.9999999933941852  \\
            0.18637274549098196  0.9999999929566092  \\
            0.18837675350701402  0.999999992493342  \\
            0.1903807615230461  0.9999999920031999  \\
            0.19238476953907815  0.9999999914849788  \\
            0.19438877755511022  0.9999999909374572  \\
            0.1963927855711423  0.9999999903594015  \\
            0.19839679358717435  0.9999999897495705  \\
            0.20040080160320642  0.9999999891067216  \\
            0.20240480961923848  0.9999999884296175  \\
            0.20440881763527055  0.999999987717033  \\
            0.20641282565130262  0.9999999869677646  \\
            0.20841683366733466  0.9999999861806389  \\
            0.21042084168336672  0.9999999853545235  \\
            0.2124248496993988  0.9999999844883386  \\
            0.21442885771543085  0.9999999835810698  \\
            0.21643286573146292  0.9999999826317821  \\
            0.218436873747495  0.9999999816396352  \\
            0.22044088176352705  0.9999999806039007  \\
            0.22244488977955912  0.9999999795239795  \\
            0.22444889779559118  0.9999999783994222  \\
            0.22645290581162325  0.9999999772299503  \\
            0.22845691382765532  0.9999999760154784  \\
            0.23046092184368738  0.9999999747561394  \\
            0.23246492985971945  0.9999999734523097  \\
            0.23446893787575152  0.9999999721046372  \\
            0.23647294589178355  0.99999997071407  \\
            0.23847695390781562  0.9999999692818874  \\
            0.24048096192384769  0.999999967809731  \\
            0.24248496993987975  0.9999999662996386  \\
            0.24448897795591182  0.9999999647540777  \\
            0.24649298597194388  0.9999999631759808  \\
            0.24849699398797595  0.9999999615687809  \\
            0.250501002004008  0.9999999599364472  \\
            0.25250501002004005  0.9999999582835207  \\
            0.2545090180360721  0.9999999566151491  \\
            0.2565130260521042  0.9999999549371208  \\
            0.25851703406813625  0.9999999532558965  \\
            0.2605210420841683  0.999999951578639  \\
            0.2625250501002004  0.9999999499132383  \\
            0.26452905811623245  0.9999999482683339  \\
            0.2665330661322645  0.9999999466533307  \\
            0.2685370741482966  0.9999999450784074  \\
            0.27054108216432865  0.9999999435545188  \\
            0.2725450901803607  0.9999999420933866  \\
            0.2745490981963928  0.9999999407074797  \\
            0.27655310621242485  0.9999999394099816  \\
            0.2785571142284569  0.9999999382147422  \\
            0.280561122244489  0.999999937136213  \\
            0.28256513026052105  0.9999999361893624  \\
            0.2845691382765531  0.9999999353895692  \\
            0.2865731462925852  0.9999999347524925  \\
            0.28857715430861725  0.9999999342939142  \\
            0.2905811623246493  0.999999934029553  \\
            0.2925851703406814  0.9999999339748463  \\
            0.29458917835671344  0.9999999341446983  \\
            0.2965931863727455  0.9999999345531931  \\
            0.2985971943887776  0.9999999352132692  \\
            0.30060120240480964  0.9999999361363568  \\
            0.3026052104208417  0.999999937331976  \\
            0.3046092184368738  0.9999999388072973  \\
            0.3066132264529058  0.9999999405666675  \\
            0.30861723446893785  0.9999999426111027  \\
            0.3106212424849699  0.9999999449377576  \\
            0.312625250501002  0.9999999475393776  \\
            0.31462925851703405  0.9999999504037478  \\
            0.3166332665330661  0.9999999535131548  \\
            0.3186372745490982  0.999999956843883  \\
            0.32064128256513025  0.9999999603657744  \\
            0.3226452905811623  0.9999999640418876  \\
            0.3246492985971944  0.9999999678283009  \\
            0.32665330661322645  0.9999999716741153  \\
            0.3286573146292585  0.9999999755217287  \\
            0.3306613226452906  0.9999999793074651  \\
            0.33266533066132264  0.999999982962663  \\
            0.3346693386773547  0.9999999864153524  \\
            0.3366733466933868  0.9999999895926693  \\
            0.33867735470941884  0.9999999924241979  \\
            0.3406813627254509  0.9999999948464557  \\
            0.342685370741483  0.999999996808793  \\
            0.34468937875751504  0.9999999982810178  \\
            0.3466933867735471  0.9999999992631247  \\
            0.3486973947895792  0.9999999997975725  \\
            0.35070140280561124  0.9999999999846378  \\
            0.3527054108216433  1.0000000000014677  \\
            0.35470941883767537  1.0000000001255631  \\
            0.35671342685370744  1.0000000007635585  \\
            0.3587174348697395  1.0000000024863036  \\
            0.36072144288577157  1.0000000060714382  \\
            0.3627254509018036  1.0000000125548396  \\
            0.36472945891783565  1.000000023292564  \\
            0.3667334669338677  1.0000000400351612  \\
            0.3687374749498998  1.0000000650165588  \\
            0.37074148296593185  1.0000001010600592  \\
            0.3727454909819639  1.0000001517044041  \\
            0.374749498997996  1.000000221353326  \\
            0.37675350701402804  1.0000003154525396  \\
            0.3787575150300601  1.0000004406987355  \\
            0.3807615230460922  1.0000006052858152  \\
            0.38276553106212424  1.0000008191943721  \\
            0.3847695390781563  1.0000010945312436  \\
            0.3867735470941884  1.0000014459269018  \\
            0.38877755511022044  1.000001890999703  \\
            0.3907815631262525  1.0000024508978662  \\
            0.3927855711422846  1.00000315093212  \\
            0.39478957915831664  1.0000040213127925  \\
            0.3967935871743487  1.0000050980049757  \\
            0.39879759519038077  1.0000064237174842  \\
            0.40080160320641284  1.0000080490455767  \\
            0.4028056112224449  1.0000100337905506  \\
            0.40480961923847697  1.0000124484810895  \\
            0.40681362725450904  1.0000153761234176  \\
            0.4088176352705411  1.0000189142102285  \\
            0.41082164328657317  1.0000231770216106  \\
            0.41282565130260523  1.0000282982544948  \\
            0.4148296593186373  1.0000344340204224  \\
            0.4168336673346693  1.000041766254594  \\
            0.4188376753507014  1.000050506582091  \\
            0.42084168336673344  1.0000609006896908  \\
            0.4228456913827655  1.0000732332535773  \\
            0.4248496993987976  1.0000878334742025  \\
            0.42685370741482964  1.000105081269166  \\
            0.4288577154308617  1.0001254141728342  \\
            0.4308617234468938  1.000149334986908  \\
            0.43286573146292584  1.0001774202186369  \\
            0.4348697394789579  1.0002103293320694  \\
            0.43687374749499  1.0002488148217439  \\
            0.43887775551102204  1.0002937330965689  \\
            0.4408817635270541  1.000346056133268  \\
            0.44288577154308617  1.0004068838225681  \\
            0.44488977955911824  1.0004774568862485  \\
            0.4468937875751503  1.000559170188344  \\
            0.44889779559118237  1.0006535861985508  \\
            0.45090180360721444  1.000762448290086  \\
            0.4529058116232465  1.0008876934683868  \\
            0.45490981963927857  1.0010314640326046  \\
            0.45691382765531063  1.0011961175715727  \\
            0.4589178356713427  1.001384234594126  \\
            0.46092184368737477  1.0015986229964933  \\
            0.46292585170340683  1.0018423184852188  \\
            0.4649298597194389  1.0021185800130767  \\
            0.46693386773547096  1.0024308792600918  \\
            0.46893787575150303  1.0027828832158359  \\
            0.4709418837675351  1.0031784290070427  \\
            0.4729458917835671  1.0036214902796672  \\
            0.4749498997995992  1.0041161346975669  \\
            0.47695390781563124  1.0046664724665535  \\
            0.4789579158316633  1.005276596230681  \\
            0.48096192384769537  1.0059505132052626  \\
            0.48296593186372744  1.0066920709844687  \\
            0.4849699398797595  1.0075048790541672  \\
            0.48697394789579157  1.008392228605279  \\
            0.48897795591182364  1.0093570137232801  \\
            0.4909819639278557  1.0104016573665917  \\
            0.49298597194388777  1.0115280456854163  \\
            0.49498997995991983  1.0127374741304493  \\
            0.4969939879759519  1.0140306084352846  \\
            0.49899799599198397  1.0154074629312275  \\
            0.501002004008016  1.0168673978020535  \\
            0.503006012024048  1.018409135870637  \\
            0.5050100200400801  1.0200307984134167  \\
            0.5070140280561122  1.0217299584189306  \\
            0.5090180360721442  1.0235037087402001  \\
            0.5110220440881763  1.0253487418221734  \\
            0.5130260521042084  1.0272614371764286  \\
            0.5150300601202404  1.0292379525574276  \\
            0.5170340681362725  1.0312743148669972  \\
            0.5190380761523046  1.0333665071460094  \\
            0.5210420841683366  1.0355105485508243  \\
            0.5230460921843687  1.0377025648894813  \\
            0.5250501002004008  1.0399388480370262  \\
            0.5270541082164328  1.0422159032935103  \\
            0.5290581162324649  1.0445304844360184  \\
            0.531062124248497  1.0468796168066963  \\
            0.533066132264529  1.0492606092476902  \\
            0.5350701402805611  1.051671056032145  \\
            0.5370741482965932  1.0541088301517514  \\
            0.5390781563126252  1.0565720694189833  \\
            0.5410821643286573  1.0590591568452479  \\
            0.5430861723446894  1.0615686966863325  \\
            0.5450901803607214  1.0640994874255032  \\
            0.5470941883767535  1.0666504928121916  \\
            0.5490981963927856  1.0692208119073008  \\
            0.5511022044088176  1.071809648918092  \\
            0.5531062124248497  1.0744162834462807  \\
            0.5551102204408818  1.0770400416290071  \\
            0.5571142284569138  1.0796802685276699  \\
            0.5591182364729459  1.0823363020159487  \\
            0.561122244488978  1.0850074483357812  \\
            0.56312625250501  1.0876929594275189  \\
            0.5651302605210421  1.0903920120961637  \\
            0.5671342685370742  1.09310368904724  \\
            0.5691382765531062  1.0958269618110121  \\
            0.5711422845691383  1.0985606755699848  \\
            0.5731462925851704  1.1013035359094774  \\
            0.5751503006012024  1.1040540975222173  \\
            0.5771543086172345  1.106810754913219  \\
            0.5791583166332666  1.1095717351685837  \\
            0.5811623246492986  1.1123350928693716  \\
            0.5831663326653307  1.115098707247458  \\
            0.5851703406813628  1.1178602816925807  \\
            0.5871743486973948  1.1206173457268729  \\
            0.5891783567134269  1.1233672595635746  \\
            0.591182364729459  1.1261072213588577  \\
            0.593186372745491  1.1288342772486566  \\
            0.5951903807615231  1.1315453342351425  \\
            0.5971943887775552  1.1342371759495318  \\
            0.5991983967935872  1.1369064812692535  \\
            0.6012024048096193  1.1395498457086815  \\
            0.6032064128256514  1.1421638054348526  \\
            0.6052104208416834  1.1447448636848296  \\
            0.6072144288577155  1.1472895192821824  \\
            0.6092184368737475  1.1497942968698431  \\
            0.6112224448897795  1.1522557783991514  \\
            0.6132264529058116  1.154670635344538  \\
            0.6152304609218436  1.1570356610543389  \\
            0.6172344689378757  1.1593478026049575  \\
            0.6192384769539078  1.1616041915017177  \\
            0.6212424849699398  1.1638021725682128  \\
            0.6232464929859719  1.1659393303885852  \\
            0.625250501002004  1.168013512714447  \\
            0.627254509018036  1.1700228503190877  \\
            0.6292585170340681  1.1719657728737163  \\
            0.6312625250501002  1.1738410205299226  \\
            0.6332665330661322  1.1756476510142502  \\
            0.6352705410821643  1.1773850421689922  \\
            0.6372745490981964  1.179052890001754  \\
            0.6392785571142284  1.1806512024287719  \\
            0.6412825651302605  1.1821802890075244  \\
            0.6432865731462926  1.1836407470476313  \\
            0.6452905811623246  1.1850334445608575  \\
            0.6472945891783567  1.1863595005574017  \\
            0.6492985971943888  1.1876202632126636  \\
            0.6513026052104208  1.1888172864116662  \\
            0.6533066132264529  1.1899523051196654  \\
            0.655310621242485  1.1910272099137957  \\
            0.657314629258517  1.192044020817229  \\
            0.6593186372745491  1.1930048602597447  \\
            0.6613226452905812  1.193911924467547  \\
            0.6633266533066132  1.1947674517221825  \\
            0.6653306613226453  1.195573684484808  \\
            0.6673346693386774  1.196332819951568  \\
            0.6693386773547094  1.1970469395048866  \\
            0.6713426853707415  1.1977179006187757  \\
            0.6733466933867736  1.198347163208582  \\
            0.6753507014028056  1.198935503224519  \\
            0.6773547094188377  1.1994825349148157  \\
            0.6793587174348698  1.1999859130686858  \\
            0.6813627254509018  1.200440009556222  \\
            0.6833667334669339  1.2008337485630693  \\
            0.685370741482966  1.2011471521128638  \\
            0.687374749498998  1.2013460616746312  \\
            0.6893787575150301  1.201374713824161  \\
            0.6913827655310621  1.2011470810103058  \\
            0.6933867735470942  1.2005418545838946  \\
            0.6953907815631263  1.199415459201694  \\
            0.6973947895791583  1.197749755757698  \\
            0.6993987975951904  1.1965701689741701  \\
            0.7014028056112225  1.19661680789105  \\
            0.7034068136272545  1.1982477957615192  \\
            0.7054108216432866  1.2013044986414145  \\
            0.7074148296593187  1.2052596440984376  \\
            0.7094188376753507  1.2096403822609834  \\
            0.7114228456913828  1.2142038412167173  \\
            0.7134268537074149  1.2188649386422232  \\
            0.7154308617234469  1.2236010058378435  \\
            0.717434869739479  1.2284068766603051  \\
            0.7194388777555111  1.2332814846027018  \\
            0.7214428857715431  1.2382248862576088  \\
            0.7234468937875751  1.243237517087235  \\
            0.7254509018036072  1.2483198993122804  \\
            0.7274549098196392  1.253472548331079  \\
            0.7294589178356713  1.2586959623055305  \\
            0.7314629258517034  1.2639906294960486  \\
            0.7334669338677354  1.2693570341788982  \\
            0.7354709418837675  1.274795659443987  \\
            0.7374749498997996  1.2803069882298377  \\
            0.7394789579158316  1.2858915036144034  \\
            0.7414829659318637  1.2915496888435003  \\
            0.7434869739478958  1.2972820272867924  \\
            0.7454909819639278  1.303089002386512  \\
            0.7474949899799599  1.3089710976168907  \\
            0.749498997995992  1.3149287964563228  \\
            0.751503006012024  1.3209625823698175  \\
            0.7535070140280561  1.3270729387987503  \\
            0.7555110220440882  1.3332603491554487  \\
            0.7575150300601202  1.3395252968208395  \\
            0.7595190380761523  1.3458682651439697  \\
            0.7615230460921844  1.3522897374426317  \\
            0.7635270541082164  1.358790197004599  \\
            0.7655310621242485  1.3653701270891818  \\
            0.7675350701402806  1.372030010928918  \\
            0.7695390781563126  1.3787703317312894  \\
            0.7715430861723447  1.3855915726804109  \\
            0.7735470941883767  1.392494216938649  \\
            0.7755511022044088  1.3994787476481603  \\
            0.7775551102204409  1.4065456479323388  \\
            0.779559118236473  1.4136954008971736  \\
            0.781563126252505  1.4209284896325136  \\
            0.7835671342685371  1.42824539721325  \\
            0.7855711422845691  1.4356466067004137  \\
            0.7875751503006012  1.4431326011421972  \\
            0.7895791583166333  1.4507038635749043  \\
            0.7915831663326653  1.4583608770238303  \\
            0.7935871743486974  1.4661041245040807  \\
            0.7955911823647295  1.4739340890213306  \\
            0.7975951903807615  1.481851253572529  \\
            0.7995991983967936  1.4898561011465503  \\
            0.8016032064128257  1.497949114724802  \\
            0.8036072144288577  1.5061307772817856  \\
            0.8056112224448898  1.514401571785615  \\
            0.8076152304609219  1.5227619811985005  \\
            0.8096192384769539  1.5312124884771947  \\
            0.811623246492986  1.5397535765734056  \\
            0.8136272545090181  1.5483857284341807  \\
            0.8156312625250501  1.5571094270022598  \\
            0.8176352705410822  1.5659251552164055  \\
            0.8196392785571143  1.5748333960117047  \\
            0.8216432865731463  1.5838346323198507  \\
            0.8236472945891784  1.5929293470694037  \\
            0.8256513026052105  1.6021180231860306  \\
            0.8276553106212425  1.6114011435927273  \\
            0.8296593186372746  1.6207791912100253  \\
            0.8316633266533067  1.6302526489561808  \\
            0.8336673346693386  1.6398219997473507  \\
            0.8356713426853707  1.6494877264977565  \\
            0.8376753507014028  1.6592503121198319  \\
            0.8396793587174348  1.6691102395243638  \\
            0.8416833667334669  1.6790679916206188  \\
            0.843687374749499  1.6891240513164623  \\
            0.845691382765531  1.6992789015184675  \\
            0.8476953907815631  1.7095330251320169  \\
            0.8496993987975952  1.7198869050613947  \\
            0.8517034068136272  1.7303410242098738  \\
            0.8537074148296593  1.7408958654797935  \\
            0.8557114228456913  1.7515519117726344  \\
            0.8577154308617234  1.7623096459890855  \\
            0.8597194388777555  1.7731695510291055  \\
            0.8617234468937875  1.784132109791982  \\
            0.8637274549098196  1.7951978051763842  \\
            0.8657314629258517  1.80636712008041  \\
            0.8677354709418837  1.8176405374016338  \\
            0.8697394789579158  1.8290185400371466  \\
            0.8717434869739479  1.8405016108835937  \\
            0.87374749498998  1.8520902328372117  \\
            0.875751503006012  1.8637848887938606  \\
            0.8777555110220441  1.875586061649052  \\
            0.8797595190380761  1.8874942342979801  \\
            0.8817635270541082  1.8995098896355442  \\
            0.8837675350701403  1.9116335105563738  \\
            0.8857715430861723  1.9238655799548487  \\
            0.8877755511022044  1.9362065807251225  \\
            0.8897795591182365  1.948656995761135  \\
            0.8917835671342685  1.9612173079566353  \\
            0.8937875751503006  1.973888000205193  \\
            0.8957915831663327  1.9866695554002152  \\
            0.8977955911823647  1.999562456434957  \\
            0.8997995991983968  2.0125671862025367  \\
            0.9018036072144289  2.025684227595944  \\
            0.9038076152304609  2.038914063508053  \\
            0.905811623246493  2.0522571768316302  \\
            0.9078156312625251  2.0657140504593423  \\
            0.9098196392785571  2.0792851672837664  \\
            0.9118236472945892  2.0929710101973953  \\
            0.9138276553106213  2.1067720620926456  \\
            0.9158316633266533  2.120688805861864  \\
            0.9178356713426854  2.1347217243973313  \\
            0.9198396793587175  2.1488713005912703  \\
            0.9218436873747495  2.1631380173358474  \\
            0.9238476953907816  2.1775223575231815  \\
            0.9258517034068137  2.192024804045343  \\
            0.9278557114228457  2.206645839794361  \\
            0.9298597194388778  2.2213859476622266  \\
            0.9318637274549099  2.236245610540893  \\
            0.9338677354709419  2.2512253113222833  \\
            0.935871743486974  2.2663255328982888  \\
            0.9378757515030061  2.2815467581607742  \\
            0.9398797595190381  2.2968894700015783  \\
            0.9418837675350702  2.3123541513125163  \\
            0.9438877755511023  2.327941284985383  \\
            0.9458917835671342  2.3436513539119526  \\
            0.9478957915831663  2.359484840983983  \\
            0.9498997995991983  2.375442229093215  \\
            0.9519038076152304  2.3915240011313728  \\
            0.9539078156312625  2.4077306399901675  \\
            0.9559118236472945  2.424062628561299  \\
            0.9579158316633266  2.440520449736453  \\
            0.9599198396793587  2.4571045864073073  \\
            0.9619238476953907  2.4738155214655273  \\
            0.9639278557114228  2.49065373780277  \\
            0.9659318637274549  2.5076197183106856  \\
            0.9679358717434869  2.524713945880915  \\
            0.969939879759519  2.5419369034050936  \\
            0.9719438877755511  2.559289073774849  \\
            0.9739478957915831  2.576770939881804  \\
            0.9759519038076152  2.594382984617576  \\
            0.9779559118236473  2.612125690873776  \\
            0.9799599198396793  2.6299995415420128  \\
            0.9819639278557114  2.6480050195138904  \\
            0.9839679358717435  2.666142607681007  \\
            0.9859719438877755  2.68441278893496  \\
            0.9879759519038076  2.7028160461673423  \\
            0.9899799599198397  2.7213528622697436  \\
            0.9919839679358717  2.7400237201337525  \\
            0.9939879759519038  2.758829102650953  \\
            0.9959919839679359  2.7777694927129293  \\
            0.9979959919839679  2.796845373211261  \\
            1.0  2.8160572270375273  \\
        }
        ;
    \node[right, , color={rgb,1:red,0.0;green,0.3608;blue,0.6706}, draw opacity={1.0}, rotate={0.0}, font={{\fontsize{8 pt}{10.4 pt}\selectfont}}]  at (axis cs:0.05,2.5) {Nominal};
    \node[right, , color={rgb,1:red,0.7529;green,0.3255;blue,0.4039}, draw opacity={1.0}, rotate={0.0}, font={{\fontsize{8 pt}{10.4 pt}\selectfont}}]  at (axis cs:0.05,2.3) {Smoothed};
\end{axis}
\end{tikzpicture}

         % \label{fig:cdtranslim-smoothed}
     \end{subfigure}
     \caption{Nominal and smoothed transonic lift and drag coefficient limits across a range of Mach numbers for a Nominal lift coefficient of unity.}
        \label{fig:translim-smoothed}
\end{figure}


\subsection{Combined Implementation}

In DuctAPE, these corrections are applied as follows.
%
First, it is assumed that the user inputs airfoil data that is already pre-processed with the stall limits applied.
%
Ideally, the airfoil data also inherently has Reynolds number dependencies (data at various Reynolds numbers) already as well.
%
Then during computation, the corrections are applied on-the-fly, beginning with the solidity/stagger correction.
%
The Prandtl-Glauert compressibility correction is applied next, followed by the Reynolds number drag correction if data at multiple Reynolds numbers was not provided.
%
Finally, the transonic effect lift limiter and drag addition corrections are applied.



% \subsection{Rotation (3D) Corrections}


\section{Comparison of Corrected Airfoil Polars to Experimental Cascade Data}

To see how the corrections actually fare (especially the solidity and stagger corrections), we compare to experimental data produced by NACA for their NACA 65-410 airfoil.
The NACA 65-series compressor blade airfoils are base on a basic thickness form and mean line.
%
The basic thickness form comes from the \(65_2\)-016 airfoil which is first scaled down to 10\% thickness and then the y-coordinates are increased by 0.0015 times the chord-wise coordinate to slightly thicken the trailing edge.
%
There are also directly derived values for the coordinates; they are slightly different than the scaled values used in the study.
%
The basic mean line comes from the NACA 6-series method to obtain a design lift coefficient of 1.0, and then scaled based on the desired lift coefficient.
%
For example the 65-410 mean camber line takes the basic mean camber line and scales it by 0.4, while the 65-(12)10 mean camber line is the basic mean camber line scaled by a factor of 1.2.
%
Tests were run for at solidities from 0.5 to 1.5 and inflow angles of \(30^\circ\) to \(70^\circ\), although not every combination was tested.
%
Tests at solidities of 1.0 and above were performed at a Reynolds number of 2.45e5; for solidities less than 1.0 tests were performed at a Reynolds number of 2e5.
%
The experimental data for lift and drag coefficients in the NACA report is given for each tested combination of inflow angle and solidity across a range of angles of attack, generally ranging from negative to positive stall.
%
Note that the lift and drag forces were not measured directly, but rather calculated from pressure and velocity measurements.
%
In order to apply our airfoil corrections we calculate the stagger angles, \(\gamma\), from the provided inflow angles, \(\beta_1\), and angles of attack, \(\alpha\), as

\begin{equation}
    \gamma = \beta_1 - \alpha
\end{equation}

\begin{figure}[hb!]
    \centering
        \begin{tikzpicture}[scale=7]
        %Airfoil
        \draw[ultra thick,primary] plot[smooth] file{ductape/figures/naca-65410scaled.dat};
\end{tikzpicture}
%
        \caption{NACA 65-410 compressor series airfoil geometry (using the scaled ordinates).}
    \label{fig:naca65410scaled}
\end{figure}


As can be seen in \cref{fig:naca65410comps}, the method of corrected airfoil data does not do especially well at matching actual cascade data.
%
In general, the lift curve slopes of the cascades are much shallower than that of the isolated and corrected XFOIL outputs.
%
In addition, the drag ``bucket'' of the isolated airfoil is much narrower than for the cascades.
%
We note that we did not apply drag corrections in an attempt to capture cascade effects.
%
Such corrections would increase the drag due to blockage from solidity and increase the discrepancies we already see in \cref{fig:naca65410comps}.

\clearpage
\newpage

\begin{figure}[h!]
     \centering
     \begin{subfigure}[t]{\textwidth}
         \centering
        % Recommended preamble:
% \usetikzlibrary{arrows.meta}
% \usetikzlibrary{backgrounds}
% \usepgfplotslibrary{patchplots}
% \usepgfplotslibrary{fillbetween}
% \pgfplotsset{%
%     layers/standard/.define layer set={%
%         background,axis background,axis grid,axis ticks,axis lines,axis tick labels,pre main,main,axis descriptions,axis foreground%
%     }{
%         grid style={/pgfplots/on layer=axis grid},%
%         tick style={/pgfplots/on layer=axis ticks},%
%         axis line style={/pgfplots/on layer=axis lines},%
%         label style={/pgfplots/on layer=axis descriptions},%
%         legend style={/pgfplots/on layer=axis descriptions},%
%         title style={/pgfplots/on layer=axis descriptions},%
%         colorbar style={/pgfplots/on layer=axis descriptions},%
%         ticklabel style={/pgfplots/on layer=axis tick labels},%
%         axis background@ style={/pgfplots/on layer=axis background},%
%         3d box foreground style={/pgfplots/on layer=axis foreground},%
%     },
% }

\begin{tikzpicture}[/tikz/background rectangle/.style={fill={rgb,1:red,1.0;green,1.0;blue,1.0}, fill opacity={1.0}, draw opacity={1.0}}, show background rectangle]
\begin{axis}[point meta max={nan}, point meta min={nan}, legend cell align={left}, legend columns={1}, title={}, title style={at={{(0.5,1)}}, anchor={south}, font={{\fontsize{14 pt}{18.2 pt}\selectfont}}, color={rgb,1:red,0.0;green,0.0;blue,0.0}, draw opacity={1.0}, rotate={0.0}, align={center}}, legend style={color={rgb,1:red,0.0;green,0.0;blue,0.0}, draw opacity={0.0}, line width={1}, solid, fill={rgb,1:red,0.0;green,0.0;blue,0.0}, fill opacity={0.0}, text opacity={1.0}, font={{\fontsize{8 pt}{10.4 pt}\selectfont}}, text={rgb,1:red,0.0;green,0.0;blue,0.0}, cells={anchor={center}}, at={(1.02, 1)}, anchor={north west}}, axis background/.style={fill={rgb,1:red,0.0;green,0.0;blue,0.0}, opacity={0.0}}, anchor={north west}, xshift={0.0mm}, yshift={-0.0mm}, width={58.5mm}, height={50.8mm}, scaled x ticks={false}, xlabel={}, x tick style={color={rgb,1:red,0.0;green,0.0;blue,0.0}, opacity={1.0}}, x tick label style={color={rgb,1:red,0.0;green,0.0;blue,0.0}, opacity={1.0}, rotate={0}}, xlabel style={at={(ticklabel cs:0.5)}, anchor=near ticklabel, at={{(ticklabel cs:0.5)}}, anchor={near ticklabel}, font={{\fontsize{11 pt}{14.3 pt}\selectfont}}, color={rgb,1:red,0.0;green,0.0;blue,0.0}, draw opacity={1.0}, rotate={0.0}}, xmajorgrids={false}, xmin={-6.0}, xmax={24.0}, xticklabels={{$-5$,$0$,$5$,$10$,$15$,$20$}}, xtick={{-5.0,0.0,5.0,10.0,15.0,20.0}}, xtick align={inside}, xticklabel style={font={{\fontsize{8 pt}{10.4 pt}\selectfont}}, color={rgb,1:red,0.0;green,0.0;blue,0.0}, draw opacity={1.0}, rotate={0.0}}, x grid style={color={rgb,1:red,0.0;green,0.0;blue,0.0}, draw opacity={0.1}, line width={0.5}, solid}, axis x line*={left}, x axis line style={color={rgb,1:red,0.0;green,0.0;blue,0.0}, draw opacity={1.0}, line width={1}, solid}, scaled y ticks={false}, ylabel={$c_\ell$}, y tick style={color={rgb,1:red,0.0;green,0.0;blue,0.0}, opacity={1.0}}, y tick label style={color={rgb,1:red,0.0;green,0.0;blue,0.0}, opacity={1.0}, rotate={0}}, ylabel style={{rotate=-90}}, ymajorgrids={false}, ymin={-0.1}, ymax={1.2}, yticklabels={{$0.00$,$0.25$,$0.50$,$0.75$,$1.00$}}, ytick={{0.0,0.25,0.5,0.75,1.0}}, ytick align={inside}, yticklabel style={font={{\fontsize{8 pt}{10.4 pt}\selectfont}}, color={rgb,1:red,0.0;green,0.0;blue,0.0}, draw opacity={1.0}, rotate={0.0}}, y grid style={color={rgb,1:red,0.0;green,0.0;blue,0.0}, draw opacity={0.1}, line width={0.5}, solid}, axis y line*={left}, y axis line style={color={rgb,1:red,0.0;green,0.0;blue,0.0}, draw opacity={1.0}, line width={1}, solid}, colorbar={false}]
    \addplot[color={rgb,1:red,0.0;green,0.3608;blue,0.6706}, name path={c4f26fc3-4458-43e4-b8fe-5a4ce5189e39}, only marks, draw opacity={1.0}, line width={0}, solid, mark={triangle*}, mark size={3.0 pt}, mark repeat={1}, mark options={color={rgb,1:red,0.0;green,0.0;blue,0.0}, draw opacity={0.0}, fill={rgb,1:red,0.0;green,0.3608;blue,0.6706}, fill opacity={1.0}, line width={0.75}, rotate={0}, solid}, forget plot]
        table[row sep={\\}]
        {
            \\
            -3.022252591018093  0.0116625546259916  \\
            0.0034424375571671  0.104580042079042  \\
            3.025323106929635  0.2047342165751638  \\
            6.04938341013227  0.3007531413323876  \\
            7.502654266395366  0.3435753779221256  \\
            9.076713164080155  0.3905691914812645  \\
            12.023699193232757  0.4772595861722991  \\
            15.022027708162602  0.5443196536804025  \\
            18.02308076538015  0.6062106590148834  \\
            21.030127815630657  0.6567297275673946  \\
        }
        ;
    \addplot[color={rgb,1:red,0.7529;green,0.3255;blue,0.4039}, name path={6492796e-92f4-49e9-a49c-bb29a5beafc3}, only marks, draw opacity={1.0}, line width={0}, solid, mark={triangle*}, mark size={3.0 pt}, mark repeat={1}, mark options={color={rgb,1:red,0.0;green,0.0;blue,0.0}, draw opacity={0.0}, fill={rgb,1:red,0.7529;green,0.3255;blue,0.4039}, fill opacity={1.0}, line width={0.75}, rotate={0}, solid}, forget plot]
        table[row sep={\\}]
        {
            \\
            -3.0084588836107535  -0.0172901516791686  \\
            0.0007539384368496  0.0730255600412629  \\
            3.0078272054609574  0.1669173851581106  \\
            5.978456718584834  0.241722341793915  \\
            8.009821576393573  0.2864410794564511  \\
            9.991905350161101  0.3335296290164415  \\
            13.01752142738885  0.3964284713643483  \\
            15.995282790591048  0.4593130500120987  \\
            18.97874963385591  0.5126613262694055  \\
            22.041451331490936  0.5135742030794309  \\
        }
        ;
    \addplot[color={rgb,1:red,0.5608;green,0.651;blue,0.3176}, name path={64c2ea61-8afb-4434-86bc-5f489ba96bcb}, only marks, draw opacity={1.0}, line width={0}, solid, mark={triangle*}, mark size={3.0 pt}, mark repeat={1}, mark options={color={rgb,1:red,0.0;green,0.0;blue,0.0}, draw opacity={0.0}, fill={rgb,1:red,0.5608;green,0.651;blue,0.3176}, fill opacity={1.0}, line width={0.75}, rotate={0}, solid}, forget plot]
        table[row sep={\\}]
        {
            \\
            -5.030781513350483  -0.078608178834383  \\
            -1.9854874478843243  0.0188193027588042  \\
            1.0325556639758702  0.0942340104674887  \\
            4.018238268428991  0.1566335491883259  \\
            7.00108223188149  0.2240317572961943  \\
            9.001188680919014  0.2719693071941807  \\
            11.990845382773  0.3273707087731748  \\
            14.979366628226739  0.3847715781069811  \\
            18.013306129690417  0.4321937372482923  \\
            21.01374966734676  0.4686001951565687  \\
        }
        ;
    \addplot[color={rgb,1:red,0.5098;green,0.5098;blue,0.5098}, name path={8077acad-193d-4dc3-8f00-27fa29d1f54d}, draw opacity={1.0}, line width={1.0}, dashed, forget plot]
        table[row sep={\\}]
        {
            \\
            -15.0  -0.5665510105403683  \\
            -14.505050505050505  -0.5223707485770781  \\
            -14.01010101010101  -0.48288290323857036  \\
            -13.515151515151516  -0.4476923804233114  \\
            -13.02020202020202  -0.41640408602976736  \\
            -12.525252525252526  -0.38862292595640446  \\
            -12.030303030303031  -0.363953806101689  \\
            -11.535353535353535  -0.342001632364087  \\
            -11.04040404040404  -0.3223713106420649  \\
            -10.545454545454545  -0.31264275688078474  \\
            -10.05050505050505  -0.3229772463157242  \\
            -9.555555555555555  -0.37730526563488587  \\
            -9.06060606060606  -0.4398893615283618  \\
            -8.565656565656566  -0.44816433443905423  \\
            -8.070707070707071  -0.44277801246942355  \\
            -7.575757575757576  -0.4298637871228801  \\
            -7.08080808080808  -0.4108534742848187  \\
            -6.585858585858586  -0.38042950578354084  \\
            -6.090909090909091  -0.33981331145726257  \\
            -5.595959595959596  -0.2900177741710254  \\
            -5.101010101010101  -0.23727259949514537  \\
            -4.606060606060606  -0.1815499910867863  \\
            -4.111111111111111  -0.12544341290574643  \\
            -3.6161616161616164  -0.06876672586757099  \\
            -3.121212121212121  -0.012021506605778243  \\
            -2.6262626262626263  0.04483434055713697  \\
            -2.1313131313131315  0.10175187222194775  \\
            -1.6363636363636365  0.15885001136591717  \\
            -1.1414141414141414  0.21614787322246745  \\
            -0.6464646464646465  0.27326335201173685  \\
            -0.15151515151515152  0.33008787367812703  \\
            0.3434343434343434  0.3867358841059261  \\
            0.8383838383838383  0.4433265167789935  \\
            1.3333333333333333  0.4990846154557274  \\
            1.8282828282828283  0.5514163812619709  \\
            2.323232323232323  0.5956790139480467  \\
            2.8181818181818183  0.6300960032070431  \\
            3.313131313131313  0.6778677494717945  \\
            3.808080808080808  0.7295565593987938  \\
            4.303030303030303  0.7780694571148149  \\
            4.797979797979798  0.8255216383729825  \\
            5.292929292929293  0.8720063083639519  \\
            5.787878787878788  0.9167960654895222  \\
            6.282828282828283  0.957800935223836  \\
            6.777777777777778  0.995808680488238  \\
            7.2727272727272725  1.0336100742677  \\
            7.767676767676767  1.0699644187909074  \\
            8.262626262626263  1.0992053861622957  \\
            8.757575757575758  1.1047006399431818  \\
            9.252525252525253  1.1001338863619823  \\
            9.747474747474747  1.0881279859286563  \\
            10.242424242424242  1.0582474472708032  \\
            10.737373737373737  1.0246110095345213  \\
            11.232323232323232  0.9907201888390117  \\
            11.727272727272727  0.9564933909437671  \\
            12.222222222222221  0.9272047502821661  \\
            12.717171717171718  0.9210081124572284  \\
            13.212121212121213  0.9191664492896328  \\
            13.707070707070708  0.9170774786104494  \\
            14.202020202020202  0.9180098805020722  \\
            14.696969696969697  0.9220997625655921  \\
            15.191919191919192  0.9319796370148837  \\
            15.686868686868687  0.9447055160240829  \\
            16.181818181818183  0.9589114265942978  \\
            16.67676767676768  0.9734513568690647  \\
            17.171717171717173  0.9887075264975685  \\
            17.666666666666668  1.0042483369086435  \\
            18.161616161616163  1.0198102993993334  \\
            18.656565656565657  1.035398900104723  \\
            19.151515151515152  1.051295046008166  \\
            19.646464646464647  1.0691577941475554  \\
            20.141414141414142  1.0864846355610962  \\
            20.636363636363637  1.1030653457193929  \\
            21.13131313131313  1.1194857325791487  \\
            21.626262626262626  1.135774015505571  \\
            22.12121212121212  1.1517956055832994  \\
            22.616161616161616  1.167218420827846  \\
            23.11111111111111  1.1821984594240826  \\
            23.606060606060606  1.1966270886411536  \\
            24.1010101010101  1.2114379401269704  \\
            24.595959595959595  1.2269249451976803  \\
            25.09090909090909  1.2422736504402454  \\
            25.585858585858585  1.2573850331402625  \\
            26.08080808080808  1.2723818841719159  \\
            26.575757575757574  1.2872454497362291  \\
            27.07070707070707  1.3019543464686127  \\
            27.565656565656564  1.316387026323919  \\
            28.060606060606062  1.3308240336463104  \\
            28.555555555555557  1.345627164736373  \\
            29.050505050505052  1.3603996856804472  \\
            29.545454545454547  1.3745594103327088  \\
            30.04040404040404  1.3878745383906563  \\
            30.535353535353536  1.4003714583324913  \\
            31.03030303030303  1.412153211455866  \\
            31.525252525252526  1.423563342761  \\
            32.02020202020202  1.4349005047644947  \\
            32.515151515151516  1.446141174796294  \\
            33.01010101010101  1.4571919141366922  \\
            33.505050505050505  1.466658196592202  \\
            34.0  1.4743512710405462  \\
        }
        ;
    \addplot[color={rgb,1:red,0.0;green,0.3608;blue,0.6706}, name path={05fc8692-f1f2-43e2-997b-e1f3ad292d37}, draw opacity={1.0}, line width={1.0}, solid, forget plot]
        table[row sep={\\}]
        {
            \\
            -3.022252591018093  -0.0031796796074362176  \\
            0.0034424375571671  0.25339357364752846  \\
            3.025323106929635  0.46860199543912806  \\
            6.04938341013227  0.6755392889220804  \\
            7.502654266395366  0.7504283840183834  \\
            9.076713164080155  0.7735493735637641  \\
            12.023699193232757  0.7754022386227456  \\
            15.022027708162602  0.7790347480558709  \\
            18.02308076538015  0.7825701741516083  \\
            21.030127815630657  0.78610645832313  \\
        }
        ;
    \addplot[color={rgb,1:red,0.7529;green,0.3255;blue,0.4039}, name path={d408b04b-37a6-4b6c-88dc-c705a59b15cf}, draw opacity={1.0}, line width={1.0}, solid, forget plot]
        table[row sep={\\}]
        {
            \\
            -3.0084588836107535  -0.0014498313539509197  \\
            0.0007539384368496  0.20853432736896713  \\
            3.0078272054609574  0.3844851543193228  \\
            5.978456718584834  0.5530043217341231  \\
            8.009821576393573  0.6343605889917966  \\
            9.991905350161101  0.6403113293097329  \\
            13.01752142738885  0.6433314453517147  \\
            15.995282790591048  0.6463191863101216  \\
            18.97874963385591  0.6493052027751073  \\
            22.041451331490936  0.6523695447899834  \\
        }
        ;
    \addplot[color={rgb,1:red,0.5608;green,0.651;blue,0.3176}, name path={44615b86-5340-410c-ab4b-d8d6314e78bb}, draw opacity={1.0}, line width={1.0}, solid, forget plot]
        table[row sep={\\}]
        {
            \\
            -5.030781513350483  -0.12699240113790117  \\
            -1.9854874478843243  0.061686212972750655  \\
            1.0325556639758702  0.24118803959554475  \\
            4.018238268428991  0.38604882927975614  \\
            7.00108223188149  0.520888040896493  \\
            9.001188680919014  0.5579663158932002  \\
            11.990845382773  0.5603371667518198  \\
            14.979366628226739  0.5629511851534315  \\
            18.013306129690417  0.5656129985473709  \\
            21.01374966734676  0.5682455923799825  \\
        }
        ;
\end{axis}
\begin{axis}[point meta max={nan}, point meta min={nan}, legend cell align={left}, legend columns={1}, title={}, title style={at={{(0.5,1)}}, anchor={south}, font={{\fontsize{14 pt}{18.2 pt}\selectfont}}, color={rgb,1:red,0.0;green,0.0;blue,0.0}, draw opacity={1.0}, rotate={0.0}, align={center}}, legend style={color={rgb,1:red,0.0;green,0.0;blue,0.0}, draw opacity={0.0}, line width={1}, solid, fill={rgb,1:red,0.0;green,0.0;blue,0.0}, fill opacity={0.0}, text opacity={1.0}, font={{\fontsize{8 pt}{10.4 pt}\selectfont}}, text={rgb,1:red,0.0;green,0.0;blue,0.0}, cells={anchor={center}}, at={(1.02, 1)}, anchor={north west}}, axis background/.style={fill={rgb,1:red,0.0;green,0.0;blue,0.0}, opacity={0.0}}, anchor={north west}, xshift={63.5mm}, yshift={-0.0mm}, width={58.5mm}, height={50.8mm}, scaled x ticks={false}, xlabel={}, x tick style={color={rgb,1:red,0.0;green,0.0;blue,0.0}, opacity={1.0}}, x tick label style={color={rgb,1:red,0.0;green,0.0;blue,0.0}, opacity={1.0}, rotate={0}}, xlabel style={at={(ticklabel cs:0.5)}, anchor=near ticklabel, at={{(ticklabel cs:0.5)}}, anchor={near ticklabel}, font={{\fontsize{11 pt}{14.3 pt}\selectfont}}, color={rgb,1:red,0.0;green,0.0;blue,0.0}, draw opacity={1.0}, rotate={0.0}}, xmajorgrids={false}, xmin={-6.0}, xmax={24.0}, xticklabels={{$-5$,$0$,$5$,$10$,$15$,$20$}}, xtick={{-5.0,0.0,5.0,10.0,15.0,20.0}}, xtick align={inside}, xticklabel style={font={{\fontsize{8 pt}{10.4 pt}\selectfont}}, color={rgb,1:red,0.0;green,0.0;blue,0.0}, draw opacity={1.0}, rotate={0.0}}, x grid style={color={rgb,1:red,0.0;green,0.0;blue,0.0}, draw opacity={0.1}, line width={0.5}, solid}, axis x line*={left}, x axis line style={color={rgb,1:red,0.0;green,0.0;blue,0.0}, draw opacity={1.0}, line width={1}, solid}, scaled y ticks={false}, ylabel={$c_d$}, y tick style={color={rgb,1:red,0.0;green,0.0;blue,0.0}, opacity={1.0}}, y tick label style={color={rgb,1:red,0.0;green,0.0;blue,0.0}, opacity={1.0}, rotate={0}}, ylabel style={{rotate=-90}}, ymajorgrids={false}, ymin={0.0}, ymax={0.05}, yticklabels={{$0.00$,$0.01$,$0.02$,$0.03$,$0.04$}}, ytick={{0.0,0.010000000000000002,0.020000000000000004,0.030000000000000006,0.04000000000000001}}, ytick align={inside}, yticklabel style={font={{\fontsize{8 pt}{10.4 pt}\selectfont}}, color={rgb,1:red,0.0;green,0.0;blue,0.0}, draw opacity={1.0}, rotate={0.0}}, y grid style={color={rgb,1:red,0.0;green,0.0;blue,0.0}, draw opacity={0.1}, line width={0.5}, solid}, axis y line*={left}, y axis line style={color={rgb,1:red,0.0;green,0.0;blue,0.0}, draw opacity={1.0}, line width={1}, solid}, colorbar={false}]
    \addplot[color={rgb,1:red,0.0;green,0.3608;blue,0.6706}, name path={186bdab7-78c6-461c-93f3-93c7fac80a0a}, only marks, draw opacity={1.0}, line width={0}, solid, mark={triangle*}, mark size={3.0 pt}, mark repeat={1}, mark options={color={rgb,1:red,0.0;green,0.0;blue,0.0}, draw opacity={0.0}, fill={rgb,1:red,0.0;green,0.3608;blue,0.6706}, fill opacity={1.0}, line width={0.75}, rotate={0}, solid}, forget plot]
        table[row sep={\\}]
        {
            \\
            -3.0159617966864865  0.0174737251259478  \\
            -0.0369917919413592  0.0124818019364153  \\
            2.9962543791456753  0.0125602479580814  \\
            5.976637825722623  0.0148097406334239  \\
            9.017233894335131  0.0135435491524397  \\
            12.026451554281216  0.0105179254712314  \\
            15.004008117194523  0.013284586416824  \\
            18.047148381104304  0.0190529434926148  \\
            21.01509353956123  0.0385779765570576  \\
        }
        ;
    \addplot[color={rgb,1:red,0.7529;green,0.3255;blue,0.4039}, name path={0432f065-4814-41d2-9a71-4935caf86828}, only marks, draw opacity={1.0}, line width={0}, solid, mark={triangle*}, mark size={3.0 pt}, mark repeat={1}, mark options={color={rgb,1:red,0.0;green,0.0;blue,0.0}, draw opacity={0.0}, fill={rgb,1:red,0.7529;green,0.3255;blue,0.4039}, fill opacity={1.0}, line width={0.75}, rotate={0}, solid}, forget plot]
        table[row sep={\\}]
        {
            \\
            -3.0559617648107853  0.0184804773284282  \\
            0.0132681813318136  0.010357537743492  \\
            2.9939009166458614  0.0090176756225192  \\
            5.988229839141134  0.0133925068404506  \\
            8.004422727110608  0.0129763221049735  \\
            10.009202125762386  0.0144645595870763  \\
            12.999821664229412  0.0114583280257211  \\
            16.039947212611906  0.0161916650956134  \\
            19.054249741413134  0.0172337574327801  \\
            22.039447872454257  0.0311321264247754  \\
        }
        ;
    \addplot[color={rgb,1:red,0.5608;green,0.651;blue,0.3176}, name path={dced0768-2b35-4fd0-8bdd-d1b814d20e44}, only marks, draw opacity={1.0}, line width={0}, solid, mark={triangle*}, mark size={3.0 pt}, mark repeat={1}, mark options={color={rgb,1:red,0.0;green,0.0;blue,0.0}, draw opacity={0.0}, fill={rgb,1:red,0.5608;green,0.651;blue,0.3176}, fill opacity={1.0}, line width={0.75}, rotate={0}, solid}, forget plot]
        table[row sep={\\}]
        {
            \\
            -5.015664845173042  0.0219398907103825  \\
            -1.9168736545785716  0.0147851465474416  \\
            0.9916873654578584  0.0135214853452558  \\
            4.012584865043886  0.0135630899155489  \\
            6.992151018380527  0.0139027570789866  \\
            8.983474085113427  0.0131961003477397  \\
            12.045173041894353  0.0130396174863388  \\
            14.992415962907765  0.0119777694982613  \\
            18.021791687365457  0.0179197714853452  \\
            20.973273720814703  0.0235581222056632  \\
        }
        ;
    \addplot[color={rgb,1:red,0.5098;green,0.5098;blue,0.5098}, name path={6bf8f23f-0a71-4a6d-9bf6-e6bde9043f19}, draw opacity={1.0}, line width={1.0}, dashed, forget plot]
        table[row sep={\\}]
        {
            \\
            -15.0  0.19342858851356226  \\
            -14.505050505050505  0.1825511736008694  \\
            -14.01010101010101  0.17197853129710508  \\
            -13.515151515151516  0.1616563195018482  \\
            -13.02020202020202  0.1515301961146778  \\
            -12.525252525252526  0.14154581903517283  \\
            -12.030303030303031  0.13164884616291223  \\
            -11.535353535353535  0.12178493539747501  \\
            -11.04040404040404  0.11189974463844013  \\
            -10.545454545454545  0.10260591529963403  \\
            -10.05050505050505  0.09471085227657969  \\
            -9.555555555555555  0.09416157778087088  \\
            -9.06060606060606  0.09630965356893784  \\
            -8.565656565656566  0.08894797599132932  \\
            -8.070707070707071  0.07877076441051206  \\
            -7.575757575757576  0.06623868565832924  \\
            -7.08080808080808  0.05357972313981393  \\
            -6.585858585858586  0.04255296615648547  \\
            -6.090909090909091  0.03253944698920481  \\
            -5.595959595959596  0.023406117632169647  \\
            -5.101010101010101  0.016776300040109642  \\
            -4.606060606060606  0.013850321888362755  \\
            -4.111111111111111  0.012656846449273838  \\
            -3.6161616161616164  0.012264738992753578  \\
            -3.121212121212121  0.011948845063480705  \\
            -2.6262626262626263  0.011762030042036908  \\
            -2.1313131313131315  0.011583891345618498  \\
            -1.6363636363636365  0.011413100451654246  \\
            -1.1414141414141414  0.011247243670881814  \\
            -0.6464646464646465  0.011013411470531236  \\
            -0.15151515151515152  0.01060225166893664  \\
            0.3434343434343434  0.010109225448796005  \\
            0.8383838383838383  0.009640289631027204  \\
            1.3333333333333333  0.00933945565625332  \\
            1.8282828282828283  0.009416855456670179  \\
            2.323232323232323  0.01071708511846682  \\
            2.8181818181818183  0.013554397772055355  \\
            3.313131313131313  0.01435536227855909  \\
            3.808080808080808  0.01481071236793852  \\
            4.303030303030303  0.015324584893773962  \\
            4.797979797979798  0.015858271345099734  \\
            5.292929292929293  0.016492627038757736  \\
            5.787878787878788  0.017491751445786515  \\
            6.282828282828283  0.018897474567010977  \\
            6.777777777777778  0.021972874142274355  \\
            7.2727272727272725  0.027241859959554082  \\
            7.767676767676767  0.03397550640952604  \\
            8.262626262626263  0.04147859597103443  \\
            8.757575757575758  0.04945732004721046  \\
            9.252525252525253  0.05658865265697755  \\
            9.747474747474747  0.0621346186006454  \\
            10.242424242424242  0.06893042373082457  \\
            10.737373737373737  0.0788501395331888  \\
            11.232323232323232  0.09392029406724697  \\
            11.727272727272727  0.11195153937922153  \\
            12.222222222222221  0.1285801129900183  \\
            12.717171717171718  0.144430224252929  \\
            13.212121212121213  0.1597092671731783  \\
            13.707070707070708  0.17408150235480008  \\
            14.202020202020202  0.18706652054100284  \\
            14.696969696969697  0.1983850371525059  \\
            15.191919191919192  0.2082050866621403  \\
            15.686868686868687  0.21750167082134184  \\
            16.181818181818183  0.22638212587983422  \\
            16.67676767676768  0.2350571497121742  \\
            17.171717171717173  0.2435070367984886  \\
            17.666666666666668  0.25167944705150597  \\
            18.161616161616163  0.25906347280265807  \\
            18.656565656565657  0.26610890999685555  \\
            19.151515151515152  0.27309670918117884  \\
            19.646464646464647  0.28007245495534044  \\
            20.141414141414142  0.28703035106457303  \\
            20.636363636363637  0.29391654988319665  \\
            21.13131313131313  0.30070447485248264  \\
            21.626262626262626  0.30719465460675804  \\
            22.12121212121212  0.31369383321626276  \\
            22.616161616161616  0.32049925607999213  \\
            23.11111111111111  0.32768986260367317  \\
            23.606060606060606  0.3375489411247478  \\
            24.1010101010101  0.3453852748283662  \\
            24.595959595959595  0.35123258565238763  \\
            25.09090909090909  0.3570556623876905  \\
            25.585858585858585  0.3628768386138644  \\
            26.08080808080808  0.3686689132485595  \\
            26.575757575757574  0.37398501777229176  \\
            27.07070707070707  0.3788652100429109  \\
            27.565656565656564  0.38344684577088217  \\
            28.060606060606062  0.3877560216364902  \\
            28.555555555555557  0.39152907694958067  \\
            29.050505050505052  0.39567200457093665  \\
            29.545454545454547  0.4004694123817664  \\
            30.04040404040404  0.4052559415557995  \\
            30.535353535353536  0.4099596549758129  \\
            31.03030303030303  0.4146334333936847  \\
            31.525252525252526  0.4192734078645425  \\
            32.02020202020202  0.42391175269810183  \\
            32.515151515151516  0.42856171286311906  \\
            33.01010101010101  0.4332103222552433  \\
            33.505050505050505  0.43731191674448666  \\
            34.0  0.4407944528314002  \\
        }
        ;
    \addplot[color={rgb,1:red,0.0;green,0.3608;blue,0.6706}, name path={ba75a273-2ff7-47e6-8b30-ead0440043e0}, draw opacity={1.0}, line width={1.0}, solid, forget plot]
        table[row sep={\\}]
        {
            \\
            -3.022252591018093  0.01260648400627696  \\
            0.0034424375571671  0.010509561196732296  \\
            3.025323106929635  0.014132794880365443  \\
            6.04938341013227  0.020429947566584058  \\
            7.502654266395366  0.03450590742113849  \\
            9.076713164080155  0.05001431988831809  \\
            12.023699193232757  0.05515778082066728  \\
            15.022027708162602  0.0603908512855357  \\
            18.02308076538015  0.06562867697178017  \\
            21.030127815630657  0.07087696415070674  \\
        }
        ;
    \addplot[color={rgb,1:red,0.7529;green,0.3255;blue,0.4039}, name path={1943bf3b-e066-40ff-8081-8910dabee190}, draw opacity={1.0}, line width={1.0}, solid, forget plot]
        table[row sep={\\}]
        {
            \\
            -3.0084588836107535  0.012594618065409478  \\
            0.0007539384368496  0.010511789666196765  \\
            3.0078272054609574  0.014106168006709451  \\
            5.978456718584834  0.020119725943641895  \\
            8.009821576393573  0.04112527889118674  \\
            9.991905350161101  0.05161163158655949  \\
            13.01752142738885  0.056892327838161706  \\
            15.995282790591048  0.06208950184993604  \\
            18.97874963385591  0.06729663380314062  \\
            22.041451331490936  0.07264205666615683  \\
        }
        ;
    \addplot[color={rgb,1:red,0.5608;green,0.651;blue,0.3176}, name path={77814fa3-60d1-42a4-ae97-712264eec697}, draw opacity={1.0}, line width={1.0}, solid, forget plot]
        table[row sep={\\}]
        {
            \\
            -5.030781513350483  0.019954173973229156  \\
            -1.9854874478843243  0.011814321028041499  \\
            1.0325556639758702  0.009536767207884588  \\
            4.018238268428991  0.01544487399267513  \\
            7.00108223188149  0.028096753782872696  \\
            9.001188680919014  0.04988250479798578  \\
            11.990845382773  0.05510044010416824  \\
            14.979366628226739  0.06031639365480861  \\
            18.013306129690417  0.06561161701416485  \\
            21.01374966734676  0.07084837888939488  \\
        }
        ;
\end{axis}
\end{tikzpicture}
%
         \caption{Inflow angle of \(\beta_1=30^\circ\).}
         \label{}
     \end{subfigure}

     \begin{subfigure}[t]{\textwidth}
         \centering
        % Recommended preamble:
% \usetikzlibrary{arrows.meta}
% \usetikzlibrary{backgrounds}
% \usepgfplotslibrary{patchplots}
% \usepgfplotslibrary{fillbetween}
% \pgfplotsset{%
%     layers/standard/.define layer set={%
%         background,axis background,axis grid,axis ticks,axis lines,axis tick labels,pre main,main,axis descriptions,axis foreground%
%     }{
%         grid style={/pgfplots/on layer=axis grid},%
%         tick style={/pgfplots/on layer=axis ticks},%
%         axis line style={/pgfplots/on layer=axis lines},%
%         label style={/pgfplots/on layer=axis descriptions},%
%         legend style={/pgfplots/on layer=axis descriptions},%
%         title style={/pgfplots/on layer=axis descriptions},%
%         colorbar style={/pgfplots/on layer=axis descriptions},%
%         ticklabel style={/pgfplots/on layer=axis tick labels},%
%         axis background@ style={/pgfplots/on layer=axis background},%
%         3d box foreground style={/pgfplots/on layer=axis foreground},%
%     },
% }

\begin{tikzpicture}[/tikz/background rectangle/.style={fill={rgb,1:red,1.0;green,1.0;blue,1.0}, fill opacity={1.0}, draw opacity={1.0}}, show background rectangle]
\begin{axis}[point meta max={nan}, point meta min={nan}, legend cell align={left}, legend columns={1}, title={}, title style={at={{(0.5,1)}}, anchor={south}, font={{\fontsize{14 pt}{18.2 pt}\selectfont}}, color={rgb,1:red,0.0;green,0.0;blue,0.0}, draw opacity={1.0}, rotate={0.0}, align={center}}, legend style={color={rgb,1:red,0.0;green,0.0;blue,0.0}, draw opacity={0.0}, line width={1}, solid, fill={rgb,1:red,0.0;green,0.0;blue,0.0}, fill opacity={0.0}, text opacity={1.0}, font={{\fontsize{8 pt}{10.4 pt}\selectfont}}, text={rgb,1:red,0.0;green,0.0;blue,0.0}, cells={anchor={center}}, at={(1.02, 1)}, anchor={north west}}, axis background/.style={fill={rgb,1:red,0.0;green,0.0;blue,0.0}, opacity={0.0}}, anchor={north west}, xshift={0.0mm}, yshift={-0.0mm}, width={58.5mm}, height={50.8mm}, scaled x ticks={false}, xlabel={}, x tick style={color={rgb,1:red,0.0;green,0.0;blue,0.0}, opacity={1.0}}, x tick label style={color={rgb,1:red,0.0;green,0.0;blue,0.0}, opacity={1.0}, rotate={0}}, xlabel style={at={(ticklabel cs:0.5)}, anchor=near ticklabel, at={{(ticklabel cs:0.5)}}, anchor={near ticklabel}, font={{\fontsize{11 pt}{14.3 pt}\selectfont}}, color={rgb,1:red,0.0;green,0.0;blue,0.0}, draw opacity={1.0}, rotate={0.0}}, xmajorgrids={false}, xmin={-6.0}, xmax={24.0}, xticklabels={{$-5$,$0$,$5$,$10$,$15$,$20$}}, xtick={{-5.0,0.0,5.0,10.0,15.0,20.0}}, xtick align={inside}, xticklabel style={font={{\fontsize{8 pt}{10.4 pt}\selectfont}}, color={rgb,1:red,0.0;green,0.0;blue,0.0}, draw opacity={1.0}, rotate={0.0}}, x grid style={color={rgb,1:red,0.0;green,0.0;blue,0.0}, draw opacity={0.1}, line width={0.5}, solid}, axis x line*={left}, x axis line style={color={rgb,1:red,0.0;green,0.0;blue,0.0}, draw opacity={1.0}, line width={1}, solid}, scaled y ticks={false}, ylabel={$c_\ell$}, y tick style={color={rgb,1:red,0.0;green,0.0;blue,0.0}, opacity={1.0}}, y tick label style={color={rgb,1:red,0.0;green,0.0;blue,0.0}, opacity={1.0}, rotate={0}}, ylabel style={{rotate=-90}}, ymajorgrids={false}, ymin={-0.1}, ymax={1.2}, yticklabels={{$0.00$,$0.25$,$0.50$,$0.75$,$1.00$}}, ytick={{0.0,0.25,0.5,0.75,1.0}}, ytick align={inside}, yticklabel style={font={{\fontsize{8 pt}{10.4 pt}\selectfont}}, color={rgb,1:red,0.0;green,0.0;blue,0.0}, draw opacity={1.0}, rotate={0.0}}, y grid style={color={rgb,1:red,0.0;green,0.0;blue,0.0}, draw opacity={0.1}, line width={0.5}, solid}, axis y line*={left}, y axis line style={color={rgb,1:red,0.0;green,0.0;blue,0.0}, draw opacity={1.0}, line width={1}, solid}, colorbar={false}]
    \addplot[color={rgb,1:red,0.0;green,0.3608;blue,0.6706}, name path={5201e8fe-69f6-4092-b1b5-5c3a84e118c6}, only marks, draw opacity={1.0}, line width={0}, solid, mark={triangle*}, mark size={3.0 pt}, mark repeat={1}, mark options={color={rgb,1:red,0.0;green,0.0;blue,0.0}, draw opacity={0.0}, fill={rgb,1:red,0.0;green,0.3608;blue,0.6706}, fill opacity={1.0}, line width={0.75}, rotate={0}, solid}, forget plot]
        table[row sep={\\}]
        {
            \\
            0.9715774365763954  0.1238928154630615  \\
            5.017505514165627  0.282302738393145  \\
            7.0283426686564265  0.3267014018678255  \\
            9.037810198567028  0.3812866481577122  \\
            11.01537688929724  0.4231343858517781  \\
            14.028550965728002  0.512652192398013  \\
            17.015074430535776  0.5503839228901362  \\
            19.535468996550257  0.5549493381574555  \\
        }
        ;
    \addplot[color={rgb,1:red,0.7529;green,0.3255;blue,0.4039}, name path={551c9734-f0e2-4919-b4a7-9600c5124673}, only marks, draw opacity={1.0}, line width={0}, solid, mark={triangle*}, mark size={3.0 pt}, mark repeat={1}, mark options={color={rgb,1:red,0.0;green,0.0;blue,0.0}, draw opacity={0.0}, fill={rgb,1:red,0.7529;green,0.3255;blue,0.4039}, fill opacity={1.0}, line width={0.75}, rotate={0}, solid}, forget plot]
        table[row sep={\\}]
        {
            \\
            0.9359379426837711  0.1027775885008912  \\
            2.9413102184414655  0.1626465451389513  \\
            4.950813910536186  0.2156857541448649  \\
            6.961006172020411  0.2675866718787544  \\
            8.973952711062655  0.3149344245245463  \\
            11.032258758638495  0.3622972396257338  \\
            13.001222927926147  0.4073533472720817  \\
            15.02036659147393  0.4444564784696545  \\
            17.045707379527247  0.471314988219008  \\
            19.029820075383995  0.49132868788082  \\
            22.057890036903007  0.5105505327447021  \\
        }
        ;
    \addplot[color={rgb,1:red,0.5608;green,0.651;blue,0.3176}, name path={d1d8f2fa-3f64-4c76-ac6b-f6b706a926bd}, only marks, draw opacity={1.0}, line width={0}, solid, mark={triangle*}, mark size={3.0 pt}, mark repeat={1}, mark options={color={rgb,1:red,0.0;green,0.0;blue,0.0}, draw opacity={0.0}, fill={rgb,1:red,0.5608;green,0.651;blue,0.3176}, fill opacity={1.0}, line width={0.75}, rotate={0}, solid}, forget plot]
        table[row sep={\\}]
        {
            \\
            -3.0963491202232385  -0.024188047438183  \\
            -0.0279048135803421  0.0702271141771955  \\
            3.0006976203395084  0.14824432214557  \\
            5.98666770017828  0.21688628788466  \\
            8.009921711495235  0.258363692736997  \\
            10.033175722812185  0.299841097589334  \\
            13.071544841485153  0.35327881559569  \\
            16.020463529958917  0.3985001162700564  \\
            19.06673901247965  0.432040151926207  \\
            22.070847221145648  0.4550344934501199  \\
        }
        ;
    \addplot[color={rgb,1:red,0.5098;green,0.5098;blue,0.5098}, name path={094adbd8-c202-4325-ba85-f01db0ba642d}, draw opacity={1.0}, line width={1.0}, dashed, forget plot]
        table[row sep={\\}]
        {
            \\
            -15.0  -0.5665510105403683  \\
            -14.505050505050505  -0.5223707485770781  \\
            -14.01010101010101  -0.48288290323857036  \\
            -13.515151515151516  -0.4476923804233114  \\
            -13.02020202020202  -0.41640408602976736  \\
            -12.525252525252526  -0.38862292595640446  \\
            -12.030303030303031  -0.363953806101689  \\
            -11.535353535353535  -0.342001632364087  \\
            -11.04040404040404  -0.3223713106420649  \\
            -10.545454545454545  -0.31264275688078474  \\
            -10.05050505050505  -0.3229772463157242  \\
            -9.555555555555555  -0.37730526563488587  \\
            -9.06060606060606  -0.4398893615283618  \\
            -8.565656565656566  -0.44816433443905423  \\
            -8.070707070707071  -0.44277801246942355  \\
            -7.575757575757576  -0.4298637871228801  \\
            -7.08080808080808  -0.4108534742848187  \\
            -6.585858585858586  -0.38042950578354084  \\
            -6.090909090909091  -0.33981331145726257  \\
            -5.595959595959596  -0.2900177741710254  \\
            -5.101010101010101  -0.23727259949514537  \\
            -4.606060606060606  -0.1815499910867863  \\
            -4.111111111111111  -0.12544341290574643  \\
            -3.6161616161616164  -0.06876672586757099  \\
            -3.121212121212121  -0.012021506605778243  \\
            -2.6262626262626263  0.04483434055713697  \\
            -2.1313131313131315  0.10175187222194775  \\
            -1.6363636363636365  0.15885001136591717  \\
            -1.1414141414141414  0.21614787322246745  \\
            -0.6464646464646465  0.27326335201173685  \\
            -0.15151515151515152  0.33008787367812703  \\
            0.3434343434343434  0.3867358841059261  \\
            0.8383838383838383  0.4433265167789935  \\
            1.3333333333333333  0.4990846154557274  \\
            1.8282828282828283  0.5514163812619709  \\
            2.323232323232323  0.5956790139480467  \\
            2.8181818181818183  0.6300960032070431  \\
            3.313131313131313  0.6778677494717945  \\
            3.808080808080808  0.7295565593987938  \\
            4.303030303030303  0.7780694571148149  \\
            4.797979797979798  0.8255216383729825  \\
            5.292929292929293  0.8720063083639519  \\
            5.787878787878788  0.9167960654895222  \\
            6.282828282828283  0.957800935223836  \\
            6.777777777777778  0.995808680488238  \\
            7.2727272727272725  1.0336100742677  \\
            7.767676767676767  1.0699644187909074  \\
            8.262626262626263  1.0992053861622957  \\
            8.757575757575758  1.1047006399431818  \\
            9.252525252525253  1.1001338863619823  \\
            9.747474747474747  1.0881279859286563  \\
            10.242424242424242  1.0582474472708032  \\
            10.737373737373737  1.0246110095345213  \\
            11.232323232323232  0.9907201888390117  \\
            11.727272727272727  0.9564933909437671  \\
            12.222222222222221  0.9272047502821661  \\
            12.717171717171718  0.9210081124572284  \\
            13.212121212121213  0.9191664492896328  \\
            13.707070707070708  0.9170774786104494  \\
            14.202020202020202  0.9180098805020722  \\
            14.696969696969697  0.9220997625655921  \\
            15.191919191919192  0.9319796370148837  \\
            15.686868686868687  0.9447055160240829  \\
            16.181818181818183  0.9589114265942978  \\
            16.67676767676768  0.9734513568690647  \\
            17.171717171717173  0.9887075264975685  \\
            17.666666666666668  1.0042483369086435  \\
            18.161616161616163  1.0198102993993334  \\
            18.656565656565657  1.035398900104723  \\
            19.151515151515152  1.051295046008166  \\
            19.646464646464647  1.0691577941475554  \\
            20.141414141414142  1.0864846355610962  \\
            20.636363636363637  1.1030653457193929  \\
            21.13131313131313  1.1194857325791487  \\
            21.626262626262626  1.135774015505571  \\
            22.12121212121212  1.1517956055832994  \\
            22.616161616161616  1.167218420827846  \\
            23.11111111111111  1.1821984594240826  \\
            23.606060606060606  1.1966270886411536  \\
            24.1010101010101  1.2114379401269704  \\
            24.595959595959595  1.2269249451976803  \\
            25.09090909090909  1.2422736504402454  \\
            25.585858585858585  1.2573850331402625  \\
            26.08080808080808  1.2723818841719159  \\
            26.575757575757574  1.2872454497362291  \\
            27.07070707070707  1.3019543464686127  \\
            27.565656565656564  1.316387026323919  \\
            28.060606060606062  1.3308240336463104  \\
            28.555555555555557  1.345627164736373  \\
            29.050505050505052  1.3603996856804472  \\
            29.545454545454547  1.3745594103327088  \\
            30.04040404040404  1.3878745383906563  \\
            30.535353535353536  1.4003714583324913  \\
            31.03030303030303  1.412153211455866  \\
            31.525252525252526  1.423563342761  \\
            32.02020202020202  1.4349005047644947  \\
            32.515151515151516  1.446141174796294  \\
            33.01010101010101  1.4571919141366922  \\
            33.505050505050505  1.466658196592202  \\
            34.0  1.4743512710405462  \\
        }
        ;
    \addplot[color={rgb,1:red,0.0;green,0.3608;blue,0.6706}, name path={28e280ec-6b32-4992-9e7a-34998671cce0}, draw opacity={1.0}, line width={1.0}, solid, forget plot]
        table[row sep={\\}]
        {
            \\
            0.9715774365763954  0.3666332489921926  \\
            5.017505514165627  0.6619368912465308  \\
            7.0283426686564265  0.7836641410079228  \\
            9.037810198567028  0.8272798108412613  \\
            11.01537688929724  0.8202083666761489  \\
            14.028550965728002  0.8107671011242511  \\
            17.015074430535776  0.8030476417933076  \\
            19.535468996550257  0.7978369047484327  \\
        }
        ;
    \addplot[color={rgb,1:red,0.7529;green,0.3255;blue,0.4039}, name path={368065b7-13e0-4017-bbb2-3c84e0cbf11d}, draw opacity={1.0}, line width={1.0}, solid, forget plot]
        table[row sep={\\}]
        {
            \\
            0.9359379426837711  0.30658197003398696  \\
            2.9413102184414655  0.4251797886076548  \\
            4.950813910536186  0.549538766957086  \\
            6.961006172020411  0.6513058658015438  \\
            8.973952711062655  0.689736990159885  \\
            11.032258758638495  0.6813404004094882  \\
            13.001222927926147  0.6743777771033937  \\
            15.02036659147393  0.6683391649852679  \\
            17.045707379527247  0.6634165521231536  \\
            19.029820075383995  0.6597087177564925  \\
            22.057890036903007  0.6561829318012258  \\
        }
        ;
    \addplot[color={rgb,1:red,0.5608;green,0.651;blue,0.3176}, name path={031f7247-af78-4f7f-b9fe-4675a60961b6}, draw opacity={1.0}, line width={1.0}, solid, forget plot]
        table[row sep={\\}]
        {
            \\
            -3.0963491202232385  -0.00784047098086171  \\
            -0.0279048135803421  0.2053752732385813  \\
            3.0006976203395084  0.37440987615116605  \\
            5.98666770017828  0.5289546350699523  \\
            8.009921711495235  0.5982998628543609  \\
            10.033175722812185  0.5956996108383177  \\
            13.071544841485153  0.5853283876436531  \\
            16.020463529958917  0.5779040939891485  \\
            19.06673901247965  0.5730108327746676  \\
            22.070847221145648  0.570980090527877  \\
        }
        ;
\end{axis}
\begin{axis}[point meta max={nan}, point meta min={nan}, legend cell align={left}, legend columns={1}, title={}, title style={at={{(0.5,1)}}, anchor={south}, font={{\fontsize{14 pt}{18.2 pt}\selectfont}}, color={rgb,1:red,0.0;green,0.0;blue,0.0}, draw opacity={1.0}, rotate={0.0}, align={center}}, legend style={color={rgb,1:red,0.0;green,0.0;blue,0.0}, draw opacity={0.0}, line width={1}, solid, fill={rgb,1:red,0.0;green,0.0;blue,0.0}, fill opacity={0.0}, text opacity={1.0}, font={{\fontsize{8 pt}{10.4 pt}\selectfont}}, text={rgb,1:red,0.0;green,0.0;blue,0.0}, cells={anchor={center}}, at={(1.02, 1)}, anchor={north west}}, axis background/.style={fill={rgb,1:red,0.0;green,0.0;blue,0.0}, opacity={0.0}}, anchor={north west}, xshift={63.5mm}, yshift={-0.0mm}, width={58.5mm}, height={50.8mm}, scaled x ticks={false}, xlabel={}, x tick style={color={rgb,1:red,0.0;green,0.0;blue,0.0}, opacity={1.0}}, x tick label style={color={rgb,1:red,0.0;green,0.0;blue,0.0}, opacity={1.0}, rotate={0}}, xlabel style={at={(ticklabel cs:0.5)}, anchor=near ticklabel, at={{(ticklabel cs:0.5)}}, anchor={near ticklabel}, font={{\fontsize{11 pt}{14.3 pt}\selectfont}}, color={rgb,1:red,0.0;green,0.0;blue,0.0}, draw opacity={1.0}, rotate={0.0}}, xmajorgrids={false}, xmin={-6.0}, xmax={24.0}, xticklabels={{$-5$,$0$,$5$,$10$,$15$,$20$}}, xtick={{-5.0,0.0,5.0,10.0,15.0,20.0}}, xtick align={inside}, xticklabel style={font={{\fontsize{8 pt}{10.4 pt}\selectfont}}, color={rgb,1:red,0.0;green,0.0;blue,0.0}, draw opacity={1.0}, rotate={0.0}}, x grid style={color={rgb,1:red,0.0;green,0.0;blue,0.0}, draw opacity={0.1}, line width={0.5}, solid}, axis x line*={left}, x axis line style={color={rgb,1:red,0.0;green,0.0;blue,0.0}, draw opacity={1.0}, line width={1}, solid}, scaled y ticks={false}, ylabel={$c_d$}, y tick style={color={rgb,1:red,0.0;green,0.0;blue,0.0}, opacity={1.0}}, y tick label style={color={rgb,1:red,0.0;green,0.0;blue,0.0}, opacity={1.0}, rotate={0}}, ylabel style={{rotate=-90}}, ymajorgrids={false}, ymin={0.0}, ymax={0.05}, yticklabels={{$0.00$,$0.01$,$0.02$,$0.03$,$0.04$}}, ytick={{0.0,0.010000000000000002,0.020000000000000004,0.030000000000000006,0.04000000000000001}}, ytick align={inside}, yticklabel style={font={{\fontsize{8 pt}{10.4 pt}\selectfont}}, color={rgb,1:red,0.0;green,0.0;blue,0.0}, draw opacity={1.0}, rotate={0.0}}, y grid style={color={rgb,1:red,0.0;green,0.0;blue,0.0}, draw opacity={0.1}, line width={0.5}, solid}, axis y line*={left}, y axis line style={color={rgb,1:red,0.0;green,0.0;blue,0.0}, draw opacity={1.0}, line width={1}, solid}, colorbar={false}]
    \addplot[color={rgb,1:red,0.0;green,0.3608;blue,0.6706}, name path={8b6c5ac7-52a4-4124-be77-123dbf229719}, only marks, draw opacity={1.0}, line width={0}, solid, mark={triangle*}, mark size={3.0 pt}, mark repeat={1}, mark options={color={rgb,1:red,0.0;green,0.0;blue,0.0}, draw opacity={0.0}, fill={rgb,1:red,0.0;green,0.3608;blue,0.6706}, fill opacity={1.0}, line width={0.75}, rotate={0}, solid}, forget plot]
        table[row sep={\\}]
        {
            \\
            0.973154362416107  0.0141036451271066  \\
            5.033557046979865  0.0139006791171477  \\
            7.0134228187919465  0.0140107792755324  \\
            9.026845637583891  0.0130177413657547  \\
            11.241610738255034  0.0115179293763745  \\
            14.060402684563758  0.0146098494775584  \\
            16.97986577181208  0.0213533061382619  \\
            19.630872483221477  0.0449014938298333  \\
        }
        ;
    \addplot[color={rgb,1:red,0.7529;green,0.3255;blue,0.4039}, name path={a81cc571-f538-4e61-a557-8a4c1f7cd0b6}, only marks, draw opacity={1.0}, line width={0}, solid, mark={triangle*}, mark size={3.0 pt}, mark repeat={1}, mark options={color={rgb,1:red,0.0;green,0.0;blue,0.0}, draw opacity={0.0}, fill={rgb,1:red,0.7529;green,0.3255;blue,0.4039}, fill opacity={1.0}, line width={0.75}, rotate={0}, solid}, forget plot]
        table[row sep={\\}]
        {
            \\
            0.9631656250899976  0.0129756933444691  \\
            3.0142556806727523  0.0107639028885753  \\
            5.002102352907294  0.0115267977997293  \\
            7.022780289721509  0.0143520145148749  \\
            8.96881029864931  0.014540794286208  \\
            11.058261094951472  0.0134755061486623  \\
            13.023644270368344  0.0104589177202431  \\
            15.05399879042709  0.0116814503355125  \\
            17.034933617486967  0.0135891196037209  \\
            19.03349364974224  0.0200776142614405  \\
            22.08230855628834  0.0251176453647438  \\
        }
        ;
    \addplot[color={rgb,1:red,0.5608;green,0.651;blue,0.3176}, name path={650fd5ae-f0f9-4361-9a39-1525281c348f}, only marks, draw opacity={1.0}, line width={0}, solid, mark={triangle*}, mark size={3.0 pt}, mark repeat={1}, mark options={color={rgb,1:red,0.0;green,0.0;blue,0.0}, draw opacity={0.0}, fill={rgb,1:red,0.5608;green,0.651;blue,0.3176}, fill opacity={1.0}, line width={0.75}, rotate={0}, solid}, forget plot]
        table[row sep={\\}]
        {
            \\
            -3.041891444485559  0.0233222973104563  \\
            0.0107503846378556  0.0148238455995482  \\
            2.999045708610046  0.0102058542855473  \\
            6.049506300270707  0.0121779266558903  \\
            8.02952460708513  0.0150596919002084  \\
            10.029719360430018  0.0135891093929538  \\
            13.053459793171946  0.0113251017586227  \\
            16.015580267591098  0.012353399419635  \\
            19.044773793989915  0.0189130815821762  \\
            22.02434416811109  0.0261771865931797  \\
        }
        ;
    \addplot[color={rgb,1:red,0.5098;green,0.5098;blue,0.5098}, name path={f0a9721d-9888-4635-942a-39584131adc0}, draw opacity={1.0}, line width={1.0}, dashed, forget plot]
        table[row sep={\\}]
        {
            \\
            -15.0  0.19342858851356226  \\
            -14.505050505050505  0.1825511736008694  \\
            -14.01010101010101  0.17197853129710508  \\
            -13.515151515151516  0.1616563195018482  \\
            -13.02020202020202  0.1515301961146778  \\
            -12.525252525252526  0.14154581903517283  \\
            -12.030303030303031  0.13164884616291223  \\
            -11.535353535353535  0.12178493539747501  \\
            -11.04040404040404  0.11189974463844013  \\
            -10.545454545454545  0.10260591529963403  \\
            -10.05050505050505  0.09471085227657969  \\
            -9.555555555555555  0.09416157778087088  \\
            -9.06060606060606  0.09630965356893784  \\
            -8.565656565656566  0.08894797599132932  \\
            -8.070707070707071  0.07877076441051206  \\
            -7.575757575757576  0.06623868565832924  \\
            -7.08080808080808  0.05357972313981393  \\
            -6.585858585858586  0.04255296615648547  \\
            -6.090909090909091  0.03253944698920481  \\
            -5.595959595959596  0.023406117632169647  \\
            -5.101010101010101  0.016776300040109642  \\
            -4.606060606060606  0.013850321888362755  \\
            -4.111111111111111  0.012656846449273838  \\
            -3.6161616161616164  0.012264738992753578  \\
            -3.121212121212121  0.011948845063480705  \\
            -2.6262626262626263  0.011762030042036908  \\
            -2.1313131313131315  0.011583891345618498  \\
            -1.6363636363636365  0.011413100451654246  \\
            -1.1414141414141414  0.011247243670881814  \\
            -0.6464646464646465  0.011013411470531236  \\
            -0.15151515151515152  0.01060225166893664  \\
            0.3434343434343434  0.010109225448796005  \\
            0.8383838383838383  0.009640289631027204  \\
            1.3333333333333333  0.00933945565625332  \\
            1.8282828282828283  0.009416855456670179  \\
            2.323232323232323  0.01071708511846682  \\
            2.8181818181818183  0.013554397772055355  \\
            3.313131313131313  0.01435536227855909  \\
            3.808080808080808  0.01481071236793852  \\
            4.303030303030303  0.015324584893773962  \\
            4.797979797979798  0.015858271345099734  \\
            5.292929292929293  0.016492627038757736  \\
            5.787878787878788  0.017491751445786515  \\
            6.282828282828283  0.018897474567010977  \\
            6.777777777777778  0.021972874142274355  \\
            7.2727272727272725  0.027241859959554082  \\
            7.767676767676767  0.03397550640952604  \\
            8.262626262626263  0.04147859597103443  \\
            8.757575757575758  0.04945732004721046  \\
            9.252525252525253  0.05658865265697755  \\
            9.747474747474747  0.0621346186006454  \\
            10.242424242424242  0.06893042373082457  \\
            10.737373737373737  0.0788501395331888  \\
            11.232323232323232  0.09392029406724697  \\
            11.727272727272727  0.11195153937922153  \\
            12.222222222222221  0.1285801129900183  \\
            12.717171717171718  0.144430224252929  \\
            13.212121212121213  0.1597092671731783  \\
            13.707070707070708  0.17408150235480008  \\
            14.202020202020202  0.18706652054100284  \\
            14.696969696969697  0.1983850371525059  \\
            15.191919191919192  0.2082050866621403  \\
            15.686868686868687  0.21750167082134184  \\
            16.181818181818183  0.22638212587983422  \\
            16.67676767676768  0.2350571497121742  \\
            17.171717171717173  0.2435070367984886  \\
            17.666666666666668  0.25167944705150597  \\
            18.161616161616163  0.25906347280265807  \\
            18.656565656565657  0.26610890999685555  \\
            19.151515151515152  0.27309670918117884  \\
            19.646464646464647  0.28007245495534044  \\
            20.141414141414142  0.28703035106457303  \\
            20.636363636363637  0.29391654988319665  \\
            21.13131313131313  0.30070447485248264  \\
            21.626262626262626  0.30719465460675804  \\
            22.12121212121212  0.31369383321626276  \\
            22.616161616161616  0.32049925607999213  \\
            23.11111111111111  0.32768986260367317  \\
            23.606060606060606  0.3375489411247478  \\
            24.1010101010101  0.3453852748283662  \\
            24.595959595959595  0.35123258565238763  \\
            25.09090909090909  0.3570556623876905  \\
            25.585858585858585  0.3628768386138644  \\
            26.08080808080808  0.3686689132485595  \\
            26.575757575757574  0.37398501777229176  \\
            27.07070707070707  0.3788652100429109  \\
            27.565656565656564  0.38344684577088217  \\
            28.060606060606062  0.3877560216364902  \\
            28.555555555555557  0.39152907694958067  \\
            29.050505050505052  0.39567200457093665  \\
            29.545454545454547  0.4004694123817664  \\
            30.04040404040404  0.4052559415557995  \\
            30.535353535353536  0.4099596549758129  \\
            31.03030303030303  0.4146334333936847  \\
            31.525252525252526  0.4192734078645425  \\
            32.02020202020202  0.42391175269810183  \\
            32.515151515151516  0.42856171286311906  \\
            33.01010101010101  0.4332103222552433  \\
            33.505050505050505  0.43731191674448666  \\
            34.0  0.4407944528314002  \\
        }
        ;
    \addplot[color={rgb,1:red,0.0;green,0.3608;blue,0.6706}, name path={a7eda40d-9ba9-4dee-85a6-e6e6c98c51d9}, draw opacity={1.0}, line width={1.0}, solid, forget plot]
        table[row sep={\\}]
        {
            \\
            0.9715774365763954  0.009591121889139234  \\
            5.017505514165627  0.01709785014308344  \\
            7.0283426686564265  0.02839055990104187  \\
            9.037810198567028  0.04994642140431407  \\
            11.01537688929724  0.05339792640605925  \\
            14.028550965728002  0.05865690726619068  \\
            17.015074430535776  0.06386937402953864  \\
            19.535468996550257  0.06826829239201485  \\
        }
        ;
    \addplot[color={rgb,1:red,0.7529;green,0.3255;blue,0.4039}, name path={8dfdea52-5273-4772-95fa-1a60317b2aa8}, draw opacity={1.0}, line width={1.0}, solid, forget plot]
        table[row sep={\\}]
        {
            \\
            0.9359379426837711  0.009624079210363783  \\
            2.9413102184414655  0.013975174870431074  \\
            4.950813910536186  0.016981042572162496  \\
            6.961006172020411  0.02767495021644838  \\
            8.973952711062655  0.049827003390487325  \\
            11.032258758638495  0.05342739082648312  \\
            13.001222927926147  0.056863881590281974  \\
            15.02036659147393  0.060387952089980025  \\
            17.045707379527247  0.0639228386114652  \\
            19.029820075383995  0.06738576853865276  \\
            22.057890036903007  0.07267074761957777  \\
        }
        ;
    \addplot[color={rgb,1:red,0.5608;green,0.651;blue,0.3176}, name path={fa9a0b8e-cd3e-413b-b34f-79a9652f225f}, draw opacity={1.0}, line width={1.0}, solid, forget plot]
        table[row sep={\\}]
        {
            \\
            -3.0963491202232385  0.01267861641785704  \\
            -0.0279048135803421  0.010535545771553285  \\
            3.0006976203395084  0.014095204206766212  \\
            5.98666770017828  0.020154178709194066  \\
            8.009921711495235  0.04112640918763189  \\
            10.033175722812185  0.05168366197533272  \\
            13.071544841485153  0.05698661648309254  \\
            16.020463529958917  0.062133450530920654  \\
            19.06673901247965  0.06745020423951058  \\
            22.070847221145648  0.07269336217225228  \\
        }
        ;
\end{axis}
\end{tikzpicture}
%
         \caption{Inflow angle of \(\beta_1=45^\circ\).}
         \label{}
     \end{subfigure}

     \begin{subfigure}[t]{\textwidth}
         \centering
        % Recommended preamble:
% \usetikzlibrary{arrows.meta}
% \usetikzlibrary{backgrounds}
% \usepgfplotslibrary{patchplots}
% \usepgfplotslibrary{fillbetween}
% \pgfplotsset{%
%     layers/standard/.define layer set={%
%         background,axis background,axis grid,axis ticks,axis lines,axis tick labels,pre main,main,axis descriptions,axis foreground%
%     }{
%         grid style={/pgfplots/on layer=axis grid},%
%         tick style={/pgfplots/on layer=axis ticks},%
%         axis line style={/pgfplots/on layer=axis lines},%
%         label style={/pgfplots/on layer=axis descriptions},%
%         legend style={/pgfplots/on layer=axis descriptions},%
%         title style={/pgfplots/on layer=axis descriptions},%
%         colorbar style={/pgfplots/on layer=axis descriptions},%
%         ticklabel style={/pgfplots/on layer=axis tick labels},%
%         axis background@ style={/pgfplots/on layer=axis background},%
%         3d box foreground style={/pgfplots/on layer=axis foreground},%
%     },
% }

\begin{tikzpicture}[/tikz/background rectangle/.style={fill={rgb,1:red,1.0;green,1.0;blue,1.0}, fill opacity={1.0}, draw opacity={1.0}}, show background rectangle]
\begin{axis}[point meta max={nan}, point meta min={nan}, legend cell align={left}, legend columns={1}, title={}, title style={at={{(0.5,1)}}, anchor={south}, font={{\fontsize{14 pt}{18.2 pt}\selectfont}}, color={rgb,1:red,0.0;green,0.0;blue,0.0}, draw opacity={1.0}, rotate={0.0}, align={center}}, legend style={color={rgb,1:red,0.0;green,0.0;blue,0.0}, draw opacity={0.0}, line width={1}, solid, fill={rgb,1:red,0.0;green,0.0;blue,0.0}, fill opacity={0.0}, text opacity={1.0}, font={{\fontsize{8 pt}{10.4 pt}\selectfont}}, text={rgb,1:red,0.0;green,0.0;blue,0.0}, cells={anchor={center}}, at={(1.02, 1)}, anchor={north west}}, axis background/.style={fill={rgb,1:red,0.0;green,0.0;blue,0.0}, opacity={0.0}}, anchor={north west}, xshift={0.0mm}, yshift={-0.0mm}, width={58.5mm}, height={50.8mm}, scaled x ticks={false}, xlabel={}, x tick style={color={rgb,1:red,0.0;green,0.0;blue,0.0}, opacity={1.0}}, x tick label style={color={rgb,1:red,0.0;green,0.0;blue,0.0}, opacity={1.0}, rotate={0}}, xlabel style={at={(ticklabel cs:0.5)}, anchor=near ticklabel, at={{(ticklabel cs:0.5)}}, anchor={near ticklabel}, font={{\fontsize{11 pt}{14.3 pt}\selectfont}}, color={rgb,1:red,0.0;green,0.0;blue,0.0}, draw opacity={1.0}, rotate={0.0}}, xmajorgrids={false}, xmin={-6.0}, xmax={24.0}, xticklabels={{$-5$,$0$,$5$,$10$,$15$,$20$}}, xtick={{-5.0,0.0,5.0,10.0,15.0,20.0}}, xtick align={inside}, xticklabel style={font={{\fontsize{8 pt}{10.4 pt}\selectfont}}, color={rgb,1:red,0.0;green,0.0;blue,0.0}, draw opacity={1.0}, rotate={0.0}}, x grid style={color={rgb,1:red,0.0;green,0.0;blue,0.0}, draw opacity={0.1}, line width={0.5}, solid}, axis x line*={left}, x axis line style={color={rgb,1:red,0.0;green,0.0;blue,0.0}, draw opacity={1.0}, line width={1}, solid}, scaled y ticks={false}, ylabel={$c_\ell$}, y tick style={color={rgb,1:red,0.0;green,0.0;blue,0.0}, opacity={1.0}}, y tick label style={color={rgb,1:red,0.0;green,0.0;blue,0.0}, opacity={1.0}, rotate={0}}, ylabel style={{rotate=-90}}, ymajorgrids={false}, ymin={-0.1}, ymax={1.2}, yticklabels={{$0.00$,$0.25$,$0.50$,$0.75$,$1.00$}}, ytick={{0.0,0.25,0.5,0.75,1.0}}, ytick align={inside}, yticklabel style={font={{\fontsize{8 pt}{10.4 pt}\selectfont}}, color={rgb,1:red,0.0;green,0.0;blue,0.0}, draw opacity={1.0}, rotate={0.0}}, y grid style={color={rgb,1:red,0.0;green,0.0;blue,0.0}, draw opacity={0.1}, line width={0.5}, solid}, axis y line*={left}, y axis line style={color={rgb,1:red,0.0;green,0.0;blue,0.0}, draw opacity={1.0}, line width={1}, solid}, colorbar={false}]
    \addplot[color={rgb,1:red,0.0;green,0.3608;blue,0.6706}, name path={4174a5ad-5d7b-494c-a586-b6f9b156f561}, only marks, draw opacity={1.0}, line width={0}, solid, mark={triangle*}, mark size={3.0 pt}, mark repeat={1}, mark options={color={rgb,1:red,0.0;green,0.0;blue,0.0}, draw opacity={0.0}, fill={rgb,1:red,0.0;green,0.3608;blue,0.6706}, fill opacity={1.0}, line width={0.75}, rotate={0}, solid}, forget plot]
        table[row sep={\\}]
        {
            \\
            0.9552621969437096  0.112069690498058  \\
            3.963837956866696  0.2477105487379803  \\
            5.996525434044109  0.3395597038873091  \\
            8.010003208994036  0.3975733365792125  \\
            10.015690874284893  0.4672112735280124  \\
            12.042624293191402  0.5051465625034581  \\
            15.034734594062257  0.5403569729227298  \\
        }
        ;
    \addplot[color={rgb,1:red,0.7529;green,0.3255;blue,0.4039}, name path={2180c97b-4881-41ad-8434-26bcd61117a2}, only marks, draw opacity={1.0}, line width={0}, solid, mark={triangle*}, mark size={3.0 pt}, mark repeat={1}, mark options={color={rgb,1:red,0.0;green,0.0;blue,0.0}, draw opacity={0.0}, fill={rgb,1:red,0.7529;green,0.3255;blue,0.4039}, fill opacity={1.0}, line width={0.75}, rotate={0}, solid}, forget plot]
        table[row sep={\\}]
        {
            \\
            -0.0276428947522378  -0.0559556543614345  \\
            1.9934329842625633  0.0571301117416487  \\
            4.030304803135785  0.1450476803369801  \\
            5.946293804481364  0.2255718715263559  \\
            7.917276077926635  0.2851401158368924  \\
            9.971260164979816  0.3457921371321594  \\
            11.945533259228919  0.4001170069619143  \\
            13.970850640613113  0.4397779792897678  \\
            15.957628853916804  0.4741780260925526  \\
            17.958886678757388  0.4855072251798983  \\
            19.97133329433101  0.4790089510325865  \\
        }
        ;
    \addplot[color={rgb,1:red,0.5608;green,0.651;blue,0.3176}, name path={d5459804-7162-4a89-9ae2-9d8edb30ebb8}, only marks, draw opacity={1.0}, line width={0}, solid, mark={triangle*}, mark size={3.0 pt}, mark repeat={1}, mark options={color={rgb,1:red,0.0;green,0.0;blue,0.0}, draw opacity={0.0}, fill={rgb,1:red,0.5608;green,0.651;blue,0.3176}, fill opacity={1.0}, line width={0.75}, rotate={0}, solid}, forget plot]
        table[row sep={\\}]
        {
            \\
            2.97174239465558  0.1350683941267651  \\
            6.000878122854724  0.216966691975364  \\
            11.037713193432031  0.3145960028429833  \\
            13.030275832458038  0.3608369335561792  \\
            16.07532450520688  0.4106607025467988  \\
            19.05453822138884  0.4056830558675343  \\
        }
        ;
    \addplot[color={rgb,1:red,0.5098;green,0.5098;blue,0.5098}, name path={b5c2b1ed-c414-4fd7-9da2-e04325f042e3}, draw opacity={1.0}, line width={1.0}, dashed, forget plot]
        table[row sep={\\}]
        {
            \\
            -15.0  -0.5665510105403683  \\
            -14.505050505050505  -0.5223707485770781  \\
            -14.01010101010101  -0.48288290323857036  \\
            -13.515151515151516  -0.4476923804233114  \\
            -13.02020202020202  -0.41640408602976736  \\
            -12.525252525252526  -0.38862292595640446  \\
            -12.030303030303031  -0.363953806101689  \\
            -11.535353535353535  -0.342001632364087  \\
            -11.04040404040404  -0.3223713106420649  \\
            -10.545454545454545  -0.31264275688078474  \\
            -10.05050505050505  -0.3229772463157242  \\
            -9.555555555555555  -0.37730526563488587  \\
            -9.06060606060606  -0.4398893615283618  \\
            -8.565656565656566  -0.44816433443905423  \\
            -8.070707070707071  -0.44277801246942355  \\
            -7.575757575757576  -0.4298637871228801  \\
            -7.08080808080808  -0.4108534742848187  \\
            -6.585858585858586  -0.38042950578354084  \\
            -6.090909090909091  -0.33981331145726257  \\
            -5.595959595959596  -0.2900177741710254  \\
            -5.101010101010101  -0.23727259949514537  \\
            -4.606060606060606  -0.1815499910867863  \\
            -4.111111111111111  -0.12544341290574643  \\
            -3.6161616161616164  -0.06876672586757099  \\
            -3.121212121212121  -0.012021506605778243  \\
            -2.6262626262626263  0.04483434055713697  \\
            -2.1313131313131315  0.10175187222194775  \\
            -1.6363636363636365  0.15885001136591717  \\
            -1.1414141414141414  0.21614787322246745  \\
            -0.6464646464646465  0.27326335201173685  \\
            -0.15151515151515152  0.33008787367812703  \\
            0.3434343434343434  0.3867358841059261  \\
            0.8383838383838383  0.4433265167789935  \\
            1.3333333333333333  0.4990846154557274  \\
            1.8282828282828283  0.5514163812619709  \\
            2.323232323232323  0.5956790139480467  \\
            2.8181818181818183  0.6300960032070431  \\
            3.313131313131313  0.6778677494717945  \\
            3.808080808080808  0.7295565593987938  \\
            4.303030303030303  0.7780694571148149  \\
            4.797979797979798  0.8255216383729825  \\
            5.292929292929293  0.8720063083639519  \\
            5.787878787878788  0.9167960654895222  \\
            6.282828282828283  0.957800935223836  \\
            6.777777777777778  0.995808680488238  \\
            7.2727272727272725  1.0336100742677  \\
            7.767676767676767  1.0699644187909074  \\
            8.262626262626263  1.0992053861622957  \\
            8.757575757575758  1.1047006399431818  \\
            9.252525252525253  1.1001338863619823  \\
            9.747474747474747  1.0881279859286563  \\
            10.242424242424242  1.0582474472708032  \\
            10.737373737373737  1.0246110095345213  \\
            11.232323232323232  0.9907201888390117  \\
            11.727272727272727  0.9564933909437671  \\
            12.222222222222221  0.9272047502821661  \\
            12.717171717171718  0.9210081124572284  \\
            13.212121212121213  0.9191664492896328  \\
            13.707070707070708  0.9170774786104494  \\
            14.202020202020202  0.9180098805020722  \\
            14.696969696969697  0.9220997625655921  \\
            15.191919191919192  0.9319796370148837  \\
            15.686868686868687  0.9447055160240829  \\
            16.181818181818183  0.9589114265942978  \\
            16.67676767676768  0.9734513568690647  \\
            17.171717171717173  0.9887075264975685  \\
            17.666666666666668  1.0042483369086435  \\
            18.161616161616163  1.0198102993993334  \\
            18.656565656565657  1.035398900104723  \\
            19.151515151515152  1.051295046008166  \\
            19.646464646464647  1.0691577941475554  \\
            20.141414141414142  1.0864846355610962  \\
            20.636363636363637  1.1030653457193929  \\
            21.13131313131313  1.1194857325791487  \\
            21.626262626262626  1.135774015505571  \\
            22.12121212121212  1.1517956055832994  \\
            22.616161616161616  1.167218420827846  \\
            23.11111111111111  1.1821984594240826  \\
            23.606060606060606  1.1966270886411536  \\
            24.1010101010101  1.2114379401269704  \\
            24.595959595959595  1.2269249451976803  \\
            25.09090909090909  1.2422736504402454  \\
            25.585858585858585  1.2573850331402625  \\
            26.08080808080808  1.2723818841719159  \\
            26.575757575757574  1.2872454497362291  \\
            27.07070707070707  1.3019543464686127  \\
            27.565656565656564  1.316387026323919  \\
            28.060606060606062  1.3308240336463104  \\
            28.555555555555557  1.345627164736373  \\
            29.050505050505052  1.3603996856804472  \\
            29.545454545454547  1.3745594103327088  \\
            30.04040404040404  1.3878745383906563  \\
            30.535353535353536  1.4003714583324913  \\
            31.03030303030303  1.412153211455866  \\
            31.525252525252526  1.423563342761  \\
            32.02020202020202  1.4349005047644947  \\
            32.515151515151516  1.446141174796294  \\
            33.01010101010101  1.4571919141366922  \\
            33.505050505050505  1.466658196592202  \\
            34.0  1.4743512710405462  \\
        }
        ;
    \addplot[color={rgb,1:red,0.0;green,0.3608;blue,0.6706}, name path={e51bc4e7-ac72-460c-b18c-3a9110e94ebe}, draw opacity={1.0}, line width={1.0}, solid, forget plot]
        table[row sep={\\}]
        {
            \\
            0.9552621969437096  0.41831279314592795  \\
            3.963837956866696  0.6656241238276845  \\
            5.996525434044109  0.820760578243245  \\
            8.010003208994036  0.9184137389332372  \\
            10.015690874284893  0.9142798240629044  \\
            12.042624293191402  0.9027350338781513  \\
            15.034734594062257  0.8865733259649003  \\
        }
        ;
    \addplot[color={rgb,1:red,0.7529;green,0.3255;blue,0.4039}, name path={245de660-9b8d-4609-8030-6ca474fe1038}, draw opacity={1.0}, line width={1.0}, solid, forget plot]
        table[row sep={\\}]
        {
            \\
            -0.0276428947522378  0.2824203637228194  \\
            1.9934329842625633  0.45402532505720816  \\
            4.030304803135785  0.5878720810231691  \\
            5.946293804481364  0.7135809845482748  \\
            7.917276077926635  0.8008158569482177  \\
            9.971260164979816  0.7937871844264768  \\
            11.945533259228919  0.7784790554879751  \\
            13.970850640613113  0.7637123725456174  \\
            15.957628853916804  0.7501864713163902  \\
            17.958886678757388  0.7375546709419596  \\
            19.97133329433101  0.7258842098044365  \\
        }
        ;
    \addplot[color={rgb,1:red,0.5608;green,0.651;blue,0.3176}, name path={5ca30905-9688-4340-98af-1205cc5c8737}, draw opacity={1.0}, line width={1.0}, solid, forget plot]
        table[row sep={\\}]
        {
            \\
            2.97174239465558  0.454926096065697  \\
            6.000878122854724  0.6373218854725252  \\
            11.037713193432031  0.693903455395807  \\
            13.030275832458038  0.67853175116211  \\
            16.07532450520688  0.6571557964503238  \\
            19.05453822138884  0.6387796233637454  \\
        }
        ;
\end{axis}
\begin{axis}[point meta max={nan}, point meta min={nan}, legend cell align={left}, legend columns={1}, title={}, title style={at={{(0.5,1)}}, anchor={south}, font={{\fontsize{14 pt}{18.2 pt}\selectfont}}, color={rgb,1:red,0.0;green,0.0;blue,0.0}, draw opacity={1.0}, rotate={0.0}, align={center}}, legend style={color={rgb,1:red,0.0;green,0.0;blue,0.0}, draw opacity={0.0}, line width={1}, solid, fill={rgb,1:red,0.0;green,0.0;blue,0.0}, fill opacity={0.0}, text opacity={1.0}, font={{\fontsize{8 pt}{10.4 pt}\selectfont}}, text={rgb,1:red,0.0;green,0.0;blue,0.0}, cells={anchor={center}}, at={(1.02, 1)}, anchor={north west}}, axis background/.style={fill={rgb,1:red,0.0;green,0.0;blue,0.0}, opacity={0.0}}, anchor={north west}, xshift={63.5mm}, yshift={-0.0mm}, width={58.5mm}, height={50.8mm}, scaled x ticks={false}, xlabel={}, x tick style={color={rgb,1:red,0.0;green,0.0;blue,0.0}, opacity={1.0}}, x tick label style={color={rgb,1:red,0.0;green,0.0;blue,0.0}, opacity={1.0}, rotate={0}}, xlabel style={at={(ticklabel cs:0.5)}, anchor=near ticklabel, at={{(ticklabel cs:0.5)}}, anchor={near ticklabel}, font={{\fontsize{11 pt}{14.3 pt}\selectfont}}, color={rgb,1:red,0.0;green,0.0;blue,0.0}, draw opacity={1.0}, rotate={0.0}}, xmajorgrids={false}, xmin={-6.0}, xmax={24.0}, xticklabels={{$-5$,$0$,$5$,$10$,$15$,$20$}}, xtick={{-5.0,0.0,5.0,10.0,15.0,20.0}}, xtick align={inside}, xticklabel style={font={{\fontsize{8 pt}{10.4 pt}\selectfont}}, color={rgb,1:red,0.0;green,0.0;blue,0.0}, draw opacity={1.0}, rotate={0.0}}, x grid style={color={rgb,1:red,0.0;green,0.0;blue,0.0}, draw opacity={0.1}, line width={0.5}, solid}, axis x line*={left}, x axis line style={color={rgb,1:red,0.0;green,0.0;blue,0.0}, draw opacity={1.0}, line width={1}, solid}, scaled y ticks={false}, ylabel={$c_d$}, y tick style={color={rgb,1:red,0.0;green,0.0;blue,0.0}, opacity={1.0}}, y tick label style={color={rgb,1:red,0.0;green,0.0;blue,0.0}, opacity={1.0}, rotate={0}}, ylabel style={{rotate=-90}}, ymajorgrids={false}, ymin={0.0}, ymax={0.05}, yticklabels={{$0.00$,$0.01$,$0.02$,$0.03$,$0.04$}}, ytick={{0.0,0.010000000000000002,0.020000000000000004,0.030000000000000006,0.04000000000000001}}, ytick align={inside}, yticklabel style={font={{\fontsize{8 pt}{10.4 pt}\selectfont}}, color={rgb,1:red,0.0;green,0.0;blue,0.0}, draw opacity={1.0}, rotate={0.0}}, y grid style={color={rgb,1:red,0.0;green,0.0;blue,0.0}, draw opacity={0.1}, line width={0.5}, solid}, axis y line*={left}, y axis line style={color={rgb,1:red,0.0;green,0.0;blue,0.0}, draw opacity={1.0}, line width={1}, solid}, colorbar={false}]
    \addplot[color={rgb,1:red,0.0;green,0.3608;blue,0.6706}, name path={9b2ba10b-14c6-406b-a7d1-6dcf27e74306}, only marks, draw opacity={1.0}, line width={0}, solid, mark={triangle*}, mark size={3.0 pt}, mark repeat={1}, mark options={color={rgb,1:red,0.0;green,0.0;blue,0.0}, draw opacity={0.0}, fill={rgb,1:red,0.0;green,0.3608;blue,0.6706}, fill opacity={1.0}, line width={0.75}, rotate={0}, solid}, forget plot]
        table[row sep={\\}]
        {
            \\
            0.9405786094041284  0.0124244976869433  \\
            3.9943105710142923  0.0119309535317269  \\
            5.989035420838222  0.013187323068425  \\
            7.971166194849133  0.0109510460539943  \\
            10.02090727059311  0.0106226610508872  \\
            11.988897327901677  0.0100788510667679  \\
            14.979879859145202  0.0170956293585583  \\
        }
        ;
    \addplot[color={rgb,1:red,0.7529;green,0.3255;blue,0.4039}, name path={22223243-cf44-4c09-86b7-c8515d2b5d1c}, only marks, draw opacity={1.0}, line width={0}, solid, mark={triangle*}, mark size={3.0 pt}, mark repeat={1}, mark options={color={rgb,1:red,0.0;green,0.0;blue,0.0}, draw opacity={0.0}, fill={rgb,1:red,0.7529;green,0.3255;blue,0.4039}, fill opacity={1.0}, line width={0.75}, rotate={0}, solid}, forget plot]
        table[row sep={\\}]
        {
            \\
            0.0583636243957022  0.0160495578845746  \\
            2.0046954194062705  0.0108762588770488  \\
            4.016704576205485  0.0100060704020361  \\
            6.023886666325359  0.0102895508633867  \\
            8.028435810983769  0.0112023052900263  \\
            10.03693437383437  0.0111711487687323  \\
            12.011936019425285  0.009145755472504  \\
            14.02043458227589  0.0091145989512101  \\
            15.983149149046653  0.0100258174929971  \\
            18.015490495798254  0.0142962356193639  \\
            20.033789466755405  0.0219227815606053  \\
        }
        ;
    \addplot[color={rgb,1:red,0.5608;green,0.651;blue,0.3176}, name path={949b1ccf-08e3-4dd4-b894-42cac9c5328b}, only marks, draw opacity={1.0}, line width={0}, solid, mark={triangle*}, mark size={3.0 pt}, mark repeat={1}, mark options={color={rgb,1:red,0.0;green,0.0;blue,0.0}, draw opacity={0.0}, fill={rgb,1:red,0.5608;green,0.651;blue,0.3176}, fill opacity={1.0}, line width={0.75}, rotate={0}, solid}, forget plot]
        table[row sep={\\}]
        {
            \\
            2.9515296274850424  0.0120877967573827  \\
            5.937922215788459  0.0100133902721482  \\
            9.009723026442774  0.0095762159814708  \\
            11.025984365952521  0.0094133613202085  \\
            12.982049797336426  0.0088592935726694  \\
            15.988588110403402  0.0091202229299363  \\
            19.045068519590817  0.0186436016213086  \\
        }
        ;
    \addplot[color={rgb,1:red,0.5098;green,0.5098;blue,0.5098}, name path={aef06601-d4c7-46a8-b387-fd6f76f334e3}, draw opacity={1.0}, line width={1.0}, dashed, forget plot]
        table[row sep={\\}]
        {
            \\
            -15.0  0.19342858851356226  \\
            -14.505050505050505  0.1825511736008694  \\
            -14.01010101010101  0.17197853129710508  \\
            -13.515151515151516  0.1616563195018482  \\
            -13.02020202020202  0.1515301961146778  \\
            -12.525252525252526  0.14154581903517283  \\
            -12.030303030303031  0.13164884616291223  \\
            -11.535353535353535  0.12178493539747501  \\
            -11.04040404040404  0.11189974463844013  \\
            -10.545454545454545  0.10260591529963403  \\
            -10.05050505050505  0.09471085227657969  \\
            -9.555555555555555  0.09416157778087088  \\
            -9.06060606060606  0.09630965356893784  \\
            -8.565656565656566  0.08894797599132932  \\
            -8.070707070707071  0.07877076441051206  \\
            -7.575757575757576  0.06623868565832924  \\
            -7.08080808080808  0.05357972313981393  \\
            -6.585858585858586  0.04255296615648547  \\
            -6.090909090909091  0.03253944698920481  \\
            -5.595959595959596  0.023406117632169647  \\
            -5.101010101010101  0.016776300040109642  \\
            -4.606060606060606  0.013850321888362755  \\
            -4.111111111111111  0.012656846449273838  \\
            -3.6161616161616164  0.012264738992753578  \\
            -3.121212121212121  0.011948845063480705  \\
            -2.6262626262626263  0.011762030042036908  \\
            -2.1313131313131315  0.011583891345618498  \\
            -1.6363636363636365  0.011413100451654246  \\
            -1.1414141414141414  0.011247243670881814  \\
            -0.6464646464646465  0.011013411470531236  \\
            -0.15151515151515152  0.01060225166893664  \\
            0.3434343434343434  0.010109225448796005  \\
            0.8383838383838383  0.009640289631027204  \\
            1.3333333333333333  0.00933945565625332  \\
            1.8282828282828283  0.009416855456670179  \\
            2.323232323232323  0.01071708511846682  \\
            2.8181818181818183  0.013554397772055355  \\
            3.313131313131313  0.01435536227855909  \\
            3.808080808080808  0.01481071236793852  \\
            4.303030303030303  0.015324584893773962  \\
            4.797979797979798  0.015858271345099734  \\
            5.292929292929293  0.016492627038757736  \\
            5.787878787878788  0.017491751445786515  \\
            6.282828282828283  0.018897474567010977  \\
            6.777777777777778  0.021972874142274355  \\
            7.2727272727272725  0.027241859959554082  \\
            7.767676767676767  0.03397550640952604  \\
            8.262626262626263  0.04147859597103443  \\
            8.757575757575758  0.04945732004721046  \\
            9.252525252525253  0.05658865265697755  \\
            9.747474747474747  0.0621346186006454  \\
            10.242424242424242  0.06893042373082457  \\
            10.737373737373737  0.0788501395331888  \\
            11.232323232323232  0.09392029406724697  \\
            11.727272727272727  0.11195153937922153  \\
            12.222222222222221  0.1285801129900183  \\
            12.717171717171718  0.144430224252929  \\
            13.212121212121213  0.1597092671731783  \\
            13.707070707070708  0.17408150235480008  \\
            14.202020202020202  0.18706652054100284  \\
            14.696969696969697  0.1983850371525059  \\
            15.191919191919192  0.2082050866621403  \\
            15.686868686868687  0.21750167082134184  \\
            16.181818181818183  0.22638212587983422  \\
            16.67676767676768  0.2350571497121742  \\
            17.171717171717173  0.2435070367984886  \\
            17.666666666666668  0.25167944705150597  \\
            18.161616161616163  0.25906347280265807  \\
            18.656565656565657  0.26610890999685555  \\
            19.151515151515152  0.27309670918117884  \\
            19.646464646464647  0.28007245495534044  \\
            20.141414141414142  0.28703035106457303  \\
            20.636363636363637  0.29391654988319665  \\
            21.13131313131313  0.30070447485248264  \\
            21.626262626262626  0.30719465460675804  \\
            22.12121212121212  0.31369383321626276  \\
            22.616161616161616  0.32049925607999213  \\
            23.11111111111111  0.32768986260367317  \\
            23.606060606060606  0.3375489411247478  \\
            24.1010101010101  0.3453852748283662  \\
            24.595959595959595  0.35123258565238763  \\
            25.09090909090909  0.3570556623876905  \\
            25.585858585858585  0.3628768386138644  \\
            26.08080808080808  0.3686689132485595  \\
            26.575757575757574  0.37398501777229176  \\
            27.07070707070707  0.3788652100429109  \\
            27.565656565656564  0.38344684577088217  \\
            28.060606060606062  0.3877560216364902  \\
            28.555555555555557  0.39152907694958067  \\
            29.050505050505052  0.39567200457093665  \\
            29.545454545454547  0.4004694123817664  \\
            30.04040404040404  0.4052559415557995  \\
            30.535353535353536  0.4099596549758129  \\
            31.03030303030303  0.4146334333936847  \\
            31.525252525252526  0.4192734078645425  \\
            32.02020202020202  0.42391175269810183  \\
            32.515151515151516  0.42856171286311906  \\
            33.01010101010101  0.4332103222552433  \\
            33.505050505050505  0.43731191674448666  \\
            34.0  0.4407944528314002  \\
        }
        ;
    \addplot[color={rgb,1:red,0.0;green,0.3608;blue,0.6706}, name path={4117d232-a3c1-42cf-b749-b92354e50a2c}, draw opacity={1.0}, line width={1.0}, solid, forget plot]
        table[row sep={\\}]
        {
            \\
            0.9552621969437096  0.009606153355000535  \\
            3.963837956866696  0.015361819404820759  \\
            5.996525434044109  0.020195678951129517  \\
            8.010003208994036  0.041127329141117226  \\
            10.015690874284893  0.051653145157693085  \\
            12.042624293191402  0.05519081135125723  \\
            15.034734594062257  0.060413028985198354  \\
        }
        ;
    \addplot[color={rgb,1:red,0.7529;green,0.3255;blue,0.4039}, name path={71c57cad-61aa-4fcb-9891-918aa521394d}, draw opacity={1.0}, line width={1.0}, solid, forget plot]
        table[row sep={\\}]
        {
            \\
            -0.0276428947522378  0.010535328346800198  \\
            1.9934329842625633  0.009752667545764714  \\
            4.030304803135785  0.015463544173662256  \\
            5.946293804481364  0.019985797755888844  \\
            7.917276077926635  0.040047831786895936  \\
            9.971260164979816  0.051575598940888034  \\
            11.945533259228919  0.05502135552942963  \\
            13.970850640613113  0.058556201200930365  \\
            15.957628853916804  0.062023783332836646  \\
            17.958886678757388  0.06551663715458902  \\
            19.97133329433101  0.06902901910082608  \\
        }
        ;
    \addplot[color={rgb,1:red,0.5608;green,0.651;blue,0.3176}, name path={8c836bfd-9b68-4a72-8965-b81565b34c17}, draw opacity={1.0}, line width={1.0}, solid, forget plot]
        table[row sep={\\}]
        {
            \\
            2.97174239465558  0.014043797639244446  \\
            6.000878122854724  0.02021405377278831  \\
            11.037713193432031  0.05343691061109698  \\
            13.030275832458038  0.05691458847442883  \\
            16.07532450520688  0.062229200995788624  \\
            19.05453822138884  0.06742890984192829  \\
        }
        ;
\end{axis}
\end{tikzpicture}
%
         \caption{Inflow angle of \(\beta_1=60^\circ\).}
         \label{}
     \end{subfigure}

     \begin{subfigure}[t]{\textwidth}
         \centering
         \raisebox{-4em}{% Recommended preamble:
% \usetikzlibrary{arrows.meta}
% \usetikzlibrary{backgrounds}
% \usepgfplotslibrary{patchplots}
% \usepgfplotslibrary{fillbetween}
% \pgfplotsset{%
%     layers/standard/.define layer set={%
%         background,axis background,axis grid,axis ticks,axis lines,axis tick labels,pre main,main,axis descriptions,axis foreground%
%     }{
%         grid style={/pgfplots/on layer=axis grid},%
%         tick style={/pgfplots/on layer=axis ticks},%
%         axis line style={/pgfplots/on layer=axis lines},%
%         label style={/pgfplots/on layer=axis descriptions},%
%         legend style={/pgfplots/on layer=axis descriptions},%
%         title style={/pgfplots/on layer=axis descriptions},%
%         colorbar style={/pgfplots/on layer=axis descriptions},%
%         ticklabel style={/pgfplots/on layer=axis tick labels},%
%         axis background@ style={/pgfplots/on layer=axis background},%
%         3d box foreground style={/pgfplots/on layer=axis foreground},%
%     },
% }

\begin{tikzpicture}[/tikz/background rectangle/.style={fill={rgb,1:red,1.0;green,1.0;blue,1.0}, fill opacity={1.0}, draw opacity={1.0}}, show background rectangle]
\begin{axis}[point meta max={nan}, point meta min={nan}, legend cell align={left}, legend columns={1}, title={}, title style={at={{(0.5,1)}}, anchor={south}, font={{\fontsize{14 pt}{18.2 pt}\selectfont}}, color={rgb,1:red,0.0;green,0.0;blue,0.0}, draw opacity={1.0}, rotate={0.0}, align={center}}, legend style={color={rgb,1:red,0.0;green,0.0;blue,0.0}, draw opacity={0.0}, line width={1}, solid, fill={rgb,1:red,0.0;green,0.0;blue,0.0}, fill opacity={0.0}, text opacity={1.0}, font={{\fontsize{8 pt}{10.4 pt}\selectfont}}, text={rgb,1:red,0.0;green,0.0;blue,0.0}, cells={anchor={center}}, at={(1.02, 1)}, anchor={north west}}, axis background/.style={fill={rgb,1:red,0.0;green,0.0;blue,0.0}, opacity={0.0}}, anchor={north west}, xshift={0.0mm}, yshift={-0.0mm}, width={58.5mm}, height={50.8mm}, scaled x ticks={false}, xlabel={$\alpha~\mathrm{(degrees)}$}, x tick style={color={rgb,1:red,0.0;green,0.0;blue,0.0}, opacity={1.0}}, x tick label style={color={rgb,1:red,0.0;green,0.0;blue,0.0}, opacity={1.0}, rotate={0}}, xlabel style={at={(ticklabel cs:0.5)}, anchor=near ticklabel, at={{(ticklabel cs:0.5)}}, anchor={near ticklabel}, font={{\fontsize{11 pt}{14.3 pt}\selectfont}}, color={rgb,1:red,0.0;green,0.0;blue,0.0}, draw opacity={1.0}, rotate={0.0}}, xmajorgrids={false}, xmin={-6.0}, xmax={24.0}, xticklabels={{$-5$,$0$,$5$,$10$,$15$,$20$}}, xtick={{-5.0,0.0,5.0,10.0,15.0,20.0}}, xtick align={inside}, xticklabel style={font={{\fontsize{8 pt}{10.4 pt}\selectfont}}, color={rgb,1:red,0.0;green,0.0;blue,0.0}, draw opacity={1.0}, rotate={0.0}}, x grid style={color={rgb,1:red,0.0;green,0.0;blue,0.0}, draw opacity={0.1}, line width={0.5}, solid}, axis x line*={left}, x axis line style={color={rgb,1:red,0.0;green,0.0;blue,0.0}, draw opacity={1.0}, line width={1}, solid}, scaled y ticks={false}, ylabel={$c_\ell$}, y tick style={color={rgb,1:red,0.0;green,0.0;blue,0.0}, opacity={1.0}}, y tick label style={color={rgb,1:red,0.0;green,0.0;blue,0.0}, opacity={1.0}, rotate={0}}, ylabel style={{rotate=-90}}, ymajorgrids={false}, ymin={-0.1}, ymax={1.2}, yticklabels={{$0.00$,$0.25$,$0.50$,$0.75$,$1.00$}}, ytick={{0.0,0.25,0.5,0.75,1.0}}, ytick align={inside}, yticklabel style={font={{\fontsize{8 pt}{10.4 pt}\selectfont}}, color={rgb,1:red,0.0;green,0.0;blue,0.0}, draw opacity={1.0}, rotate={0.0}}, y grid style={color={rgb,1:red,0.0;green,0.0;blue,0.0}, draw opacity={0.1}, line width={0.5}, solid}, axis y line*={left}, y axis line style={color={rgb,1:red,0.0;green,0.0;blue,0.0}, draw opacity={1.0}, line width={1}, solid}, colorbar={false}]
    \addplot[color={rgb,1:red,0.0;green,0.3608;blue,0.6706}, name path={3e3e13cf-23a7-4981-8e33-2a6f9a3dfccc}, only marks, draw opacity={1.0}, line width={0}, solid, mark={triangle*}, mark size={3.0 pt}, mark repeat={1}, mark options={color={rgb,1:red,0.0;green,0.0;blue,0.0}, draw opacity={0.0}, fill={rgb,1:red,0.0;green,0.3608;blue,0.6706}, fill opacity={1.0}, line width={0.75}, rotate={0}, solid}, forget plot]
        table[row sep={\\}]
        {
            \\
            -1.4967175748469774  -0.1015323675291747  \\
            0.532188254979129  0.0248952099642306  \\
            2.5066630242703134  0.1840155712444191  \\
            4.534202955463169  0.3162823653949582  \\
            6.523702609720075  0.4111713434466746  \\
            8.52167083550312  0.4698571782241611  \\
            10.472310673644131  0.5308718701713351  \\
            12.570672448970882  0.5603752806494849  \\
            14.539068969344111  0.545480156053919  \\
            16.511290005890437  0.5142352248183784  \\
        }
        ;
    \addplot[color={rgb,1:red,0.7529;green,0.3255;blue,0.4039}, name path={e490c857-27b2-474a-8fe2-09e5648afd21}, only marks, draw opacity={1.0}, line width={0}, solid, mark={triangle*}, mark size={3.0 pt}, mark repeat={1}, mark options={color={rgb,1:red,0.0;green,0.0;blue,0.0}, draw opacity={0.0}, fill={rgb,1:red,0.7529;green,0.3255;blue,0.4039}, fill opacity={1.0}, line width={0.75}, rotate={0}, solid}, forget plot]
        table[row sep={\\}]
        {
            \\
            1.9641713409545245  0.0628069017720242  \\
            3.968033693690197  0.1857762552646392  \\
            5.954525163519206  0.2720241445568  \\
            7.977019971663141  0.3449964093705725  \\
            10.005919686353666  0.4007821749509928  \\
            12.006385497738872  0.4495322477340218  \\
            14.013547347786425  0.4803146167730917  \\
            16.033519010927154  0.4767239873454575  \\
            18.03262620577218  0.445786347844652  \\
        }
        ;
    \addplot[color={rgb,1:red,0.5608;green,0.651;blue,0.3176}, name path={8dbf193d-b8e5-494e-ae61-1548afd3d3bb}, only marks, draw opacity={1.0}, line width={0}, solid, mark={triangle*}, mark size={3.0 pt}, mark repeat={1}, mark options={color={rgb,1:red,0.0;green,0.0;blue,0.0}, draw opacity={0.0}, fill={rgb,1:red,0.5608;green,0.651;blue,0.3176}, fill opacity={1.0}, line width={0.75}, rotate={0}, solid}, forget plot]
        table[row sep={\\}]
        {
            \\
            -0.0098389717831298  -0.0629398729502143  \\
            1.9132146550450573  0.0435581326636135  \\
            3.9597207859358825  0.1350517063081695  \\
            5.933764219234746  0.1994881075491209  \\
            7.945235633032944  0.2729457822425764  \\
            11.98572905894519  0.3537538779731126  \\
            14.04739252474516  0.3881548234598907  \\
            15.993049194858916  0.409514699364751  \\
            17.948278918599495  0.3948160732752253  \\
        }
        ;
    \addplot[color={rgb,1:red,0.5098;green,0.5098;blue,0.5098}, name path={216217b3-0a2e-4ba9-83c6-dc0c74e922ab}, draw opacity={1.0}, line width={1.0}, dashed, forget plot]
        table[row sep={\\}]
        {
            \\
            -15.0  -0.5665510105403683  \\
            -14.505050505050505  -0.5223707485770781  \\
            -14.01010101010101  -0.48288290323857036  \\
            -13.515151515151516  -0.4476923804233114  \\
            -13.02020202020202  -0.41640408602976736  \\
            -12.525252525252526  -0.38862292595640446  \\
            -12.030303030303031  -0.363953806101689  \\
            -11.535353535353535  -0.342001632364087  \\
            -11.04040404040404  -0.3223713106420649  \\
            -10.545454545454545  -0.31264275688078474  \\
            -10.05050505050505  -0.3229772463157242  \\
            -9.555555555555555  -0.37730526563488587  \\
            -9.06060606060606  -0.4398893615283618  \\
            -8.565656565656566  -0.44816433443905423  \\
            -8.070707070707071  -0.44277801246942355  \\
            -7.575757575757576  -0.4298637871228801  \\
            -7.08080808080808  -0.4108534742848187  \\
            -6.585858585858586  -0.38042950578354084  \\
            -6.090909090909091  -0.33981331145726257  \\
            -5.595959595959596  -0.2900177741710254  \\
            -5.101010101010101  -0.23727259949514537  \\
            -4.606060606060606  -0.1815499910867863  \\
            -4.111111111111111  -0.12544341290574643  \\
            -3.6161616161616164  -0.06876672586757099  \\
            -3.121212121212121  -0.012021506605778243  \\
            -2.6262626262626263  0.04483434055713697  \\
            -2.1313131313131315  0.10175187222194775  \\
            -1.6363636363636365  0.15885001136591717  \\
            -1.1414141414141414  0.21614787322246745  \\
            -0.6464646464646465  0.27326335201173685  \\
            -0.15151515151515152  0.33008787367812703  \\
            0.3434343434343434  0.3867358841059261  \\
            0.8383838383838383  0.4433265167789935  \\
            1.3333333333333333  0.4990846154557274  \\
            1.8282828282828283  0.5514163812619709  \\
            2.323232323232323  0.5956790139480467  \\
            2.8181818181818183  0.6300960032070431  \\
            3.313131313131313  0.6778677494717945  \\
            3.808080808080808  0.7295565593987938  \\
            4.303030303030303  0.7780694571148149  \\
            4.797979797979798  0.8255216383729825  \\
            5.292929292929293  0.8720063083639519  \\
            5.787878787878788  0.9167960654895222  \\
            6.282828282828283  0.957800935223836  \\
            6.777777777777778  0.995808680488238  \\
            7.2727272727272725  1.0336100742677  \\
            7.767676767676767  1.0699644187909074  \\
            8.262626262626263  1.0992053861622957  \\
            8.757575757575758  1.1047006399431818  \\
            9.252525252525253  1.1001338863619823  \\
            9.747474747474747  1.0881279859286563  \\
            10.242424242424242  1.0582474472708032  \\
            10.737373737373737  1.0246110095345213  \\
            11.232323232323232  0.9907201888390117  \\
            11.727272727272727  0.9564933909437671  \\
            12.222222222222221  0.9272047502821661  \\
            12.717171717171718  0.9210081124572284  \\
            13.212121212121213  0.9191664492896328  \\
            13.707070707070708  0.9170774786104494  \\
            14.202020202020202  0.9180098805020722  \\
            14.696969696969697  0.9220997625655921  \\
            15.191919191919192  0.9319796370148837  \\
            15.686868686868687  0.9447055160240829  \\
            16.181818181818183  0.9589114265942978  \\
            16.67676767676768  0.9734513568690647  \\
            17.171717171717173  0.9887075264975685  \\
            17.666666666666668  1.0042483369086435  \\
            18.161616161616163  1.0198102993993334  \\
            18.656565656565657  1.035398900104723  \\
            19.151515151515152  1.051295046008166  \\
            19.646464646464647  1.0691577941475554  \\
            20.141414141414142  1.0864846355610962  \\
            20.636363636363637  1.1030653457193929  \\
            21.13131313131313  1.1194857325791487  \\
            21.626262626262626  1.135774015505571  \\
            22.12121212121212  1.1517956055832994  \\
            22.616161616161616  1.167218420827846  \\
            23.11111111111111  1.1821984594240826  \\
            23.606060606060606  1.1966270886411536  \\
            24.1010101010101  1.2114379401269704  \\
            24.595959595959595  1.2269249451976803  \\
            25.09090909090909  1.2422736504402454  \\
            25.585858585858585  1.2573850331402625  \\
            26.08080808080808  1.2723818841719159  \\
            26.575757575757574  1.2872454497362291  \\
            27.07070707070707  1.3019543464686127  \\
            27.565656565656564  1.316387026323919  \\
            28.060606060606062  1.3308240336463104  \\
            28.555555555555557  1.345627164736373  \\
            29.050505050505052  1.3603996856804472  \\
            29.545454545454547  1.3745594103327088  \\
            30.04040404040404  1.3878745383906563  \\
            30.535353535353536  1.4003714583324913  \\
            31.03030303030303  1.412153211455866  \\
            31.525252525252526  1.423563342761  \\
            32.02020202020202  1.4349005047644947  \\
            32.515151515151516  1.446141174796294  \\
            33.01010101010101  1.4571919141366922  \\
            33.505050505050505  1.466658196592202  \\
            34.0  1.4743512710405462  \\
        }
        ;
    \addplot[color={rgb,1:red,0.0;green,0.3608;blue,0.6706}, name path={c7179110-cdc8-4e54-969c-c94d24386e96}, draw opacity={1.0}, line width={1.0}, solid, forget plot]
        table[row sep={\\}]
        {
            \\
            -1.4967175748469774  0.17407247017677072  \\
            0.532188254979129  0.41022029402126386  \\
            2.5066630242703134  0.6105614188462789  \\
            4.534202955463169  0.7895041957436538  \\
            6.523702609720075  0.9327135735019897  \\
            8.52167083550312  0.9956332479959229  \\
            10.472310673644131  0.9844403945015514  \\
            12.570672448970882  0.9715093060510268  \\
            14.539068969344111  0.9594139156866938  \\
            16.511290005890437  0.9474157329136081  \\
        }
        ;
    \addplot[color={rgb,1:red,0.7529;green,0.3255;blue,0.4039}, name path={3c8422ca-e3bc-40c5-9584-34412b34a8ee}, draw opacity={1.0}, line width={1.0}, solid, forget plot]
        table[row sep={\\}]
        {
            \\
            1.9641713409545245  0.5201334002753634  \\
            3.968033693690197  0.6695769274518379  \\
            5.954525163519206  0.8148053359782522  \\
            7.977019971663141  0.9069064953984333  \\
            10.005919686353666  0.8955247919419036  \\
            12.006385497738872  0.8762697200831823  \\
            14.013547347786425  0.8574670865512557  \\
            16.033519010927154  0.8391912280996394  \\
            18.03262620577218  0.8218410469896203  \\
        }
        ;
    \addplot[color={rgb,1:red,0.5608;green,0.651;blue,0.3176}, name path={350071ab-7c4f-482c-95eb-f8877394be1c}, draw opacity={1.0}, line width={1.0}, solid, forget plot]
        table[row sep={\\}]
        {
            \\
            -0.0098389717831298  0.29917382288135164  \\
            1.9132146550450573  0.46938879278593  \\
            3.9597207859358825  0.606930126525406  \\
            5.933764219234746  0.7373658971867093  \\
            7.945235633032944  0.8229753790072039  \\
            11.98572905894519  0.7886432117080772  \\
            14.04739252474516  0.7677539984775523  \\
            15.993049194858916  0.7489528819958252  \\
            17.948278918599495  0.7310032635716652  \\
        }
        ;
\end{axis}
\begin{axis}[point meta max={nan}, point meta min={nan}, legend cell align={left}, legend columns={1}, title={}, title style={at={{(0.5,1)}}, anchor={south}, font={{\fontsize{14 pt}{18.2 pt}\selectfont}}, color={rgb,1:red,0.0;green,0.0;blue,0.0}, draw opacity={1.0}, rotate={0.0}, align={center}}, legend style={color={rgb,1:red,0.0;green,0.0;blue,0.0}, draw opacity={0.0}, line width={1}, solid, fill={rgb,1:red,0.0;green,0.0;blue,0.0}, fill opacity={0.0}, text opacity={1.0}, font={{\fontsize{8 pt}{10.4 pt}\selectfont}}, text={rgb,1:red,0.0;green,0.0;blue,0.0}, cells={anchor={center}}, at={(1.02, 1)}, anchor={north west}}, axis background/.style={fill={rgb,1:red,0.0;green,0.0;blue,0.0}, opacity={0.0}}, anchor={north west}, xshift={63.5mm}, yshift={-0.0mm}, width={58.5mm}, height={50.8mm}, scaled x ticks={false}, xlabel={$\alpha~\mathrm{(degrees)}$}, x tick style={color={rgb,1:red,0.0;green,0.0;blue,0.0}, opacity={1.0}}, x tick label style={color={rgb,1:red,0.0;green,0.0;blue,0.0}, opacity={1.0}, rotate={0}}, xlabel style={at={(ticklabel cs:0.5)}, anchor=near ticklabel, at={{(ticklabel cs:0.5)}}, anchor={near ticklabel}, font={{\fontsize{11 pt}{14.3 pt}\selectfont}}, color={rgb,1:red,0.0;green,0.0;blue,0.0}, draw opacity={1.0}, rotate={0.0}}, xmajorgrids={false}, xmin={-6.0}, xmax={24.0}, xticklabels={{$-5$,$0$,$5$,$10$,$15$,$20$}}, xtick={{-5.0,0.0,5.0,10.0,15.0,20.0}}, xtick align={inside}, xticklabel style={font={{\fontsize{8 pt}{10.4 pt}\selectfont}}, color={rgb,1:red,0.0;green,0.0;blue,0.0}, draw opacity={1.0}, rotate={0.0}}, x grid style={color={rgb,1:red,0.0;green,0.0;blue,0.0}, draw opacity={0.1}, line width={0.5}, solid}, axis x line*={left}, x axis line style={color={rgb,1:red,0.0;green,0.0;blue,0.0}, draw opacity={1.0}, line width={1}, solid}, scaled y ticks={false}, ylabel={$c_d$}, y tick style={color={rgb,1:red,0.0;green,0.0;blue,0.0}, opacity={1.0}}, y tick label style={color={rgb,1:red,0.0;green,0.0;blue,0.0}, opacity={1.0}, rotate={0}}, ylabel style={{rotate=-90}}, ymajorgrids={false}, ymin={0.0}, ymax={0.05}, yticklabels={{$0.00$,$0.01$,$0.02$,$0.03$,$0.04$}}, ytick={{0.0,0.010000000000000002,0.020000000000000004,0.030000000000000006,0.04000000000000001}}, ytick align={inside}, yticklabel style={font={{\fontsize{8 pt}{10.4 pt}\selectfont}}, color={rgb,1:red,0.0;green,0.0;blue,0.0}, draw opacity={1.0}, rotate={0.0}}, y grid style={color={rgb,1:red,0.0;green,0.0;blue,0.0}, draw opacity={0.1}, line width={0.5}, solid}, axis y line*={left}, y axis line style={color={rgb,1:red,0.0;green,0.0;blue,0.0}, draw opacity={1.0}, line width={1}, solid}, colorbar={false}]
    \addplot[color={rgb,1:red,0.0;green,0.3608;blue,0.6706}, name path={05e03c07-4fad-458c-ad19-491189eb894d}, only marks, draw opacity={1.0}, line width={0}, solid, mark={triangle*}, mark size={3.0 pt}, mark repeat={1}, mark options={color={rgb,1:red,0.0;green,0.0;blue,0.0}, draw opacity={0.0}, fill={rgb,1:red,0.0;green,0.3608;blue,0.6706}, fill opacity={1.0}, line width={0.75}, rotate={0}, solid}, forget plot]
        table[row sep={\\}]
        {
            \\
            -1.4956977792346144  0.0143721489251277  \\
            0.4804283099784197  0.0104382346781757  \\
            2.4925018438088973  0.012332349968587  \\
            4.510038515119231  0.0113127919364091  \\
            6.524297303941651  0.0120414378977847  \\
            8.495397306673219  0.0107881031076147  \\
            10.558278019066343  0.0105850564979468  \\
            12.526318664809192  0.0109633787684267  \\
            14.533803163156598  0.0153049796498129  \\
            16.58586686333961  0.0458710062188715  \\
        }
        ;
    \addplot[color={rgb,1:red,0.7529;green,0.3255;blue,0.4039}, name path={1edcfa55-0c9a-4e41-b624-352ae3c25db1}, only marks, draw opacity={1.0}, line width={0}, solid, mark={triangle*}, mark size={3.0 pt}, mark repeat={1}, mark options={color={rgb,1:red,0.0;green,0.0;blue,0.0}, draw opacity={0.0}, fill={rgb,1:red,0.7529;green,0.3255;blue,0.4039}, fill opacity={1.0}, line width={0.75}, rotate={0}, solid}, forget plot]
        table[row sep={\\}]
        {
            \\
            1.9572788314346408  0.0097556078530629  \\
            4.001002684371022  0.0116765272075484  \\
            6.01391589619783  0.0135184729993161  \\
            7.9761594754889895  0.0116760404675624  \\
            9.99905085702743  0.0103050156121851  \\
            11.98942800750553  0.0094035731582367  \\
            14.00307132931124  0.0110104235367988  \\
            16.010873771285745  0.0144980372210067  \\
            18.0278269249958  0.0250391217263694  \\
        }
        ;
    \addplot[color={rgb,1:red,0.5608;green,0.651;blue,0.3176}, name path={8b6e56f7-00b0-45f0-b2a7-462fa6a2663c}, only marks, draw opacity={1.0}, line width={0}, solid, mark={triangle*}, mark size={3.0 pt}, mark repeat={1}, mark options={color={rgb,1:red,0.0;green,0.0;blue,0.0}, draw opacity={0.0}, fill={rgb,1:red,0.5608;green,0.651;blue,0.3176}, fill opacity={1.0}, line width={0.75}, rotate={0}, solid}, forget plot]
        table[row sep={\\}]
        {
            \\
            0.0191039847875674  0.0182305965941545  \\
            1.9793187572091382  0.0142457353228404  \\
            4.0457994450115535  0.014168567606511  \\
            5.956662256731908  0.0130879984669642  \\
            7.951635668141968  0.0106060282211257  \\
            9.981036052137958  0.0097271122543365  \\
            11.96605947147116  0.0102500396157092  \\
            13.952409556414613  0.010372313962787  \\
            16.013251915373473  0.0119979215572106  \\
            17.981692014578584  0.0175290115972685  \\
        }
        ;
    \addplot[color={rgb,1:red,0.5098;green,0.5098;blue,0.5098}, name path={e0c18771-a880-4905-9763-6686ff7e7275}, draw opacity={1.0}, line width={1.0}, dashed, forget plot]
        table[row sep={\\}]
        {
            \\
            -15.0  0.19342858851356226  \\
            -14.505050505050505  0.1825511736008694  \\
            -14.01010101010101  0.17197853129710508  \\
            -13.515151515151516  0.1616563195018482  \\
            -13.02020202020202  0.1515301961146778  \\
            -12.525252525252526  0.14154581903517283  \\
            -12.030303030303031  0.13164884616291223  \\
            -11.535353535353535  0.12178493539747501  \\
            -11.04040404040404  0.11189974463844013  \\
            -10.545454545454545  0.10260591529963403  \\
            -10.05050505050505  0.09471085227657969  \\
            -9.555555555555555  0.09416157778087088  \\
            -9.06060606060606  0.09630965356893784  \\
            -8.565656565656566  0.08894797599132932  \\
            -8.070707070707071  0.07877076441051206  \\
            -7.575757575757576  0.06623868565832924  \\
            -7.08080808080808  0.05357972313981393  \\
            -6.585858585858586  0.04255296615648547  \\
            -6.090909090909091  0.03253944698920481  \\
            -5.595959595959596  0.023406117632169647  \\
            -5.101010101010101  0.016776300040109642  \\
            -4.606060606060606  0.013850321888362755  \\
            -4.111111111111111  0.012656846449273838  \\
            -3.6161616161616164  0.012264738992753578  \\
            -3.121212121212121  0.011948845063480705  \\
            -2.6262626262626263  0.011762030042036908  \\
            -2.1313131313131315  0.011583891345618498  \\
            -1.6363636363636365  0.011413100451654246  \\
            -1.1414141414141414  0.011247243670881814  \\
            -0.6464646464646465  0.011013411470531236  \\
            -0.15151515151515152  0.01060225166893664  \\
            0.3434343434343434  0.010109225448796005  \\
            0.8383838383838383  0.009640289631027204  \\
            1.3333333333333333  0.00933945565625332  \\
            1.8282828282828283  0.009416855456670179  \\
            2.323232323232323  0.01071708511846682  \\
            2.8181818181818183  0.013554397772055355  \\
            3.313131313131313  0.01435536227855909  \\
            3.808080808080808  0.01481071236793852  \\
            4.303030303030303  0.015324584893773962  \\
            4.797979797979798  0.015858271345099734  \\
            5.292929292929293  0.016492627038757736  \\
            5.787878787878788  0.017491751445786515  \\
            6.282828282828283  0.018897474567010977  \\
            6.777777777777778  0.021972874142274355  \\
            7.2727272727272725  0.027241859959554082  \\
            7.767676767676767  0.03397550640952604  \\
            8.262626262626263  0.04147859597103443  \\
            8.757575757575758  0.04945732004721046  \\
            9.252525252525253  0.05658865265697755  \\
            9.747474747474747  0.0621346186006454  \\
            10.242424242424242  0.06893042373082457  \\
            10.737373737373737  0.0788501395331888  \\
            11.232323232323232  0.09392029406724697  \\
            11.727272727272727  0.11195153937922153  \\
            12.222222222222221  0.1285801129900183  \\
            12.717171717171718  0.144430224252929  \\
            13.212121212121213  0.1597092671731783  \\
            13.707070707070708  0.17408150235480008  \\
            14.202020202020202  0.18706652054100284  \\
            14.696969696969697  0.1983850371525059  \\
            15.191919191919192  0.2082050866621403  \\
            15.686868686868687  0.21750167082134184  \\
            16.181818181818183  0.22638212587983422  \\
            16.67676767676768  0.2350571497121742  \\
            17.171717171717173  0.2435070367984886  \\
            17.666666666666668  0.25167944705150597  \\
            18.161616161616163  0.25906347280265807  \\
            18.656565656565657  0.26610890999685555  \\
            19.151515151515152  0.27309670918117884  \\
            19.646464646464647  0.28007245495534044  \\
            20.141414141414142  0.28703035106457303  \\
            20.636363636363637  0.29391654988319665  \\
            21.13131313131313  0.30070447485248264  \\
            21.626262626262626  0.30719465460675804  \\
            22.12121212121212  0.31369383321626276  \\
            22.616161616161616  0.32049925607999213  \\
            23.11111111111111  0.32768986260367317  \\
            23.606060606060606  0.3375489411247478  \\
            24.1010101010101  0.3453852748283662  \\
            24.595959595959595  0.35123258565238763  \\
            25.09090909090909  0.3570556623876905  \\
            25.585858585858585  0.3628768386138644  \\
            26.08080808080808  0.3686689132485595  \\
            26.575757575757574  0.37398501777229176  \\
            27.07070707070707  0.3788652100429109  \\
            27.565656565656564  0.38344684577088217  \\
            28.060606060606062  0.3877560216364902  \\
            28.555555555555557  0.39152907694958067  \\
            29.050505050505052  0.39567200457093665  \\
            29.545454545454547  0.4004694123817664  \\
            30.04040404040404  0.4052559415557995  \\
            30.535353535353536  0.4099596549758129  \\
            31.03030303030303  0.4146334333936847  \\
            31.525252525252526  0.4192734078645425  \\
            32.02020202020202  0.42391175269810183  \\
            32.515151515151516  0.42856171286311906  \\
            33.01010101010101  0.4332103222552433  \\
            33.505050505050505  0.43731191674448666  \\
            34.0  0.4407944528314002  \\
        }
        ;
    \addplot[color={rgb,1:red,0.0;green,0.3608;blue,0.6706}, name path={d40f53af-78fd-4a7c-991f-83d72d0455f5}, draw opacity={1.0}, line width={1.0}, solid, forget plot]
        table[row sep={\\}]
        {
            \\
            -1.4967175748469774  0.011579518059453009  \\
            0.532188254979129  0.010016944200912918  \\
            2.5066630242703134  0.011940214094511523  \\
            4.534202955463169  0.016272734955427037  \\
            6.523702609720075  0.02354536900139248  \\
            8.52167083550312  0.04685021478371466  \\
            10.472310673644131  0.05245009705229581  \\
            12.570672448970882  0.056112429244627636  \\
            14.539068969344111  0.05954792927141587  \\
            16.511290005890437  0.06299010433676155  \\
        }
        ;
    \addplot[color={rgb,1:red,0.7529;green,0.3255;blue,0.4039}, name path={ddc52191-54f2-4334-a8fa-1622b9cad4df}, draw opacity={1.0}, line width={1.0}, solid, forget plot]
        table[row sep={\\}]
        {
            \\
            1.9641713409545245  0.009715059108937335  \\
            3.968033693690197  0.015368105087339946  \\
            5.954525163519206  0.020019916035599713  \\
            7.977019971663141  0.04075332131828034  \\
            10.005919686353666  0.05163609121753359  \\
            12.006385497738872  0.055127562721445  \\
            14.013547347786425  0.05863072101291913  \\
            16.033519010927154  0.06215623664376715  \\
            18.03262620577218  0.06564533690812248  \\
        }
        ;
    \addplot[color={rgb,1:red,0.5608;green,0.651;blue,0.3176}, name path={f62ad62b-35c5-4ff6-9a60-87a9324c86a1}, draw opacity={1.0}, line width={1.0}, solid, forget plot]
        table[row sep={\\}]
        {
            \\
            -0.0098389717831298  0.010520562226905703  \\
            1.9132146550450573  0.009654347138969496  \\
            3.9597207859358825  0.01535567232895031  \\
            5.933764219234746  0.01993407545323115  \\
            7.945235633032944  0.0403824940798203  \\
            11.98572905894519  0.05509151043454102  \\
            14.04739252474516  0.05868979199028788  \\
            15.993049194858916  0.06208560348997746  \\
            17.948278918599495  0.06549812312048688  \\
        }
        ;
\end{axis}
\end{tikzpicture}
}%
         \caption{Inflow angle of \(\beta_1=70^\circ\).}
         \label{}
     \end{subfigure}

     \caption{Comparison of NACA experimental data (\(\blacktriangle\) markers) and XFOIL airfoil outputs with applied corrections (lines) for angles of attack vs lift \((c_\ell\)) and drag \((c_d\)) coefficients at various inflow angles (\(\beta_1\)) and solidities. \primary{Blue indicates solidity = 1.0}, \secondary{red indicates solidity = 1.25}, and \tertiary{green indicates solidity = 1.5}; \gray{Grey dashed lines indicate the uncorrected, smoothed XFOIL outputs}.}
     \label{fig:naca65410comps}
\end{figure}

\clearpage
\newpage
